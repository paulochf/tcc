%!TEX root = tcc.tex

% http://tex.stackexchange.com/a/99664
% \newglossaryentry{www}{
%     name={World Wide Web},
%     description={World Wide Web (WWW) é nome que se dá à rede mundial de
%             computadores interligados, que originou a Internet},
%     first={World Wide Web (WWW)},
%     long={World Wide Web}
% }

%%%%%%%%%%%%%%%%%%%%%%%%%%%%%%%%%%%%%%%%%%%%%%%%%%%%%%%%%%%%%%%%%%%%%%%%%%%%%%%%%%%%%%%%
%%%% CAPITULO 2
%%%%%%%%%%%%%%%%%%%%%%%%%%%%%%%%%%%%%%%%%%%%%%%%%%%%%%%%%%%%%%%%%%%%%%%%%%%%%%%%%%%%%%%%

\newglossaryentry{isp}{
    name={ISP},
    description={do inglês \emph{Internet Service Provider}; fornecedores de
    acesso a Internet, que são empresas que vendem serviço e equipamento que permitem
    o acesso de um computador pessoal acessar a Internet},
    first={fornecedor de acesso a Internet (\glsentryname{isp})},
    long={fornecedor de acesso a Internet},
    plural={\glsentryname{isp}s},
    firstplural={fornecedores de acesso a Internet (\glsentryname{isp}s)}
}

\newglossaryentry{mp3}{
    name={MP3},
    description={do inglês \emph{MPEG-1/2 Audio Layer 3}; formato patenteado de
    compressão de dados de áudio digital que usa um método de compressão de dados com
    perdas},
    long={formato de áudio \glsentryname{mp3}},
    first={\glsentrylong{mp3} (\glsentryname{mp3})},
    plural={\glsentryname{mp3}s}
}

\newglossaryentry{peer}{
    name={peer},
    description={do inglês \emph{peer-to-peer}; como são chamados cada nó da rede desse
    tipo},
    first={\glsentryname{peer} (um nó da rede)},
    firstplural={\glsplural{peer} (nós da rede)},
    plural={\glsentryname{peer}s}
}

\newglossaryentry{p2p}{
    name={P2P},
    description={do inglês \emph{peer-to-peer}; redes de arquitetura descentralizada e
    distribuída, onde cada nó (\emph{peer}) fornece e consome recursos},
    long={peer-to-peer},
    first={rede \glsentrylong{p2p} (\glsentryname{p2p})},
    plural={redes \glsentrylong{p2p}},
    firstplural={\glsentryplural{p2p} (\glsentryname{p2p})}
}

\newglossaryentry{audiogalaxy}{
    name={Audiogalaxy},
    description={rede P2P de compartilhamento de músicas MP3 criado em 1998},
    first={\glsentryname{audiogalaxy}.com},
    long={\glsentryname{audiogalaxy}}
}

\newglossaryentry{riaa}{
    name={RIAA},
    description={do inglês \emph{Recording Industry Association of America}; Associação
    da Indústria de Gravação da América, organização que representa as gravadoras
    musicais e distribuidores, e tem sido autora de ações judiciais devido a quebra de
    direitos autorais causada por compartilhamento indevido de música},
    first={RIAA (do inglês \emph{Recording Industry Association of America})},
    long={\glsentryname{riaa}}
}

\newglossaryentry{gnutella}{
    name={Gnutella},
    description={software de compartilhamento P2P desenvolvido por 3 programadores da
    empresa Nullsoft, recém adquirida da AOL Inc., lançado em 2000 sob a licença GPL.
    No dia seguinte, a AOL ordenou indisponibilizar o software alegando problemas
    legais e proibindo a continuação do desenvolvimento. Alguns dias depois, o
    protocolo já tinha sido alvo de engenharia reversa e já havia softwares que o
    implementavam},
    first={\glsentryname{gnutella}}
}

\newglossaryentry{anycast}{
    name={anycast},
    description={método de endereçamento e roteamento de rede onde os datagramas de um
    único remetente são roteados para um membro de um grupo de receptores potenciais que
    estão definidos pelo mesmo intervalo no endereço de destino. Geralmente é usado
    para serviços que demandem alta disponibilidade},
    first={\glsentryname{anycast}}
}

\newglossaryentry{edonkey}{
    name={eDonkey},
    description={lançado em 6 de setembro de 2000, o protocolo foi inaugurado juntamente
    com o software que o utilizava, o eDonkey2000, mas inúmeros softwares cliente para
    diferentes plataforas surgiram nos dias seguintes ao lançamento},
    first={\glsentryname{edonkey}}
}

\newglossaryentry{swarming}{
    name={swarming},
    description={também chamado de transmissão de \emph{arquivos por segmentação} ou de
    \emph{múltiplas fontes}, é a transmissão coordenada de um arquivo a partir de
    um ou vários locais onde este está disponível para um único destino, inclusive no
    caso de um arquivo em um local sendo transmitido em várias partes paralelas. Cabe ao
    software que faz o download juntar as partes no ponto de destino},
    first={\glsentryname{swarming}},
    plural={\glsentryname{swarming}s}
}

\newglossaryentry{hashtable}{
    name={tabela de hash},
    description={ou \emph{mapa de hash}, é uma estrutura de dados que cria uma lista de
    correspondência chave-valor, onde os dados são guardados como os valores e
    indexados por seus respectivos \emph{valores hash}},
    first={\glsentryname{hashtable}},
    firstplural={tabelas de hash},
    plural={\glsentryfirstplural{hashtable}},
}

\newglossaryentry{hashfunction}{
    name={função de hash},
    description={é uma função ou algoritmo matemático que mapeia um dado de comprimento
    variável em outro de comprimento fixo},
    first={\glsentryname{hashfunction}},
    firstplural={funções de hash},
    plural={\glsentryfirstplural{hashfunction}}
}

\newglossaryentry{hashvalue}{
    name={valor hash},
    description={ou \emph{hash}; valores gerados por uma tabela hash},
    first={\glsentryname{hashvalue}},
    firstplural={valores hash},
    plural={\glsentryfirstplural{hashvalue}}
}

\newglossaryentry{dht}{
    name={DHT},
    description={do inglês \emph{distributed hash table}; tabela hash distribuída, é um
    serviço de busca similar a uma tabela hash, mas na forma de sistema distribuído e
    descentralizada},
    first={tabela hash distribuída (\glsentryname{dht})},
    firstplural={tabelas hash distribuídas (\glsentryplural{dht})},
    long={DHT},
    plural={\glsentrylong{dht}s}
}

\newglossaryentry{kademlia}{
    name={Kademlia},
    description={DHT usado em redes P2P que especifica a estrutura da rede e a troca de
    informações através de buscas de nós, guardando as localizações de recursos que
    estão na rede},
    first={\glsentryname{kademlia}}
}

\newglossaryentry{betatester}{
    name={beta tester},
    description={usuários de uma versão beta de um software},
    first={\glsentryname{betatester}},
    plural={\glsentryname{betatester}s}
}

\newglossaryentry{tracker}{
    name={tracker},
    description={em português, rastreador; servidor que funciona como um ponto de
    encontro de peers},
    first={\glsentryname{tracker}},
    plural={\glsentryname{tracker}s}
}

%%%%%%%%%%%%%%%%%%%%%%%%%%%%%%%%%%%%%%%%%%%%%%%%%%%%%%%%%%%%%%%%%%%%%%%%%%%%%%%%%%%%%%%%
%%%% CAPITULO 3
%%%%%%%%%%%%%%%%%%%%%%%%%%%%%%%%%%%%%%%%%%%%%%%%%%%%%%%%%%%%%%%%%%%%%%%%%%%%%%%%%%%%%%%%

\newglossaryentry{magnetlink}{
    name={magnet link},
    description={em português, link magnético; padrão aberto, definido por convenção,
    de esquema de caminho de Internet (URI) utilizado para localizar recursos de rede
    BitTorrent para download},
    first={link magnético (\glsentryname{magnetlink})},
    long={link magnético},
    plural={links magnéticos},
    firstplural={\glsentryplural{magnetlink} (\glsentryname{magnetlink}s)}
}

\newglossaryentry{uri}{
    name={URI},
    description={do inglês \emph{Uniform Resource Identifier}; Identificador Uniforme
    de Recursos, é uma cadeia de caracteres usada para identificar algum recurso,
    especificando algum protocolo e um caminho. Por exemplo, o URI `file:///arquivo.
    txt' indica um arquivo no computador local (nome de esquema `file'), enquanto
    `http://pagina.com' se refere a uma página de Internet (nome de esquema `http')},
    first={identificador uniforme de recursos (\glsentryname{uri})},
    firstplural={identificadores uniformes de recursos (\glsentryname{uri})},
    long={identificador uniforme de recursos},
    plural={\glsentrylong{uri}s}
}

\newglossaryentry{url}{
    name={URL},
    description={do inglês \emph{Uniform Resource Locator}; Localizador Uniforme
    de Recursos, é uma cadeia de caracteres usada para identificar algum recurso na
    Internet, especificando algum protocolo de comunicação},
    first={localizador uniforme de recursos (\glsentryname{url})},
    firstplural={localizadores uniformes de recursos (\glsentryname{url})},
    long={localizador uniforme de recursos},
    plural={\glsentrylong{url}s}
}

\newglossaryentry{metadata}{
    name={metadado},
    description={dados sobre outros dados; informação sobre outra informação},
    first={\glsentryname{metadata}},
    plural={\glsentryname{metadata}s},
    firstplural={\glsentryplural{metadata}}
}

\newglossaryentry{announce}{
    name={announce},
    description={endereço web (URL) do tracker},
    first={\glsentryname{announce}},
    plural={\glsentryname{announce}s},
    firstplural={\glsentryplural{announce}}
}