%!TEX root = tcc.tex

% http://tex.stackexchange.com/a/99664
% \newglossaryentry{www}{
%     name={World Wide Web},
%     description={World Wide Web (WWW) é nome que se dá à rede mundial de
%             computadores interligados, que originou a Internet},
%     first={World Wide Web (WWW)},
%     long={World Wide Web}
% }

%%%%%%%%%%%%%%%%%%%%%%%%%%%%%%%%%%%%%%%%%%%%%%%%%%%%%%%%%%%%%%%%%%%%%%%%%%%%%%%%%%%%%%%%
%%%% CAPITULO 2
%%%%%%%%%%%%%%%%%%%%%%%%%%%%%%%%%%%%%%%%%%%%%%%%%%%%%%%%%%%%%%%%%%%%%%%%%%%%%%%%%%%%%%%%

\newglossaryentry{isp}{
    name={ISP},
    description={do inglês \emph{Internet Service Provider}; fornecedores de
    acesso à Internet, que são empresas que vendem serviço e equipamento que permitem
    o acesso de um computador pessoal à Internet},
    first={fornecedor de acesso à Internet (\glsentryname{isp})},
    long={fornecedor de acesso à Internet},
    plural={\glsentryname{isp}s},
    firstplural={fornecedores de acesso a Internet (\glsentryname{isp}s)}
}

\newglossaryentry{mp3}{
    name={MP3},
    description={do inglês \emph{MPEG-1/2 Audio Layer 3}; formato patenteado de
    compressão de dados de áudio digital, que usa um método de compressão de dados com
    perdas},
    long={formato de áudio \glsentryname{mp3}},
    first={\glsentrylong{mp3} (\glsentryname{mp3})},
    plural={\glsentryname{mp3}s},
    firstplural={\glsentryplural{mp3}}
}

\newglossaryentry{peer}{
    name={peer},
    description={em português, significa par, colega; nome que se dá a cada nó da rede,
    ou seja, a um computador conectado},
    long={\glsentryname{peer}},
    first={nó da rede (\emph{\glsentryname{peer}})},
    plural={\glsentryname{peer}s},
    firstplural={nós da rede (\emph{\glsplural{peer}})}
}

\newglossaryentry{p2p}{
    name={P2P},
    description={do inglês \emph{peer-to-peer}; redes de arquitetura descentralizada e
    distribuída, onde cada nó (\glink{peer}) fornece e consome recursos},
    long={peer-to-peer},
    first={rede \glsentrylong{p2p} (\glsentryname{p2p})},
    plural={redes \glsentrylong{p2p}},
    firstplural={\glsentryplural{p2p} (\glsentryname{p2p})}
}

\newglossaryentry{audiogalaxy}{
    name={Audiogalaxy},
    description={Rede \glink{p2p} de compartilhamento de músicas \glink{mp3} criado em
    1998},
    first={\glsentryname{audiogalaxy}.com},
    long={\glsentryname{audiogalaxy}},
}

\newglossaryentry{riaa}{
    name={RIAA},
    description={do inglês \emph{Recording Industry Association of America}; Associação
    da Indústria de Gravação da América, organização que representa as gravadoras
    musicais e distribuidores, e tem sido autora de ações judiciais devido a quebra de
    direitos autorais causada por compartilhamento indevido de música},
    long={\glsentryname{riaa} (do inglês \emph{Recording Industry Association of
    America})},
    first={\glsentrylong{riaa}},
}

\newglossaryentry{gnutella}{
    name={Gnutella},
    description={software de compartilhamento \glink{p2p} desenvolvido por 3
    programadores da empresa Nullsoft, recém adquirida da AOL Inc., lançado em 2000 sob
    a licença GPL. No dia seguinte ao lançamento, a AOL ordenou indisponibilizar o
    software, alegando problemas legais e proibindo a continuação do desenvolvimento.
    Alguns dias depois, o protocolo já tinha sido alvo de engenharia reversa e já havia
    softwares que o implementavam},
    first={\glsentryname{gnutella}},
}

\newglossaryentry{anycast}{
    name={anycast},
    description={método de endereçamento e roteamento de rede onde os datagramas de um
    único remetente são roteados para um membro de um grupo de receptores potenciais que
    estão definidos pelo mesmo intervalo no endereço de destino. Geralmente é usado
    para serviços que demandem alta disponibilidade},
    first={\glsentryname{anycast}}
}

\newglossaryentry{edonkey}{
    name={eDonkey},
    description={lançado em 6 de setembro de 2000, o protocolo foi inaugurado juntamente
    com o software que o utilizava, o eDonkey2000, mas inúmeros softwares cliente para
    diferentes plataformas surgiram nos dias seguintes ao lançamento},
    first={\glsentryname{edonkey}}
}

\newglossaryentry{swarming}{
    name={swarming},
    description={também chamado de transmissão de arquivos por segmentação ou de
    múltiplas fontes, é a transmissão em paralelo de um arquivo, a partir de
    um ou vários locais onde estiver disponível, para um único destino. Cabe ao
    software do destinatário juntar as partes recebidas},
    first={enxame de \glsentryplural{peer} (\emph{\glsentryname{swarming}})},
    firstplural={enxames de \glsentryplural{peer} (\emph{\glsentryname{swarming}s})},
    plural={\glsentryname{swarming}s}
}

\newglossaryentry{hashtable}{
    name={tabela hash},
    description={ou \emph{mapa de hash}, é uma estrutura de dados que cria uma lista de
    correspondência chave-valor, onde os dados são guardados como valores e indexados
    por seus respectivos \emph{valores hash}},
    first={\glsentryname{hashtable}},
    firstplural={tabelas hash},
    plural={\glsentryfirstplural{hashtable}},
}

\newglossaryentry{hashfunction}{
    name={função de hash},
    description={é uma função ou algoritmo matemático que mapeia um dado de comprimento
    variável em outro de comprimento fixo},
    first={\glsentryname{hashfunction}},
    firstplural={funções de hash},
    plural={\glsentryfirstplural{hashfunction}}
}

\newglossaryentry{hashvalue}{
    name={valor hash},
    description={ou \emph{hash}; valores gerados por uma \glink{hashfunction}},
    first={\glsentryname{hashvalue}},
    firstplural={valores hash},
    plural={\glsentryfirstplural{hashvalue}}
}

\newglossaryentry{dht}{
    name={DHT},
    description={do inglês \emph{distributed hash table}; \glink{hashtable}
    distribuída, ou seja, é um serviço de busca similar a uma \glink{hashtable}, mas
    descentralizada e na forma de sistema distribuído},
    long={\glink{hashtable} distribuída},
    first={\glsentrylong{dht} (\glsentryname{dht})},
    firstplural={\glinkpl{hashtable} distribuídas (\glsentryname{dht}s)},
    plural={\glsentryname{dht}s}
}

\newglossaryentry{kademlia}{
    name={Kademlia},
    description={\glink{dht} usado em \glink{p2p} que especifica a estrutura da rede e
    a troca de informações através de buscas de nós, guardando as localizações de
    recursos que estão na rede},
    first={\glsentryname{kademlia}}
}

\newglossaryentry{betatester}{
    name={beta tester},
    description={usuários de uma versão beta de um software},
    long={usuário da versão não final (\emph{\glsentryname{betatester}})}
    first={\glsentrylong{betatester}},
    plural={\glsentryname{betatester}s}
}

\newglossaryentry{tracker}{
    name={tracker},
    description={em português, rastreador; servidor que funciona como um ponto de
    encontro de \glink{peer}},
    first={rastreador (\emph{\glsentryname{tracker}})},
    firstplural={rastreadores (\emph{\glsentryname{tracker}s})},
    plural={\glsentryname{tracker}s}
}

%%%%%%%%%%%%%%%%%%%%%%%%%%%%%%%%%%%%%%%%%%%%%%%%%%%%%%%%%%%%%%%%%%%%%%%%%%%%%%%%%%%%%%%%
%%%% CAPITULO 3
%%%%%%%%%%%%%%%%%%%%%%%%%%%%%%%%%%%%%%%%%%%%%%%%%%%%%%%%%%%%%%%%%%%%%%%%%%%%%%%%%%%%%%%%

\newglossaryentry{leecher}{
    name={leecher},
    description={em português, sugador; nome dado ao \glink{peer} que ainda não
    terminou um download de um torrent},
    long={sugador (\emph{\glsentryname{leecher}})},
    first={\glsentrylong{leecher}},
    firstplural={sugadores (\emph{\glsentryname{leecher}s})},
    plural={\glsentryname{leecher}s}
}

\newglossaryentry{seeder}{
    name={seeder},
    description={em português, semeador; nome dado ao \glink{peer} que já terminou um
    download de um torrent e que, por ainda estar conectado à rede, fornece partes a
    possíveis interessados},
    long={semeador (\emph{\glsentryname{seeder}})},
    first={\glsentrylong{seeder}},
    firstplural={semeadores (\emph{\glsentryname{seeder}s})},
    plural={\glsentryname{seeder}s}
}

\newglossaryentry{metadata}{
    name={metadado},
    description={dados sobre outros dados; informação sobre outra informação},
    first={\glsentryname{metadata}},
    plural={\glsentryname{metadata}s},
    firstplural={\glsentryplural{metadata}}
}

\newglossaryentry{torrentfile}{
    name={arquivo .torrent},
    description={arquivo que contém \glinkpl{metadata}, como a lista dos nomes dos
    arquivos a serem baixados e seus tamanhos, \glinkpl{checksum} das partes desses
    arquivos, endereços de um ou mais \glinkpl{tracker}, etc, formando um pacote chamado
    \glink{torrent}},
    first={arquivo de extensão .torrent},
    firstplural={arquivos de extensão .torrent},
    plural={arquivos .torrent}
}

\newglossaryentry{torrent}{
    name={torrent},
    description={conjunto de um ou mais arquivos definidos por um \glink{torrentfile}},
    first={torrent},
    plural={\glsentryname{torrent}s},
    firstplural={\glsentryplural{torrent}}
}

\newglossaryentry{checksum}{
    name={checksum},
    description={em português, soma de verificação; bloco de dados de tamanho fixo
    gerado por algum algoritmo de soma para verificação, usado no certificado de
    integridade contra problemas durante a transmissão ou de armazenamento (leitura
    ou escrita)},
    first={verificação por soma (\emph{\glsentryname{checksum}})},
    firstplural={verificações por soma (\emph{\glsentryname{checksum}s})},
    plural={\glsentryname{checksum}s}
}

\newglossaryentry{swarm}{
    name={swarm},
    description={em português, enxame; grupo de \glinkpl{peer} que estão compartilhando
    dados de um mesmo torrent num determinado momento},
    first={enxame (\emph{\glsentryname{swarm}})},
    firstplural={enxames (\emph{\glsentryname{swarm}s})},
    plural={\glsentryname{swarm}s}
}

\newglossaryentry{bencode}{
    name={bencode},
    description={``codificação B'', pronunciado \emph{bê encode}; formato de
    codificação compacta de arquivos \glink{torrent} para transmissão de
    \glinkpl{metadata}},
    first={codificação B (\glsentryname{bencode})}
}

\newglossaryentry{string}{
    name={string},
    description={sequência de caracteres},
    first={sequência de caracteres (\emph{\glsentryname{string}})},
    firstplural={sequências de caracteres (\emph{\glsentryname{string}s})},
    plural={\glsentryname{string}s}
}

\newglossaryentry{magnetlink}{
    name={magnet link},
    description={em português, link magnético; padrão aberto, definido por convenção,
    de esquema de \glink{uri} utilizado para localizar recursos de rede BitTorrent para
    download},
    first={link magnético (\emph{\glsentryname{magnetlink}})},
    plural={\glsentryname{magnetlink}s},
    firstplural={links magnéticos (\glsentryname{magnetlink}s)}
}

% \textcolor{Bittersweet}{\texttt{file}} não é substituível por novo comando
\newglossaryentry{uri}{
    name={URI},
    description={do inglês \emph{Uniform Resource Identifier}; Identificador
    Uniforme de Recursos, é uma \glink{string} usada para identificar algum recurso,
    especificando algum protocolo e um caminho. Por exemplo, o URI
    \textcolor{Bittersweet}{\texttt{file:///arquivo.txt}} indica um arquivo computador
    local (nome de esquema \textcolor{Bittersweet}{\texttt{file}}), enquanto
    \textcolor{Bittersweet}{\texttt{http://pagina.com}} se refere a uma página de
    Internet (nome de esquema \textcolor{Bittersweet}{\texttt{http}})},
    first={identificador uniforme de recursos (\glsentryname{uri})},
    firstplural={identificadores uniformes de recursos (\glsentryname{uri}s)},
    long={identificador uniforme de recursos},
    plural={\glsentryname{uri}s}
}

\newglossaryentry{url}{
    name={URL},
    description={do inglês \emph{Uniform Resource Locator}; Localizador Uniforme de
    Recursos, é uma \glink{string} usada para identificar algum recurso na Internet,
    especificando algum protocolo de comunicação},
    first={localizador uniforme de recursos (\glsentryname{url})},
    firstplural={localizadores uniformes de recursos (\glsentryname{url}s)},
    long={localizador uniforme de recursos},
    plural={\glsentryname{url}s}
}

\newglossaryentry{caseinsensitive}{
    name={case insensitive},
    description={em português, insensível ao tamanho das letras (maiúsculas ou
    minúsculas); uma \glink{string} \glsentryname{caseinsensitive} não é
    diferenciada com letras iguais em caixas alta ou baixa. Assim,
    \textcolor{Bittersweet}{\texttt{FOO}}, \textcolor{Bittersweet}{\texttt{Foo}}
    e \textcolor{Bittersweet}{\texttt{foo}} seriam \glinkpl{string} dadas como iguais},
    first={insensível ao tamanho das letras (\emph{\glsentryname{caseinsensitive}})},
}

\newglossaryentry{querystring}{
    name={query string},
    description={em português, \glink{string} de busca; parte de uma \glink{url} que
    possui um dicionário de dados a serem transmitidos para alguma aplicação de
    Internet, determinado por um caractere \textcolor{Bittersweet}{\texttt{?}}. Por
    exemplo, em \textcolor{Bittersweet}{\texttt{http://um/endereco/qualquer/?key1=value1\&key2=value2}}},
    first={\glink{string} de busca (\emph{\glsentryname{querystring}})},
    firstplural={\glinkpl{string} de busca (\emph{\glsentryname{querystring}s})},
    plural={\glsentryname{querystring}s}
}

\newglossaryentry{httpget}{
    name={HTTP GET},
    description={método de requisição do protocolo HTTP que permite uso de
    \glinkpl{querystring} para troca de informações},
    first={requisição de método GET do protocolo HTTP (HTTP GET)},
    firstplural={requisições de método GET do protocolo HTTP (HTTP GET)}
}

\newglossaryentry{urlencode}{
    name={URL encode},
    description={em português, codificação de \glink{url}; por possuir caracteres
    especiais, \glinkpl{url} devem converter esses caracteres quando se desejar
    transmití-los. Letras (A-Z e a-z), números (0-9), caracteres
    (\textcolor{Bittersweet}{\texttt{.\~{}\textunderscore},}); espaços são convertidos
    para \textcolor{Bittersweet}{\texttt{+}} ou \textcolor{Bittersweet}{\texttt{\%20}};
    o restante é convertido para o respectivo valor em hexadecimal do caractere
    convertido para UTF-8},
    first={codificação de endereços \glsentryname{uri} e \glsentryname{url} (\emph{\glsentryname{urlencode}})}
}

\newglossaryentry{urn}{
    name={URN},
    description={do inglês \emph{Uniform Resource Name}; Nome Uniforme de Recursos, é o
    nome histórico dado a uma \glink{uri} que usa o esquema
    \textcolor{Bittersweet}{\texttt{urn}}. Sua sintaxe é
    \textcolor{Bittersweet}{\texttt{urn:<NID>:<NSS>}}, onde
    \textcolor{Bittersweet}{\texttt{urn}} é o prefixo \glink{caseinsensitive},
    \textcolor{Bittersweet}{\texttt{<NID>}} é o identificador de espaço de nomes, que
    determina a interpretação sintática de \textcolor{Bittersweet}{\texttt{<NSS>}},
    que é a \glink{string} específica do espaço de nomes usado},
    first={nome uniforme de recursos (\glsentryname{urn})},
    firstplural={nomes uniformes de recursos (\glsentryname{urn})},
    long={nome uniforme de recursos},
    plural={\glsentrylong{urn}s}
}

\newglossaryentry{announce}{
    name={announce},
    description={endereço \glink{url} do \glink{tracker} para troca de informações,
    onde este recebe de \glinkpl{peer} informações sobre as respectivas situações com
    relação a um torrent específico naquele momento, as processa e então responde
    informando sobre a situação geral daquele torrent e uma lista de outros
    \glsentryplural{peer} conectados naquele momento},
    first={endereço web de contato do \glink{tracker} (\emph{\glsentryname{announce}})},
    firstplural={endereços web de contato do \glink{tracker}
    (\emph{\glsentryname{announce}s})},
    plural={\glsentryname{announce}s}
}

\newglossaryentry{scrape}{
    name={scrape},
    description={do inglês \emph{screen scraping}; sondagem de tela, que era um
    processo automatizado para recolhimento de informações sobre torrents que acessavam
    páginas web dos \glinkpl{tracker}. Por questões práticas e econômicas, foi
    estabelecido que neste endereço \glink{url} do \glsentryname{tracker}, ao contrário
    do \glink{announce}, o \glsentryname{tracker} somente informa as quantidades de
    \glinkpl{peer} que estão participando de uma lista de torrents naquele momento},
    first={endereço web do \glink{tracker} para informações básicas
    (\emph{\glsentryname{scrape}})},
    firstplural={endereços web do \glink{tracker} para informações básicas
    (\emph{\glsentryname{scrape}s})},
    plural={\glsentryname{scrape}s}
}

\newglossaryentry{proxy}{
    name={proxy},
    description={servidor que funciona como intermediário de acessos, repassando
    requisições de um computador cliente para o servidor de destino},
    first={servidor \emph{\glsentryname{proxy}}},
    firstplural={servidores \emph{\glsentryname{proxy}} (\emph{proxies})}
    plural={proxies},
}

\newglossaryentry{nat}{
    name={gateway NAT},
    description={do inglês \emph{Network Address Translation}; tradução de endereço de
    rede, é uma funcionalidade empregada por roteadores de rede, que por associarem um
    endereço de Internet (o fornecido pelo \glink{isp}) para vários dispositivos dentro
    dessa rede, precisa manter uma tabela de tradução de endereços para que saiba rotear
    os pacotes de dados transmitidos entre as redes interna e externa},
    first={tradução de endereço de rede (\emph{\glsentryname{nat}})}
}

\newglossaryentry{tcp}{
    name={TCP},
    description={do inglês \emph{Transmission Control Protocol}; protocolo de controle
    de transmissão, é um dos protocolos principais de Internet (IP). Pertencente à
    camada de transporte de dados de rede, provê conexões cujos pacotes de dados são
    pesados, mas com entregas confiáveis, ordenadas e com verificação de erros e de
    congestionamento},
    first={protocolo \glsentryname{tcp}}
}

\newglossaryentry{udp}{
    name={UDP},
    description={do inglês \emph{Transmission Control Protocol}; protocolo de controle
    de transmissão, é um dos protocolos principais de Internet (IP). Pertencente à
    camada de transporte de dados de rede, provê conexões cujos pacotes de dados são
    leves, mas com entregas não confiáveis, desordenadas e sem verificação de erros
    (feita na camada de aplicação) e de congestionamento},
    first={protocolo \glsentryname{udp}}
}

\newglossaryentry{thread}{
    name={thread},
    description={\todo{escrever}},
    first={\glsentryname{thread}}
}

\newglossaryentry{pool}{
    name={pool},
    description={em português, poço de recursos; conjunto de recursos alocados em
    memória que são pré-computados, a fim de estarem prontos para serem utilizados},
    first={poço de recursos (\emph{\glsentryname{pool}})},
    plural={\glsentryname{pool}s},
    firstplural={poços de recursos (\emph{\glsentryplural{pool}})}
}

\newglossaryentry{kbucket}{
    name={$k$-bucket},
    description={\todo{escrever}},
    first={$k$-balde (\emph{\glsentryname{kbucket}})},
    plural={$k$-\glsentryplural{bucket}},
    firstplural={$k$-baldes (\emph{\glsentryplural{kbucket}})}
}

\newglossaryentry{bucket}{
    name={bucket},
    description={\todo{escrever}},
    first={balde (\emph{\glsentryname{bucket}})},
    plural={\glsentryname{bucket}s},
    firstplural={baldes (\emph{\glsentryplural{bucket}})}
}