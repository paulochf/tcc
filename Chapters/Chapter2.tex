%!TEX root = ../tcc.tex

\chapter{Napster, Gnutella, eDonkey e BitTorrent}

Para entendermos como e por que o BitTorrent se tornou o que é hoje, devemos voltar um
pouco no tempo e rever a história que precedeu à sua criação, que é no fim da década dos
anos 1990.

\section{Período pré-torrent}

Entre o final dos anos 80 e o início dos 90 \cite{wiki:fs,wiki:fs-timeline}, a
Internet deixou de ser uma rede de computadores usada somente por entidades
governamentais, laboratórios de pesquisa e universidades, passando a ter seu acesso
comercializado para o público em geral pelos \glspl{isp} \cite{wiki:isp}. Com o
advento do \gls{mp3} \cite{wiki:mp3}, no final de 1991, e do seu primeiro reprodutor
de áudio \gls*{mp3} Winamp, o tráfego da Internet cresceu devido ao aumento da troca
direta desse tipo de arquivo.

Entre 1998 e 1999, dois sites de compartilhamento gratuito de músicas foram criados: o
MP3.com \cite{wiki:mp3.com}, que era um site de divulgação de bandas independentes,
e o \gls{audiogalaxy} \cite{wiki:audiogalaxy.com,revista:pnp}. Mais popular que o
primeiro, o \gls*{audiogalaxy} era um site de busca de músicas, sendo que o download e
upload eram feitos a partir de um software cliente. A lista de músicas procuradas era
enviada pelo site para o computador onde o usuário tinha instalado o cliente, que então
conectava com o computador do outro usuário, que era indicado pelo servidor. A lista
possuía todos os arquivos que um dia passaram pela sua rede. Se algum arquivo fosse
requisitado, mas o usuário que o possuísse não estivesse conectado, o servidor central
do \gls*{audiogalaxy} fazia a ponte, pegando o arquivo para si e enviando-o para o
cliente do requisitante em seu próximo login.

Os 3 anos seguintes à criação desses dois sites foram muito produtivo ao mundo das
\glspl{p2p} de modo geral, onde surgiram alguns protocolos desse paradigma e inúmeros
softwares que os implementavam. Os mais relevantes foram o Napster, o Gnutella, o
eDonkey e o BitTorrent.

\subsection{Napster}

Em maio de 1999, surgiu o Napster \cite{wiki:napster}, um programa de compartilhamento
de \gls*{mp3} que inovou por desfigurar o usual modelo cliente-servidor, no qual um
servidor central localizava os arquivos nos usuários e fazia a conexão entre estes,
onde ocorriam as transferências. O Napster foi contemporâneo ao \gls*{audiogalaxy}, e
ambos fizeram muito sucesso por cerca de 2 anos, até que começaram as ações judiciais
contra ambos os programas.

Não demorou muito tempo para a indústria da música entrar em ação contra a troca de
arquivos protegidos por direitos autorais sem autorização dos detentores de tais
direitos pela Internet. Seu primeiro alvo foi o Napster, em dezembro de 1999, quando a
\gls{riaa} entrou com processo judicial representando várias gravadoras, alegando
quebra de direitos autorais \cite{site:napster-riaa}. Em abril de 2000, foi a vez da
banda Metallica processar, como retaliação à descoberta de que uma música ainda não
lançada oficialmente já circulava na rede
\cite{site:napster-metallica,site:napster-metallica-orig}. Um mês depois, outra ação
judicial, agora encabeçada pelo rapper Dr. Dre, que tinha feito um pedido formal para a
retirada de seu material de circulação \cite{site:napster-drdre-orig}. Isso fez com que
o Napster recebesse atenção da mídia, ganhando popularidade e atingindo os 20 milhões
de usuários em meados do ano 2000 \cite{site:napster-use-2000}.

Em 2001, esses imbróglios judiciais resultaram numa liminar federal que ordenou que o
Napster retirasse o conteúdo protegido das entidades representadas pela \gls*{riaa}. O
Napster tentou cumprir a ordem judicial, mas a juíza do caso não ficou satisfeita,
ordenando então, em julho daquele ano, o desligamento da rede enquanto não conseguisse
controlar o conteúdo que trafegava ali \cite{wiki:napster}. Em setembro, o Napster fez
um acordo \cite{wiki:napstervsriaa}, onde pagou 26 milhões de dólares pelos danos já
causados, pelo uso indevido de músicas, e mais 10 milhões de dólares pelos danos futuros
envolvendo royalties. Para pagar esse valor, o Napster tentou cobrar o serviço que
prestava de seus usuários, que acabaram migrando de rede \gls*{p2p}, inclusive para o
\gls*{audiogalaxy}. Não conseguindo quitar o acordo, em 2002, o Napster decreta
falência e é forçado a liquidar seus ativos. De lá para cá, foi negociado algumas vezes,
e, atualmente, pertence ao site Rhapsody \cite{site:napster-rhapsody}.

O sucesso do Napster, mesmo que por curto período tempo, mostrou o potencial que as
redes \gls*{p2p} poderiam ter, e com isso, novos softwares e protocolos de redes foram
sendo lançados, sempre tentando se diferenciar dos seus antecessores, a fim de não serem
novos alvos de ações judiciais. A solução para isso foi tentar descentralizar os
mecanismos de indexação e de busca, que foram os pontos fracos do Napster.

\subsection{Gnutella}

O sucessor foi o \gls{gnutella}, que em março de 2000 \cite{wiki:gnutella}, surgiu como
uma resposta de domínio público, feita com ``gambiarras'', para os problemas que o
Napster encontrou com relação às acusações de violação de direitos autorais. Enquanto o
Napster possuía em sua estrutura um servidor central, fato este que foi explorado em
seu julgamento como prova de que o sistema encorajava a violação de direitos autorais,
o \gls*{gnutella} foi modelado como um sistema \gls*{p2p} puro, onde todos os
\glspl*{peer} são completamente iguais, sendo responsáveis pelos seus próprios atos.

O \gls*{gnutella} disponibiliza arquivos da mesma forma que o Napster \cite{book:birman}
, mas sem a limitação de ser de em formato de música, ou seja, qualquer arquivo pode
ser compartilhado. A diferença mais significativa entre os dois protocolos é o
algoritmo de busca: a abordagem do \gls*{gnutella} é baseada numa forma de \gls{anycast}
. Isso envolve duas partes: a primeira, sobre como cada usuário é conectado a outros
nós e mantém a lista dessas conexões atualizada; a segunda, sobre como ele trata as
buscas e trabalha inundando de pedidos para todos os nós que estão a uma certa
distância do usuário (nó-cliente). Por exemplo, se a distância limite for de 4, então
todos os nós que estiverem a 4 passos a partir do cliente serão verificados, começando
a partir dos mais próximos. Eventualmente, algum nó possuirá o arquivo requisitado e
responderá, e assim será feita a transferência desse arquivo. Muitos softwares que
implementam o protocolo vão além dessa funcionalidade básica de download simples,
tentando transferir de forma paralela partes diferentes do arquivo desejado de nós
diferentes, de forma a amenizar eventuais problemas de velocidade de rede.

Assim, experiências sugerem que o sistema escala para um tamanho maior, tornando o
mecanismo de \gls*{anycast} extremamente caro, e, em alguns casos, até proibitivo. O
problema ocorre nas buscas por arquivos menos populares, onde será necessário um maior
número de nós perguntados.

O \gls*{gnutella} ainda teve uma segunda versão \cite{wiki:gnutella2}, no final de 2002,
onde utilizou o mesmo protocolo que o original, porém, organizando a rede de
\glspl*{peer} em folhas (\emph{leafs}) e \emph{hubs}. Um \emph{hub} poderia
ter centenas de conexões à outras folhas, mas apenas 7 (em média) a outros \emph{hubs},
enquanto uma folha se conectaria a apenas 2 \emph{hubs} simultaneamente. Essa nova
topologia, somada com uma nova tabela de índice de arquivos das folhas mantida pelos
\emph{hubs} onde estavam conectados, melhorou o desempenho das buscas, que era ruim na versão antiga.

\subsection{eDonkey}

O protocolo \gls{edonkey} inovou em muitos aspectos em relação aos seus precursores,
tendo papel fundamental na história das redes \gls*{p2p}, consolidando-se como
ferramenta de compartilhamento especializado em arquivos grandes.

O \gls*{edonkey} implementou o primeiro método de download por \gls{swarming}, onde
\glspl*{peer} fazem downloads de diferentes partes de um arquivo e de \glspl*{peer}
diferentes, utilizando de forma efetiva a largura de banda de rede para todos os
\glspl*{peer}, ao invés de ficar limitado somente à banda de um único \gls*{peer}.

Outra melhoria deu-se na busca de arquivos: no seu lançamento, os servidores eram
separados entre si, porém, nas versões seguintes, permitiu-se que eles formassem uma
rede de buscas. Isso possibilitou que os servidores repassassem buscas de seus clientes
conectados localmente a outros servidores, facilitando a localização de \glspl*{peer}
conectados em qualquer servidor da rede de buscas, aumentando a capacidade de download
do enxame.

Diferentemente do Napster, o \gls*{edonkey} utilizou-se de \glspl{hashvalue} de arquivos
nos resultados de busca ao invés dos simples nomes dos arquivos. As buscas geradas
pelos usuários eram baseadas em palavras-chave e comparadas com a lista de nomes de
arquivos armazenada no servidor, mas o servidor retornava uma lista de pares de nomes
de arquivos com seus respectivos \glspl*{hashvalue}. Enfim, quando o usuário
selecionasse o arquivo desejado, o cliente iniciaria o download do arquivo usando o seu
\gls*{hashvalue}. Desse modo, um arquivo poderia ter muitos nomes entre os diferentes
\glspl*{peer} e servidores, mas seria considerado idêntico para download se possuísse o
mesmo \gls*{hashvalue}.

A arquitetura da rede em dois níveis, usando cliente e servidor, alcançou um meio termo
entre as redes centralizadas (como o Napster) e as descentralizadas (como o
\gls*{gnutella}), já que o servidor central no primeiro era um alvo garantido para ações
judiciais, enquanto o segundo mostrou-se inviável à propositura de tais ações, devido
ao tráfego massivo de buscas entre \glspl*{peer}.

Por fim, a inovação mais importante foi o uso de \glspl{dht}, em específico o
\gls{kademlia}, como algoritmo de indexação e busca nos servidores centrais dos
arquivos através da rede \gls*{edonkey}. Além de ser uma das razões da melhoria no
desempenho das pesquisas, os \glspl*{dht} possuem ainda outras características, tais
como tolerância a falhas e escalabilidade.

\section{Nascimento do BitTorrent}

Em meados dos anos 1990, Bram Cohen era um programador que tinha largado a faculdade no
segundo ano do curso de Ciência da Computação, da Universidade de Buffalo -- Nova
Iorque, para trabalhar em empresas ``ponto com''. A última delas foi a MojoNation, uma
empresa que desenvolvia um software de distribuição de arquivos criptografados por
\gls*{swarming}.

Em abril de 2001, Bram saiu da MojoNation e começou a modelar o protocolo BitTorrent,
lançando a primeira implementação usando a linguagem Python, em julho de 2001. Em
fevereiro de 2002, ele apresentou o seu trabalho na CodeCon \cite{site:codecon}, e na
mesma época começou a testá-lo, usando como chamariz uma coleção de material
pornográfico para atrair \glspl{betatester} \cite{site:bramcohen}. Assim, o software
começou a ser usado imediatamente.

Nesse meio tempo, Bram ainda passou pela Valve \cite{wiki:bramcohen}, empresa de
desenvolvimento de jogos, trabalhando no sistema de distribuição online do jogo Half
Life 2. Em 2004, saiu da Valve e voltou o foco ao Torrent. Em setembro, fundou a
BitTorrent Inc. com seu irmão Ross Cohen e o parceiro de negócios Ashwin Navin, sendo
então responsável pelo desenvolvimento do protocolo. Ainda naquele ano, surgiram os
primeiros programas de televisão e filmes compartilhados na rede através do BitTorrent,
o que popularizou o protocolo.

Em maio de 2005, a empresa lançou uma nova versão do BitTorrent, que não precisava de
\glspl{tracker}, juntamente com um site de buscas de conteúdo torrent na Internet. Em
setembro, a empresa recebeu investimento na ordem de \$8.5 milhões de dólares. No final
desse ano, a BitTorrent Inc. e a MPAA (\emph{Motion Picture Association of America},
associação americana de produtoras de filmes) fizeram um acordo \cite{wiki:mpaa}
visando a retirada dos conteúdos não autorizados dos representados pela associação, o
que não evitou a pirataria, pois já havia outros sites de busca de torrent sem
restrições de conteúdo, como o TorrentSpy, Mininova, The Pirate Bay, etc.

\section{Mundo pós-torrent}

Desde o fechamento de seu site de buscas, a BitTorrent Inc. tem desenvolvido outros
softwares baseados na tecnologia \gls*{p2p} \cite{site:bittorrent}, como transmissão de
vídeos ao vivo (BitTorrent Live), sincronização de arquivos entre computadores ligados
à Internet (BitTorrent Sync), publicação e distribuição de conteúdo de artistas a seus
fãs (BitTorrent Bundles), entre outros serviços comerciais.

Como protocolo, o BitTorrent criou um novo paradigma de transmissão de informações pela
Internet, sendo utilizado de inúmeras formas e motivos, tais como:

\begin{itemize}
    \item alguns softwares de podcasting, como o Miro \cite{site:miro}, passaram a
        usar o protocolo como forma de lidar com a grande quantidade de downloads de
        programas online;

    \item o site da gravadora DGM Live fornece o conteúdo via torrent após a venda
        \cite{site:dgm};

    \item VODO \cite{site:vodo} é um site de divulgação e distribuição de filmes sob a
        licença Creative Commons e que faz a publicação em outros sites de busca de
        torrents;

    \item canais como a americana CBC \cite{site:cbc} e a holandesa VPRO
        \cite{site:vpro} já disponibilizaram programas de sua grade para download. A
        norueguesa NRK o faz para conteúdos em HD \cite{site:nrk} e, apesar de algumas
        restrições de direitos, tem aumentado a oferta;

    \item o serviço Amazon S3, de armazenamento de conteúdo via web service, permite o
        uso de torrent para a transmissão de arquivos \cite{site:aws-s3};

    \item as empresas de desenvolvimento de jogos CCP Games (Eve Online) e Blizzard
        (Diablo III, StarCraft II e World of Warcraft) usaram o protocolo para
        distribuir o instalador de seu jogo \cite{site:eve}, e distribuir os jogos e
        suas eventuais atualizações \cite{site:blizzard}, respectivamente;

    \item o governo britânico distribuiu os detalhes de seus gastos \cite{site:gov-uk},
        enquanto a Universidade do Estado da Flórida utiliza para transmitir grandes
        conjuntos de dados científicos aos seus pesquisadores \cite{site:univ-fl};

    \item Facebook \cite{site:facebook-torrent} e Twitter \cite{site:twitter-torrent}
        o usam para atualizar os seus sites, enviando de forma eficiente o código novo
        para seus servidores de aplicação \cite{site:twitter-torrent-power}.
\end{itemize}

Em 2013, o BitTorrent é um dos maiores geradores de tráfego de rede do mundo, de forma
crescente, ao lado do NetFlix, Youtube, Facebook e acessos HTTP
\cite{report:internet-usage-2013}.

\subsection{Questões legais}

Desde que surgiu, o BitTorrent, bem como os outros protocolos \gls*{p2p}, chamou a
atenção dos defensores de direitos autorais, por conta do compartilhamento não
autorizado de arquivos protegidos por tais direitos, sendo alvo de medidas judiciais.
Porém, assim como o \gls*{gnutella}, e ao contrário do Napster, por possuir uma
estrutura descentralizada e não armazenar dados sobre os compartilhamentos realizados,
dificulta o trabalho de identificação das pessoas que compatilham esses dados.

Ainda assim, não existe um consenso sobre os efeitos financeiros do compartilhamento de
arquivos protegidos por direitos autorais, onde o principal argumento utilizado pelos
reclamantes é que estes têm grandes prejuízos e, por isso, entram com ações
indenizatórias de valores vultosos. Existem alguns estudos que tentam medir esses
prejuízos; um dos mais recentes, mostrou que não existem evidências de diminuição das
receitas das empresas cujo conteúdo é pirateado, e que o combate aos usuários
infratores não tem o impacto esperado, que é o de reduzir o compartilhamento desses
arquivos \cite{report:lse-piracy}.

\subsection{Estudos acadêmicos}

%\todoquestion{tá beeem resumido; desenvolve mais??}

Academicamente, o protocolo é bastante estudado desde o seu surgimento, sendo focos de
pesquisa os efeitos do algoritmo original e ajustes finos de seu funcionamento. Os
pontos principais são a parte algorítmica da troca de pedaços dos arquivos, estudos
sobre as topologias das redes formadas pelos \glspl*{peer} e melhoria da eficiência
do protocolo com alterações nessas topologias.

\afterpage{\clearpage}