%!TEX root = ../tcc.tex

\chapter{Napster, Gnutella e BitTorrent}

\begin{comment}
Vou contar a história do BitTorrent, desde o aumento do uso de
transferências de dados via internet, passando pela criação de protocolos e softwares
usados para baixar arquivos de forma ilegal, até a sua criação e o boom de usuários que
o utilizam para essa motivação. Em paralelo, discutirei os motivos legais pelos quais os
softwares anteriores foram descontinuados, que influenciaram diretamente na sua criação.

Separarei em 3 períodos: o antes, a criação e as consequências disso no mundo.
\end{comment}

Para entendermos como e por que o BitTorrent se tornou o que é hoje, devemos voltar um
pouco no tempo e rever a história que precedeu à sua criação, que é o fim da década dos
anos 1990.

% http://tex.stackexchange.com/a/99664
% \newglossaryentry{www}{
%     name={World Wide Web},
%     description={World Wide Web (WWW) é nome que se dá à rede mundial de
%             computadores interligados, que originou a Internet},
%     first={World Wide Web (WWW)},
%     long={World Wide Web}
% }

\section{Período pré-torrent}

\newglossaryentry{isp}{
    name={ISP},
    description={do inglês \emph{Internet Service Provider}; fornecedores de
    acesso a Internet, que são empresas que vendem serviço e equipamento que permitem
    o acesso de um computador pessoal acessar a Internet},
    first={fornecedor de acesso a Internet (\glsentryname{isp})},
    long={fornecedor de acesso a Internet},
    plural={\glsentryname{isp}s},
    firstplural={fornecedores de acesso a Internet (\glsentryname{isp})}
}

\newglossaryentry{mp3}{
    name={MP3},
    description={do inglês \emph{MPEG-1/2 Audio Layer 3}; formato patenteado de
    compressão de dados de áudio digital que usa um método de compressão de dados com
    perdas},
    first={\glsentrylong{mp3} (\glsentryname{mp3})},
    long={formato de áudio \glsentryname{mp3}},
    plural={\glsentryname{mp3}s}
}

\newglossaryentry{p2p}{
    name={P2P},
    description={do inglês \emph{peer-to-peer}; redes de arquitetura descentralizada e
    distribuída, onde cada nó (\emph{peer}) faz fornece e consome recursos},
    first={\glsentrylong{p2p} (\glsentryname{p2p})},
    long={rede \glsentryname{p2p}},
    plural={redes \glsentryname{p2p}}
}

\newglossaryentry{audiogalaxy}{
    name={Audiogalaxy},
    description={rede P2P de compartilhamento de músicas MP3 criado em 1998},
    first={\glsentryname{audiogalaxy}.com},
    long={\glsentryname{audiogalaxy}}
}

\newglossaryentry{riaa}{
    name={RIAA},
    description={do inglês \emph{Recording Industry Association of America}; Associação
    da Indústria de Gravação da América, organização que representa as gravadoras
    musicais e distribuidores, e tem sido autora de ações judiciais devido a quebra de
    direitos autorais causada por compartilhamento indevido de música},
    first={RIAA (do inglês \emph{Recording Industry Association of America})},
    long={\glsentryname{riaa}}
}

\todorefs{\cite{site:wiki-fs,site:wiki-fs-timeline}}

Entre o final dos anos 80 e o início dos 90, a Internet deixou de ser uma rede de
computadores usada somente por entidades governamentais, laboratórios de pesquisa e
universidades, passando a ter seu acesso comercializado para o público em geral pelos
\glspl{isp} \cite{site:wiki-isp}. Com o advento do \gls{mp3} \cite{site:wiki-mp3} no
final de 1991 \todoquestion{Como fazer referência a uma coisa deste tipo?} e do seu
primeiro reprodutor de áudio \gls*{mp3} Winamp, o tráfego da Internet aumentou devido
ao aumento da troca direta desse tipo de arquivo.

Entre 1998 e 1999, dois sites de compartilhamento gratuito de músicas foram criados: o
MP3.com \cite{site:wiki-mp3.com}, que era um site de divulgação de bandas independentes,
e o \gls{audiogalaxy} \cite{site:wiki-audiogalaxy.com,revista:pnp}. Mais popular que o
primeiro, o \gls*{audiogalaxy} era um site de busca de músicas, sendo que o download e
upload eram feitos a partir de um software cliente. A lista de músicas procuradas ia da
página para o computador onde usuário tinha instalado o cliente, que então conectava
com o do outro usuário, que era indicado pelo servidor. A lista possuía todos os
arquivos que um dia passaram pela sua rede. Se algum arquivo fosse requisitado mas o
usuário que possuísse não estivesse conectado, o servidor central do gls*{audiogalaxy}
fazia a ponte, pegando o arquivo para si e enviando-o para o cliente do requisitante em
seu próximo login.

Em maio de 1999 surgiu o Napster \cite{site:wiki-napster}, um programa de
compartilhamento de \gls*{mp3} que inovou por desfigurar o usual modelo
cliente-servidor, onde um servidor central localizava os arquivos nos usuários e fazia
a conexão entre os usuários, onde ocorria a transferências. O Napster foi contemporâneo
do \gls*{audiogalaxy} e ambos fizeram muito sucesso por cerca de 2 anos, até que
começaram as ações judiciais.

Não demorou muito tempo para a indústria da música entrar em ação contra a troca de
arquivos protegidos por direitos autorais sem autorização pela Internet. Seu primeiro
alvo foi o Napster, em dezembro de 1999, quando a \gls{riaa} entrou com processo
representando várias gravadoras alegando quebra de direitos autorais
\cite{site:napster-riaa}. Em abril de 2000, foi a vez da banda Metallica processar,
como resposta à sua descoberto que uma música ainda não lançada oficialmente já
circulava na rede \cite{site:napster-metallica,site:napster-metallica-orig}. Um mês
depois, outra ação, agora encabeçada pelo rapper Dr. Dre, que tinha feito pedido formal
para a retirada de seu material de circular, também abriu processo
\cite{site:napster-drdre-orig}. Isso fez com que o Napster tivesse atenção da mídia,
ganhando popularidade e atingindo o 20 milhões de usuários em meados de 2000.

\todorefs{referenciar o parágrafo seguinte melhor (?)}

Em 2001, esses imbróglios judiciais resultaram numa liminar federal que ordenava a
retirada de conteúdo protegido das entidades representadas pela \gls*{riaa}. O Napster
tentou, mas a juíza do caso não ficou satisfeita ordenando então, em julho, o
desligamento da rede enquanto não conseguisse controlar o conteúdo que trafegava ali. Em
setembro, o Napster fez um acordo, onde pagou 26 milhões de dólares por danos já
causados, uso indevido de música e também 10 milhões de dólares pelos danos futuros
envolvendo royalties. Para pagar esse valor, o Napster tentou cobrar o serviço de seus
usuários, que acabaram migrando de rede \gls*{p2p}, inclusive para o \gls*{audiogalaxy}.
Não conseguindo, em 2002, o Napster decreta falência e é forçado a liquidar seus ativos.
De lá para cá, foi negociado algumas vezes e atualmente pertence ao site Rhapsody
\cite{site:napster-rhapsody}.

\begin{comment}
\todo{revisar}

O sucesso do Napster, mesmo que por curto período tempo, mostrou a eficiência que redes
P2P poderiam ter, e com isso novos softwares e modelos de redes foram sendo lançados,
porém tentando contornar o ponto fraco do antecessor a fim de não serem novos alvos de
medidas judiciais. A solução para isso foi tentar descentralizar o mecanismo de indexação
e busca, que foi o calcanhar de Aquiles do Napster.

A pioneira nessa tentativa foi a rede Gnutella, que foi lançada em 2003 mas que em sua
primeira versão não conseguiu manter o bom desempenho do Napster. A busca era demorada e
inconsistente, pois era repassada aleatória e finitamente de peers para seus izinhos, o
que podia terminar em buscas sem resultados mesmo quando um arquivo estava sendo
compartilhado por alguém conectado à rede.
\end{comment}

continua...

\section{Nascimento do BitTorrent}

Aqui contarei a história do surgimento do protocolo.

\section{Mundo pós-torrent}

Aqui discorrerei sobre as consequências do uso do BitTorrent, desde as influências acadêmico-tecnológicas da sua difusão, bem como explicarei as questões jurídicas envolvidas no seu uso.

\clearpage
