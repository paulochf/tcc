%!TEX root = ../tcc.tex

\chapter{Napster, Gnutella, eDonkey e BitTorrent}

\begin{comment}
Vou contar a história do BitTorrent, desde o aumento do uso de
transferências de dados via internet, passando pela criação de protocolos e softwares
usados para baixar arquivos de forma ilegal, até a sua criação e o boom de usuários que
o utilizam para essa motivação. Em paralelo, discutirei os motivos legais pelos quais os
softwares anteriores foram descontinuados, que influenciaram diretamente na sua criação.

Separarei em 3 períodos: o antes, a criação e as consequências disso no mundo.
\end{comment}

Para entendermos como e por que o BitTorrent se tornou o que é hoje, devemos voltar um
pouco no tempo e rever a história que precedeu à sua criação, que é o fim da década dos
anos 1990.

% http://tex.stackexchange.com/a/99664
% \newglossaryentry{www}{
%     name={World Wide Web},
%     description={World Wide Web (WWW) é nome que se dá à rede mundial de
%             computadores interligados, que originou a Internet},
%     first={World Wide Web (WWW)},
%     long={World Wide Web}
% }

\todo[inline]{adicionar firstplural ao glossário}

\section{Período pré-torrent}

\newglossaryentry{isp}{
    name={ISP},
    description={do inglês \emph{Internet Service Provider}; fornecedores de
    acesso a Internet, que são empresas que vendem serviço e equipamento que permitem
    o acesso de um computador pessoal acessar a Internet},
    first={fornecedor de acesso a Internet (\glsentryname{isp})},
    long={fornecedor de acesso a Internet},
    plural={\glsentryname{isp}s},
    firstplural={fornecedores de acesso a Internet (\glsentryname{isp}s)}
}

\newglossaryentry{mp3}{
    name={MP3},
    description={do inglês \emph{MPEG-1/2 Audio Layer 3}; formato patenteado de
    compressão de dados de áudio digital que usa um método de compressão de dados com
    perdas},
    first={\glsentrylong{mp3} (\glsentryname{mp3})},
    long={formato de áudio \glsentryname{mp3}},
    plural={\glsentryname{mp3}s}
}

\newglossaryentry{peer}{
    name={peer},
    description={do inglês \emph{peer-to-peer}; como são chamados cada nó da rede desse
    tipo},
    first={\glsentrylong{peer} (um nó da rede)},
    long={\glsentryname{peer}},
    plural={\glsentryname{peer}s}
}

\newglossaryentry{p2p}{
    name={P2P},
    description={do inglês \emph{peer-to-peer}; redes de arquitetura descentralizada e
    distribuída, onde cada nó (\emph{peer}) faz fornece e consome recursos},
    first={\glsentrylong{p2p} (\glsentryname{p2p})},
    long={rede \glsentryname{p2p}},
    plural={redes \glsentryname{p2p}}
}

\newglossaryentry{audiogalaxy}{
    name={Audiogalaxy},
    description={rede P2P de compartilhamento de músicas MP3 criado em 1998},
    first={\glsentryname{audiogalaxy}.com},
    long={\glsentryname{audiogalaxy}}
}

\newglossaryentry{riaa}{
    name={RIAA},
    description={do inglês \emph{Recording Industry Association of America}; Associação
    da Indústria de Gravação da América, organização que representa as gravadoras
    musicais e distribuidores, e tem sido autora de ações judiciais devido a quebra de
    direitos autorais causada por compartilhamento indevido de música},
    first={RIAA (do inglês \emph{Recording Industry Association of America})},
    long={\glsentryname{riaa}}
}

\newglossaryentry{gnutella}{
    name={Gnutella},
    description={software de compartilhamento P2P desenvolvido por 3 programadores da
    empresa Nullsoft, recém adquirida da AOL Inc., lançado em 2000 sob a licença GPL.
    No dia seguinte, a AOL ordenou indisponibilizar o software alegando problemas
    legais e proibindo a continuação do desenvolvimento. Alguns dias depois, o
    protocolo já tinha sido alvo de engenharia reversa e já havia softwares que o
    implementavam},
    first={\glsentryname{gnutella}}
}

\newglossaryentry{anycast}{
    name={anycast},
    description={método de endereçamento e roteamento de rede onde os datagramas de um
    único remetente são roteados para um membro de um grupo de receptores potenciais que
    estão definidos pelo mesmo intervalo no endereço de destino. Geralmente é usado
    para serviços que demandem alta disponibilidade},
    first={\glsentryname{anycast}}
}

\newglossaryentry{edonkey}{
    name={eDonkey},
    description={lançado em 6 de setembro de 2000, o protocolo foi inaugurado juntamente
    com o software que o utilizava, o eDonkey2000, mas inúmeros softwares cliente para
    diferentes plataforas surgiram nos dias seguintes ao lançamento},
    first={\glsentryname{edonkey}}
}

\newglossaryentry{swarming}{
    name={swarming},
    description={também chamado de transmissão de \emph{arquivos por segmentação} ou de
    \emph{múltiplas fontes}, é a transmissão coordenada de um arquivo a partir de
    um ou vários locais onde este está disponível para um único destino, inclusive no
    caso de um arquivo em um local sendo transmitido em várias partes paralelas. Cabe ao
    software que faz o download juntar as partes no ponto de destino},
    first={\glsentryname{swarming}}
}

\newglossaryentry{hashtable}{
    name={tabela de hash},
    description={ou \emph{mapa de hash}, é uma estrutura de dados que cria uma lista de
    correspondência chave-valor, onde os dados são guardados como os valores e
    indexados por seus respectivos \emph{valores hash}},
    first={\glsentryname{hashtable}},
    plural={tabelas de hash}
}

\newglossaryentry{hashfunction}{
    name={função de hash},
    description={é uma função ou algoritmo matemático que mapeia um dado de comprimento
    variável em outro de comprimento fixo},
    first={\glsentryname{hashfunction}},
    plural={funções de hash}
}

\newglossaryentry{hashvalue}{
    name={valor hash},
    description={ou \emph{hash}; valores gerados por uma tabela hash},
    first={\glsentryname{hashvalue}},
    plural={valores hash}
}

\newglossaryentry{dht}{
    name={DHT},
    description={do inglês \emph{distributed hash table}; tabela hash distribuída, é um
    serviço de busca similar a uma tabela hash, mas na forma de sistema distribuído e
    descentralizada},
    first={\glsentryname{dht}},
    plural={\glsentryname{dht}s}
}

\newglossaryentry{kademlia}{
    name={Kademlia},
    description={DHT usado em redes P2P que especifica a estrutura da rede e a troca de
    informações através de buscas de nós, guardando as localizações de recursos que
    estão na rede},
    first={\glsentryname{kademlia}}
}

\newglossaryentry{betatester}{
    name={beta tester},
    description={blablabla},
    first={\glsentryname{betatester}},
    plural={\glsentryname{betatester}s}
}


Entre o final dos anos 80 e o início dos 90 \cite{site:wiki-fs,site:wiki-fs-timeline}, a
Internet deixou de ser uma rede de computadores usada somente por entidades
governamentais, laboratórios de pesquisa e universidades, passando a ter seu acesso
comercializado para o público em geral pelos \glspl{isp} \cite{site:wiki-isp}. Com o
advento do \gls{mp3} \cite{site:wiki-mp3} no final de 1991 e do seu primeiro reprodutor
de áudio \gls*{mp3} Winamp, o tráfego da Internet aumentou devido ao aumento da troca
direta desse tipo de arquivo.

Entre 1998 e 1999, dois sites de compartilhamento gratuito de músicas foram criados: o
MP3.com \cite{site:wiki-mp3.com}, que era um site de divulgação de bandas independentes,
e o \gls{audiogalaxy} \cite{site:wiki-audiogalaxy.com,revista:pnp}. Mais popular que o
primeiro, o \gls*{audiogalaxy} era um site de busca de músicas, sendo que o download e
upload eram feitos a partir de um software cliente. A lista de músicas procuradas ia da
página para o computador onde usuário tinha instalado o cliente, que então conectava
com o do outro usuário, que era indicado pelo servidor. A lista possuía todos os
arquivos que um dia passaram pela sua rede. Se algum arquivo fosse requisitado mas o
usuário que o possuísse não estivesse conectado, o servidor central do
\gls*{audiogalaxy} fazia a ponte, pegando o arquivo para si e enviando-o para o cliente
do requisitante em seu próximo login.

O período dos 3 anos seguintes à criação desses dois sites foi muito produtivo ao
mundo \gls*{p2p} de modo geral, onde surgiram alguns protocolos desse paradigma e
inúmeros softwares que os implementavam. Os mais relevantes foram o Napster, o Gnutella,
o eDonkey e o BitTorrent.

\subsection{Napster}

Em maio de 1999 surgiu o Napster \cite{site:wiki-napster}, um programa de
compartilhamento de \gls*{mp3} que inovou por desfigurar o usual modelo
cliente-servidor, onde um servidor central localizava os arquivos nos usuários e fazia
a conexão entre os usuários, onde ocorria a transferências. O Napster foi contemporâneo
do \gls*{audiogalaxy} e ambos fizeram muito sucesso por cerca de 2 anos, até que
começaram as ações judiciais.

Não demorou muito tempo para a indústria da música entrar em ação contra a troca de
arquivos protegidos por direitos autorais sem autorização pela Internet. Seu primeiro
alvo foi o Napster, em dezembro de 1999, quando a \gls{riaa} entrou com processo
representando várias gravadoras alegando quebra de direitos autorais
\cite{site:napster-riaa}. Em abril de 2000, foi a vez da banda Metallica processar,
como resposta à sua descoberto que uma música ainda não lançada oficialmente já
circulava na rede \cite{site:napster-metallica,site:napster-metallica-orig}. Um mês
depois, outra ação, agora encabeçada pelo rapper Dr. Dre, que tinha feito pedido formal
para a retirada de seu material de circular, também abriu processo
\cite{site:napster-drdre-orig}. Isso fez com que o Napster tivesse atenção da mídia,
ganhando popularidade e atingindo o 20 milhões de usuários em meados de 2000
\cite{site:napster-use-2000}.

Em 2001, esses imbróglios judiciais resultaram numa liminar federal que ordenava a
retirada de conteúdo protegido das entidades representadas pela \gls*{riaa}. O Napster
tentou, mas a juíza do caso não ficou satisfeita ordenando então, em julho, o
desligamento da rede enquanto não conseguisse controlar o conteúdo que trafegava ali
\cite{site:wiki-napster}. Em setembro, o Napster fez um acordo, onde pagou 26 milhões de
dólares por danos já causados, uso indevido de música e também 10 milhões de dólares
pelos danos futuros envolvendo royalties. Para pagar esse valor, o Napster tentou cobrar
o serviço de seus usuários, que acabaram migrando de rede \gls*{p2p}, inclusive para o
\gls*{audiogalaxy}. Não conseguindo, em 2002, o Napster decreta falência e é forçado a
liquidar seus ativos. De lá para cá, foi negociado algumas vezes e atualmente pertence
ao site Rhapsody \cite{site:napster-rhapsody}.

O sucesso do Napster, mesmo que por curto período tempo, mostrou o potencial das redes
\gls*{p2p} poderiam ter, e com isso novos softwares e protocolos de redes foram sendo
lançados, sempre tentando se diferenciar dos outros softwares a fim de não serem novos
alvos de ações judiciais. A solução para isso foi tentar descentralizar os mecanismos de
indexação e de busca, que foram os pontos fracos do Napster.

\subsection{Gnutella}

O tal sucessor foi o \gls{gnutella}, em março de 2000 \cite{wiki:gnutella}, foi uma
resposta de domínio público feita com ``gambiarras'' para os problemas que o Napster
encontrou com relação ás acusações deviolação de direitos autorais. Enquanto o Napster
possuía um servidor central como estrutura que, no julgamento, foi usado como prova de
que o sistema encorajava a violação de direitos autorais, o \gls*{gnutella} foi modelado
como um sistema \gls*{p2p} puro, onde todos os \glspl*{peer} são completamente iguais,
sendo responsáveis pelos seus próprios atos.

O \gls*{gnutella} disponibiliza arquivos da mesma forma que o Napster
\cite{book:birman}, mas sem a limitação de ser de formato de música, ou seja, qualquer
arquivo pode ser compartilhado. A diferença mais significativa entre os dois protocolos
é o algoritmo de busca: a abordagem do \gls*{gnutella} é baseada numa forma de
\gls{anycast}. Isso envolve duas partes: a primeira é como cada usuário é conectado a
outros nós e mantém a lista dessas conexões atualizada. A segunda parte é como ele
trata as buscas e trabalha inundando de pedidos para todos os nós que estão a uma certa
distância do usuário (nó- cliente). Por exemplo, se a distância limite for de 4, então
todos os nós que estiverem a 4 passos a partir do cliente serão verificados, começando
a partir dos mais próximos. Eventualmente, algum nó possuirá o arquivo requisitado e
responderá, e assim será feita a transferência desse arquivo. Muitos softwares que
implementam o protocolo vão além dessa funcionalidade básica de download simples
tentando transferir de forma paralela partes diferentes do arquivo desejado de nós
diferentes, tentando amenizar eventuais problemas de velocidade de rede.

Experiências sugerem assim que o sistema escala para um tamanho maior, o mecanismo de
\gls*{anycast} se torna extremamente caro e em algumas vezes até proibitivo. O problema
ocorre nas buscas por arquivos menos populares, onde será necessário um maior número de
nós perguntados.

O \gls*{gnutella} ainda teve uma segunda versão \cite{wiki:gnutella2}, no final de 2002,
onde utilizou o mesmo protocolo que o original, porém organizando a rede de
\glspl*{peer} em \emph{leafs} (folhas, em inglês) e \emph{hubs}. Um \emph{hub} poderia
ter centenas de conexões de folhas outras 7, em média, a outros \emph{hubs}, enquanto
uma folha se conectaria apenas a 2 \emph{hubs}. Essa nova topologia, somada com uma
nova tabela de índice de arquivos das folhas mantida pelos \emph{hubs} onde estavam
conectados, melhorou o desempenho das buscas, que era ruim na versão antiga.

\subsection{eDonkey}

O protocolo \gls{edonkey} inovou em muitos aspectos de seus precursores, tendo papel
fundamental na história das redes \gls*{p2p} e sua consolidação como ferramenta de
compartilhamento especializado em arquivos grandes.

O \gls*{edonkey} implementou o primeiro método de download por \gls{swarming}, que é
chamado o método onde \glspl*{peer} fazem downloads de diferentes partes de um arquivo e
de \glspl*{peer} diferentes, utilizando de forma efetiva a largura de banda de rede
para todos os \glspl*{peer} ao invés de ficar limitado somente à banda de um único
\gls*{peer}.

Outra melhoria foi a busca: no seu lançamento, os servidores eram separados entre si,
porém nas versões seguintes permitiu que eles formassem uma rede de buscas. Isso
permitiu que os servidores repassassem buscas de seus clientes conectados localmente a
outros servidores, facilitando a localização de \glspl*{peer} conectados em qualquer
servidor da rede de buscas, aumentando a capacidade de download do enxame.

Uma terceira diferença com o Napster foi o uso de \glspl{hashvalue} de arquivos
nos resultados de busca ao invés dos simples nomes dos arquivos. As buscas geradas
pelos usuários eram baseadas em palavras-chave e comparadas com a lista de nomes de
arquivos armazenada no servidor, mas o servidor retornava uma lista de pares de nomes
de arquivos com seus repectivos valores \gls*{hashvalue}. Enfim, quando o usuário
selecionasse o arquivo desejado, o cliente iniciaria o download do arquivo usando o seu
valor \gls*{hashvalue}. Desse modo, um arquivo poderia ter muitos nomes entre os
diferentes \gls*{peer} e servidores, mas seria considerado idêntico para download se
possuísse o mesmo \gls*{hashvalue}.

A arquitetura da rede em dois níveis usando cliente e servidor alcançou um meio termo
entre as redes centralizadas, como o Napster, e as descentralizadas, como o
\gls*{gnutella}, já que o servidor central no primeiro era um alvo estável para ações
legais, enquanto o segundo rapidamente mostrou-se inviável devido ao tráfego massivo de
buscas entre \glspl*{peer}.

Por fim, a inovação mais importante foi o uso de \gls{dht}, em específico o
\gls{kademlia}, como algoritmo de indexação e busca nos servidores centrais dos
arquivos através da rede \gls*{edonkey}. Além de ser uma das causas da melhora no
desempenho nas buscas, \glspl*{dht} possuem ainda outras características como
tolerância a falhas e escalabilidade. O \gls*{kademlia} em específico ainda oferece
outras vantagens como armazenamento de dados eficiente; anonimato; segurança de rede,
conteúdo e usuário; e autenticação.

\section{Nascimento do BitTorrent}

Em meados dos anos 90, Bram Cohen era um programador que tinha largado a faculdade no
segundo ano do curso de Ciência da Computação da Universidade de Buffalo, Nova Iorque,
para trabalhar em empresas \emph{pontocom}. A última foi a MojoNation, uma empresa que
desenvolvia um software de distribuição de arquivos criptografados por \gls*{swarming},
que ele já tinha percebido ser uma vantagem com relação ao Kazaa, que fazia
transferências de uma única origem.

Em abril de 2001, Bram saiu da MojoNation e começou a modelar o protocolo BitTorrent,
lançando a primeira implementação em Python em julho de 2001. Em fevereiro de 2002, ele
apresentou o seu trabalho na CodeCon \cite{site:codecon} e na mesma época começou a
testá-lo, baixando uma coleção de material pornográfico para atrair \glspl{betatester}.

\section{Mundo pós-torrent}

Aqui discorrerei sobre as consequências do uso do BitTorrent, desde as influências acadêmico-tecnológicas da sua difusão, bem como explicarei as questões jurídicas envolvidas no seu uso.

\clearpage