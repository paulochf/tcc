%!TEX root = ../../tcc.tex

\subsection*{UDP}

O protocolo \gls*{udp} é um protocolo de conexão que pertence à camada de transporte e
que especifica conexões quase que diretas (se comparado com o \gls*{tcp}) entre as
camadas de aplicação e de rede.

Quando as mensagens são recebidas através dos \glspl{socket}, elas são anexadas aos
endereços IP e número de porta de origem e de destino, e aos comprimentos e
\gls{checksum} do cabeçalho e do corpo de dados \gls*{udp}, para enfim serem repassadas
como segmentos para a camada de rede. Por sua vez, a camada de rede encapsula os
segmentos em datagramas fazem o melhor possível para entregá-los ao destinatário.

Já quando estão sendo recebidas, o protocolo \gls*{udp} utiliza a porta de destino
contida no cabeçalho para entregar os dados do segmento para o processo correto de
aplicação. Como não existe nenhum \emph{handshake} entre as camadas de transporte das
partes durante esse procedimento, o protocolo \gls*{udp} é dito ``sem conexão'' ou ``não
orientado a conexão''. Além disso, é considerado não confiável, já que não se possui
garantia de entregas dos pacotes, muito menos na ordem correta.

Apesar de ser considerado não confiável, existem vantagens em se escolher o método
\gls*{udp} em vez do \gls*{tcp}:

\begin{itemize}
    \item controle avançado de quais dados são enviados e quando: como o \gls*{tcp}
        possui controle de congestionamento e confirmação de recebimento de segmentos,
        pode ser que a aplicação seja comprometida pelo atraso de algum datagrama. No
        caso de aplicações em tempo real, é possível suportar alguma perda de dados, ou
        ainda implementar o seu próprio método de verificação de integridade;

    \item dispensa conexões: enquanto o \gls*{tcp} utiliza \emph{handshakes} antes da
        transferência dos dados, o \gls*{udp} já os envia sem a necessidade de
        contatos anteriores. Essa é o principal motivo pelo qual servidores de DNS
        (\emph{Domain Name Service}) utilizam \gls*{udp};

    \item não mantém estado da conexão: essas informações de estado são necessárias para
        se conseguir uma conexão de dados confiável, que é como o \gls*{tcp} controla
        buffers de entrada e saída, congestionamento de dados e parâmetros de
        confirmação. Por conta disso, um servidor dedicado a uma aplicação consegue
        aguentar muito mais conexões do que se fosse usado \gls*{tcp};

    \item pouco \gls{overhead} de cabeçalho de pacotes: enquanto o \gls*{tcp} possui
        \gls*{overhead} de cabeçalho de 20 bytes por segmento, o \gls*{udp} possui 8
        bytes.
\end{itemize}

Originalmente, os \glspl*{tracker} utilizavam \gls*{tcp}, porém com o tempo perceberam
que com \gls*{udp} se tornariam mais eficientes, reduzindo o consumo da largura da
conexão de rede pela metade \cite{site:tracker-udp}. Por esses motivos, os servidores
dos \glspl{tracker} utilizam prioritariamente o protocolo \gls*{udp}.