%!TEX root = ../../tcc.tex

\newpage
\subsection*{UDP}

O protocolo \gls*{udp} é um protocolo de conexão que pertence à camada de transporte e
que especifica conexões quase que diretas (se comparado com o \gls*{tcp}) entre as
camadas de aplicação e de rede.

Quando mensagens são enviadas por uma aplicação, elas são recebidas através dos
\glspl*{socket}. Depois, são anexadas aos endereços IP e número de porta de origem e de
destino, e aos comprimentos e \gls{checksum} do cabeçalho e do corpo de dados
\gls*{udp}, para enfim serem repassadas como segmentos para a camada de rede. Por sua
vez, a camada de rede encapsula os segmentos em datagramas e faz o melhor possível para
entregá-los ao destinatário.

Já quando datagramas são recebidos, o protocolo \gls*{udp} utiliza a porta de destino
contida no cabeçalho para entregar os dados do segmento para o processo correto de
aplicação. Como não existe nenhum protocolo de \emph{handshake} entre as camadas de
transporte de ambos remetente e destinatário, o protocolo \gls*{udp} é dito ``sem
conexão'' ou ``não orientado a conexão''. Além disso, é considerado não confiável, já
que não há garantia de entregas dos pacotes, muito menos a entrega na ordem correta.

Apesar de ser considerado não confiável, existem vantagens em se escolher o \gls*{udp}
ao invés do \gls*{tcp}:

\begin{itemize}
    \item controle avançado de quais dados são enviados e quando: como o \gls*{tcp}
        possui controle de congestionamento e confirmação de recebimento de segmentos,
        pode ser que a aplicação seja comprometida pelo atraso de algum datagrama. No
        caso de aplicações em tempo real, é possível suportar alguma perda de dados, ou
        ainda implementar o seu próprio método de verificação de integridade;

    \item dispensa conexões: enquanto o \gls*{tcp} utiliza protocolos de
        \emph{handshake} antes da transferência dos dados, o \gls*{udp} já os envia sem
        a necessidade de contatos anteriores. Essa agilidade é o principal motivo pelo
        qual servidores de DNS (\emph{Domain Name Service}) utilizam \gls*{udp};

    \item não mantém estado da conexão: essas informações de estado são necessárias para
        se conseguir uma conexão de dados confiável, como faz o \gls*{tcp}, que usa
        buffers de entrada e saída, gerencia congestionamento de dados e possui
        parâmetros de confirmação. Por conta disso, um servidor dedicado a uma
        aplicação consegue suportar muito mais conexões do que se fosse usado
        \gls*{tcp};

    \item pouco \gls{overhead} de cabeçalho de pacotes: enquanto o \gls*{tcp} possui
        \gls*{overhead} de cabeçalho de 20 bytes por segmento, o \gls*{udp} possui 8
        bytes.
\end{itemize}

Originalmente, os \glspl*{tracker} utilizavam o protocolo \gls*{tcp}, porém, com o
tempo, percebeu-se que com o protocolo \gls*{udp} eles se tornariam mais eficientes,
reduzindo o consumo da largura de rede pela metade \cite{site:tracker-udp}. Por esses
motivos, os servidores dos \glspl*{tracker} utilizam prioritariamente o protocolo
\gls*{udp}.

No Transmission, \glspl*{tracker} podem ser adicionados por \glspl*{torrentfile} ou
manualmente. Quando o programa precisa atualizar as informações sobre cada um desses
\glspl*{tracker}, procura pelos respectivos endereços de \gls{announce}. Caso os possua
com esquemas \gls{uri} para \gls*{tcp} e \gls*{udp}, o Transmission escolhe aquele em
\gls*{udp}.

\cfile[label="./libtransmission/announcer.c:555"]{./Codes/chap4/015-udpvstcp.c}