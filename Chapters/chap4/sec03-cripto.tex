%!TEX root = ../../tcc.tex

\section{Criptografia}

A criptografia é o estudo e prática de técnicas que visam a segurança de informações de
várias maneiras, a fim de que todas as partes de uma transação confiem que os objetivos
daquela segurança tenham sido alcançadas. Com início há mais de quatro mil anos atrás,
no Egito antigo, os quatro objetivos principais que a criptografia oferece são:

\begin{description}
    \item[privacidade ou confidencialidade:] manter o sigilo da informação para todos,
        exceto àqueles que forem autorizados a vê-la;

    \item[integridade de dados:] assegurar que a informação não foi alterada por pessoas
        não autorizadas ou por outros meios, como falhas de rede ou de dispositivos;

    \item[autenticação ou identificação de entidades:] comprovação da identidade de uma
        entidade, como pessoas, computadores, cartões de crédito, etc;

    \item[autenticação de mensagens:] também chamado de autenticação da origem dos
        dados, é a comprovação da fonte da informação;

    \item[assinatura:] forma de vincular uma informação a uma entidade;

    \item[autorização:] transmissão, a outra entidade, de uma aprovação oficial para
        ser ou fazer algo;

    \item[validação:] um meio de proporcionar prontidão da autorização para a
        utilização ou manipulação de informações ou recursos;

    \item[controle de acesso:] restrição de acesso a recursos para entidades
        privilegiadas;

    \item[certificação:] endosso de informações por uma entidade confiável;

    \item[registro de horário:] registro do tempo da criação ou existência de
        informações;

    \item[testemunho:] verificação da criação ou existência de uma informação por uma
        entidade que não seja a criadora;

    \item[recibo:] reconhecimento de que a informação foi recebida;

    \item[confirmação:] reconhecimento que serviços foram fornecidos;

    \item[propriedade:] um meio de permitir que uma entidade com o direito legal use ou
        transfira um recurso a outros;

    \item[anonimato:] ocultação da identidade de uma entidade envolvida em algum
        processo;

    \item[não-aceitação:] prevenção da recusa de compromissos ou ações anteriores;

    \item[revogação:] retração de certificação ou autorização.
\end{description}
