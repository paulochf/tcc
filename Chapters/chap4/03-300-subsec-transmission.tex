%!TEX root = ../../tcc.tex

\subsection*{Criptografia no BitTorrent}

O BitTorrent usa criptografia somente quando o usuário habilita a opção correspondente
no programa cliente, criptografando comunicações com \glspl*{tracker} e \glspl*{peer}.
O método escolhido pelo protocolo é o que está definido em uma de suas propostas de
melhoria. Porém, esta está suspensa, sendo então usada de forma extraoficial atualmente
pelos programas cliente em geral, inclusive o Transmission.

Além disso, como é dito na própria proposta \cite{site:bittorrent-cripto}, o objetivo
dessa melhoria é impedir que \glspl{isp} ou outros administradores de rede de bloquear
ou quebrar conexões BitTorrent que ocorram entre o \gls*{peer} receptor de uma resposta
de um \gls{tracker} e qualquer outro \gls*{peer} cujo endereço IP e porta apareça nessa
resposta. Por isso, essa especificação é chamada de ofuscação de \glspl*{peer}. A idéia
proposta é usar o \bverb|info_hash| de um \gls*{torrent} como chave compartilhada entre
o \gls*{peer} e o \gls*{tracker}, não impedindo ataques de homem no meio por espiões que
saibam desses \glspl{hashvalue}.

%!TEX root = ../../tcc.tex

\subsubsection*{Comunicação com trackers}

Quando criptografia for habilitada, as comunicações com \glspl*{tracker}, ou seja, as
requisições de \glspl{announce}, não devem enviar o \bverb|info_hash| do \gls*{torrent}.
Ao invés disso, deve enviar o \bverb|sha_ih|, que é o \gls*{hashvalue} SHA-1 do
\bverb|info_hash| (que também é um \gls*{hashvalue} SHA-1) na forma \gls*{urlencode}.

Já a resposta de \glspl*{tracker} a \glspl*{announce} se mantêm no mesmo formato,
exceto pela lista de \glspl*{peer}, que será ofuscada.

Para isso, a requisição do \gls*{announce} deve passar como parâmetros

\begin{description}
    \item[supportcrypto:] valor 1 indica que o \gls*{peer} pode criar e receber
        conexões criptografadas. Neste caso, se o \gls*{tracker} aceitar esta extensão
        do BitTorrent, as listas de \glspl*{peer} que enviará em suas respostas
        priorizarão outros \glspl*{peer} que também enviaram \bverb|supportcrypto=1|
        antes dos que não o fizeram.

    \item[requirecrypto:] valor 1 indica que o \gls*{peer} irá criar e aceitar somente
        conexões criptografadas. Neste caso, as listas de \glspl*{peer} que o
        \gls*{tracker} enviará conterão somente \glspl*{peer} que também enviaram
        \bverb|supportcrypto=1| e \bverb|requirecrypto=1|.

    \item[cryptoport:] quando o parâmetro de \bverb|requirecrypto=1|, é inteiro que
        representa a porta na qual o cliente irá utilizar para conexões criptografadas.
\end{description}

\cfile[label="./libtransmission/announcer-http.c:58"]{./Codes/chap4/007-cripto-announce.c}

%!TEX root = ../../tcc.tex

\subsubsection*{Comunicação com peers}

A comunicação entre \glspl*{peer} é criptografada usando RC4 e troca de chaves
Diffie-Hellman-Merkle \cite{wikivuze:encription}. Para isso, o protocolo de
\emph{handshake} para mensagens entre \glspl*{peer} é estendido, de forma a efetuar
esses cinco procedimentos criptográficos:

\cfile[label="./libtransmission/crypto.c:60-79"]{./Codes/chap4/008-cripto-peer1.c}

\newpage
1) $A$ envia $Y_A$ para $B$:

O Transmission, nesse caso o \gls*{peer} $A$, envia sua chave pública ($Y_A$) com um
trecho de dados aleatórios, com comprimento qualquer entre 0 e 512 bytes;

\cfile[label="./libtransmission/handshake.c:313"]{./Codes/chap4/009-cripto-peer2-send-ya.c}

2) $A$ recebe $Y_B$ de $B$:

O outro \gls*{peer}, $B$, responde com sua chave pública ($Y_B$). Assim, a chave
compartilhada $S$ já pode ser calculada;

\cfile[label="./libtransmission/handshake.c:391"]{./Codes/chap4/010-cripto-peer3-read-yb.c}

\newpage
3) $A$ envia para $B$ as opções de criptografia e a mensagem de \emph{handshake}:

Então, $A$ envia dados de forma criptografada: a mensagem de \emph{handshake} e outras
informações sobre a criptografia para $B$, na forma:

\begin{verbatim}
HASH('req1', S), HASH('req2', SKEY) xor HASH('req3', S),
    ENCRYPT(VC, crypto_provide, len(PadC), PadC, len(IA)), ENCRYPT(IA)
\end{verbatim}

onde:

\begin{description}
    \item[HASH():] é a função que calcula o \gls*{hashvalue} SHA-1 de todos os
        parâmetros de entrada concatenados;

    \item[ENCRYPT():] é a função RC4 com chave \bverb|HASH('keyA', S, SKEY)| (se
        $A \rightarrow B$), ou \bverb|HASH('keyB', S, SKEY)| (se $B \rightarrow A$). Os
        primeiros 1024 bytes da encriptação RC4 são descartados. O uso seguido desta
        função por uma das partes encripta o fluxo de dados, sem reinicializações ou
        trocas;

    \item[VC:] é uma constante de verificação (\emph{verification constant}), que é uma
        string de 8 bytes de valor 0x00, usada para verificar se a outra parte conhece
        $S$ e SKEY, evitando ataques de repetição do SKEY;

    \item[crypto\_provide/crypto\_select:] são bitfield de 32 bits. Dois valores são
        usados atualmente, com o restante sendo reservado para uso futuro. O \gls*{peer}
        $A$ deve ligar os bits de todos os métodos suportados por ele, enquanto o
        \gls*{peer} $B$ deve ligar o bit do método escolhido dentre os oferecidos, e
        enviar como resposta. Por enquanto, 0x01 indica sem encriptação, e 0x02 indica o
        RC4;

    \item[PadC/PadD:] reservados para futuras extensões do handshake. Hoje, possuem 0
         bytes;

    \item[IA:] conjunto de dados inicial de $A$. Pode ser 0 bytes, se quiser esperar por
        negociação de encriptação;
\end{description}

\newpage
\cfile[label="./libtransmission/handshake.c:391", samepage=false]{./Codes/chap4/011-cripto-peer4-hashashash.c}

\newpage
4) $A$ recebe de $B$ a opção de criptografia escolhida e a mensagem de resposta ao
\emph{handshake} encriptada por RC4:

Aqui, $B$ envia como resposta:

\begin{verbatim}
ENCRYPT(VC, crypto_select, len(padD), padD), ENCRYPT2(Payload Stream)
\end{verbatim}

\cfile[label="./libtransmission/handshake.c:493", samepage=false]{./Codes/chap4/012-cripto-peer5-readvc.c}

\newpage
5) $A$ envia para $B$ o fim da mensagem de \emph{handshake} encriptada por RC4:

\cfile[label="./libtransmission/handshake.c:587"]{./Codes/chap4/013-cripto-peer6-handshake.c}
\cfile[label="./libtransmission/peer-io.c:1083"]{./Codes/chap4/014-cripto-peer7-write.c}