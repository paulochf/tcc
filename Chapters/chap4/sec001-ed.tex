%!TEX root = ../../tcc.tex

\section{Estruturas de dados}

Conjuntos são tão fundamentais na Ciência da Computação quando na matemática. Porém, os
conjuntos manipulados por algoritmos são adaptáveis, dinâmicos. Alguns algoritmos
utilizam conjuntos, realizando diversas operações sobre eles, como inserção, remoção e
testes de existência de elementos. Esses conjuntos dinâmicos que permitem essas
operações são chamados de dicionários.

Na linguagem C, geralmente são implementados usando estruturas: coleção de variáveis
(membros), independentes de tipo, agrupadas sobre o mesmo nome. Com isso, várias
implementações de dicionário foram criadas, cada um com suas peculiaridades, onde
algumas se tornarão. No Transmission, algumas dessas estruturas são utilizadas:

\begin{itemize}
    \item vetores (\emph{arrays})
    \item listas ligadas (\emph{linked lists})
    \item filas (\emph{queues})
    \item \glspl*{hashtable} (\emph{hash tables})
\end{itemize}

%!TEX root = ../../tcc.tex

\subsection*{Vetores}

Vetores são a implementação de vetores matemáticos de maneira virtual. Na prática,
consistem de listas de variáveis do mesmo tipo. Na linguagem C, podem ser declaradas de
forma estática ou dinâmica.

Vetores dinâmicos têm tamanho fixo

%!TEX root = ../../tcc.tex

\newpage
\subsection*{Listas ligadas}

Listas ligadas é uma estrutura de dado que organiza os objetos de forma linear, assim
como os vetores. Porém, enquanto estes possuem índices que determinam a sua posição, as
listas possuem ponteiros para os outros elementos. Por causa disso, elas crescem
organicamente, conforme novos elementos vão sendo criados e associados, então evitando
desperdício de memória.

Por conter ponteiros, elementos de listas ligadas sempre são definidos usando-se
estruturas.

\begin{ccode*}{label=./libtransmission/makemeta.c:41}
struct FileList {
    uint64_t          size;
    char *            filename;
    struct FileList * next;      // ponteiro para o próximo elemento
};
\end{ccode*}

Essa estrutura é utilizada como se fosse um tipo definido pelo usuário, que aloca em
memória de forma dinâmica.

\begin{ccode*}{label=./libtransmission/makemeta.c:41}
struct FileList {
    uint64_t          size;
    char *            filename;
    struct FileList * next;      // ponteiro para o próximo elemento
};
\end{ccode*}

Feito isso, para a estrutura poder ser usada, o Transmission utiliza de uma função que
aloca uma variável dessa estrutura de forma dinâmica e seta valores iniciais nulos para
os seus campos.

\begin{ccode*}{label=./libtransmission/list.c:19}
static tr_list * recycled_nodes = NULL;

static tr_list* node_alloc(void) {
    tr_list * ret;                  // ponteiro que apontará para a região alocada

    if (recycled_nodes == NULL) {   // Se não houver elementos reciclados,...
        ret = tr_new(tr_list, 1);   // ... aloque um novo.
    }
    else {   // Caso contrário, reutilize, reapontando as tomando o controle do central
        ret = recycled_nodes;           // ... referências dos elementos adjacentes...
        recycled_nodes = recycled_nodes->next;   // ... e tomando o controle do central
    }

    *ret = TR_LIST_CLEAR;           // limpa campos do elemento
    return ret;                     // devolve o ponteiro para o elemento
}
\end{ccode*}

Existem vários tipos de listas ligadas, algumas podendo, inclusive, serem combinadas
entre si:

\begin{itemize}
    \item simplesmente ligada: possui somente um ponteiro para o próximo elemento
    \item duplasmente ligada: possui 2 ponteiros (um para o elemento anterior e outro
        para o posterior)
    \item multiplamente ligada: possui ponteiros vários elementos, porém ligando-os em
        ordens diferentes
    \item circularmente ligada: quando o último elemento liga a lista de volta ao
        1º elemento
    \item com cabeça: quando possui um elemento falso somente para ajudar a manipular as
        listas
\end{itemize}

\todoquestion{mostro algoritmos?}

Comparando-se vetores e listas ligadas, cada um tem suas vantagens e desvantagens em
relação à complexidade de seus algoritmos de manipulação.

\newpage
\begin{table}
    \centering
    \begin{tabular}{| l | c | c | c |}
        \hline
        \textbf{Ação} & \textbf{Vetor (est.)} & \textbf{Vetor (din.)} & \textbf{Lista ligada} \\
        \hline
        Busca por posição & $\Theta(1)$ & $\Theta(1)$ & $\Theta(n)$ \\
        \hline
        Inserção/Remoção (início) & $\Theta(n)$ & $\Theta(n)$ & $\Theta(1)$ \\
        \hline
        Inserção/Remoção (fim) & $\Theta(1)$ & $\Theta(1)$ & \parbox[t]{.3\textwidth}{\centering $\Theta(1)$ (c/ cabeça) \\ $\Theta(n)$ (s/ cabeça)} \\
        \hline
        Inserção/Remoção (meio) & $\Theta(n)$ & $\Theta(n)$ & $\Theta(n)$ \\
        \hline
        Redimensionamento & $\Theta(n)$ & ? & não necessita \\
        \hline
    \end{tabular}
    \caption{tabela de comparação de complexidades dos algoritmos de manipulaçãp de
    vetores e listas ligadas. OBS: tempos de buscas considerados lineares.
    Redimensionamento de vetor dinâmico depende da implementação da linguagem C.}
\end{table}

%!TEX root = ../../tcc.tex

\newpage
\subsection*{Tabelas hash}

\Glspl{hashtable} são estruturas de dados eficientes na implementação de dicionários.
Apesar de buscas demorarem tanto quanto procurar um elemento em uma lista ligada -
$\Theta(n)$ no pior caso -, o espalhamento é bastante eficiente. Isso faz com que o
tempo médio de uma busca seja $O(1)$.

Uma \gls*{hashtable} generaliza a noção do vetor de elementos comum. Nele, o
endereçamento direto nos permite avaliar o conteúdo de uma posição em $O(1)$. O que
torna esta tabela especial é a vantagem de transformar um certo conteúdo possuir uma
chave específica e exclusiva, fornecendo um meio de se encontrar essa chave. Esse meio é
uma \gls{hashfunction}.

Às vezes, \glspl*{hashfunction} fazem com que 2 conteúdos possuam a mesma chave, ou
seja, as chaves colidem. Para esses casos, existem várias técnicas de solução de
conflitos, porém colisões podem ser evitadas com boa \gls*{hashfunction}, descritas a
seguir.

A \gls*{hashtable} usada pelo Transmission aparece no \gls*{dht}, porém de uma forma
mais simples: não existe ``a \gls*{hashfunction} do \gls*{dht}'' como de costume, onde
existe uma função característica para uma modelagem de tabela. Ao invés disso, as chaves
já estão calculadas, sendo os IDs do \glspl*{torrent} e dos \glspl*{peer} do Kademlia.