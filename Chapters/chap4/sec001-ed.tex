%!TEX root = ../../tcc.tex

\section{Estruturas de dados, listas ligadas e árvores}

\begin{comment}
Aqui vou falar de tipos de estruturas de dados utilizadas no programa, como
\emph{structs} e a sua utilização na implementação de listas ligadas e árvores e em como
estes são usados na implementação de filas.
\end{comment}

Conjuntos são tão fundamentais na Ciência da Computação quando na matemática. Porém, os
conjuntos manipulados por algoritmos são adaptáveis, dinâmicos. Na linguagem C,
geralmente são implementados usando estruturas: coleção de variáveis (membros),
independentes de tipo, agrupadas sobre o mesmo nome.

Existem formatos de estruturas de dados bastante conhecidos e que viraram padrões. No
Transmission, algumas dessas estruturas são utilizadas:

\begin{itemize}
    \item vetores (\emph{arrays})
    \item listas ligadas (\emph{linked lists})
    \item filas (\emph{queues})
\end{itemize}

%!TEX root = ../../tcc.tex

\subsection*{Vetores}

Vetores (ou \emph{arrays}) são a implementação de vetores matemáticos de maneira
virtual. Na prática, consistem de listas de variáveis do mesmo tipo. Na linguagem C,
podem ser declaradas de forma estática ou dinâmica.

Vetores estáticos têm tamanho fixo estabelecido na sua declaração em tempo de
compilação, tendo espaços de memória reservados na pilha de memória de acordo com o
tipo, não podendo ser alterado em tempo de execução.

\begin{ccode}
    char announce[1024]; // URL de announce
\end{ccode}

Por outro lado, vetores dinâmicos são construídos por ponteiros para o tipo (ao invés do
tipo) juntamente com alocação e liberação de memória programáticas. Com isso, a memória
é reservada em tempo de execução na memória heap. Assim, pode ter seu tamanho
redimensionado conforme a necessidade.

\begin{ccode}
    int *a = malloc( 3*sizeof(int) ); // aloca memória para um vetor de 3 inteiros
    free(a);                          // desaloca a memória alocada
\end{ccode}

Apesar dessa diferença, ambos os tipos de vetores funcionam da mesma maneira, usufruindo
da aritmética de ponteiros e acesso instantâneo ao valor armazenado.

\begin{ccode}
    int b[3];
    int *c = malloc( 3*sizeof(int) );

    b[0] = 1; b[1] = 3; b[2] = 5;
    c[0] = 2; c[1] = 4; c[2] = 6;

    printf("b[1] = \%d, *(c+2) = \%d\n", b[1], *(c+2)); // b[1] = 3, *(c+2) = 6
\end{ccode}

O Transmission não aloca seus vetores dinâmicos literalmente desta forma, pois possui
suas funções próprias onde encapsula o código mostrado.

%!TEX root = ../../tcc.tex

\newpage
\subsection*{Listas ligadas}

Listas ligadas é uma estrutura de dado que organiza os objetos de forma linear, assim
como os vetores. Porém, enquanto estes possuem índices que determinam a sua posição, as
listas possuem ponteiros para os outros elementos. Por causa disso, elas crescem
organicamente, conforme novos elementos vão sendo criados e associados, então evitando
desperdício de memória.

Por conter ponteiros, elementos de listas ligadas sempre são definidos usando-se
estruturas.

\begin{ccode*}{label=./libtransmission/makemeta.c:41}
struct FileList {
    uint64_t          size;
    char *            filename;
    struct FileList * next;      // ponteiro para o próximo elemento
};
\end{ccode*}

Essa estrutura é utilizada como se fosse um tipo definido pelo usuário, que aloca em
memória de forma dinâmica.

\begin{ccode*}{label=./libtransmission/makemeta.c:41}
struct FileList {
    uint64_t          size;
    char *            filename;
    struct FileList * next;      // ponteiro para o próximo elemento
};
\end{ccode*}

Feito isso, para a estrutura poder ser usada, o Transmission utiliza de uma função que
aloca uma variável dessa estrutura de forma dinâmica e seta valores iniciais nulos para
os seus campos.

\begin{ccode*}{label=./libtransmission/list.c:19}
static tr_list * recycled_nodes = NULL;

static tr_list* node_alloc(void) {
    tr_list * ret;                  // ponteiro que apontará para a região alocada

    if (recycled_nodes == NULL) {   // Se não houver elementos reciclados,...
        ret = tr_new(tr_list, 1);   // ... aloque um novo.
    }
    else {   // Caso contrário, reutilize, reapontando as tomando o controle do central
        ret = recycled_nodes;           // ... referências dos elementos adjacentes...
        recycled_nodes = recycled_nodes->next;   // ... e tomando o controle do central
    }

    *ret = TR_LIST_CLEAR;           // limpa campos do elemento
    return ret;                     // devolve o ponteiro para o elemento
}
\end{ccode*}

Existem vários tipos de listas ligadas, algumas podendo, inclusive, serem combinadas
entre si:

\begin{itemize}
    \item simplesmente ligada: possui somente um ponteiro para o próximo elemento
    \item duplasmente ligada: possui 2 ponteiros (um para o elemento anterior e outro
        para o posterior)
    \item multiplamente ligada: possui ponteiros vários elementos, porém ligando-os em
        ordens diferentes
    \item circularmente ligada: quando o último elemento liga a lista de volta ao
        1º elemento
    \item com cabeça: quando possui um elemento falso somente para ajudar a manipular as
        listas
\end{itemize}

\todoquestion{mostro algoritmos?}

Comparando-se vetores e listas ligadas, cada um tem suas vantagens e desvantagens em
relação à complexidade de seus algoritmos de manipulação.

\newpage
\begin{table}
    \centering
    \begin{tabular}{| l | c | c | c |}
        \hline
        \textbf{Ação} & \textbf{Vetor (est.)} & \textbf{Vetor (din.)} & \textbf{Lista ligada} \\
        \hline
        Busca por posição & $\Theta(1)$ & $\Theta(1)$ & $\Theta(n)$ \\
        \hline
        Inserção/Remoção (início) & $\Theta(n)$ & $\Theta(n)$ & $\Theta(1)$ \\
        \hline
        Inserção/Remoção (fim) & $\Theta(1)$ & $\Theta(1)$ & \parbox[t]{.3\textwidth}{\centering $\Theta(1)$ (c/ cabeça) \\ $\Theta(n)$ (s/ cabeça)} \\
        \hline
        Inserção/Remoção (meio) & $\Theta(n)$ & $\Theta(n)$ & $\Theta(n)$ \\
        \hline
        Redimensionamento & $\Theta(n)$ & ? & não necessita \\
        \hline
    \end{tabular}
    \caption{tabela de comparação de complexidades dos algoritmos de manipulaçãp de
    vetores e listas ligadas. OBS: tempos de buscas considerados lineares.
    Redimensionamento de vetor dinâmico depende da implementação da linguagem C.}
\end{table}

%!TEX root = ../../tcc.tex

\subsection*{Filas}
