%!TEX root = ../../tcc.tex

\section{Funções de hash}
\label{sec:sha1}

\begin{comment}
Aqui vou explicar como funciona o algoritmo da \gls{hashfunction} SHA-1 e mostrar como
e para que é usado na identificação de torrents e na verificação de integridade de
partes.
\end{comment}

\Glspl{hashfunction} são funções matemáticas usadas para gerar conteúdo de comprimento
fixo para referência ao conteúdo original.

Isso é útil quando existe grandes quantidades de dados a serem indexados. Por exemplo,
numa busca em uma tabela de dados ou tarefas de comparação de dados tal como detecção de
duplicatas ou de trechos de sequências de DNA semelhantes. Outro uso é na criptografia,
quando é utilizado para comparar um conjunto de dados recebido com outro já existente,
verificando sua igualdade.

Em geral, \glspl*{hashfunction} não são inversíveis, ou seja, não é possível recuperar o
valor de entrada para um dado \gls{hashvalue}. Quando usado para fins criptográficos,
são construídas de forma que essa reconstrução seja impossível sem que um imenso poder
computional seja utilizado. Por conta disso, é igualmente difícil fingir um
\gls*{hashvalue} para esconder dados maliciosos, sendo então usado pelo algoritmo
\gls{pgp}.

Outra característica importante é o determinismo. Quando a função é executada para 2
dados iguais, deve produzir o mesmo valor. Essa condição é fundamental no caso de uma
\gls*{hashtable}, pois a busca deve encontrar o mesmo local onde o algoritmo de inserção
armazenou o dado, logo precisando do mesmo \gls*{hashvalue}.

Outros usos para \glspl*{hashfunction} são em \glspl{checksum}, códigos de correção de
erros e cifras.

%!TEX root = ../../tcc.tex

\subsection*{SHA-1}

O SHA-1 é uma \gls*{hashfunction} criada pela NSA, a Agência de Segurança Nacional
americana, em 1995, e tem seu nome da abreviação de \emph{Secure Hash Algorithm}
(algoritmo de hash seguro). Ela produz um \gls*{hashvalue} de 160 bits (ou 20 bytes),
que forma um número hexadecimal de 40 caracteres.

Seu uso foi difundido depois que seu predecessor, o algoritmo MD5, foi constatado
com colisão de \gls*{hashvalue} prática realizada em um computador comum.

O seu algoritmo é relativamente simples se comparado com seus irmãos SHA-2 e SHA-3, o
que lhe confere a melhor vazão dentre as 3 versões.

\begin{algorithm}
    \caption{SHA1 (M)}
    \label{sha1}
    V $\leftarrow$ SHF1 (5A827999 || 6ED9EBA1 || 8F1BBCDC || CA62C1D6, M)
\end{algorithm}

\begin{algorithm}
    \caption{SHF1 (K,M)}
    y $\leftarrow$ shapad (M) \\
    Parse y as $M_1$ || $M_2$ || \ldots || $M_n$, onde $|M_i| = 512 (1 \leq i \leq n)$\\
    V $\leftarrow$ 67452301 || EFCDAB89 || 98BADCFE || 10325476 || C3D2E1F0 \\
    \Para{$i \leftarrow 1$ \Ate $n$}{
        V $\leftarrow$ shf1 (K, $M_i || V$)
    }
    \Retorna{V}
\end{algorithm}



\begin{comment}

    \FOR{$i=0$ to $79$}
    \IF{$0 \leq i \leq 19$}
    \STATE $T = a \lll 5 + f_{if}(b,c,d) + e + W[i] + $K0
    \ELSIF{$20 \leq i \leq 39$}
    \STATE $T = a \lll 5 + f_{xor}(b,c,d) + e + W[i] + $K1
    \ELSIF{$40 \leq i \leq 59$}
    \STATE $T = a \lll 5 + f_{maj}(b,c,d) + e + W[i] + $K2
    \ELSIF{$60 \leq i \leq 79$}
    \STATE $T = a \lll 5 + f_{xor}(b,c,d) + e + W[i] + $K3
    \ENDIF
    \STATE $e = d$, $d = c$, $c = b \lll 30$, $b = a$, $a = T$
    \ENDFOR
    \STATE $H_0 = a + H_0$, $H_1 = b + H_1$, $H_2 = c + H_2$, $H_3 = d + H_3$, $H_4 = e + H_4$
    \ENDFOR
    \RETURN concat($H_0$, $H_1$, $H_2$, $H_3,$ $H_4$)

    \SetAlgoLined
    \LinesNumbered
    \Dados{audio, tamanho, nivel, filtro, ch}
    \Entrada{i, j, inicio, comprimento}
    \BlankLine
    $inicio \leftarrow 0$\;
      $comprimento \leftarrow tamanho$\;
    \Para{$i \leftarrow 0$ \Ate $i<nivel$}{
        $inicio \leftarrow 0$\;
    $comprimento \leftarrow tamanho/2^{i-1}$\;
    \Para{$j \leftarrow 0$ \Ate $j<2^{i-1}$}{
        \eSe{j é par}{
                transformada\_wavelet(audio[inicio],comprimento, 1,'n', filtro, ch)\;
            }{
                transformada\_wavelet(audio[inicio],comprimento, 1,'i', filtro, ch)\;
            }
        }
    }
\end{comment}