%!TEX root = ../../tcc.tex

\subsection*{SHA-1}

O SHA-1 é uma \gls*{hashfunction} criada pela NSA, a Agência de Segurança Nacional
americana, em 1995, e tem seu nome da abreviação de \emph{Secure Hash Algorithm}
(algoritmo de hash seguro). Ela produz um \gls*{hashvalue} de 160 bits (ou 20 bytes),
que forma um número hexadecimal de 40 caracteres.

Seu uso foi difundido depois que seu predecessor, o algoritmo MD5, foi constatado
com colisão de \gls*{hashvalue} prática realizada em um computador comum.

O seu algoritmo é relativamente simples se comparado com seus irmãos SHA-2 e SHA-3, o
que lhe confere a melhor vazão dentre as 3 versões.

\begin{algorithm}
    \caption{SHA1 (M)}
    \label{sha1}
    V $\leftarrow$ SHF1 (5A827999 || 6ED9EBA1 || 8F1BBCDC || CA62C1D6, M)
\end{algorithm}

\begin{algorithm}
    \caption{SHF1 (K,M)}
    y $\leftarrow$ shapad (M) \\
    Parse y as $M_1$ || $M_2$ || \ldots || $M_n$, onde $|M_i| = 512 (1 \leq i \leq n)$\\
    V $\leftarrow$ 67452301 || EFCDAB89 || 98BADCFE || 10325476 || C3D2E1F0 \\
    \Para{$i \leftarrow 1$ \Ate $n$}{
        V $\leftarrow$ shf1 (K, $M_i || V$)
    }
    \Retorna{V}
\end{algorithm}



\begin{comment}

    \FOR{$i=0$ to $79$}
    \IF{$0 \leq i \leq 19$}
    \STATE $T = a \lll 5 + f_{if}(b,c,d) + e + W[i] + $K0
    \ELSIF{$20 \leq i \leq 39$}
    \STATE $T = a \lll 5 + f_{xor}(b,c,d) + e + W[i] + $K1
    \ELSIF{$40 \leq i \leq 59$}
    \STATE $T = a \lll 5 + f_{maj}(b,c,d) + e + W[i] + $K2
    \ELSIF{$60 \leq i \leq 79$}
    \STATE $T = a \lll 5 + f_{xor}(b,c,d) + e + W[i] + $K3
    \ENDIF
    \STATE $e = d$, $d = c$, $c = b \lll 30$, $b = a$, $a = T$
    \ENDFOR
    \STATE $H_0 = a + H_0$, $H_1 = b + H_1$, $H_2 = c + H_2$, $H_3 = d + H_3$, $H_4 = e + H_4$
    \ENDFOR
    \RETURN concat($H_0$, $H_1$, $H_2$, $H_3,$ $H_4$)

    \SetAlgoLined
    \LinesNumbered
    \Dados{audio, tamanho, nivel, filtro, ch}
    \Entrada{i, j, inicio, comprimento}
    \BlankLine
    $inicio \leftarrow 0$\;
      $comprimento \leftarrow tamanho$\;
    \Para{$i \leftarrow 0$ \Ate $i<nivel$}{
        $inicio \leftarrow 0$\;
    $comprimento \leftarrow tamanho/2^{i-1}$\;
    \Para{$j \leftarrow 0$ \Ate $j<2^{i-1}$}{
        \eSe{j é par}{
                transformada\_wavelet(audio[inicio],comprimento, 1,'n', filtro, ch)\;
            }{
                transformada\_wavelet(audio[inicio],comprimento, 1,'i', filtro, ch)\;
            }
        }
    }
\end{comment}