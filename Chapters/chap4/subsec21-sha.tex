%!TEX root = ../../tcc.tex

\subsection*{SHA-1}

O SHA-1 é uma \gls*{hashfunction} criada pela NSA, a Agência de Segurança Nacional
americana, em 1995, e tem seu nome da abreviação de \emph{Secure Hash Algorithm}
(algoritmo de hash seguro). Seu uso foi difundido depois que seu predecessor, o
algoritmo MD5, foi constatado com colisão de \gls*{hashvalue} prática realizada em um
computador comum \cite{report:md5-attack}.

Pertencente a uma família de algoritmos, que conta ainda com as versões SHA-0, SHA-2
(com funcionamento para vários comprimentos de bits de saída) e SHA-3, o SHA-1 ainda é
confiável o suficiente e com vários estudos que, até agora, somente comprovaram ataques
por colisão de difícil realização atualmente. Essa família é caracterizada por possuir
algoritmos iterativos, baseada no desenho do algoritmo MD4 \cite{report:md4}.

O resulta dessa função é um \gls*{hashvalue} de 160 bits (ou 20 bytes), que forma um
número hexadecimal de 40 caracteres. A função de compressão do algoritmo consiste de
três partes:

\begin{enumerate}
    \item expansão da mensagem: a mensagem de entrada é expandida para que o bloco de
        dado total seja múltiplo de 512 bits;

    \item transformação de estado: consiste de passos simples de operações de números
        binários usando alguns valores pré-definidos, onde uma variável de encadeamento
        é usada de mensagem de entrada para iteração seguinte e os blocos da mensagem
        expandida se tornam as novas chaves de iteração;

    \item retroalimentação: ao final do processamento de um bloco de 512 bits, a
        mensagem de entrada da transformação de estado é adicionada ao valor de saída.
        Esta operação é chamada de construção de Davies-Meyer e garante que se a
        mensagem de entrada for fixada então a função de compressão é náo-inversível na
        variável de encadeamento.
\end{enumerate}

\todo[inline]{falar da implementação em C e o uso no Transmission}