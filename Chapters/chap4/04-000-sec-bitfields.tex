%!TEX root = ../../tcc.tex

\section{Bitfields}
\label{sec:bitfield}

Apesar de ser um simples array de bits usado no gerenciamento de partes que o programa
já baixou ou não, foi percebido que o seu uso de forma não-convencional, chamado de
\emph{lazy bitfield}, pode ajudar a evitar o controle de banda (chamado de modelagem de
tráfego, ou \emph{traffic shaping} em inglês), feito por \glspl{isp}.