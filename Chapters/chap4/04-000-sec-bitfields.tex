%!TEX root = ../../tcc.tex

\section{Bitfields e o traffic shaping}
\label{sec:bitfield}

No BitTorrent, os bitfields são usados como uma tabela de controle de quais partes de um
\gls*{torrent} o programa cliente já recebeu de outros \glspl*{peer}, permitindo
conhecer quais são as partes mais raras. Com eles, o \gls*{peer} consegue utilizar a
estratégia de \emph{Rarest First} (\pageref{subsubsec:rarest-first}), priorizando
partes raras antes das mais comuns.

Com o poder crescente da tecnologia de banda larga e a disseminação do uso do
BitTorrent em todo o mundo, as \glspl{isp} têm tido trabalho em manter a qualidade de
suas infraestruturas com a imensa quantidade de dados que trafegam pelos seus cabos.
Para evitar que o tráfego em excesso cause perda de desempenho no oferecimento do
serviço, as empresas realizam o chamado \emph{traffic shaping} (modelagem de tráfego),
que foi especificado em 1998 \cite{site:rfcshaping}.

Para conseguir controlar o grande volume da sua rede, as \glspl*{isp} tentam
classificar os dados de acordo com seus protocolos, as portas utilizadas e as
informações de cabeçalho dos pacotes. Feito isso, os pacotes entram na fila
correspondente ao seu tipo, até esta fila ser enviada depois de seguir o contrato de
tráfego da fornecedora da conexão. É dessa forma que os pacotes enviados durante o uso
de BitTorrent se atrasam ou, até mesmo, se perdem, causando limitação das taxas de
transmissão.

Especificamente, o controle do tráfego é admissível porque, durante a análise dos
pacotes, é possível perceber o distinto protocolo de \emph{handshake} do BitTorrent,
onde ocorre o envio dos bitfields. Para se evitar isso, foi desenvolvida a técnica de
\emph{lazy bitfield} (bitfield preguiçoso): o \gls*{peer} envia um bitfield indicando
que possui menos partes que a realidade, enviando a diferença em seguida, utilizando
mensagens \bverb|have|. Assim, o \gls*{peer} finge não ser um potencial \gls{seeder},
por exemplo, evitando ter sua rede regulada.

A proposta de extensão \emph{Fast} do protocolo BitTorrent \cite{site:bittorrent-fast},
que surgiu em 2008, permite o uso de uma mensagem simples para avisar outros
\glspl*{peer}, que também suportarem o protocolo, que o remetente completou o
\gls*{torrent} sem a necessidade de enviar o bitfield. Por conta disso, a probabilidade
da conexão ser controlada é menor.