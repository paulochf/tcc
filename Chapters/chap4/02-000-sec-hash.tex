%!TEX root = ../../tcc.tex

\section{Funções de hash}
\label{sec:sha1}

\Glspl{hashfunction} são funções matemáticas usadas para gerar conteúdo de comprimento
fixo que referencia o conteúdo original.

Isso é útil quando existem grandes quantidades de dados a serem indexados. Por exemplo,
numa busca em uma tabela de dados, ou tarefas de comparação de dados, tais como
detecção de duplicatas ou de trechos de sequências de DNA semelhantes. Outro uso é na
criptografia, quando é utilizado para comparar um conjunto de dados recebido com outro
já existente, verificando sua igualdade.

Em geral, \glspl*{hashfunction} não são inversíveis, ou seja, não é possível recuperar o
valor de entrada para um dado \gls{hashvalue}. Quando usadas para fins criptográficos,
são construídas de forma que essa recuperação seja impossível sem que um imenso poder
computacional seja utilizado. Por conta disso, é igualmente difícil fingir um
\gls*{hashvalue} para esconder dados maliciosos.

Outra característica importante é o determinismo. Quando a função é executada para dois
dados de entrada iguais, ela deve produzir o mesmo valor. Essa condição é fundamental
no caso de uma \gls*{hashtable}, pois a busca deve encontrar o mesmo local onde o
algoritmo de inserção armazenou o dado, logo, precisa do mesmo \gls*{hashvalue}.

Outros usos para \glspl*{hashfunction} são nas \glspl{checksum}, códigos de correção de
erros e cifras.

%!TEX root = ../../tcc.tex

\subsection*{SHA-1}

O SHA-1 é uma \gls*{hashfunction} criada pela NSA, a Agência de Segurança Nacional
americana, em 1995, e tem seu nome da abreviação de \emph{Secure Hash Algorithm}
(algoritmo de hash seguro). Seu uso foi difundido depois que seu predecessor, o
algoritmo MD5, foi constatado com colisão de \gls*{hashvalue} ocorrida na prática,
em um computador comum \cite{report:md5-attack}.

Pertencente a uma família de algoritmos, que conta ainda com as versões SHA-0, SHA-2
(com funcionamento para vários comprimentos de bits de saída) e SHA-3, o SHA-1 teve
falhas expostas comprovadas por colisão, ainda que de difícil realização atualmente.
Essa família é caracterizada por possuir algoritmos iterativos, baseada no desenho do
algoritmo MD4 \cite{report:md4}.

O resultado dessa função é um \gls*{hashvalue} de 160 bits (ou 20 bytes) na forma
hexadecimal. A função de compressão do algoritmo consiste de três partes:

\begin{enumerate}
    \item expansão da mensagem: a mensagem de entrada é expandida para que o bloco de
        dado total seja múltiplo de 512 bits;

    \item transformação de estado: consiste em passos simples de operações de números
        binários, utilizando alguns valores pré-definidos. Uma variável de encadeamento
        é usada como mensagem de entrada para a iteração seguinte e os blocos da
        mensagem expandida se tornam as novas chaves de iteração; e

    \item retroalimentação: ao final do processamento de um bloco de 512 bits, a
        mensagem de entrada da transformação de estado é adicionada ao valor de saída.
        Esta operação é chamada de construção de Davies-Meyer, e garante que, se a
        mensagem de entrada for fixada, a função de compressão será não-inversível na
        variável de encadeamento.
\end{enumerate}

Uma das implementações conhecidas para a linguagem C é a biblioteca OpenSSL
\cite{site:openssl}, de código aberto, e é usada pelo Transmission no desenvolvimento de
códigos de \glspl*{hashfunction}, criptografia de dados e dados pseudo-aleatórios. O
OpenSSL foi programado de forma otimizada, possuindo um código bastante diferente do
usual.

O Transmission, seguindo a \gls{api} do OpenSSL, possui a sua \gls*{hashfunction}
SHA-1. Na sua versão, calcula o \gls*{hashvalue} de um bloco de dados em partes,
utilizando-na para obter o \gls*{hashvalue} do \gls*{torrent}, na criação de um ID para
a \gls*{dht} que ele possui, e na verificação da integridade das partes obtidas de
outros \glspl*{peer}. Neste último, a parte completada tem seu \gls*{hashvalue}
calculado e comparado com o valor que está contido no \gls*{torrentfile}: se ambos os
valores coincidirem, então a parte foi adquirida sem perdas, caso contrário será baixada
novamente.

\cfile[label="\<openssl/sha.h\>:101"]{./Codes/chap4/003-sha-ctx.c}

\cfile[label="./libtransmission/crypto.c:38"]{./Codes/chap4/004-trsha.c}