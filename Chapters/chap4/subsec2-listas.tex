%!TEX root = ../../tcc.tex

\newpage
\subsection*{Listas ligadas}

Listas ligadas é uma estrutura de dado que organiza os objetos de forma linear, assim
como os vetores. Porém, enquanto estes possuem índices que determinam a sua posição, as
listas possuem ponteiros para os outros elementos. Por causa disso, elas crescem
organicamente, conforme novos elementos vão sendo criados e associados, então evitando
desperdício de memória.

Por conter ponteiros, elementos de listas ligadas sempre são definidos usando-se
estruturas, que são utilizadas como se fossem tipos definidos pelo usuário.

\cfile[label="./libtransmission/list.h:28"]{./Codes/chap4/001-lista-struct.c}

Feito isso, para a estrutura poder ser usada, o Transmission utiliza de uma função que
aloca memória para um objeto de forma dinâmica e seta valores iniciais nulos para os
seus campos.

\cfile[label="./libtransmission/list.c:19"]{./Codes/chap4/002-lista-code.c}

Existem vários tipos de listas ligadas, algumas podendo, inclusive, serem combinadas
entre si:

\begin{itemize}
    \item simplesmente ligada: possui somente um ponteiro para o próximo elemento
    \item duplasmente ligada: possui 2 ponteiros (um para o elemento anterior e outro
        para o próximo elemento)
    \item multiplamente ligada: possui ponteiros vários elementos, porém ligando-os em
        ordens diferentes
    \item circularmente ligada: quando o último elemento liga a lista de volta ao
        1º elemento
    \item com cabeça: quando possui um elemento falso somente para ajudar a manipular as
        listas
\end{itemize}

Comparando-se vetores e listas ligadas, cada um tem suas vantagens e desvantagens em
relação à complexidade de seus algoritmos de manipulação.

\begin{table}
    \centering
    \begin{tabular}{| l | c | c | c |}
        \hline
        \textbf{Ação} & \textbf{Vetor} & \textbf{Lista ligada} \\
        \hline
        Busca por posição & $\Theta(1)$ & $\Theta(n)$ \\
        \hline
        Inserção/Remoção (início) & $\Theta(n)$ & $\Theta(1)$ \\
        \hline
        Inserção/Remoção (fim) & $\Theta(1)$ & \parbox[t]{.3\textwidth}{\centering $\Theta(1)$ (c/ cabeça) \\ $\Theta(n)$ (s/ cabeça)} \\
        \hline
        Inserção/Remoção (meio) & $\Theta(n)$ & $\Theta(n)$ \\
        \hline
        Redimensionamento & \parbox[t]{.25\textwidth}{\centering $\Theta(n)$ (estático) \\ ? (dinâmico)} & não necessita \\
        \hline
    \end{tabular}
    \caption{tabela de comparação de complexidades dos algoritmos de manipulaçãp de
    vetores e listas ligadas. OBS: tempos de buscas considerados lineares.
    Redimensionamento de vetor dinâmico depende da implementação da linguagem C.}
\end{table}

Por conta da agilidade que é conseguida na manipulação de listas ligadas de tamanhos
imprevisíveis, o Transmission as utiliza em várias partes do seu código.