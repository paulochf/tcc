%!TEX root = ../../tcc.tex

\subsection*{Vetores}

Vetores (ou \emph{arrays}) são a implementação de vetores matemáticos de maneira
virtual. Na prática, consistem de sequências ou listas de variáveis do mesmo tipo. Na
linguagem C, podem ser declaradas de forma estática ou dinâmica.

Vetores estáticos têm tamanho fixo estabelecido na sua declaração em tempo de
compilação, tendo espaços de memória reservados na pilha de execução de acordo com o
tipo, não podendo ser alterado durante a execução do programa.

\begin{ccode}
    char announce[1024]; // URL de announce
\end{ccode}

Já os vetores dinâmicos são blocos de memória alocados e liberados durante a execução do
programa. Para isso, são utilizados ponteiros para um objeto do mesmo tipo de cada
elemento do vetor. Dessa forma, a memória é reservada em tempo de execução na memória
\emph{heap}. Assim, pode ter seu tamanho redimensionado conforme a necessidade.

\begin{ccode}
    int *a = malloc( 3*sizeof(int) ); // aloca memoria para um vetor de 3 inteiros
    free(a);                          // desaloca a memoria alocada
\end{ccode}

Apesar dessa diferença, ambos os tipos de vetores funcionam da mesma maneira, usufruindo
da aritmética de ponteiros e acesso instantâneo ao valor armazenado.

\begin{ccode}
    int b[3];
    int *c = malloc( 3*sizeof(int) );

    b[0] = 1; b[1] = 3; b[2] = 5;
    c[0] = 2; c[1] = 4; c[2] = 6;

    printf("b[1] = \%d, *(c+2) = \%d\n", b[1], *(c+2)); // b[1] = 3, *(c+2) = 6
\end{ccode}

O Transmission não aloca seus vetores dinâmicos literalmente desta forma, pois possui
funções próprias onde encapsula o código mostrado.