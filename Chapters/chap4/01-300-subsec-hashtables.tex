%!TEX root = ../../tcc.tex

\subsection*{Tabelas hash}

\Glspl{hashtable} são estruturas de dados eficientes na implementação de dicionários.
Apesar de buscas demorarem tanto quanto procurar um elemento em uma lista ligada
($\Theta(n)$ no pior caso), o espalhamento é bastante eficiente. Isso faz com que o
tempo médio de uma busca seja $\Oh(1)$~\cite{book:clrs}.

\todo[inline]{verificar se referências do tcc estão na mesma linha.}

Uma \gls*{hashtable} generaliza a noção do \emph{array} comum. Nele, o endereçamento
direto nos permite avaliar o conteúdo de uma posição em $\Oh(1)$. O que torna esta
tabela especial é a vantagem de transformar um certo conteúdo em um valor único,
chamado de chave, fornecendo um meio de se encontrar essa chave. Esse meio é uma
\gls{hashfunction}.

Às vezes, \glspl*{hashfunction} fazem com que dois conteúdos possuam a mesma chave, ou
seja, as chaves colidem. Para esses casos, existem várias técnicas de solução de
conflitos, porém colisões podem ser evitadas com boas \glspl*{hashfunction}, descritas a
seguir.

A \gls*{hashtable} usada pelo Transmission aparece na \gls*{dht}, porém de uma forma
mais simples: não existe ``a \gls*{hashfunction} da \gls*{dht}'', onde há uma função
característica para uma modelagem de tabela. Ao invés disso, as chaves já estão
calculadas, sendo os IDs dos \glspl*{torrent} e dos \glspl*{peer} do Kademlia.

Outra utilização de \glspl*{hashtable} no BitTorrent (apesar de não ser usada no
Transmission) é nas \textbf{árvores hash} ou \textbf{árvores de Merkle}
\cite{site:merkletree}. Essas árvores são usadas para organizar o grande
\gls*{hashvalue} das partes do \gls*{torrent}, contido no \gls*{torrentfile}, em uma
árvore cujas folhas possuem o \gls*{hashvalue} de uma parte e cada nó que não é uma
folha possui como valor os \glspl*{hashvalue} dos seus nós filhos. Dessa forma, o
cálculo de um \gls*{hashvalue} de um conjunto de partes contínuo pode ser adquirido em
$\Oh(\log n)$ \cite{artigo:merkletree-cripto}.
