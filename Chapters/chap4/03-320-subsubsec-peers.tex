%!TEX root = ../../tcc.tex

\subsubsection*{Comunicação com peers}

A comunicação entre \glspl*{peer} é criptografada usando RC4 e troca de chaves
Diffie-Hellman-Merkle \cite{wikivuze:encription}. Para isso, o protocolo de mensagens
de \glspl*{peer} é estendido de forma a efetuar esses procedimentos criptográficos:

\cfile[label="./libtransmission/crypto.c:60-79"]{./Codes/chap4/008-cripto-peer1.c}

1) envio do $Y_A$

A chave pública de $A$ ($Y_A$) é enviada juntamente com um trecho de dados aleatórios
cujo comprimento qualquer entre 0 e 512 bytes.

\cfile[label="./libtransmission/handshake.c:313"]{./Codes/chap4/009-cripto-peer2-send-ya.c}

2) recebimento do $Y_B$

\cfile[label="./libtransmission/handshake.c:313"]{./Codes/chap4/010-cripto-peer3-read-yb.c}

3) envio das chaves e métodos

\cfile[label="./libtransmission/handshake.c:313"]{./Codes/chap4/011-cripto-peer4-hashashash.c}

\toto[inline]{parei aqui}

\cfile[label="./libtransmission/handshake.c:313"]{./Codes/chap4/012-cripto-peer5-naosei.c}