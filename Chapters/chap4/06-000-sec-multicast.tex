%!TEX root = ../../tcc.tex

\newpage
\section{Multicast}

\begin{comment}
Apesar de não ser utilizado pelo protocolo BitTorrent, o multicast, que é uma forma de
entregar dados a um grupo de computadores simultaneamente numa só transmissão, é usado
pelo Transmission para tentar descobrir \glspl*{peer} que estão na mesma rede local,
otimizando as conexões.
\end{comment}

Na área de redes de computadores, \emph{multicast} é um conceito de entrega de
informação simultânea para um grupo de computadores a partir de uma única fonte,
utilizando algoritmos específicos para seu roteamento. Geralmente, é implementado sobre
os protocolos da camada de rede ou de enlace, é usado em aplicações que necessitam de
distribuição de ``um para muitos'' ou ``muitos para muitos'', como por exemplo as de
distribuição de dados em massa (atualização de servidores), transmissão de áudio e vídeo
contínuo (transmissões de vídeo ao vivo), aplicações de dados compartilhados
(teleconferência), fontes de dados (bolsa de valores), atualização de cache de Internet
e jogos multi-jogadores interativos \cite{book:kurose}.

Os endereços de \emph{multicast} são definidos pelo protocolo IP como a sub-rede formada
entre os endereços IP \sverb|224.0.0.0| até \sverb|239.255.255.255|, antigamente chamada
de \textbf{rede classe D}, onde cada trecho desse intervalo é reservado a um uso
específico, controlado pela IANA (\emph{Internet Assigned Numbers Authority})
\cite{site:iana-multicast}. Dentre estes, endereços contidos em no intervalo de
\sverb|239.0.0.0| a \sverb|239.255.255.255|, chamados de
\textbf{endereços de multicast de escopo administrativo}, são destinados a uso privado
dentro dos limites de organizações, porém podendo não ser globalmente únicos
\cite{site:rfcmulticast}. Cada endereço deles é de um grupo específico de
\emph{multicast}, cujas mensagens encaminhadas a esse endereço serão repassadas a todos
os nós de rede que estiverem conectados a ele.

Um tipo de mensagem enviada em \emph{multicast} é a que utiliza o padrão SSDP
(\emph{Simple Service Discovery Protocol}), que serve para anúncio e descoberta de
equipamentos conectados a uma rede, sem a necessidade de mecanismos de configuração
baseados em servidor. Esse protocolo envia mensagens semelhantes às HTTP utilizando o
protocolo \gls{udp}. Apesar de não ser especificado por nenhum órgao técnico, o SSDP é
amplamente utilizado e faz parte da especificação do UPnP
(\emph{Universal Plug and Play}) \cite{site:upnp}.

O BitTorrent utiliza o \emph{multicast} de forma não oficial (que não possui
especificação formal) para a \textbf{descoberta de \glspl*{peer} local}. Inicialmente
desenvolvido pelo programa $\mu$Torrent e adotado por outros, um \gls*{peer} se
conecta a um grupo de \emph{multicast} no endereço \sverb|239.192.152.143:6771| e
espera por mensagens SSDP \cite{site:utorrent-forum}, onde estão indicadas o endereço
IP e porta do \gls*{peer} remetente, adicionando à sua lista de \glspl*{peer}.

\cfile[label="./libtransmission/tr-lpd.c:425"]{./Codes/chap4/016-rec-lpd.c}

\newpage
Da mesma forma, ele envia mensagens frequentemente para o grupo, para avisar da sua
existência na rede.

\cfile[label="./libtransmission/tr-lpd.c:425"]{./Codes/chap4/017-send-lpd.c}