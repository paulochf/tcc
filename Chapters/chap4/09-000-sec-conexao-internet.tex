%!TEX root = ../../tcc.tex

\section{Conexão com a Internet}

Programas que se conectam à redes ou à Internet são comuns atualmente. Para isso, a
linguagem C permite escrever programas criar conexões de Internet via \glspl{socket}
\cite{site:beej}.

Através de um \gls*{socket}, a aplicação consegue enviar mensagens para a camada de
transporte, que as transforma em segmentos e as repassa para as camadas inferiores, que
transmitem os dados.

Quando o protocolo \gls{tcp} for usado, os passos para se criar uma conexão são

\begin{enumerate}
    \item configuração do \gls*{socket} em uma variável \textbf{sadd} do tipo
        \sverb|struct sockaddr_in| (para IPv4) ou \sverb|struct sockaddr_in6| (para
        IPv6), que conterá informações do computador de destino da conexão.

        Enquanto isso, ocorre a criação de um \gls*{socket} usando a função
        \sverb|socket()|. Essa função, passando o parâmetro \bverb|SOCK_STREAM|, pede
        para o sistema operacional (SO) a criação de um descritor de arquivo especial
        para \gls*{socket} \gls*{tcp}, que será usado para o fluxo de dados da conexão;

    \item ligação do \gls*{socket} à variável \textbf{sadd} usando a função
        \sverb|bind()|, que serve para registrar no SO que ele deve manipular os dados
        que chegam na porta indicada em \textbf{sadd} usando o \gls*{socket} indicado;

    \item aguarda conexões no \gls*{socket} usando a função \sverb|listen()|; e

    \item recebe dados pelo \gls*{socket} usando a função \sverb|accept()|.
\end{enumerate}

\cfile[label="./libtransmission/net.c:317"]{./Codes/chap4/018-net-tcp.c}

\cfile[label="./libtransmission/net.c:417"]{./Codes/chap4/019-net-tcp2.c}

Já quando o protocolo \gls{udp} for usado, os passos para se criar uma conexão são
exatamente os mesmos, exceto pelo parâmetro \bverb|SOCK_DGRAM| na função
\sverb|socket()|.

\cfile[label="./libtransmission/tr-udp.c:190"]{./Codes/chap4/020-net-udp.c}

As funções \sverb|send()| e \sverb|recv()|, que enviam e recebem dados respectivamente,
possuem as suas versões mais complexas \sverb|sendto()| e \sverb|recvfrom()|, recebendo
outro \gls*{socket} como parâmetro de entrada.