%!TEX root = ../../tcc.tex

\subsection*{BitTorrent prefere UDP}

Em 2008, a substituição do \gls*{tcp} pelo \gls*{udp}, como protocolo utilizado na
troca de partes no BitTorrent, foi desenvolvida pioneiramente pelo programa cliente
$\mu$Torrent. Essa mudança instantaneamente causou discussões acaloradas pela Internet.

Richard Bennett, arquiteto de redes que escreveu o primeiro padrão Ethernet para cabos
de par trançado e ajudou no desenvolvimento de protocolos wifi, publicou um artigo
\cite{site:register-bennett} que dizia que a substituição para o protocolo \gls*{udp}
poderia causar um colapso da Internet como um todo, pois os \glspl{isp} não teriam como
controlar o tráfego de dados BitTorrent por ele, já que necessitavam que os dados
fossem transmitidos pelo protocolo \gls*{tcp}. Esse colapso afetaria usuários de outros
serviços \gls*{udp} (na sua maioria, aplicações de tempo real), como jogos online ou
sistemas de comunicação VoIP.

Enquanto isso, desenvolvedores de programas cliente responderam dizendo que não existiam
motivos para preocupações. Simon Morris, que na época era chefe da gestão de produto da
empresa BitTorrent, afirmou \cite{site:dslreports-bennett} que a substituição de
protocolo permitiria um melhor controle de congestionamentos das transmissões de dados.
Com o \gls*{tcp}, esses congestionamentos só seriam conhecidos depois de suas
ocorrências. Por outro lado, com o uso do \gls*{udp}, seria possível prevê-los
medindo-se as taxas de transmissão entre \glspl*{peer}.

Outro depoimento foi o de um gerente de comunidade do $\mu$Torrent, chamado Firon, que
explicou a um site de notícias relacionadas a BitTorrent
\cite{site:torrentfreak-bennett} o mesmo argumento dado por Simon Morris. Além disso,
acrescentou que o uso do \gls*{tcp} no BitTorrent significaria o uso de dois protocolos
de \emph{handshake} e, portanto, a mudança para \gls*{udp} eliminaria essa redundância,
reduzindo o tráfego na Internet.

Essa grande discussão ainda causa reflexos nos dias atuais, pois \glspl*{isp} ainda
tentam controlar o fluxo de dados BitTorrent no seu fornecimento de Internet. Porém,
conforme os dados de uma pesquisa recente sobre a quantidade de
tráfego na Internet durante horários de pico \cite{report:internet-usage-2013},
60,47\% do tráfego de download e 54,97\% do tráfego total norte-americanos são gerados
pelos serviços de \emph{streaming} de vídeo Netflix e Youtube e por navegação usual de
sites. Enquanto isso, o BitTorrent consome apenas 5,57\% e 9,23\%, respectivamente.
Portanto, não se pode afirmar com confiança que o BitTorrent seja um grande responsável
pelo congestionamento de dados nos Estados Unidos
\cite{site:theguardian-bittorrentvsnetflix}.

Outra evidência de que o protocolo \gls*{udp} pode ter importância em aplicações onde o
\gls*{tcp} seria preferido, é que o Google está trabalhando atualmente no QUIC
(\emph{Quick \gls*{udp} Internet Connections}), que fará parte da especificação do
protocolo HTTP 2.0 \cite{site:chromium-quic}. O QUIC, apesar de aumentar o consumo de
banda, reduzirá o tempo de resposta de confirmação de recebimento de pacotes, entre
outras melhorias.