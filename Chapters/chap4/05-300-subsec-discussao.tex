%!TEX root = ../../tcc.tex

\subsection*{BitTorrent prefere UDP}

A substituição do \gls*{tcp} pelo \gls*{udp} para se trocar partes no BitTorrent,
desenvolvida pioneiramente pelo programa cliente $\mu$Torrent, causou discussões
acaloradas pela Internet. Richard Bennet, arquiteto de redes que escreveu o primeiro
padrão Ethernet para cabos de par trançado e ajudou no desenvolvimento de protocolos
wifi, publicou um artigo \cite{site:register-bennett} dizendo que a troca para o
\gls*{udp} poderia causar um colapso da Internet como um todo, pois os \glspl*{isp} não
teriam como controlar o tráfego de dados BitTorrent, porque necessitavam que estes
fossem pacotes \gls*{tcp}. Esse colapso afetaria usuários de outros serviços
\gls*{udp}, que na sua maioria eram aplicações de tempo real, como jogos online ou
sistemas de comunicação VoIP.

Enquanto isso, desenvolvedores de programas cliente responderam dizendo que não existiam
motivos para preocupações. Simon Morris, que na época era chefe da gestão de produto da
empresa BitTorrent, disse \cite{site:dslreports-bennett} que a troca era para se ter um
melhor controle de congestionamento das transmissões de dados, pois com o \gls*{tcp} o
conhecimento sobre o problema só surge depois que ele ocorre. Por outro lado, usando
\gls*{udp} é possível prever congestionamentos medindo as taxas de transmissão entre
\glspl*{peer} e, com isso, diminuir a velocidade de envio.

Outro depoimento é de um gerente de comunidade do $\mu$Torrent chamado Firon, que
explicou a um site de notícias relacionadas a BitTorrent
\cite{site:torrentfreak-bennett} o mesmo argumento dado por Simon Morris, e ainda
acrescentou que o BitTorrent já possui um protocolo de \emph{handshake} e que usar o
protocolo \gls*{tcp} redundaria esse processo e, por isso, a mudança para \gls*{udp}
reduziria o tráfego na Internet.

Com essa grande discussão, podemos ver que ainda causa reflexos nos dias atuais, pois
\glspl*{isp} ainda tentam controlar o fluxo de dados BitTorrent do seu fornecimento de
Internet. Porém, conforme os dados de uma pesquisa recente sobre a quantidade de
tráfego de Internet gerada nos horários de pico \cite{report:internet-usage-2013},
60,47\% do tráfego de download e 54,97\% do tráfego total norte-americano são gerados
pelos serviços de \emph{streaming} de vídeo Netflix e Youtube e por navegação de sites
via HTTP, enquanto o BitTorrent consome apenas 5,57\% e 9,23\%, respectivamente.
Portanto, não se pode afirmar contundentemente que o BitTorrent seja o responsável pelo
congestionamento de dados nos Estados Unidos.

Outro fato de que o protocolo \gls*{udp} pode ter importância em aplicações onde o
\gls*{tcp} seria preferido é que o Google está trabalhando no QUIC
(\emph{Quick \gls*{udp} Internet Connections}), que fará parte da especificação do
protocolo HTTP 2.0 \cite{site:chromium-quic}. O QUIC, apesar de aumentar o consumo de
banda, reduzirá o tempo de resposta de confirmação de recebimento de pacotes, entre
outras melhorias.