%!TEX root = ../../tcc.tex

\newpage
\subsection*{Tabelas hash}

\Glspl{hashtable} são estruturas de dados eficientes na implementação de dicionários.
Apesar de buscas demorarem tanto quanto procurar um elemento em uma lista ligada -
$\Theta(n)$ no pior caso -, o espalhamento é bastante eficiente. Isso faz com que o
tempo médio de uma busca seja $O(1)$ \cite{book:clrs}.

Uma \gls*{hashtable} generaliza a noção do vetor de elementos comum. Nele, o
endereçamento direto nos permite avaliar o conteúdo de uma posição em $O(1)$. O que
torna esta tabela especial é a vantagem de transformar um certo conteúdo possuir uma
chave específica e exclusiva, fornecendo um meio de se encontrar essa chave. Esse meio é
uma \gls{hashfunction}.

Às vezes, \glspl*{hashfunction} fazem com que 2 conteúdos possuam a mesma chave, ou
seja, as chaves colidem. Para esses casos, existem várias técnicas de solução de
conflitos, porém colisões podem ser evitadas com boa \gls*{hashfunction}, descritas a
seguir.

A \gls*{hashtable} usada pelo Transmission aparece no \gls*{dht}, porém de uma forma
mais simples: não existe ``a \gls*{hashfunction} do \gls*{dht}'' como de costume, onde
existe uma função característica para uma modelagem de tabela. Ao invés disso, as chaves
já estão calculadas, sendo os IDs do \glspl*{torrent} e dos \glspl*{peer} do Kademlia.