%!TEX root = ../../tcc.tex

\section{Criptografia}

A criptografia é o estudo e prática de técnicas que visam a segurança de informações de
várias maneiras, a fim de que todas as partes de uma transação confiem que os objetivos
daquela segurança tenham sido alcançadas. Com início há mais de quatro mil anos atrás,
no Egito antigo, a área se manteve em atividade com a necessidade de se cifrar e
quebrar mensagens, se tornando mais necessária com o advento da computação.

Os quatro objetivos principais da criptografia são de oferecer
\cite{book:applied-crypto}:

\begin{description}
    \item[confidencialidade,] para se manter um conteúdo de informação sob sigilo
        somente àqueles autorizados a visualizá-lo. Suas inúmeras formas vão desde a
        proteção fisica até os algoritmos matemáticos que tornam o conteúdo
        ininteligível;

    \item[integridade dos dados,] que foca na alteração não autorizada de dados, de
        forma a detectar essa manipulação;

    \item[autenticação,] relacionada a identificação de dados e entidades. Duas partes
        que iniciam uma comunicação devem se identificar. Da mesma forma, uma
        informação deve poder ser autenticada a partir do envio para o destinatário;

    \item[aceitação,] que previne uma entidade de negar uma ação ou compromisso
        previamente estabelecido, necessitando de um procedimento que envolva um
        terceiro para resolver alguma disputa.
\end{description}

Existem dois tipos principais de criptografia: por chaves simétricas ou por chaves
públicas. Por meio de chaves simétricas, um sistema criptográfico usa uma mesma chave
para encriptar e decriptar mensagens, sejam estas feitas por cifragem de fluxo de dados
(onde cada caractere da mensagem é computado por vez) ou de blocos de dados (onde blocos
da mensagem são computados). Uma desvantagem das chaves simétricas \cite{wiki:crypto} é
que é necessário um gerenciamento das chaves utilizadas, a fim de se manter a
criptografia segura. Para isso, idealmente, cada par de entidades que desejam se
comunicar entre si devem compartilhar a mesma chave, fazendo com que o número total de
chaves necessárias numa rede seja proporcional ao quadrado do número de membros. Este
era o único método conhecido até junho de 1976 \cite{artigo:diffiehellman}.

Já as criptografias que utilizam de chaves públicas surgiram a partir de outro trabalho
de Whitfield Diffie, Martin Hellman \cite{artigo:diffiehellman-public}, onde propuseram
o sistema criptográfico por meio da troca de chaves assimétricas (\emph{Diffie-Hellman
Key Exchange}). Nessa troca, são usadas duas chaves diferentes porém relacionadas
matematicamente, onde a chave privada - para posse apenas do seu dono - é usada para a
geração de uma chave pública para distribuição livre. Assim, enquanto a chave pública
de uma entidade é usada para a criptografia dos dados pela outra entidade, apenas a
respectiva chave privada pode descriptografá-la.

Atualmente, existem vários algoritmos e métodos de criptografia, cada qual com suas
vantagens e desvantagens, e cenários de aplicação mais adequados do que outros. Porém,
a criptografia tem como mote ``não existe sistema de segurança impenetrável'', o que nos
leva à área de criptoanálise, que é a arte e ciência de se analisar sistemas de
segurança de forma a descobrir seus aspectos ocultos, utilizando-os para quebrar os
respectivos sistemas criptográficos. Assim, muitos dos métodos existentes possuem falhas
descobertas: enquanto algumas já são práticas, outras são apenas teóricas por falta de
capacidade computacional atual.

%!TEX root = ../../tcc.tex

\subsection*{Troca de chaves Diffie-Hellman}

Este método, também conhecido por \emph{Diffie-Hellman-Merkle Key Exchange}, é uma das
formas mais conhecidas de troca de chaves criptográficas. Ele permite que duas partes
que não possuem conhecimento a priori do outro estabeleçam um segredo comum utilizando
métodos de comunicação inseguros.

O método algébrico utiliza grupos multiplicativos de inteiros módulo $p$, onde $p$ é um
número primo e $g$ é chamado de número gerador. O protocolo pode ser explicado no
seguinte exemplo, que usa duas partes (Alice e Bob)
\cite{book:schneier,artigo:diffiehellman}: suponha que Alice e Bob desejam estabelecer
um meio seguro de comunicação, que Eve deseja espionar realizando um ``ataque de homem
no meio'', tendo acesso a todas as informações que Alice e Bob trocarem.

\begin{enumerate}
    \item Alice e Bob escolhem dois números inteiros $p$ primo e $g$ gerador;

    \item Alice escolhe um número inteiro $X_{A}$ para sua chave privada, e envia para
        Bob o resultado $Y_{A} = g^{X_{A}} \bmod p$.

    \item Bob então escolhe um número inteiro $X_{B}$ para sua chave privada, e envia
        para Alice o resultado $Y_{B} = g^{X_{B}} \bmod p$.

    \item Alice, então, calcula a chave compartilhada $S_A = B^{X_{A}} \bmob p$. Bob faz
        o mesmo, ou seja, calcula $S_B = A^{X_{B}} \bmob p$.

    \item como $S_A = S_B = S$, Alice e Bob passam a utilizar a chave $S$.
\end{enumerate}

A princípio, não é óbvio ver que $S_A = S_B = S$, mas é fácil mostrar. Considere Alice e
suas chaves. A chave que ela recebeu de Bob, $Y_{B}$, foi resultado de
$Y_{B} = g^{X_{B}} \bmod p$. Então, o cálculo de $S_A$ feito por ela é equivalente a
$S_A = (g^{X_{B}})^{X_{A}} \bmod p$.

Analogamente, Bob recebeu $Y_{A}$ de Alice, que foi resultado de
$Y_{A} = g^{X_{A}} \bmod p$. Assim, o cálculo dele de $S_B$ é equivalente a
$S_B = (g^{X_{A}})^{X_{B}} \bmod p$. Porém, podemos manipular um pouco as equações,
chegando em
\begin{align*}
S_A & = (g^{X_{B}})^{X_{A}} \bmod p \\
    & = g^{(X_{B}.X_{A})}   \bmod p \\
    & = g^{(X_{A}.X_{B})}   \bmod p \\
    & = (g^{X_{A}})^{X_{B}} \bmod p \\
    & = S_B
\end{align*}

Veja que não importa que Eve tenha obtido $p$, $g$, $Y_{A}$ ou $Y_{B}$. Ela não
conseguirá obter $S$ pois este depende de $X_{A}$ e $X_{B}$. Além disso, Eve pode
tentar calcular $X_{A}$ e $X_{B}$, porém a dificuldade deste cálculo depende dos
tamanhos de $p$, $X_{A}$ e $X_{B}$: quanto maiores forem esses números, mais difíceis
serão os cálculos inversos, que são chamados de ``problemas de logaritmo discreto''. É
praticamente impossível descobrir essas chaves privadas em uma quantidade de tempo
razoável. Assim, esse método é considerado seguro, enquanto a computação quântica não
estiver desenvolvida o suficiente para que novos algoritmos possam ser usados
\cite{artigo:shor}.

%!TEX root = ../../tcc.tex

\subsection*{RC4}

O RC4 (ou ainda ``Rivest Cypher 4'', ``Ron's Code 4'' ou ``Arc Four'') é uma função
criptográfica, criada por Ron Rivest, em 1987. Inicialmente, era um segredo comercial,
porém, em 1994, seu código foi publicado na lista de discussão de criptografia
CypherPunks \cite{site:rc4-code}, se espalhando pela Internet rapidamente. Seu uso se
tornou comum, sendo utilizado por muitos softwares, chegando a protocolos como as
encriptações de placas de rede sem fio WEP e WPA, ou ainda o protocolo de segurança TLS
para conexões de Internet.

O RC4 é um algoritmo de chave simétrica que se divide em duas partes: na primeira
parte, ele executa o algoritmo de escalonamento de chaves (\emph{key scheduling}),
que utiliza uma chave de tamanho variável entre 1 e 256 bytes para inicializar uma
tabela de estados. Cada elemento dessa tabela é permutado pelo menos uma vez, e será
usado na geração de bytes pseudoaleatórios na segunda parte.

Na segunda parte, executa o algoritmo de geração pseudoaleatório, onde modifica o estado
(também permutando os elementos pelo menos uma vez) e resulta em 1 byte da chave de
fluxo, que então é mesclada usando \gls{xor} bit a bit com o próximo byte da mensagem,
para produzir ou próximo byte da mensagem cifrada (na encriptação) ou da decifrada (na
decriptação), já que o \gls*{xor} é uma função involuntária (ou seja, é uma função que
é a própria inversa).

\cfile[label="./libtransmission/crypto.c:258"]{./Codes/chap4/005-rc4enc.c}

\cfile[label="./libtransmission/crypto.c:237"]{./Codes/chap4/006-rc4dec.c}

%!TEX root = ../../tcc.tex

\subsubsection*{Falhas de segurança}

The objective is NOT to create a cryptographically secure protocol that can survive unlimited observation of passing packets and substantial computational resources on network timescales. The objective is to raise the bar sufficiently to deter attacks based on observing ip-port numbers in peer-to-tracker communications.

If a tracker observes a large number of tracker requests and responses and subsequent connections, it is possible to attack the encryption. RC4 is known to have a number of weaknesses especially in the way it is used with WEP [2] [3] [4]. However, with tracker peer obfuscation, the number of bytes transferred between the tracker and a client is likely significantly smaller than transferred between a wireless computer and a basestation. An attacker faces a much larger task in obtaining sufficient ciphertext to directly break the encryption.

Hobbling the RC4 encryption by using a bounded-length RC4 pseudorandom string for small swarms is likely to have negilgible impact on security over any other encyption method since the pseudorandom string is probably equal to or longer than the plaintext and thus no part of it is repeated in the XOR except as peers arrive or leave the swarm. Thus on the timescales of rerequest intervals, nearly the same ciphertext is handed to every peer requesting the same infohash. Intercepting the same ciphertext multiple times provides no additional information to the attacker. The attacker could correlate ip-port pairs in connections following tracker responses, but an attacker could do this regardless of the encryption method employed. Furthermore more direct methods of traffic analysis applied to peer-to-peer communication is available to network operators.

For larger swarms, hobbling RC4 may simplify breaking the encryption since the same pseudorandom string is used repeatedly across the peer list. Some study is in order taking into account that the tracker can periodically change intiailization vectors.

%!TEX root = ../../tcc.tex

\subsection*{Criptografia no BitTorrent}

O BitTorrent usa criptografia somente quando o usuário habilita a opção correspondente
no programa cliente, criptografando comunicações com \glspl*{tracker} e \glspl*{peer}.
O método escolhido pelo protocolo é o que está definido em uma de suas propostas de
melhoria. Porém, esta está suspensa, sendo então usada de forma extraoficial atualmente
pelos programas cliente em geral, inclusive o Transmission.

Além disso, como é dito na própria proposta \cite{site:bittorrent-cripto}, o objetivo
dessa melhoria é impedir que \glspl{isp} ou outros administradores de rede de bloquear
ou quebrar conexões BitTorrent que ocorram entre o \gls*{peer} receptor de uma resposta
de um \gls{tracker} e qualquer outro \gls*{peer} cujo endereço IP e porta apareça nessa
resposta. Por isso, essa especificação é chamada de ofuscação de \glspl*{peer}. A idéia
proposta é usar o \bverb|info_hash| de um \gls*{torrent} como chave compartilhada entre
o \gls*{peer} e o \gls*{tracker}, não impedindo ataques de homem no meio por espiões que
saibam desses \glspl{hashvalue}.

%!TEX root = ../../tcc.tex

\subsubsection*{Comunicação com trackers}

Quando criptografia for habilitada, as comunicações com \glspl*{tracker}, ou seja, as
requisições de \glspl{announce}, não devem enviar o \bverb|info_hash| do \gls*{torrent}.
Ao invés disso, deve enviar o \bverb|sha_ih|, que é o \gls*{hashvalue} SHA-1 do
\bverb|info_hash| (que também é um \gls*{hashvalue} SHA-1) na forma \gls*{urlencode}.

Já a resposta de \glspl*{tracker} a \glspl*{announce} se mantêm no mesmo formato,
exceto pela lista de \glspl*{peer}, que será ofuscada.

Para isso, a requisição do \gls*{announce} deve passar como parâmetros

\begin{description}
    \item[supportcrypto:] valor 1 indica que o \gls*{peer} pode criar e receber
        conexões criptografadas. Neste caso, se o \gls*{tracker} aceitar esta extensão
        do BitTorrent, as listas de \glspl*{peer} que enviará em suas respostas
        priorizarão outros \glspl*{peer} que também enviaram \bverb|supportcrypto=1|
        antes dos que não o fizeram.

    \item[requirecrypto:] valor 1 indica que o \gls*{peer} irá criar e aceitar somente
        conexões criptografadas. Neste caso, as listas de \glspl*{peer} que o
        \gls*{tracker} enviará conterão somente \glspl*{peer} que também enviaram
        \bverb|supportcrypto=1| e \bverb|requirecrypto=1|.

    \item[cryptoport:] quando o parâmetro de \bverb|requirecrypto=1|, é inteiro que
        representa a porta na qual o cliente irá utilizar para conexões criptografadas.
\end{description}

\cfile[label="./libtransmission/announcer-http.c:58"]{./Codes/chap4/007-cripto-announce.c}

%!TEX root = ../../tcc.tex

\subsubsection*{Comunicação com peers}

A comunicação entre \glspl*{peer} é criptografada usando RC4 e troca de chaves
Diffie-Hellman-Merkle \cite{wikivuze:encription}. Para isso, o protocolo de
\emph{handshake} para mensagens entre \glspl*{peer} é estendido, de forma a efetuar
esses cinco procedimentos criptográficos:

\cfile[label="./libtransmission/crypto.c:60-79"]{./Codes/chap4/008-cripto-peer1.c}

\newpage
1) $A$ envia $Y_A$ para $B$:

O Transmission, nesse caso o \gls*{peer} $A$, envia sua chave pública ($Y_A$) com um
trecho de dados aleatórios, com comprimento qualquer entre 0 e 512 bytes;

\cfile[label="./libtransmission/handshake.c:313"]{./Codes/chap4/009-cripto-peer2-send-ya.c}

2) $A$ recebe $Y_B$ de $B$:

O outro \gls*{peer}, $B$, responde com sua chave pública ($Y_B$). Assim, a chave
compartilhada $S$ já pode ser calculada;

\cfile[label="./libtransmission/handshake.c:391"]{./Codes/chap4/010-cripto-peer3-read-yb.c}

\newpage
3) $A$ envia para $B$ as opções de criptografia e a mensagem de \emph{handshake}:

Então, $A$ envia dados de forma criptografada: a mensagem de \emph{handshake} e outras
informações sobre a criptografia para $B$, na forma:

\begin{verbatim}
HASH('req1', S), HASH('req2', SKEY) xor HASH('req3', S),
    ENCRYPT(VC, crypto_provide, len(PadC), PadC, len(IA)), ENCRYPT(IA)
\end{verbatim}

onde:

\begin{description}
    \item[HASH():] é a função que calcula o \gls*{hashvalue} SHA-1 de todos os
        parâmetros de entrada concatenados;

    \item[ENCRYPT():] é a função RC4 com chave \bverb|HASH('keyA', S, SKEY)| (se
        $A \rightarrow B$), ou \bverb|HASH('keyB', S, SKEY)| (se $B \rightarrow A$). Os
        primeiros 1024 bytes da encriptação RC4 são descartados. O uso seguido desta
        função por uma das partes encripta o fluxo de dados, sem reinicializações ou
        trocas;

    \item[VC:] é uma constante de verificação (\emph{verification constant}), que é uma
        string de 8 bytes de valor 0x00, usada para verificar se a outra parte conhece
        $S$ e SKEY, evitando ataques de repetição do SKEY;

    \item[crypto\_provide/crypto\_select:] são bitfield de 32 bits. Dois valores são
        usados atualmente, com o restante sendo reservado para uso futuro. O \gls*{peer}
        $A$ deve ligar os bits de todos os métodos suportados por ele, enquanto o
        \gls*{peer} $B$ deve ligar o bit do método escolhido dentre os oferecidos, e
        enviar como resposta. Por enquanto, 0x01 indica sem encriptação, e 0x02 indica o
        RC4;

    \item[PadC/PadD:] reservados para futuras extensões do handshake. Hoje, possuem 0
         bytes;

    \item[IA:] conjunto de dados inicial de $A$. Pode ser 0 bytes, se quiser esperar por
        negociação de encriptação;
\end{description}

\newpage
\cfile[label="./libtransmission/handshake.c:391", samepage=false]{./Codes/chap4/011-cripto-peer4-hashashash.c}

\newpage
4) $A$ recebe de $B$ a opção de criptografia escolhida e a mensagem de resposta ao
\emph{handshake} encriptada por RC4:

Aqui, $B$ envia como resposta:

\begin{verbatim}
ENCRYPT(VC, crypto_select, len(padD), padD), ENCRYPT2(Payload Stream)
\end{verbatim}

\cfile[label="./libtransmission/handshake.c:493", samepage=false]{./Codes/chap4/012-cripto-peer5-readvc.c}

\newpage
5) $A$ envia para $B$ o fim da mensagem de \emph{handshake} encriptada por RC4:

\cfile[label="./libtransmission/handshake.c:587"]{./Codes/chap4/013-cripto-peer6-handshake.c}
\cfile[label="./libtransmission/peer-io.c:1083"]{./Codes/chap4/014-cripto-peer7-write.c}