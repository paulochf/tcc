%!TEX root = ../../tcc.tex

\subsection*{SHA-1}

O SHA-1 é uma \gls*{hashfunction} criada pela NSA, a Agência de Segurança Nacional
americana, em 1995, e tem seu nome da abreviação de \emph{Secure Hash Algorithm}
(algoritmo de hash seguro). Seu uso foi difundido depois que seu predecessor, o
algoritmo MD5, foi constatado com colisão de \gls*{hashvalue} ocorrida na prática,
em um computador comum \cite{report:md5-attack}.

Pertencente a uma família de algoritmos, que conta ainda com as versões SHA-0, SHA-2
(com funcionamento para vários comprimentos de bits de saída) e SHA-3, o SHA-1 teve
falhas expostas comprovadas por colisão, ainda que de difícil realização atualmente.
Essa família é caracterizada por possuir algoritmos iterativos, baseada no desenho do
algoritmo MD4 \cite{report:md4}.

O resultado dessa função é um \gls*{hashvalue} de 160 bits (ou 20 bytes) na forma
hexadecimal. A função de compressão do algoritmo consiste de três partes:

\begin{enumerate}
    \item expansão da mensagem: a mensagem de entrada é expandida para que o bloco de
        dado total seja múltiplo de 512 bits;

    \item transformação de estado: consiste em passos simples de operações de números
        binários, utilizando alguns valores pré-definidos. Uma variável de encadeamento
        é usada como mensagem de entrada para a iteração seguinte e os blocos da
        mensagem expandida se tornam as novas chaves de iteração; e

    \item retroalimentação: ao final do processamento de um bloco de 512 bits, a
        mensagem de entrada da transformação de estado é adicionada ao valor de saída.
        Esta operação é chamada de construção de Davies-Meyer, e garante que, se a
        mensagem de entrada for fixada, a função de compressão será não-inversível na
        variável de encadeamento.
\end{enumerate}

Uma das implementações conhecidas para a linguagem C é a biblioteca OpenSSL
\cite{site:openssl}, de código aberto, e é usada pelo Transmission no desenvolvimento de
códigos de \glspl*{hashfunction}, criptografia de dados e dados pseudo-aleatórios. O
OpenSSL foi programado de forma otimizada, possuindo um código bastante diferente do
usual.

O Transmission, seguindo a \gls{api} do OpenSSL, possui a sua \gls*{hashfunction}
SHA-1. Na sua versão, calcula o \gls*{hashvalue} de um bloco de dados em partes,
utilizando-na para obter o \gls*{hashvalue} do \gls*{torrent}, na criação de um ID para
a \gls*{dht} que ele possui, e na verificação da integridade das partes obtidas de
outros \glspl*{peer}. Neste último, a parte completada tem seu \gls*{hashvalue}
calculado e comparado com o valor que está contido no \gls*{torrentfile}: se ambos os
valores coincidirem, então a parte foi adquirida sem perdas, caso contrário será baixada
novamente.

\cfile[label="\<openssl/sha.h\>:101"]{./Codes/chap4/003-sha-ctx.c}

\cfile[label="./libtransmission/crypto.c:38"]{./Codes/chap4/004-trsha.c}