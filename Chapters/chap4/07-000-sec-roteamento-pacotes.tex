%!TEX root = ../../tcc.tex

\section{Configuração e roteamento de pacotes em rede}

\begin{comment}
Em alguns roteadores que são ponte entre a Internet e a rede que ele gerencia, existe
uma função de se configurar portas de comunicação de rede automaticamente usando-se o
\emph{Network Address Translation Port Mapping Protocol}. Assim, não é necessário
realizar uma configuração específica somente pare esse fim.
\end{comment}

Atualmente, o uso de roteadores de rede para uso doméstico é bastante comum, servindo
para distribuição de uma conexão de Internet para vários equipamentos eletrônicos que
conseguem acessá-la. Para que ocorra essa distribuição de dados de Internet é
necessário que no roteador funcione um protocolo chamado NAT
(\emph{Network Address Translator}).

Um roteador em que funciona um serviço de NAT parece, externamente (para a rede
exterior), que é um único dispositivo com o seu endereço IP \cite{book:kurose}, enquanto
internamente, enquanto internamente é visto como o responsável por rotear os dados e
que abstrai a conexão de Internet externa. Do ponto de vista do roteador, ele conhece os
endereços IP de cada dispositivo da rede interna e o endereço IP do modem do \gls{isp}.
Assim, ele contrói uma tabela de tradução de endereços onde, para cada endereço da rede
interna, associa a uma porta de rede interna e outra externa.

Então, quando um dos dispositivos envia dados para a Internet, esse pacote passa pelo
roteador, que troca o endereço IP e porta do dispositivo de origem, contidos no
datagrama, pela respectiva tradução da rede externa, e então repassa o pacote para a
Internet. Analogamente, quando um pacote da Internet chega ao roteador, este verifica a
porta de conexão de destino, contida no datagrama, e a procura em sua tabela de
tradução. Se houver um dispositivo da rede interna associado a essa porta, o roteador
troca o endereço e porta de destinos do datagrama e repassa o pacote para tal
dispositivo.

Enquanto o serviço de NAT parece solucionar um problema, ele cria outro: \glspl*{peer}
de redes \gls*{p2p} necessitam saber as portas com que estão se comunicando para
informar a outros \glspl*{peer}. Porém, um programa cliente que esteja sendo executado
em um dispositivo que está numa sub-rede atendida por um serviço NAT deve saber informar
qual a porta externa na tabela de NAT está associada a ele, e não a que utiliza no
dispositivo.

Para resolver esse problema, existem dois protocolos diferentes de configuração de
portas: o NAT PMP (NAT \emph{Port Mapping Protocol}), que atualmente é chamado de PCP
(\emph{Port Control Protocol}) \cite{site:rfcpcp}, e o uPnP
(\emph{Universal Plug and Play}) \cite{site:rfcupnp}. Ambos esses protocolos têm a mesma
função: configurar um serviço de NAT e conhecer as portas externas que ele lhes
reservou, fazendo a travessia de NAT (\emph{NAT Traversal}).

O UPnP \cite{wiki:upnp} é um conjunto de protocolos que possibilita a comunicação entre
dispositivos variados a partir da conexão destes a uma rede, estabelecendo serviços.
Entre os diversos protocolos, está o IGDP (\emph{Internet Gateway Device Protocol}), no
qual é possível realizar várias ações, desde o conhecimento do endereço IP externo do
roteador, conhecer o mapeamento de portas internas e externas, e adicionar ou remover
entradas nesse mapeamento. Ao adicionar uma entrada, é possível realizar a travessia de
NAT. O UPnP utiliza as portas 1900 para o pacotes \gls*{udp} e a 2869 para portas
\gls*{tcp}.

Já o PCP \cite{wiki:pcp}, introduzido em 2005 pela Apple como alternativa ao IGDP,
serve somente para configuração de travessia de NAT e conhecimento do endereço externo
do gateway NAT, automatizando a configuração de redirecionamento de portas em
roteadores. Para isso, utiliza a porta 5351 para pacotes \gls*{udp}.

Essa praticidade de não precisar se configurar manualmente um roteador tem um preço.
Ambos os protocolos não são totalmente seguros, fazendo com que um roteador que permita
configurações através deles possa oferecer meios de se invadir essas redes por conexões
externas. Assim, para efeitos de segurança, um serviço NAT não substitui
\emph{firewalls}.

O Transmission utiliza duas bibliotecas, uma para cada protocolo: o MiniUPnP
\cite{site:miniupnp} e o libnatpmp \cite{site:libnatpmp}. Ambas são desenvolvidas por
Thomas Bernard e de código aberto.