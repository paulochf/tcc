%!TEX root = ../../tcc.tex

\subsubsection*{Comunicação com trackers}

Quando criptografia for habilitada, as comunicações com \glspl*{tracker}, ou seja, as
requisições de \glspl{announce}, não devem enviar o \bverb|info_hash| do \gls*{torrent}.
Ao invés disso, deve enviar o \bverb|sha_ih|, que é o \gls*{hashvalue} SHA-1 do
\bverb|info_hash| (que também é um \gls*{hashvalue} SHA-1) na forma \gls*{urlencode}.

Já a resposta de \glspl*{tracker} a \glspl*{announce} se mantêm no mesmo formato,
exceto pela lista de \glspl*{peer}, que será ofuscada.

Para isso, a requisição do \gls*{announce} deve passar como parâmetros

\begin{description}
    \item[supportcrypto:] valor 1 indica que o \gls*{peer} pode criar e receber
        conexões criptografadas. Neste caso, se o \gls*{tracker} aceitar esta extensão
        do BitTorrent, as listas de \glspl*{peer} que enviará em suas respostas
        priorizarão outros \glspl*{peer} que também enviaram \bverb|supportcrypto=1|
        antes dos que não o fizeram.

    \item[requirecrypto:] valor 1 indica que o \gls*{peer} irá criar e aceitar somente
        conexões criptografadas. Neste caso, as listas de \glspl*{peer} que o
        \gls*{tracker} enviará conterão somente \glspl*{peer} que também enviaram
        \bverb|supportcrypto=1| e \bverb|requirecrypto=1|.

    \item[cryptoport:] quando o parâmetro de \bverb|requirecrypto=1|, é inteiro que
        representa a porta na qual o cliente irá utilizar para conexões criptografadas.
\end{description}

\cfile[label="./libtransmission/announcer-http.c:58"]{./Codes/chap4/007-cripto-announce.c}