%!TEX root = ../../tcc.tex

\subsection*{TCP}

O protocolo \gls*{tcp} é um protocolo de conexão que pertence à camada de transporte,
que determina um meio de transmissão confiável de dados e orientado a conexão.

Quando as mensagens chegam pelos \glspl*{socket}, o protocolo \gls*{tcp} também as anexa
aos endereços IP e número de porta de origem e de destino, e também aos comprimentos e
\gls*{checksum} do cabeçalho e do corpo de dados \gls*{tcp}, para enfim serem repassadas
como segmentos para a camada de rede. Porém, acrescenta também outras informações:

\begin{itemize}
    \item campos de \emph{flags} sinalizadores, onde 6 bits indicam estados da conexão
        ou do segmento:
        \begin{description}
            \item[URG:] indicador de urgência do segmento enviado;

            \item[SYN:] número de sincronização de sequência, que deve ser enviado
                sempre no primeiro pacote de uma remessa de dados por um nó;

            \item[ACK:] indica que o valor contido no campo de confirmação de
                recebimento é válido. Ele deve ser enviado depois do primeiro pacote com
                SYN;

            \item[PSH:] para indicar ao destinatário que repasse os dados para a camada
                de aplicação imediatamente ao receber o pacote;

            \item[RST:] campo de redefinição de conexão, para indicar o erro de que não
                possui um \gls*{socket} na porta indicada no cabeçalho; e

            \item[FIN:] indica fim do envio dos dados pelo nó.
        \end{description}

    \item números de sequência e de confirmação de recebimento (\emph{acknowledgement}),
        para uso na implementação do serviço de entrega confiável de dados;

    \item quantidade de bytes que o destinatário deseja receber (\emph{receive window});

    \item campos de opção, usados na negociação do tamanho da janela de dados ou do
        tamanho máximo do segmento (MSS);

    \item ponteiro de dados urgentes, que indica o fim do trecho de dados urgentes.
\end{itemize}

Na prática, PSH, URG e o ponteiro de dados urgentes não são utilizados
\cite{book:kurose}.

Munido de todos esses dados, o \gls*{tcp} implementa um protocolo de \emph{handshake} de
quatro passos que precede o envio dos dados de fato, e então passa a enviar o conjunto
de dados de forma ordenada, com checagem de erros de transmissão e de recebimento, e
controles de congestionamento de envio e de recebimento, utilizando confirmações e
limites de tempo. É um protocolo grande se comparado ao \gls*{udp} e, por consequência,
utiliza maior quantidade de dados e de banda de conexão.