%!TEX root = ../../tcc.tex

\newpage
\subsection*{Listas ligadas}

Listas ligadas é uma estrutura de dados que organiza os objetos de forma linear, assim
como os vetores. Porém, enquanto estes possuem índices que determinam a sua posição, em
listas, cada elemento possuem um ponteiro para o elemento seguinte. Por causa disso,
seu comprimento se altera organicamente conforme novos elementos vão sendo criados e
inseridos, consumindo somente a memória necessária.

Os nós de listas ligadas são definidos usando-se estruturas.

\cfile[label="./libtransmission/list.h:28"]{./Codes/chap4/001-lista-struct.c}

Para a estrutura ser usada, o Transmission utiliza uma função que aloca memória
dinamicamente para um objeto com valor nulo em todos os seus campos.

\cfile[label="./libtransmission/list.c:19"]{./Codes/chap4/002-lista-code.c}

Existem vários tipos de listas ligadas:

\begin{description}
    \item[simplesmente ligada:] possui somente um ponteiro para o próximo elemento;
    \item[duplamente ligada:] possui 2 ponteiros, um para o elemento anterior e outro
        para o próximo elemento;
    \item[multiplamente ligada:] possui ponteiros vários elementos, porém ligando-os em
        ordens diferentes;
    \item[circularmente ligada:] quando o último elemento liga a lista de volta ao
        1º elemento; e
    \item[com cabeça:] quando possui um elemento falso somente para ajudar a manipular
        as listas.
\end{description}

Comparando-se vetores e listas ligadas, cada um tem suas vantagens e desvantagens em
relação à complexidade de seus algoritmos de manipulação.

\begin{table}
    \centering
    \begin{tabular}{| l | c | c | c |}
        \hline
        \textbf{Operação} & \textbf{Vetor} & \textbf{Lista ligada} \\
        \hline
        Busca por posição & $\Theta(1)$ & $\Theta(n)$ \\
        \hline
        Inserção/Remoção (início) & $\Theta(n)$ & $\Theta(1)$ \\
        \hline
        Inserção/Remoção (fim) & $\Theta(1)$ & \parbox[t]{.3\textwidth}{\centering $\Theta(1)$ (c/ cabeça) \\ $\Theta(n)$ (s/ cabeça)} \\
        \hline
        Inserção/Remoção (meio) & $\Theta(n)$ & $\Theta(n)$ \\
        \hline
        Redimensionamento & \parbox[t]{.25\textwidth}{\centering $\Theta(n)$ (estático) \\ ? (dinâmico)} & não necessita \\
        \hline
    \end{tabular}
    \caption{tabela com os consumos de tempo de voperações sobre vetores e listas
    ligadas. OBS: tempos de buscas são considerados lineares. Redimensionamento de vetor
    dinâmico depende da implementação da linguagem C.}
\end{table}

Por conta da agilidade que é conseguida na manipulação de listas ligadas de tamanhos
imprevisíveis, o Transmission as utiliza em várias partes do seu código, como por
exemplo a lista implementada pelo \emph{framework} de criação de interfaces GTK+.