%!TEX root = ../../tcc.tex

\section{Estruturas de dados}

Conjuntos são tão fundamentais na Ciência da Computação quando na matemática. Os
conjuntos manipulados por algoritmos são, em geral, alterados ao longo do tempo, sendo
chamados de dinâmicos. Alguns algoritmos utilizam conjuntos, realizando diversas
operações sobre eles, como inserção, remoção e testes de pertinência de elementos.
Conjuntos dinâmicos que permitem essas operações são chamados de dicionários
\cite{book:clrs}.

Na linguagem C, geralmente são implementados usando estruturas (\emph{structs}), que são
coleções de variáveis (membros), independentes de tipo, agrupadas sobre o mesmo nome.
Com isso, várias implementações de dicionário foram criadas, cada uma com suas
peculiaridades. Dentre as estruturas utilizadas no Transmission encontramos:

\begin{itemize}
    \item vetores (\emph{arrays});
    \item listas ligadas (\emph{linked lists});
    \item filas (\emph{queues}); e
    \item \glspl*{hashtable} (\emph{hash tables}).
\end{itemize}

%!TEX root = ../../tcc.tex

\subsection*{Vetores}

Vetores (ou \emph{arrays}) são a implementação de vetores matemáticos de maneira
virtual. Na prática, consistem de listas de variáveis do mesmo tipo. Na linguagem C,
podem ser declaradas de forma estática ou dinâmica.

Vetores estáticos têm tamanho fixo estabelecido na sua declaração em tempo de
compilação, tendo espaços de memória reservados na pilha de memória de acordo com o
tipo, não podendo ser alterado em tempo de execução.

\begin{ccode}
    char announce[1024]; // URL de announce
\end{ccode}

Por outro lado, vetores dinâmicos são construídos por ponteiros para o tipo (ao invés do
tipo) juntamente com alocação e liberação de memória programáticas. Com isso, a memória
é reservada em tempo de execução na memória heap. Assim, pode ter seu tamanho
redimensionado conforme a necessidade.

\begin{ccode}
    int *a = malloc( 3*sizeof(int) ); // aloca memória para um vetor de 3 inteiros
    free(a);                          // desaloca a memória alocada
\end{ccode}

Apesar dessa diferença, ambos os tipos de vetores funcionam da mesma maneira, usufruindo
da aritmética de ponteiros e acesso instantâneo ao valor armazenado.

\begin{ccode}
    int b[3];
    int *c = malloc( 3*sizeof(int) );

    b[0] = 1; b[1] = 3; b[2] = 5;
    c[0] = 2; c[1] = 4; c[2] = 6;

    printf("b[1] = \%d, *(c+2) = \%d\n", b[1], *(c+2)); // b[1] = 3, *(c+2) = 6
\end{ccode}

O Transmission não aloca seus vetores dinâmicos literalmente desta forma, pois possui
suas funções próprias onde encapsula o código mostrado.

%!TEX root = ../../tcc.tex

\newpage
\subsection*{Listas ligadas}

Listas ligadas é uma estrutura de dados que organiza os objetos de forma linear, assim
como os vetores. Porém, enquanto estes possuem índices que determinam a sua posição, em
listas, cada elemento possuem um ponteiro para o elemento seguinte. Por causa disso,
seu comprimento se altera organicamente conforme novos elementos vão sendo criados e
inseridos, consumindo somente a memória necessária.

Os nós de listas ligadas são definidos usando-se estruturas.

\cfile[label="./libtransmission/list.h:28"]{./Codes/chap4/001-lista-struct.c}

Para a estrutura ser usada, o Transmission utiliza uma função que aloca memória
dinamicamente para um objeto com valor nulo em todos os seus campos.

\cfile[label="./libtransmission/list.c:19"]{./Codes/chap4/002-lista-code.c}

Existem vários tipos de listas ligadas:

\begin{description}
    \item[simplesmente ligada:] possui somente um ponteiro para o próximo elemento;
    \item[duplamente ligada]: possui 2 ponteiros, um para o elemento anterior e outro
        para o próximo elemento;
    \item[multiplamente ligada]: possui ponteiros vários elementos, porém ligando-os em
        ordens diferentes;
    \item[circularmente ligada]: quando o último elemento liga a lista de volta ao
        1º elemento; e
    \item[com cabeça]: quando possui um elemento falso somente para ajudar a manipular as
        listas.
\end{description}

Comparando-se vetores e listas ligadas, cada um tem suas vantagens e desvantagens em
relação à complexidade de seus algoritmos de manipulação.

\begin{table}
    \centering
    \begin{tabular}{| l | c | c | c |}
        \hline
        \textbf{Operação} & \textbf{Vetor} & \textbf{Lista ligada} \\
        \hline
        Busca por posição & $\Theta(1)$ & $\Theta(n)$ \\
        \hline
        Inserção/Remoção (início) & $\Theta(n)$ & $\Theta(1)$ \\
        \hline
        Inserção/Remoção (fim) & $\Theta(1)$ & \parbox[t]{.3\textwidth}{\centering $\Theta(1)$ (c/ cabeça) \\ $\Theta(n)$ (s/ cabeça)} \\
        \hline
        Inserção/Remoção (meio) & $\Theta(n)$ & $\Theta(n)$ \\
        \hline
        Redimensionamento & \parbox[t]{.25\textwidth}{\centering $\Theta(n)$ (estático) \\ ? (dinâmico)} & não necessita \\
        \hline
    \end{tabular}
    \caption{tabela com os consumos de tempo de voperações sobre vetores e listas
    ligadas. OBS: tempos de buscas são considerados lineares. Redimensionamento de vetor
    dinâmico depende da implementação da linguagem C.}
\end{table}

Por conta da agilidade que é conseguida na manipulação de listas ligadas de tamanhos
imprevisíveis, o Transmission as utiliza em várias partes do seu código.

%!TEX root = ../../tcc.tex

\subsection*{Tabelas hash}

\Glspl{hashtable} são estruturas de dados eficientes na implementação de dicionários.
Apesar de buscas demorarem tanto quanto procurar um elemento em uma lista ligada -
$\Theta(n)$ no pior caso -, o espalhamento é bastante eficiente. Isso faz com que o
tempo médio de uma busca seja $O(1)$ \cite{book:clrs}.

Uma \gls*{hashtable} generaliza a noção do vetor de elementos comum. Nele, o
endereçamento direto nos permite avaliar o conteúdo de uma posição em $O(1)$. O que
torna esta tabela especial é a vantagem de transformar um certo conteúdo possuir uma
chave específica e exclusiva, fornecendo um meio de se encontrar essa chave. Esse meio é
uma \gls{hashfunction}.

Às vezes, \glspl*{hashfunction} fazem com que 2 conteúdos possuam a mesma chave, ou
seja, as chaves colidem. Para esses casos, existem várias técnicas de solução de
conflitos, porém colisões podem ser evitadas com boa \gls*{hashfunction}, descritas a
seguir.

A \gls*{hashtable} usada pelo Transmission aparece no \gls*{dht}, porém de uma forma
mais simples: não existe ``a \gls*{hashfunction} do \gls*{dht}'' como de costume, onde
existe uma função característica para uma modelagem de tabela. Ao invés disso, as chaves
já estão calculadas, sendo os IDs do \glspl*{torrent} e dos \glspl*{peer} do Kademlia.