%!TEX root = ../../tcc.tex

\subsection*{RC4}

O RC4 (ou ainda ``Rivest Cypher 4'', ``Ron's Code 4'' ou ``Arc Four'') é uma função
criptográfica criada por Ron Rivest em 1987. Inicialmente era segredo comercial, porém
em 1994 seu código foi publicado na lista de discussão de criptografia CypherPunks
\cite{site:rc4-code}, se espalhando pela Internet rapidamente. Seu uso se tornou comum,
sendo usado por muitos softwares, chegando a protocolos como as encriptações de placas
de rede sem fio WEP e WPA ou ainda o protocolo de segurança TLS para conexões de
Internet.

O RC4 é um algoritmo de chave simétrica que se divide em duas partes: na primeira
parte, ele executa o algoritmo de escalonamento de chaves (\emph{key scheduling}),
que utiliza uma chave de tamanho variável entre 1 e 256 bytes para inicializar uma
tabela de estados. Cada elemento dessa tabela é permutado pelo menos uma vez e será
usado na geração de bytes pseudoaleatórios na segunda parte.

Na segunda parte, executa o algoritmo de geração pseudoaleatório, onde modifica o estado
(também permutando os elementos pelo menos uma vez) e resulta em 1 byte da chave de
fluxo, que então é mesclada usando \gls{xor} bit a bit com o próximo byte da mensagem
para produzir ou próximo byte da mensagem cifrada (na encriptação) ou da decifrada (na
decriptação), já que o \gls*{xor} é uma função involuntária (ou seja, é uma função que
é a própria inversa).

\cfile[label="./libtransmission/crypto.c:258"]{./Codes/chap4/005-rc4enc.c}

\cfile[label="./libtransmission/crypto.c:237"]{./Codes/chap4/006-rc4dec.c}