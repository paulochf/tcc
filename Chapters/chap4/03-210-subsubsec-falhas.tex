%!TEX root = ../../tcc.tex

\subsubsection*{Falhas de segurança}

Apesar de ainda ser bastante utilizado, o RC4 não pode ser considerado
criptograficamente seguro. Na primeira década dos anos 2000 foram publicados diversos
trabalhos sobre falhas do método e até ataques diferentes que conseguiram obter os
dados originais
\cite{blog:matthew-rc4-1,blog:matthew-rc4-2,blog:matthew-rc4-3,artigo:patterson}.

Porém, como é dito na especificação da extensão criptográfica no protocolo BitTorrent
\cite{site:bittorrent-cripto}, ``o objetivo não é de criar um protocolo
criptograficamente que consegue sobreviver a observações dos pacotes de dados e recursos
computacionais consideráveis na escala de tempo de redes. O objetivo é de aumentar a
dificuldade o suficiente a fim de deter ataques baseados na obtenção de endereços IP e
porta na comunicações entre \glspl*{peer} e \glspl*{tracker}''. Especificamente, o
objetivo é prevenir que \glspl{isp} ou outros administradores de rede consigam bloquear
ou impedir conexões BitTorrent entre esse \gls*{peer} e qualquer outro \gls*{peer} cujo
endereço IP e porta apareça em alguma lista de \glspl*{peer} enviada por um
\gls*{tracker}. Por isso, essa especificação é chamada de ofuscação de \glspl*{peer},
onde a idéia é usar o \bverb|info_hash| de um \gls*{torrent} como chave compartilhada
entre o \gls*{peer} e o \gls*{tracker}.

Por exemplo, em um dos ataques ao RC4 usado no protocolo de rede sem fio WEP, é
necessário que o atacante obtenha uma quantidade de bytes muito grande para que consiga
decifrar o texto. Porém, enquanto em redes sem fio esses dados trafegados são facilmente
alcançam aquela quantidade necessária, numa comunicação entre \glspl*{peer} e
\glspl*{tracker} os dados transmitidos são muito menores, dificultando ainda mais o
processo de quebra.