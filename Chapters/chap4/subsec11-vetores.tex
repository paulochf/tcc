%!TEX root = ../../tcc.tex

\subsection*{Vetores}

Vetores (ou \emph{arrays}) são a implementação de vetores matemáticos de maneira
virtual. Na prática, consistem de listas de variáveis do mesmo tipo. Na linguagem C,
podem ser declaradas de forma estática ou dinâmica.

Vetores estáticos têm tamanho fixo estabelecido na sua declaração em tempo de
compilação, tendo espaços de memória reservados na pilha de memória de acordo com o
tipo, não podendo ser alterado em tempo de execução.

\begin{ccode}
    char announce[1024]; // URL de announce
\end{ccode}

Por outro lado, vetores dinâmicos são construídos por ponteiros para o tipo (ao invés do
tipo) juntamente com alocação e liberação de memória programáticas. Com isso, a memória
é reservada em tempo de execução na memória heap. Assim, pode ter seu tamanho
redimensionado conforme a necessidade.

\begin{ccode}
    int *a = malloc( 3*sizeof(int) ); // aloca memória para um vetor de 3 inteiros
    free(a);                          // desaloca a memória alocada
\end{ccode}

Apesar dessa diferença, ambos os tipos de vetores funcionam da mesma maneira, usufruindo
da aritmética de ponteiros e acesso instantâneo ao valor armazenado.

\begin{ccode}
    int b[3];
    int *c = malloc( 3*sizeof(int) );

    b[0] = 1; b[1] = 3; b[2] = 5;
    c[0] = 2; c[1] = 4; c[2] = 6;

    printf("b[1] = \%d, *(c+2) = \%d\n", b[1], *(c+2)); // b[1] = 3, *(c+2) = 6
\end{ccode}

O Transmission não aloca seus vetores dinâmicos literalmente desta forma, pois possui
suas funções próprias onde encapsula o código mostrado.