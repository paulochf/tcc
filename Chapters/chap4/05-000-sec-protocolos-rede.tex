%!TEX root = ../../tcc.tex

\section{Protocolos de redes}

A Internet é o meio mais importante de comunicação que existe atualmente. Usamos de
forma tão corriqueira que nem nos damos conta de quantas camadas e protocolos existem em
uso em um único instante. Para chegar até o que é hoje, precisou passar por muitas
evoluções desde que a chamado precursora da Internet, a ARPANET, foi iniciada em outubro
de 1969. Atualmente, seus protocolos são mantidos pela IETF (Internet Engineering Task
Force).

Tecnicamente falando, a Internet é organizada em uma pilha de camadas de protocolo, que
oferecem e consomem serviços às camadas adjacentes, permitindo que dados sejam roteados
entre um computador emissor e outro receptor. Esses protocolos podem estar implementados
tanto por \emph{software}, por \emph{hardware}, ou por uma combinação de ambos. A
vantagem da modelagem da pilha é que provê um meio organizado de se discutir as partes
do sistema e até atualizá-las separadamente. Em contrapartida, uma camada pode
necessitar de um valor presente em outra, ou ainda de possuir alguma funcionalidade já
implementada em outra.

As cinco camadas que representam a pilha de camadas de protocolo da Internet são:

\begin{description}
    \item[aplicação:] é onde existem as aplicações de rede e seus protocolos; ocorrem
        as traduções de endereços de Internet para endereços de rede (DNS); e
        transmissões de documentos de Internet (HTTP), de mensagens de e-mail (SMTP) e
        de arquivos (FTP). Os pacotes de dados dessa camada são chamados de
        \textbf{mensagens};

    \item[transporte:] é a camada onde atuam os protocolos \gls{tcp} e \gls{udp}, que
        transforma as mensagens da camada superior em \textbf{segmentos};

    \item[rede:] camada responsável por transportar pacotes conhecidos como
        \textbf{datagramas} para outro computador, recebe da camada de transporte um
        segmento e um endereço de destino. Assim, funciona como um serviço de entrega,
        que sabe quais rotas o datagrama deve tomar para chegar ao destinho. Também é
        onde atua o protocolo IP, que todo componente de Internet deve possuir, e que
        define alguns dados no datagrama da mesma forma que equipamentos roteadores
        fazem.

    \item[enlace:]

    \item[física:]
\end{description}

\begin{comment}
Aqui vou explicar o que são os protocolos de rede TCP e UDP, apontar suas diferenças e
mostrar os motivos pelos quais o UDP é preferido ao TCP no uso de endereços de
\gls*{announce} de \glspl*{tracker}.
\end{comment}

%!TEX root = ../../tcc.tex

\subsection*{O protocolo TCP}



%!TEX root = ../../tcc.tex

\subsection*{O protocolo UDP}

O protocolo \gls*{udp} é um tipo de conexão que especifica conexões quase que diretas
(se comparado com o \gls*{tcp} \todo{RLY?}) entre as camadas de aplicação e de
rede.

Em sua camada, ocorrem a multiplexação e a demultiplexação dos dados. Quando as
mensagens estão sendo enviadas por um processo da aplicação, elas são recebidas através
de \glspl{socket}, anexadas aos endereços IP e número de porta de origem e de destino, e
aos comprimentos e \gls{checksum} do cabeçalho e do corpo de dados \gls*{udp}, para
enfim serem repassadas como segmentos para a camada de rede. Por sua vez, a camada de
rede encapsula os segmentos em datagramas fazem o melhor possível para entregá-los ao
destinatário.

Já quando estão sendo recebidas, o protocolo \gls*{udp} utiliza a porta de destino
contida no cabeçalho para entregar os dados do segmento para o processo correto de
aplicação. Como não existe nenhum \emph{handshake} entre as camadas de transporte das
partes durante esse procedimento, o protocolo \gls*{udp} é dito ``sem conexão'' ou ``não
orientado a conexão''. Além disso, é considerado não confiável, já que não se possui
garantia de entregas dos pacotes, muito menos na ordem correta.

Apesar de ser considerado não confiável, existem vantagens em se escolher o método
\gls*{udp} em vez do \gls*{tcp}:

\begin{itemize}
    \item controle avançado de quais dados são enviados e quando: como o \gls*{tcp}
        possui controle de congestionamento e confirmação de recebimento de segmentos,
        pode ser que a aplicação seja comprometida pelo atraso de algum datagrama. No
        caso de aplicações em tempo real, é possível suportar alguma perda de dados, ou
        ainda implementar o seu próprio método de verificação de integridade;

    \item dispensa conexões: enquanto o \gls*{tcp} utiliza \emph{handshakes} antes da
        transferência dos dados, o \gls*{udp} já os envia sem a necessidade de
        contatos anteriores. Essa é o principal motivo pelo qual servidores de DNS
        (\emph{Domain Name Service}) utilizam \gls*{udp};

    \item não mantém estado da conexão: essas informações de estado são necessárias para
        se conseguir uma conexão de dados confiável, que é como o \gls*{tcp} controla
        buffers de entrada e saída, congestionamento de dados e parâmetros de
        confirmação. Por conta disso, um servidor dedicado a uma aplicação consegue
        aguentar muito mais conexões do que se fosse usado \gls*{tcp};

    \item pouco \gls{overhead} de cabeçalho de pacotes: enquanto o \gls*{tcp} possui
        \gls*{overhead} de cabeçalho de 20 bytes por segmento, o \gls*{udp} possui 8
        bytes.
\end{itemize}

Originalmente, os \glspl*{tracker} utilizavam \gls*{tcp}, porém com o tempo perceberam
que com \gls*{udp} se tornariam mais eficientes, reduzindo o consumo da largura da
conexão de rede pela metade \cite{site:tracker-udp}. Por esses motivos, os servidores
dos \glspl{tracker} utilizam prioritariamente o protocolo \gls*{udp}.