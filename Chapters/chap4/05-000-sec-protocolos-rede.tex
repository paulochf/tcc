%!TEX root = ../../tcc.tex

\section{Protocolos TCP e UDP}

A Internet é o meio mais importante de comunicação que existe atualmente. Usamos de
forma tão corriqueira que nem nos damos conta de quantas camadas e protocolos existem em
uso em um único instante. Para chegar até o que é hoje, passou por uma grande evolução,
que começou a partir da precursora da Internet, a ARPAnet, que em outubro de 1969
\cite{book:kurose} conectou quatro universidades. Atualmente, seus protocolos são mantidos
pela IETF (Internet Engineering Task Force).

Tecnicamente falando, a Internet é organizada em uma pilha de camadas de protocolo, que
oferecem e consomem serviços às camadas adjacentes, permitindo que dados sejam roteados
entre um computador emissor e outro receptor. Esses protocolos podem estar implementados
tanto por \emph{software}, por \emph{hardware}, ou por uma combinação de ambos. A
vantagem da modelagem da pilha é ela que provê um meio organizado de se discutir as
partes do sistema, e até atualizá-las separadamente. Em contrapartida, uma camada pode
necessitar de um valor presente em outra, ou ainda possuir alguma funcionalidade já
implementada em outra.

As cinco camadas que representam a pilha de camadas de protocolo da Internet podem ser
explicadas fazendo-se uma analogia com um serviço postal \cite{site:internet-layer}:

\begin{description}
    \item[aplicação:] é onde existem as aplicações de rede e seus protocolos; ocorrem
        as traduções de endereços de Internet para endereços de rede (DNS); e
        transmissões de documentos de Internet (HTTP), de mensagens de e-mail (SMTP) e
        de arquivos (FTP). Os pacotes de dados dessa camada são chamados de
        \textbf{mensagens}.

        Nessa camada, seria onde uma pessoa escreve uma carta a uma amiga e a deposita
        na caixa de correio. A pessoa amiga recebe a carta na sua caixa de correio e a
        lê. Ambas as pessoas não sabem dos processos e rotas que a carta tomou. Bastou
        uma enviar a carta e a outra recebê-la;

    \item[transporte:] é onde atuam os protocolos \gls{tcp} e \gls{udp}, que recebem as
        mensagens através de \glspl{socket} e as transformam em \textbf{segmentos}.
        Além disso, também criam conexões entre cliente e servidor (multiplexação e
        demultiplexação dos dados) e as monitora, prevenindo erros.

        No caso do serviço postal, o remetente será avisado se enviar a carta para um
        endereço incorreto (por exemplo, se tiver errado o Estado), ou ainda se uma
        carta registrada não puder ser entregue. Nesses casos, a carta é devolvida,
        ficando a cargo da pessoa decidir o que fazer após esse problema;

    \item[rede:] é responsável por transportar pacotes, conhecidos como
        \textbf{datagramas}, para outro computador. Ele recebe da camada de transporte
        um segmento e um endereço de destino. Assim, funciona como um serviço de
        entrega, que sabe quais rotas o datagrama deve tomar para chegar ao destino.
        Também é onde atua o protocolo IP nas versões IPv4 e IPv6, que todo componente
        de Internet deve possuir, e que define alguns dados no datagrama, da mesma
        forma que equipamentos roteadores fazem.

        O serviço de entrega de correio usaria aviões para transportar suas cartas
        entre as cidades, porém seu piloto não saberia quem as enviou, para quem
        levando ou quais seus conteúdos;

    \item[enlace:] responsável pela transmissão dos dados que recebe da camada de rede.
        A cada nó da rota, a camada de rede repassa o datagrama para a camada de
        enlace, que então o transforma em \textbf{quadros} e o entrega para o próximo
        nó da rede. Esse nó recebe os quadros na sua camada de enlace e os repassa a
        sua camada de rede. Dos protocolos que atuam nesta camada, se incluem o
        Ethernet, os vários de conexões wifi, os de ponto a ponto (PPP), etc.

        Para o serviço postal, seu equivalente seria a frota de caminhões e
        entregadores, que distribui os pacotes dentro de uma cidade; e

    \item[física:] responsável pela transmissão de cada bit dos quadros, de um nó para
        outro. Os protocolos dessa camada dependem do tipo de meio pelo qual os nós
        estão ligados. Por exemplo, o protocolo Ethernet possui uma especificação para
        cabeamentos coaxiais, outra para cabos de par trançado, outra para fibra óptica,
        etc.

        Na analogia das cartas, são as canetas e papéis usados em sua escrita, ou a luz
        acesa para sua leitura.
\end{description}

Uma característica bastante forte do BitTorrent é o uso perceptível dos protocolos
\gls*{tcp} e \gls*{udp}, onde ele implementa sempre a mesma funcionalidade para cada um
desses protocolos. Ambos têm suas características , que podem ser explorados
dependendo da aplicação, para uma melhor utilização dos recursos.

%!TEX root = ../../tcc.tex

\subsection*{UDP}

O protocolo \gls*{udp} é um protocolo de conexão que pertence à camada de transporte e
que especifica conexões quase que diretas (se comparado com o \gls*{tcp}) entre as
camadas de aplicação e de rede.

Quando mensagens são enviadas por uma aplicação, elas são recebidas através dos
\glspl*{socket}. Depois, são anexadas aos endereços IP e número de porta de origem e de
destino, e aos comprimentos e \gls{checksum} do cabeçalho e do corpo de dados
\gls*{udp}, para enfim serem repassadas como segmentos para a camada de rede. Por sua
vez, a camada de rede encapsula os segmentos em datagramas e faz o melhor possível para
entregá-los ao destinatário.

Já quando datagramas são recebidos, o protocolo \gls*{udp} utiliza a porta de destino
contida no cabeçalho para entregar os dados do segmento para o processo correto de
aplicação. Como não existe nenhum protocolo de \emph{handshake} entre as camadas de
transporte de ambos remetente e destinatário, o protocolo \gls*{udp} é dito ``sem
conexão'' ou ``não orientado a conexão''. Além disso, é considerado não confiável, já
que não há garantia de entregas dos pacotes, muito menos a entrega na ordem correta.

Apesar de ser considerado não confiável, existem vantagens em se escolher o \gls*{udp}
ao invés do \gls*{tcp}:

\begin{itemize}
    \item controle avançado de quais dados são enviados e quando: como o \gls*{tcp}
        possui controle de congestionamento e confirmação de recebimento de segmentos,
        pode ser que a aplicação seja comprometida pelo atraso de algum datagrama. No
        caso de aplicações em tempo real, é possível suportar alguma perda de dados, ou
        ainda implementar o seu próprio método de verificação de integridade;

    \item dispensa conexões: enquanto o \gls*{tcp} utiliza protocolos de
        \emph{handshake} antes da transferência dos dados, o \gls*{udp} já os envia sem
        a necessidade de contatos anteriores. Essa agilidade é o principal motivo pelo
        qual servidores de DNS (\emph{Domain Name Service}) utilizam \gls*{udp};

    \item não mantém estado da conexão: essas informações de estado são necessárias para
        se conseguir uma conexão de dados confiável, como faz o \gls*{tcp}, que usa
        buffers de entrada e saída, gerencia congestionamento de dados e possui
        parâmetros de confirmação. Por conta disso, um servidor dedicado a uma
        aplicação consegue suportar muito mais conexões do que se fosse usado
        \gls*{tcp};

    \item pouco \gls{overhead} de cabeçalho de pacotes: enquanto o \gls*{tcp} possui
        \gls*{overhead} de cabeçalho de 20 bytes por segmento, o \gls*{udp} possui 8
        bytes.
\end{itemize}

Originalmente, os \glspl*{tracker} utilizavam o protocolo \gls*{tcp}, porém, com o
tempo, percebeu-se que com o protocolo \gls*{udp} eles se tornariam mais eficientes,
reduzindo o consumo da largura de rede pela metade \cite{site:tracker-udp}. Por esses
motivos, os servidores dos \glspl*{tracker} utilizam prioritariamente o protocolo
\gls*{udp}.

%!TEX root = ../../tcc.tex

\subsection*{TCP}

O protocolo \gls*{tcp} é um protocolo da camada de transporte que determina um meio de
transmissão de dados confiável e orientado a conexão.

Quando as mensagens chegam pelos \glspl*{socket}, o protocolo \gls*{tcp} também as anexa
aos endereços IP e número de porta de origem e de destino, e também aos comprimentos e
\gls*{checksum} do cabeçalho e do corpo de dados \gls*{udp}, para enfim serem repassadas
como segmentos para a camada de rede. Porém, acrescenta também outras informações:

\begin{itemize}
    \item números de sequência e de confirmação de recebimento (\emph{acknowledgement}),
        para uso na implementação do serviço de entrega de dados confiável;

    \item quantidade de bytes que o destinatário deseja receber (\emph{receive window});

    \item campos de opção, usados na negociação dos tamanhos da janela de dados ou do
        máximo do segmento (MSS);

    \item campos de \emph{flags} sinalizadores, onde 6 bits indicam estados da conexão
        ou do segmento:
        \begin{description}
            \item[URG:] indicador de urgência do segmento enviado;

            \item[ACK:] indica que o valor contido no campo de confirmação de
                recebimento é válido. Deve ser enviado depois do primeiro pacote com
                SYN;

            \item[PSH:] para indicar que o destinatário repasse os dados para a camada
                de aplicação imediatamente ao receber o pacote;

            \item[RST:] campo de redefinição de conexão, para indicar erro que não
                possui um \gls*{socket} na porta indicada no cabeçalho;

            \item[SYN:] número de sincronização de sequência, deve ser enviado sempre
                no primeiro pacote de uma remessa de dados de um nó;

            \item[FIN:] indica fim dos dados enviados.
        \end{description}

    \item ponteiro de dados urgentes, que indicam o fim do trecho de dados urgentes.
\end{itemize}

Na prática, PSH, URG e o ponteiro de dados urgentes não são utilizados
\cite{book:kurose}.

Munido de todos esses dados, o \gls*{tcp} implementa um protocolo de \emph{handshake} de
4 passos que precede o envio dos dados de fato, e então passa a enviar o conjunto de
dados de forma ordenada, com checagem de erros de transmissão e de recebimento, e
controle de congestionamento de envio e de recebimento utilizando confirmações e
limites de tempo. É um protocolo grande se comparado ao \gls*{udp} e, por consequência,
que utiliza maior quantidade de dados e de banda de conexão.

%!TEX root = ../../tcc.tex

\subsection*{BitTorrent prefere UDP}

A substituição do \gls*{tcp} pelo \gls*{udp} para se trocar partes no BitTorrent,
desenvolvida pioneiramente pelo programa cliente $\mu$Torrent, causou discussões
acaloradas pela Internet. Richard Bennet, arquiteto de redes que escreveu o primeiro
padrão Ethernet para cabos de par trançado e ajudou no desenvolvimento de protocolos
wifi, publicou um artigo \cite{site:register-bennett} dizendo que a troca para o
\gls*{udp} poderia causar um colapso da Internet como um todo, pois os \glspl*{isp} não
teriam como controlar o tráfego de dados BitTorrent, porque necessitavam que estes
fossem pacotes \gls*{tcp}. Esse colapso afetaria usuários de outros serviços
\gls*{udp}, que na sua maioria eram aplicações de tempo real, como jogos online ou
sistemas de comunicação VoIP.

Enquanto isso, desenvolvedores de programas cliente responderam dizendo que não existiam
motivos para preocupações. Simon Morris, que na época era chefe da gestão de produto da
empresa BitTorrent, disse \cite{site:dslreports-bennett} que a troca era para se ter um
melhor controle de congestionamento das transmissões de dados, pois com o \gls*{tcp} o
conhecimento sobre o problema só surge depois que ele ocorre. Por outro lado, usando
\gls*{udp} é possível prever congestionamentos medindo as taxas de transmissão entre
\glspl*{peer} e, com isso, diminuir a velocidade de envio.

Outro depoimento é de um gerente de comunidade do $\mu$Torrent chamado Firon, que
explicou a um site de notícias relacionadas a BitTorrent
\cite{site:torrentfreak-bennett} o mesmo argumento dado por Simon Morris, e ainda
acrescentou que o BitTorrent já possui um protocolo de \emph{handshake} e que usar o
protocolo \gls*{tcp} redundaria esse processo e, por isso, a mudança para \gls*{udp}
reduziria o tráfego na Internet.

Com essa grande discussão, podemos ver que ainda causa reflexos nos dias atuais, pois
\glspl*{isp} ainda tentam controlar o fluxo de dados BitTorrent do seu fornecimento de
Internet. Porém, conforme os dados de uma pesquisa recente sobre a quantidade de
tráfego de Internet gerada nos horários de pico \cite{report:internet-usage-2013},
60,47\% do tráfego de download e 54,97\% do tráfego total norte-americano são gerados
pelos serviços de \emph{streaming} de vídeo Netflix e Youtube e por navegação de sites
via HTTP, enquanto o BitTorrent consome apenas 5,57\% e 9,23\%, respectivamente.
Portanto, não se pode afirmar contundentemente que o BitTorrent seja o responsável pelo
congestionamento de dados nos Estados Unidos.

Outro fato de que o protocolo \gls*{udp} pode ter importância em aplicações onde o
\gls*{tcp} seria preferido é que o Google está trabalhando no QUIC
(\emph{Quick \gls*{udp} Internet Connections}), que fará parte da especificação do
protocolo HTTP 2.0 \cite{site:chromium-quic}. O QUIC, apesar de aumentar o consumo de
banda, reduzirá o tempo de resposta de confirmação de recebimento de pacotes, entre
outras melhorias.
