%!TEX root = ../../tcc.tex

\section{Protocolos de redes}

A Internet é o meio mais importante de comunicação que existe atualmente. Usamos de
forma tão corriqueira que nem nos damos conta de quantas camadas e protocolos existem em
uso em um único instante. Para chegar até o que é hoje, precisou passar por muitas
evoluções desde que a chamado precursora da Internet, a ARPANET, foi iniciada em outubro
de 1969 \cite{book:kurose}. Atualmente, seus protocolos são mantidos pela IETF
(Internet Engineering Task Force).

Tecnicamente falando, a Internet é organizada em uma pilha de camadas de protocolo, que
oferecem e consomem serviços às camadas adjacentes, permitindo que dados sejam roteados
entre um computador emissor e outro receptor. Esses protocolos podem estar implementados
tanto por \emph{software}, por \emph{hardware}, ou por uma combinação de ambos. A
vantagem da modelagem da pilha é que provê um meio organizado de se discutir as partes
do sistema e até atualizá-las separadamente. Em contrapartida, uma camada pode
necessitar de um valor presente em outra, ou ainda de possuir alguma funcionalidade já
implementada em outra.

As cinco camadas que representam a pilha de camadas de protocolo da Internet podem ser
explicadas fazendo-se uma analogia com um serviço postal \cite{site:internet-layer}:

\begin{description}
    \item[aplicação:] é onde existem as aplicações de rede e seus protocolos; ocorrem
        as traduções de endereços de Internet para endereços de rede (DNS); e
        transmissões de documentos de Internet (HTTP), de mensagens de e-mail (SMTP) e
        de arquivos (FTP). Os pacotes de dados dessa camada são chamados de
        \textbf{mensagens}.

        Nessa camada, seria onde uma pessoa escreve uma carta e a deposita na caixa de
        correio. A pessoa amiga recebe pega a carta na sua caixa de correio e a lê.
        Ambas as pessoas não sabem dos processos e rotas que a carta tomou. Bastou uma
        enviar a carta e a outra receber.

    \item[transporte:] é onde atuam os protocolos \gls{tcp} e \gls{udp}, que
        transformam as mensagens da camada superior em \textbf{segmentos} e criam
        conexões entre cliente e servidor e as monitora prevenindo erros.

        No caso do serviço postal, O remetente será avisado se enviar a carta para um
        endereço incorreto (por exemplo, tiver errado o estado), ou ainda se uma carta
        registrada não pôde ser entregue. Nesses casos, a carta é devolvida, ficando a
        cargo da pessoa o que fazer com esse problema.

    \item[rede:] é responsável por transportar pacotes conhecidos como
        \textbf{datagramas} para outro computador, recebe da camada de transporte um
        segmento e um endereço de destino. Assim, funciona como um serviço de entrega,
        que sabe quais rotas o datagrama deve tomar para chegar ao destino. Também é
        onde atua o protocolo IP, que todo componente de Internet deve possuir, e que
        define alguns dados no datagrama da mesma forma que equipamentos roteadores
        fazem.

        O serviço de entrega de correio usaria aviões para mover suas cartas entre as
        cidades, porém seu piloto não saberia quem as enviou, para quem está levando ou
        quais seus conteúdos.

    \item[enlace:] responsável pela transmissão dos dados que recebe da camada de rede.
        A cada nó da rota, a camada de rede repassa o datagrama para a camada de
        enlace, que então o transforma em \textbf{quadros} e o entrega para o próximo
        nó da rede; esse nó recebe os quadros na sua camada de enlace e os repassa a
        sua camada de rede. Os protocolos que atuam nesta camada se incluem o Ethernet,
        os vários de conexões wifi, os de ponto a ponto (PPP), etc.

        Para o serviço postal, seu equivalente seriam os caminhões e entregadores, que
        distribuem os pacotes dentro de uma cidade.

    \item[física:] responsável pela transmissão de cada bit dos quadros de um nó para
        outro. Os protocolos dessa camada dependem do tipo de meio pelo qual os nós
        estão ligados. Por exemplo, o protocolo Ethernet possui uma especificação para
        cabeamentos coaxiais, outra para cabos de par trançado, outra para fibra óptica,
        etc.

        Na analogia das cartas, são as canetas e papéis usados na escrita, ou a luz
        acesa para a leitura.
\end{description}

Uma característica bastante forte do BitTorrent é o uso perceptível dos protocolos
\gls*{tcp} e \gls*{udp}. Ambos têm seus pontos fortes e fracos, que podem ser explorados
dependendo da aplicação para uma melhor utilização dos recursos.

\begin{comment}
Aqui vou explicar o que são os protocolos de rede TCP e UDP, apontar suas diferenças e
mostrar os motivos pelos quais o UDP é preferido ao TCP no uso de endereços de
\gls*{announce} de \glspl*{tracker}.
\end{comment}

%!TEX root = ../../tcc.tex

\subsection*{O protocolo TCP}



%!TEX root = ../../tcc.tex

\subsection*{O protocolo UDP}

