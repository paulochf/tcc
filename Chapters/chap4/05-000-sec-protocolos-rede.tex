%!TEX root = ../../tcc.tex

\section{Protocolos de redes}

A Internet é o meio mais importante de comunicação que existe atualmente. Usamos de
forma tão corriqueira que nem nos damos conta de quantas camadas e protocolos existem em
uso em um único instante. Para chegar até o que é hoje, precisou passar por muitas
evoluções desde que a chamado precursora da Internet, a ARPANET, foi iniciada em outubro
de 1969. Atualmente, seus protocolos são mantidos pela IETF (Internet Engineering Task
Force).

Tecnicamente falando, a Internet é organizada em uma pilha de camadas de protocolo, que
oferecem e consomem serviços às camadas adjacentes, permitindo que dados sejam roteados
entre um computador emissor e outro receptor. Esses protocolos podem estar implementados
tanto por \emph{software}, por \emph{hardware}, ou por uma combinação de ambos. A
vantagem da modelagem da pilha é que provê um meio organizado de se discutir as partes
do sistema e até atualizá-las separadamente. Em contrapartida, uma camada pode
necessitar de um valor presente em outra, ou ainda de possuir alguma funcionalidade já
implementada em outra.

As cinco camadas que representam a pilha de camadas de protocolo da Internet são:

\begin{description}
    \item[aplicação:] é onde existem as aplicações de rede e seus protocolos; ocorrem
        as traduções de endereços de Internet para endereços de rede (DNS); e
        transmissões de documentos de Internet (HTTP), de mensagens de e-mail (SMTP) e
        de arquivos (FTP). Os pacotes de dados dessa camada são chamados de
        \textbf{mensagens};

    \item[transporte:] é a camada onde atuam os protocolos \gls{tcp} e \gls{udp}, que
        transforma as mensagens da camada superior em \textbf{segmentos};

    \item[rede:] camada responsável por transportar pacotes conhecidos como
        \textbf{datagramas} para outro computador, recebe da camada de transporte um
        segmento e um endereço de destino. Assim, funciona como um serviço de entrega,
        que sabe quais rotas o datagrama deve tomar para chegar ao destinho. Também é
        onde atua o protocolo IP, que todo componente de Internet deve possuir, e que
        define alguns dados no datagrama da mesma forma que equipamentos roteadores
        fazem.

    \item[enlace:]

    \item[física:]
\end{description}

\begin{comment}
Aqui vou explicar o que são os protocolos de rede TCP e UDP, apontar suas diferenças e
mostrar os motivos pelos quais o UDP é preferido ao TCP no uso de endereços de
\gls*{announce} de \glspl*{tracker}.
\end{comment}

%!TEX root = ../../tcc.tex

\subsection*{O protocolo TCP}



%!TEX root = ../../tcc.tex

\subsection*{O protocolo UDP}

