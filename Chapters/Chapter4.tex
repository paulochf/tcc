%!TEX root = ../tcc.tex

\chapter{Conceitos de Computação no BitTorrent}

O BitTorrent é um protocolo cuja existência depende de vários componentes das mais
variadas áreas de estudo da Computação. Neste capítulo, mostraremos alguns desses
componentes e de que forma o Transmission os implementa, a fim de conseguir desempenhar
bem sua função de cliente BitTorrent.

%!TEX root = ../../tcc.tex

\section{Estruturas de dados, listas ligadas e árvores}

Aqui vou falar de tipos de estruturas de dados utilizadas no programa, como
\emph{structs} e a sua utilização na implementação de listas ligadas e árvores e em como
estes são usados na implementação de filas.

%!TEX root = ../../tcc.tex

\section{Funções de hash}
\label{sec:sha1}

Aqui vou explicar como funciona o algoritmo da \gls{hashfunction} SHA-1 e mostrar como
e para que é usado na identificação de torrents e na verificação de integridade de
partes.
 % SHA1

%!TEX root = ../../tcc.tex

\section{Criptografia}

Aqui vou explicar como este algoritmo de chave simétrica funciona e como é utilizado
pelo Transmission para criptografar pacotes de dados. % RC4

%!TEX root = ../../tcc.tex

\section{Bitfields}
\label{sec:bitfield}

Apesar de ser um simples array de bits usado no gerenciamento de partes que o programa
já baixou ou não, foi percebido que o seu uso de forma não-convencional, chamado de
\emph{lazy bitfield}, pode ajudar a evitar o controle de banda (chamado de modelagem de
tráfego, ou \emph{traffic shaping} em inglês), feito por \glspl{isp}.

%!TEX root = ../../tcc.tex

\section{Protocolos de redes}

Aqui vou explicar o que são os protocolos de rede TCP e UDP, apontar suas diferenças e
mostrar os motivos pelos quais o UDP é preferido ao TCP no uso de endereços de
\gls*{announce} de \glspl*{tracker}. % HTTP e UDP

%!TEX root = ../../tcc.tex

\section{Multicast}

Apesar de não ser utilizado pelo protocolo BitTorrent, o multicast, que é uma forma de
entregar dados a um grupo de computadores simultaneamente numa só transmissão, é usado
pelo Transmission para tentar descobrir \glspl*{peer} que estão na mesma rede local,
otimizando as conexões.

%!TEX root = ../../tcc.tex

\section{Roteamento de pacotes}

Em alguns roteadores que são ponte entre a Internet e a rede que ele gerencia, existe
uma função de se configurar portas de comunicação de rede automaticamente usando-se o
\emph{Network Address Translation Port Mapping Protocol}. Assim, não é necessário
realizar uma configuração específica somente pare esse fim. %NAT PMP

%!TEX root = ../../tcc.tex

\section{Retomada de downloads}

No Transmission e em algums outros softwares que realizam downloads, existe a função de
se pausar a transferência do arquivo para que seja retomado em outro momento. Nesta
seção, mostrarei qual a idéiapor trás desse mecanismo e a forma como foi implementado no
Transmission.

%!TEX root = ../../tcc.tex

\section{Conexão com a Internet}

Aqui mostrarei a parte técnica da programação em linguagem C para utilização de
transmissão de dados por rede.t

%!TEX root = ../../tcc.tex

\section{IPv6}

O IPv6 é a versão mais recente do protocolo de Internet (IP), que foi criado para
substituir o IPv4, que atualmente é mais o usado porém sofre de exaustão de endereços.
Nesta seção, falarei sobre o novo protocolo e quais as implicações na programação de
softwares com comunicação de redes.

%!TEX root = ../../tcc.tex

\section{Threads}

Dentro de um contexto de programas que utilizam a Internet e funcionalidades que chegam
próximas ao tempo real, processamentos pesados devem ser tratados com cautela a fim de
se manter a instantaneidade do processo. Aqui explicarei como o Transmission usa o
conceito de \emph{threads} para paralelizar esses processamentos e, com isso, conseguir
utilizar as informações de rápida mudança antes que seja necessário obtê-las novamente.

%!TEX root = ../../tcc.tex

\section{Engenharia de Software}

O Transmission é um programa extenso e complexo, desenvolvido por vários programadores
que estão espalhados pelo globo. Nesta seção, abordarei alguns pontos utilizados pelos
desenvolvedores na manutenção do código aberto de qualidade e em funcionamento.

\afterpage{\clearpage}