%!TEX root = ../tcc.tex

\chapter{Conceitos de Computação no BitTorrent}

O BitTorrent é um protocolo cuja existência depende de vários componentes das mais
variadas áreas de estudo da Computação. Neste capítulo, mostraremos alguns desses
componentes e de que forma o Transmission os implementa, a fim de conseguir desempenhar
bem sua função de cliente BitTorrent.

%!TEX root = ../../tcc.tex

\section{Estruturas de dados}

Conjuntos são tão fundamentais na Ciência da Computação quando na matemática. Porém, os
conjuntos manipulados por algoritmos são adaptáveis, dinâmicos. Alguns algoritmos
utilizam conjuntos, realizando diversas operações sobre eles, como inserção, remoção e
testes de existência de elementos. Esses conjuntos dinâmicos que permitem essas
operações são chamados de dicionários.

Na linguagem C, geralmente são implementados usando estruturas: coleção de variáveis
(membros), independentes de tipo, agrupadas sobre o mesmo nome. Com isso, várias
implementações de dicionário foram criadas, cada um com suas peculiaridades, onde
algumas se tornarão. No Transmission, algumas dessas estruturas são utilizadas:

\begin{itemize}
    \item vetores (\emph{arrays})
    \item listas ligadas (\emph{linked lists})
    \item filas (\emph{queues})
    \item \glspl*{hashtable} (\emph{hash tables})
\end{itemize}

%!TEX root = ../../tcc.tex

\subsection*{Vetores}

Vetores (ou \emph{arrays}) são a implementação de vetores matemáticos de maneira
virtual. Na prática, consistem de listas de variáveis do mesmo tipo. Na linguagem C,
podem ser declaradas de forma estática ou dinâmica.

Vetores estáticos têm tamanho fixo estabelecido na sua declaração em tempo de
compilação, tendo espaços de memória reservados na pilha de memória de acordo com o
tipo, não podendo ser alterado em tempo de execução.

\begin{ccode}
    char announce[1024]; // URL de announce
\end{ccode}

Por outro lado, vetores dinâmicos são construídos por ponteiros para o tipo (ao invés do
tipo) juntamente com alocação e liberação de memória programáticas. Com isso, a memória
é reservada em tempo de execução na memória heap. Assim, pode ter seu tamanho
redimensionado conforme a necessidade.

\begin{ccode}
    int *a = malloc( 3*sizeof(int) ); // aloca memória para um vetor de 3 inteiros
    free(a);                          // desaloca a memória alocada
\end{ccode}

Apesar dessa diferença, ambos os tipos de vetores funcionam da mesma maneira, usufruindo
da aritmética de ponteiros e acesso instantâneo ao valor armazenado.

\begin{ccode}
    int b[3];
    int *c = malloc( 3*sizeof(int) );

    b[0] = 1; b[1] = 3; b[2] = 5;
    c[0] = 2; c[1] = 4; c[2] = 6;

    printf("b[1] = \%d, *(c+2) = \%d\n", b[1], *(c+2)); // b[1] = 3, *(c+2) = 6
\end{ccode}

O Transmission não aloca seus vetores dinâmicos literalmente desta forma, pois possui
suas funções próprias onde encapsula o código mostrado.

%!TEX root = ../../tcc.tex

\newpage
\subsection*{Listas ligadas}

Listas ligadas é uma estrutura de dado que organiza os objetos de forma linear, assim
como os vetores. Porém, enquanto estes possuem índices que determinam a sua posição, as
listas possuem ponteiros para os outros elementos. Por causa disso, elas crescem
organicamente, conforme novos elementos vão sendo criados e associados, então evitando
desperdício de memória.

Por conter ponteiros, elementos de listas ligadas sempre são definidos usando-se
estruturas.

\begin{ccode*}{label=./libtransmission/makemeta.c:41}
struct FileList {
    uint64_t          size;
    char *            filename;
    struct FileList * next;      // ponteiro para o próximo elemento
};
\end{ccode*}

Essa estrutura é utilizada como se fosse um tipo definido pelo usuário, que aloca em
memória de forma dinâmica.

\begin{ccode*}{label=./libtransmission/makemeta.c:41}
struct FileList {
    uint64_t          size;
    char *            filename;
    struct FileList * next;      // ponteiro para o próximo elemento
};
\end{ccode*}

Feito isso, para a estrutura poder ser usada, o Transmission utiliza de uma função que
aloca uma variável dessa estrutura de forma dinâmica e seta valores iniciais nulos para
os seus campos.

\begin{ccode*}{label=./libtransmission/list.c:19}
static tr_list * recycled_nodes = NULL;

static tr_list* node_alloc(void) {
    tr_list * ret;                  // ponteiro que apontará para a região alocada

    if (recycled_nodes == NULL) {   // Se não houver elementos reciclados,...
        ret = tr_new(tr_list, 1);   // ... aloque um novo.
    }
    else {   // Caso contrário, reutilize, reapontando as tomando o controle do central
        ret = recycled_nodes;           // ... referências dos elementos adjacentes...
        recycled_nodes = recycled_nodes->next;   // ... e tomando o controle do central
    }

    *ret = TR_LIST_CLEAR;           // limpa campos do elemento
    return ret;                     // devolve o ponteiro para o elemento
}
\end{ccode*}

Existem vários tipos de listas ligadas, algumas podendo, inclusive, serem combinadas
entre si:

\begin{itemize}
    \item simplesmente ligada: possui somente um ponteiro para o próximo elemento
    \item duplasmente ligada: possui 2 ponteiros (um para o elemento anterior e outro
        para o posterior)
    \item multiplamente ligada: possui ponteiros vários elementos, porém ligando-os em
        ordens diferentes
    \item circularmente ligada: quando o último elemento liga a lista de volta ao
        1º elemento
    \item com cabeça: quando possui um elemento falso somente para ajudar a manipular as
        listas
\end{itemize}

\todoquestion{mostro algoritmos?}

Comparando-se vetores e listas ligadas, cada um tem suas vantagens e desvantagens em
relação à complexidade de seus algoritmos de manipulação.

\newpage
\begin{table}
    \centering
    \begin{tabular}{| l | c | c | c |}
        \hline
        \textbf{Ação} & \textbf{Vetor (est.)} & \textbf{Vetor (din.)} & \textbf{Lista ligada} \\
        \hline
        Busca por posição & $\Theta(1)$ & $\Theta(1)$ & $\Theta(n)$ \\
        \hline
        Inserção/Remoção (início) & $\Theta(n)$ & $\Theta(n)$ & $\Theta(1)$ \\
        \hline
        Inserção/Remoção (fim) & $\Theta(1)$ & $\Theta(1)$ & \parbox[t]{.3\textwidth}{\centering $\Theta(1)$ (c/ cabeça) \\ $\Theta(n)$ (s/ cabeça)} \\
        \hline
        Inserção/Remoção (meio) & $\Theta(n)$ & $\Theta(n)$ & $\Theta(n)$ \\
        \hline
        Redimensionamento & $\Theta(n)$ & ? & não necessita \\
        \hline
    \end{tabular}
    \caption{tabela de comparação de complexidades dos algoritmos de manipulaçãp de
    vetores e listas ligadas. OBS: tempos de buscas considerados lineares.
    Redimensionamento de vetor dinâmico depende da implementação da linguagem C.}
\end{table}

%!TEX root = ../../tcc.tex

\newpage
\subsection*{Tabelas hash}

\Glspl{hashtable} são estruturas de dados eficientes na implementação de dicionários.
Apesar de buscas demorarem tanto quanto procurar um elemento em uma lista ligada -
$\Theta(n)$ no pior caso -, o espalhamento é bastante eficiente. Isso faz com que o
tempo médio de uma busca seja $O(1)$.

Uma \gls*{hashtable} generaliza a noção do vetor de elementos comum. Nele, o
endereçamento direto nos permite avaliar o conteúdo de uma posição em $O(1)$. O que
torna esta tabela especial é a vantagem de transformar um certo conteúdo possuir uma
chave específica e exclusiva, fornecendo um meio de se encontrar essa chave. Esse meio é
uma \gls{hashfunction}.

Às vezes, \glspl*{hashfunction} fazem com que 2 conteúdos possuam a mesma chave, ou
seja, as chaves colidem. Para esses casos, existem várias técnicas de solução de
conflitos, porém colisões podem ser evitadas com boa \gls*{hashfunction}, descritas a
seguir.

A \gls*{hashtable} usada pelo Transmission aparece no \gls*{dht}, porém de uma forma
mais simples: não existe ``a \gls*{hashfunction} do \gls*{dht}'' como de costume, onde
existe uma função característica para uma modelagem de tabela. Ao invés disso, as chaves
já estão calculadas, sendo os IDs do \glspl*{torrent} e dos \glspl*{peer} do Kademlia.

%!TEX root = ../../tcc.tex

\section{Funções de hash}
\label{sec:sha1}

\begin{comment}
Aqui vou explicar como funciona o algoritmo da \gls{hashfunction} SHA-1 e mostrar como
e para que é usado na identificação de torrents e na verificação de integridade de
partes.
\end{comment}

\Glspl{hashfunction} são funções matemáticas usadas para gerar conteúdo de comprimento
fixo para referência ao conteúdo original.

Isso é útil quando existe grandes quantidades de dados a serem indexados. Por exemplo,
numa busca em uma tabela de dados ou tarefas de comparação de dados tal como detecção de
duplicatas ou de trechos de sequências de DNA semelhantes. Outro uso é na criptografia,
quando é utilizado para comparar um conjunto de dados recebido com outro já existente,
verificando sua igualdade.

Em geral, \glspl*{hashfunction} não são inversíveis, ou seja, não é possível recuperar o
valor de entrada para um dado \gls{hashvalue}. Quando usado para fins criptográficos,
são construídas de forma que essa reconstrução seja impossível sem que um imenso poder
computional seja utilizado. Por conta disso, é igualmente difícil fingir um
\gls*{hashvalue} para esconder dados maliciosos, sendo então usado pelo algoritmo
\gls{pgp}.

Outra característica importante é o determinismo. Quando a função é executada para 2
dados iguais, deve produzir o mesmo valor. Essa condição é fundamental no caso de uma
\gls*{hashtable}, pois a busca deve encontrar o mesmo local onde o algoritmo de inserção
armazenou o dado, logo precisando do mesmo \gls*{hashvalue}.

Outros usos para \glspl*{hashfunction} são em \glspl{checksum}, códigos de correção de
erros e cifras.

%!TEX root = ../../tcc.tex

\subsection*{SHA-1}

O SHA-1 é uma \gls*{hashfunction} criada pela NSA, a Agência de Segurança Nacional
americana, em 1995, e tem seu nome da abreviação de \emph{Secure Hash Algorithm}
(algoritmo de hash seguro). Ela produz um \gls*{hashvalue} de 160 bits (ou 20 bytes),
que forma um número hexadecimal de 40 caracteres.

Seu uso foi difundido depois que seu predecessor, o algoritmo MD5, foi constatado
com colisão de \gls*{hashvalue} prática realizada em um computador comum.

O seu algoritmo é relativamente simples se comparado com seus irmãos SHA-2 e SHA-3, o
que lhe confere a melhor vazão dentre as 3 versões.

\begin{algorithm}
    \caption{SHA1 (M)}
    \label{sha1}
    V $\leftarrow$ SHF1 (5A827999 || 6ED9EBA1 || 8F1BBCDC || CA62C1D6, M)
\end{algorithm}

\begin{algorithm}
    \caption{SHF1 (K,M)}
    y $\leftarrow$ shapad (M) \\
    Parse y as $M_1$ || $M_2$ || \ldots || $M_n$, onde $|M_i| = 512 (1 \leq i \leq n)$\\
    V $\leftarrow$ 67452301 || EFCDAB89 || 98BADCFE || 10325476 || C3D2E1F0 \\
    \Para{$i \leftarrow 1$ \Ate $n$}{
        V $\leftarrow$ shf1 (K, $M_i || V$)
    }
    \Retorna{V}
\end{algorithm}



\begin{comment}

    \FOR{$i=0$ to $79$}
    \IF{$0 \leq i \leq 19$}
    \STATE $T = a \lll 5 + f_{if}(b,c,d) + e + W[i] + $K0
    \ELSIF{$20 \leq i \leq 39$}
    \STATE $T = a \lll 5 + f_{xor}(b,c,d) + e + W[i] + $K1
    \ELSIF{$40 \leq i \leq 59$}
    \STATE $T = a \lll 5 + f_{maj}(b,c,d) + e + W[i] + $K2
    \ELSIF{$60 \leq i \leq 79$}
    \STATE $T = a \lll 5 + f_{xor}(b,c,d) + e + W[i] + $K3
    \ENDIF
    \STATE $e = d$, $d = c$, $c = b \lll 30$, $b = a$, $a = T$
    \ENDFOR
    \STATE $H_0 = a + H_0$, $H_1 = b + H_1$, $H_2 = c + H_2$, $H_3 = d + H_3$, $H_4 = e + H_4$
    \ENDFOR
    \RETURN concat($H_0$, $H_1$, $H_2$, $H_3,$ $H_4$)

    \SetAlgoLined
    \LinesNumbered
    \Dados{audio, tamanho, nivel, filtro, ch}
    \Entrada{i, j, inicio, comprimento}
    \BlankLine
    $inicio \leftarrow 0$\;
      $comprimento \leftarrow tamanho$\;
    \Para{$i \leftarrow 0$ \Ate $i<nivel$}{
        $inicio \leftarrow 0$\;
    $comprimento \leftarrow tamanho/2^{i-1}$\;
    \Para{$j \leftarrow 0$ \Ate $j<2^{i-1}$}{
        \eSe{j é par}{
                transformada\_wavelet(audio[inicio],comprimento, 1,'n', filtro, ch)\;
            }{
                transformada\_wavelet(audio[inicio],comprimento, 1,'i', filtro, ch)\;
            }
        }
    }
\end{comment} % SHA1

\input{Chapters/chap4/sec003-cripto} % RC4

%!TEX root = ../../tcc.tex

\section{Bitfields}
\label{sec:bitfield}

Apesar de ser um simples array de bits usado no gerenciamento de partes que o programa
já baixou ou não, foi percebido que o seu uso de forma não-convencional, chamado de
\emph{lazy bitfield}, pode ajudar a evitar o controle de banda (chamado de modelagem de
tráfego, ou \emph{traffic shaping} em inglês), feito por \glspl{isp}.

\input{Chapters/chap4/sec005-protocolos-rede} % HTTP e UDP

\input{Chapters/chap4/sec006-multicast}

%!TEX root = ../../tcc.tex

\section{Roteamento de pacotes}

Em alguns roteadores que são ponte entre a Internet e a rede que ele gerencia, existe
uma função de se configurar portas de comunicação de rede automaticamente usando-se o
\emph{Network Address Translation Port Mapping Protocol}. Assim, não é necessário
realizar uma configuração específica somente pare esse fim. %NAT PMP

%!TEX root = ../../tcc.tex

\section{Retomada de downloads}

No Transmission e em algums outros softwares que realizam downloads, existe a função de
se pausar a transferência do arquivo para que seja retomado em outro momento. Nesta
seção, mostrarei qual a idéiapor trás desse mecanismo e a forma como foi implementado no
Transmission.

%!TEX root = ../../tcc.tex

\section{Conexão com a Internet}

Aqui mostrarei a parte técnica da programação em linguagem C para utilização de
transmissão de dados por rede.t

%!TEX root = ../../tcc.tex

\section{IPv6}

O IPv6 é a versão mais recente do protocolo de Internet (IP), que foi criado para
substituir o IPv4, que atualmente é mais o usado porém sofre de exaustão de endereços.
Nesta seção, falarei sobre o novo protocolo e quais as implicações na programação de
softwares com comunicação de redes.

%!TEX root = ../../tcc.tex

\section{Threads}

Dentro de um contexto de programas que utilizam a Internet e funcionalidades que chegam
próximas ao tempo real, processamentos pesados devem ser tratados com cautela a fim de
se manter a instantaneidade do processo. Aqui explicarei como o Transmission usa o
conceito de \emph{threads} para paralelizar esses processamentos e, com isso, conseguir
utilizar as informações de rápida mudança antes que seja necessário obtê-las novamente.

%!TEX root = ../../tcc.tex

\section{Engenharia de Software}

O Transmission é um programa extenso e complexo, desenvolvido por vários programadores
que estão espalhados pelo globo. Nesta seção, abordarei alguns pontos utilizados pelos
desenvolvedores na manutenção do código aberto de qualidade e em funcionamento.

\afterpage{\clearpage}