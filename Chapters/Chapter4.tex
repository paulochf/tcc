%!TEX root = ../tcc.tex

\chapter{Conceitos de Computação no BitTorrent}

O BitTorrent é um protocolo cuja existência depende de vários componentes das mais
variadas áreas de estudo da Computação. Neste capítulo, mostraremos alguns desses
componentes e de que forma o Transmission os implementa, a fim de conseguir desempenhar
bem sua função de cliente BitTorrent.

%!TEX root = ../../tcc.tex

\section{Estruturas de dados}

Conjuntos são tão fundamentais na Ciência da Computação quando na matemática. Os
conjuntos manipulados por algoritmos são, em geral, alterados ao longo do tempo, sendo
chamados de dinâmicos. Alguns algoritmos utilizam conjuntos, realizando diversas
operações sobre eles, como inserção, remoção e testes de pertinência de elementos.
Conjuntos dinâmicos que permitem essas operações são chamados de dicionários
\cite{book:clrs}.

Na linguagem C, geralmente são implementados usando estruturas (\emph{structs}), que são
coleções de variáveis (membros), independentes de tipo, agrupadas sobre o mesmo nome.
Com isso, várias implementações de dicionário foram criadas, cada uma com suas
peculiaridades. Dentre as estruturas utilizadas no Transmission encontramos:

\begin{itemize}
    \item vetores (\emph{arrays});
    \item listas ligadas (\emph{linked lists});
    \item filas (\emph{queues}); e
    \item \glspl*{hashtable} (\emph{hash tables}).
\end{itemize}

%!TEX root = ../../tcc.tex

\subsection*{Vetores}

Vetores (ou \emph{arrays}) são a implementação de vetores matemáticos de maneira
virtual. Na prática, consistem de sequência ou listas de variáveis do mesmo tipo. Na
linguagem C, podem ser declaradas de forma estática ou dinâmica.

Vetores estáticos têm tamanho fixo estabelecido na sua declaração em tempo de
compilação, tendo espaços de memória reservados na pilha de execução de acordo com o
tipo, não podendo ser alterado durante a execução do programa.

\begin{ccode}
    char announce[1024]; // URL de announce
\end{ccode}

Já os vetores dinâmicos são blocos de memória alocados e liberados durante a execução do
programa. Para isso, são utilizados ponteiros para um objeto do mesmo tipo de cada
elemento do vetor. Dessa forma, a memória é reservada em tempo de execução na memória
heap. Assim, pode ter seu tamanho redimensionado conforme a necessidade.

\begin{ccode}
    int *a = malloc( 3*sizeof(int) ); // aloca memoria para um vetor de 3 inteiros
    free(a);                          // desaloca a memoria alocada
\end{ccode}

Apesar dessa diferença, ambos os tipos de vetores funcionam da mesma maneira, usufruindo
da aritmética de ponteiros e acesso instantâneo ao valor armazenado.

\begin{ccode}
    int b[3];
    int *c = malloc( 3*sizeof(int) );

    b[0] = 1; b[1] = 3; b[2] = 5;
    c[0] = 2; c[1] = 4; c[2] = 6;

    printf("b[1] = \%d, *(c+2) = \%d\n", b[1], *(c+2)); // b[1] = 3, *(c+2) = 6
\end{ccode}

O Transmission não aloca seus vetores dinâmicos literalmente desta forma, pois possui
suas funções próprias onde encapsula o código mostrado.

%!TEX root = ../../tcc.tex

\newpage
\subsection*{Listas ligadas}

Listas ligadas é uma estrutura de dados que organiza os objetos de forma linear, assim
como os vetores. Porém, enquanto estes possuem índices que determinam a sua posição, em
listas, cada elemento possuem um ponteiro para o elemento seguinte. Por causa disso,
seu comprimento se altera organicamente conforme novos elementos vão sendo criados e
inseridos, consumindo somente a memória necessária.

Os nós de listas ligadas são definidos usando-se estruturas.

\cfile[label="./libtransmission/list.h:28"]{./Codes/chap4/001-lista-struct.c}

Para a estrutura ser usada, o Transmission utiliza uma função que aloca memória
dinamicamente para um objeto com valor nulo em todos os seus campos.

\cfile[label="./libtransmission/list.c:19"]{./Codes/chap4/002-lista-code.c}

Existem vários tipos de listas ligadas:

\begin{description}
    \item[simplesmente ligada:] possui somente um ponteiro para o próximo elemento;
    \item[duplamente ligada:] possui 2 ponteiros, um para o elemento anterior e outro
        para o próximo elemento;
    \item[multiplamente ligada:] possui ponteiros vários elementos, porém ligando-os em
        ordens diferentes;
    \item[circularmente ligada:] quando o último elemento liga a lista de volta ao
        1º elemento; e
    \item[com cabeça:] quando possui um elemento falso somente para ajudar a manipular
        as listas.
\end{description}

Comparando-se vetores e listas ligadas, cada um tem suas vantagens e desvantagens em
relação à complexidade de seus algoritmos de manipulação.

\begin{table}
    \centering
    \begin{tabular}{| l | c | c | c |}
        \hline
        \textbf{Operação} & \textbf{Vetor} & \textbf{Lista ligada} \\
        \hline
        Busca por posição & $\Theta(1)$ & $\Theta(n)$ \\
        \hline
        Inserção/Remoção (início) & $\Theta(n)$ & $\Theta(1)$ \\
        \hline
        Inserção/Remoção (fim) & $\Theta(1)$ & \parbox[t]{.3\textwidth}{\centering $\Theta(1)$ (c/ cabeça) \\ $\Theta(n)$ (s/ cabeça)} \\
        \hline
        Inserção/Remoção (meio) & $\Theta(n)$ & $\Theta(n)$ \\
        \hline
        Redimensionamento & \parbox[t]{.25\textwidth}{\centering $\Theta(n)$ (estático) \\ ? (dinâmico)} & não necessita \\
        \hline
    \end{tabular}
    \caption{tabela com os consumos de tempo de voperações sobre vetores e listas
    ligadas. OBS: tempos de buscas são considerados lineares. Redimensionamento de vetor
    dinâmico depende da implementação da linguagem C.}
\end{table}

Por conta da agilidade que é conseguida na manipulação de listas ligadas de tamanhos
imprevisíveis, o Transmission as utiliza em várias partes do seu código, inclusive a
implementada pelo \emph{framework} de criação de interfaces GTK.

%!TEX root = ../../tcc.tex

\subsection*{Tabelas hash}

\Glspl{hashtable} são estruturas de dados eficientes na implementação de dicionários.
Apesar de buscas demorarem tanto quanto procurar um elemento em uma lista ligada
($\Theta(n)$ no pior caso), o espalhamento é bastante eficiente. Isso faz com que o
tempo médio de uma busca seja $\Oh(1)$~\cite{book:clrs}.

Uma \gls*{hashtable} generaliza a noção do \emph{array} comum. Nele, o endereçamento
direto nos permite avaliar o conteúdo de uma posição em $\Oh(1)$. O que torna esta
tabela especial é a vantagem de transformar um certo conteúdo em um valor único,
chamado de chave, fornecendo um meio de se encontrar essa chave. Esse meio é uma
\gls{hashfunction}.

Às vezes, \glspl*{hashfunction} fazem com que dois conteúdos possuam a mesma chave, ou
seja, as chaves colidem. Para esses casos, existem várias técnicas de solução de
conflitos, porém colisões podem ser evitadas com boas \glspl*{hashfunction}, descritas a
seguir.

A \gls*{hashtable} usada pelo Transmission aparece na \gls*{dht}, porém de uma forma
mais simples: não existe ``a \gls*{hashfunction} da \gls*{dht}'', onde há uma função
característica para uma modelagem de tabela. Ao invés disso, as chaves já estão
calculadas, sendo os IDs dos \glspl*{torrent} e dos \glspl*{peer} do Kademlia.

Outra utilização de \glspl*{hashtable} no BitTorrent (apesar de não ser usada no
Transmission) é nas \textbf{árvores hash} ou \textbf{árvores de Merkle}
\cite{site:merkletree}. Essas árvores são usadas para organizar o grande
\gls*{hashvalue} das partes do \gls*{torrent}, contido no \gls*{torrentfile}, em uma
árvore cujas folhas possuem o \gls*{hashvalue} de uma parte e cada nó que não é uma
folha possui como valor os \glspl*{hashvalue} dos seus nós filhos. Dessa forma, o
cálculo de um \gls*{hashvalue} de um conjunto de partes contínuo pode ser adquirido em
$\Oh(\log n)$ \cite{artigo:merkletree-cripto}.


%!TEX root = ../../tcc.tex

\section{Funções de hash}
\label{sec:sha1}

\Glspl{hashfunction} são funções matemáticas usadas para gerar conteúdo de comprimento
fixo que referencia o conteúdo original.

Isso é útil quando existem grandes quantidades de dados a serem indexados. Por exemplo,
numa busca em uma tabela de dados, ou tarefas de comparação de dados, tais como
detecção de duplicatas ou de trechos de sequências de DNA semelhantes. Outro uso é na
criptografia, quando é utilizado para comparar um conjunto de dados recebido com outro
já existente, verificando sua igualdade.

Em geral, \glspl*{hashfunction} não são inversíveis, ou seja, não é possível recuperar o
valor de entrada para um dado \gls{hashvalue}. Quando usadas para fins criptográficos,
são construídas de forma que essa recuperação seja impossível sem que um imenso poder
computacional seja utilizado. Por conta disso, é igualmente difícil fingir um
\gls*{hashvalue} para esconder dados maliciosos.

Outra característica importante é o determinismo. Quando a função é executada para dois
dados de entrada iguais, ela deve produzir o mesmo valor. Essa condição é fundamental
no caso de uma \gls*{hashtable}, pois a busca deve encontrar o mesmo local onde o
algoritmo de inserção armazenou o dado, logo, precisa do mesmo \gls*{hashvalue}.

Outros usos para \glspl*{hashfunction} são nas \glspl{checksum}, códigos de correção de
erros e cifras.

%!TEX root = ../../tcc.tex

\subsection*{SHA-1}

O SHA-1 é uma \gls*{hashfunction} criada pela NSA, a Agência de Segurança Nacional
americana, em 1995, e tem seu nome da abreviação de \emph{Secure Hash Algorithm}
(algoritmo de hash seguro). Seu uso foi difundido depois que seu predecessor, o
algoritmo MD5, foi constatado com colisão de \gls*{hashvalue} ocorrida na prática,
em um computador comum \cite{report:md5-attack}.

Pertencente a uma família de algoritmos, que conta ainda com as versões SHA-0, SHA-2
(com funcionamento para vários comprimentos de bits de saída) e SHA-3, o SHA-1 teve
falhas expostas comprovadas por colisão, ainda que de difícil realização atualmente.
Essa família é caracterizada por possuir algoritmos iterativos, baseada no desenho do
algoritmo MD4 \cite{report:md4}.

O resultado dessa função é um \gls*{hashvalue} de 160 bits (ou 20 bytes) na forma
hexadecimal. A função de compressão do algoritmo consiste de três partes:

\begin{enumerate}
    \item expansão da mensagem: a mensagem de entrada é expandida para que o bloco de
        dado total seja múltiplo de 512 bits;

    \item transformação de estado: consiste em passos simples de operações de números
        binários, utilizando alguns valores pré-definidos. Uma variável de encadeamento
        é usada como mensagem de entrada para a iteração seguinte e os blocos da
        mensagem expandida se tornam as novas chaves de iteração; e

    \item retroalimentação: ao final do processamento de um bloco de 512 bits, a
        mensagem de entrada da transformação de estado é adicionada ao valor de saída.
        Esta operação é chamada de construção de Davies-Meyer, e garante que, se a
        mensagem de entrada for fixada, a função de compressão será não-inversível na
        variável de encadeamento.
\end{enumerate}

Uma das implementações conhecidas para a linguagem C é a biblioteca OpenSSL
\cite{site:openssl}, de código aberto, e é usada pelo Transmission no desenvolvimento de
códigos de \glspl*{hashfunction}, criptografia de dados e dados pseudo-aleatórios. O
OpenSSL foi programado de forma otimizada, possuindo um código bastante diferente do
usual.

O Transmission, seguindo a \gls{api} do OpenSSL, possui a sua \gls*{hashfunction}
SHA-1. Na sua versão, calcula o \gls*{hashvalue} de um bloco de dados em partes,
utilizando-na para obter o \gls*{hashvalue} do \gls*{torrent}, na criação de um ID para
a \gls*{dht} que ele possui, e na verificação da integridade das partes obtidas de
outros \glspl*{peer}. Neste último, a parte completada tem seu \gls*{hashvalue}
calculado e comparado com o valor que está contido no \gls*{torrentfile}: se ambos os
valores coincidirem, então a parte foi adquirida sem perdas, caso contrário será baixada
novamente.

\cfile[label="\<openssl/sha.h\>:101"]{./Codes/chap4/003-sha-ctx.c}

\cfile[label="./libtransmission/crypto.c:38"]{./Codes/chap4/004-trsha.c} % SHA1

%!TEX root = ../../tcc.tex

\section{Criptografia}

A criptografia é o estudo e prática de técnicas que visam a segurança de informações de
diversas formas, a fim de que todas as partes de uma transação confiem que os objetivos
daquela segurança tenham sido alcançadas. Com início há mais de quatro mil anos, no
Egito antigo, a área se manteve em atividade devido a necessidade de se cifrar e quebrar
mensagens, se tornando mais necessária com o advento da computação.

Os quatro objetivos principais da criptografia são oferecer \cite{book:applied-crypto}:

\begin{description}
    \item[confidencialidade,] para se manter um conteúdo de informação sob sigilo,
        com acesso somente àqueles autorizados a visualizá-lo. Suas inúmeras formas vão
        desde a proteção fisica até os algoritmos matemáticos que tornam o conteúdo
        ininteligível;

    \item[integridade dos dados,] que foca na alteração não autorizada de dados, de
        forma a detectar essa manipulação;

    \item[autenticação,] relacionada à identificação de dados e entidades. Duas partes
        que iniciam uma comunicação devem se identificar. Da mesma forma, uma
        informação deve poder ser autenticada a partir do envio para o destinatário; e

    \item[aceitação,] que previne uma entidade de negar uma ação ou compromisso
        previamente estabelecido, necessitando de um procedimento que envolva um
        terceiro para resolver alguma disputa.
\end{description}

Existem dois tipos principais de criptografia: por chaves simétricas ou por chaves
públicas. Por meio de chaves simétricas, um sistema criptográfico usa uma mesma chave
para encriptar e decriptar mensagens, sejam estas feitas por cifragem de fluxo de dados
(onde cada caractere da mensagem é computado por vez) ou de blocos de dados (onde blocos
da mensagem são computados). Uma desvantagem das chaves simétricas \cite{wiki:crypto} é
que é necessário um gerenciamento das chaves utilizadas, a fim de se manter a
criptografia segura. Para isso, idealmente, cada par de entidades que desejam se
comunicar entre si devem compartilhar a mesma chave, fazendo com que o número total de
chaves necessárias numa rede seja proporcional ao quadrado do número de membros. Este
era o único método conhecido até junho de 1976 \cite{artigo:diffiehellman}.

Já as criptografias que utilizam chaves públicas surgiram a partir de outro trabalho de
Whitfield Diffie, Martin Hellman \cite{artigo:diffiehellman-public}, onde propuseram
o sistema criptográfico por meio da troca de chaves assimétricas (\emph{Diffie-Hellman
Key Exchange}). Nessa troca, são usadas duas chaves diferentes, porém relacionadas
matematicamente, onde a chave privada --- para posse apenas do seu dono --- é usada
para a geração de uma chave pública, para distribuição livre. Assim, enquanto a chave
pública de uma entidade é usada para a criptografia dos dados pela outra entidade,
apenas a respectiva chave privada pode descriptografá-la.

Atualmente, existem vários algoritmos e métodos de criptografia, cada qual com suas
vantagens e desvantagens, e cenários de aplicação mais adequados do que outros. Porém,
a criptografia tem como mote ``não existe sistema de segurança impenetrável'', o que nos
leva à área de criptoanálise, que é a arte e ciência de se analisar sistemas de
segurança de forma a descobrir seus aspectos ocultos, utilizando-os para quebrar os
respectivos sistemas criptográficos. Assim, muitos dos métodos existentes possuem falhas
descobertas: enquanto algumas já são práticas, outras são apenas teóricas, por falta de
capacidade computacional atual.

%!TEX root = ../../tcc.tex

\subsection*{Troca de chaves Diffie-Hellman}

Este método, também conhecido por \emph{Diffie-Hellman-Merkle Key Exchange}, é uma das
formas mais conhecidas de troca de chaves criptográficas. Ele permite que duas partes
que não possuem conhecimento a priori do outro estabeleçam um segredo comum utilizando
métodos de comunicação inseguros.

O método algébrico utiliza grupos multiplicativos de inteiros módulo $p$, onde $p$ é um
número primo e $g$ é chamado de número gerador. O protocolo pode ser explicado no
seguinte exemplo, que usa duas partes (Alice e Bob)
\cite{book:schneier,artigo:diffiehellman}: suponha que Alice e Bob desejam estabelecer
um meio seguro de comunicação, que Eve deseja espionar realizando um ``ataque de homem
no meio'', tendo acesso a todas as informações que Alice e Bob trocarem.

\begin{enumerate}
    \item Alice e Bob escolhem dois números inteiros $p$ primo e $g$ gerador;

    \item Alice escolhe um número inteiro $X_{A}$ para sua chave privada, e envia para
        Bob o resultado $Y_{A} = g^{X_{A}} \bmod p$.

    \item Bob então escolhe um número inteiro $X_{B}$ para sua chave privada, e envia
        para Alice o resultado $Y_{B} = g^{X_{B}} \bmod p$.

    \item Alice, então, calcula a chave compartilhada $S_A = B^{X_{A}} \bmod p$. Bob faz
        o mesmo, ou seja, calcula $S_B = A^{X_{B}} \bmod p$.

    \item como $S_A = S_B = S$, Alice e Bob passam a utilizar a chave $S$.
\end{enumerate}

A princípio, não é óbvio ver que $S_A = S_B = S$, mas é fácil mostrar. Considere Alice e
suas chaves. A chave que ela recebeu de Bob, $Y_{B}$, foi resultado de
$Y_{B} = g^{X_{B}} \bmod p$. Então, o cálculo de $S_A$ feito por ela é equivalente a
$S_A = (g^{X_{B}})^{X_{A}} \bmod p$.

Analogamente, Bob recebeu $Y_{A}$ de Alice, que foi resultado de
$Y_{A} = g^{X_{A}} \bmod p$. Assim, o cálculo dele de $S_B$ é equivalente a
$S_B = (g^{X_{A}})^{X_{B}} \bmod p$. Porém, podemos manipular um pouco as equações,
chegando em
\begin{align*}
S_A & = (g^{X_{B}})^{X_{A}} \bmod p \\
    & = g^{(X_{B}.X_{A})}   \bmod p \\
    & = g^{(X_{A}.X_{B})}   \bmod p \\
    & = (g^{X_{A}})^{X_{B}} \bmod p \\
    & = S_B
\end{align*}

Veja que não importa que Eve tenha obtido $p$, $g$, $Y_{A}$ ou $Y_{B}$. Ela não
conseguirá obter $S$ pois este depende de $X_{A}$ e $X_{B}$. Além disso, Eve pode
tentar calcular $X_{A}$ e $X_{B}$, porém a dificuldade deste cálculo depende dos
tamanhos de $p$, $X_{A}$ e $X_{B}$: quanto maiores forem esses números, mais difíceis
serão os cálculos inversos, que são chamados de ``problemas de logaritmo discreto''. É
praticamente impossível descobrir essas chaves privadas em uma quantidade de tempo
razoável. Assim, esse método é considerado seguro, enquanto a computação quântica não
estiver desenvolvida o suficiente para que novos algoritmos possam ser usados
\cite{artigo:shor}.

%!TEX root = ../../tcc.tex

\subsection*{RC4}

O RC4 (ou ainda ``Rivest Cypher 4'', ``Ron's Code 4'' ou ``Arc Four'') é uma função
criptográfica criada por Ron Rivest em 1987. Inicialmente era segredo comercial, porém
em 1994 seu código foi publicado na lista de discussão de criptografia CypherPunks
\cite{site:rc4-code}, se espalhando pela Internet rapidamente. Seu uso se tornou comum,
sendo usado por muitos softwares, chegando a protocolos como as encriptações de placas
de rede sem fio WEP e WPA ou ainda o protocolo de segurança TLS para conexões de
Internet.

O RC4 é um algoritmo de chave simétrica que se divide em duas partes: na primeira
parte, ele executa o algoritmo de escalonamento de chaves (\emph{key scheduling}),
que utiliza uma chave de tamanho variável entre 1 e 256 bytes para inicializar uma
tabela de estados. Cada elemento dessa tabela é permutado pelo menos uma vez e será
usado na geração de bytes pseudoaleatórios na segunda parte.

Na segunda parte, executa o algoritmo de geração pseudoaleatório, onde modifica o estado
(também permutando os elementos pelo menos uma vez) e resulta em 1 byte da chave de
fluxo, que então é mesclada usando \gls{xor} bit a bit com o próximo byte da mensagem
para produzir ou próximo byte da mensagem cifrada (na encriptação) ou da decifrada (na
decriptação), já que o \gls*{xor} é uma função involuntária (ou seja, é uma função que
é a própria inversa).

\cfile[label="./libtransmission/crypto.c:258"]{./Codes/chap4/005-rc4enc.c}

\cfile[label="./libtransmission/crypto.c:237"]{./Codes/chap4/006-rc4dec.c}

%!TEX root = ../../tcc.tex

\subsubsection*{Falhas de segurança}

The objective is NOT to create a cryptographically secure protocol that can survive unlimited observation of passing packets and substantial computational resources on network timescales. The objective is to raise the bar sufficiently to deter attacks based on observing ip-port numbers in peer-to-tracker communications.

If a tracker observes a large number of tracker requests and responses and subsequent connections, it is possible to attack the encryption. RC4 is known to have a number of weaknesses especially in the way it is used with WEP [2] [3] [4]. However, with tracker peer obfuscation, the number of bytes transferred between the tracker and a client is likely significantly smaller than transferred between a wireless computer and a basestation. An attacker faces a much larger task in obtaining sufficient ciphertext to directly break the encryption.

Hobbling the RC4 encryption by using a bounded-length RC4 pseudorandom string for small swarms is likely to have negilgible impact on security over any other encyption method since the pseudorandom string is probably equal to or longer than the plaintext and thus no part of it is repeated in the XOR except as peers arrive or leave the swarm. Thus on the timescales of rerequest intervals, nearly the same ciphertext is handed to every peer requesting the same infohash. Intercepting the same ciphertext multiple times provides no additional information to the attacker. The attacker could correlate ip-port pairs in connections following tracker responses, but an attacker could do this regardless of the encryption method employed. Furthermore more direct methods of traffic analysis applied to peer-to-peer communication is available to network operators.

For larger swarms, hobbling RC4 may simplify breaking the encryption since the same pseudorandom string is used repeatedly across the peer list. Some study is in order taking into account that the tracker can periodically change intiailization vectors.

%!TEX root = ../../tcc.tex

\subsection*{Criptografia no BitTorrent}

O BitTorrent usa criptografia somente quando o usuário habilita a opção correspondente
no programa cliente, criptografando comunicações com \glspl*{tracker} e \glspl*{peer}.
O método escolhido pelo protocolo é o que está definido em uma de suas propostas de
melhoria. Porém, esta está suspensa, sendo então usada de forma extraoficial atualmente
pelos programas cliente em geral, inclusive o Transmission.

Além disso, como é dito na própria proposta \cite{site:bittorrent-cripto}, o objetivo
dessa melhoria é impedir que \glspl{isp} ou outros administradores de rede de bloquear
ou quebrar conexões BitTorrent que ocorram entre o \gls*{peer} receptor de uma resposta
de um \gls{tracker} e qualquer outro \gls*{peer} cujo endereço IP e porta apareça nessa
resposta. Por isso, essa especificação é chamada de ofuscação de \glspl*{peer}. A idéia
proposta é usar o \bverb|info_hash| de um \gls*{torrent} como chave compartilhada entre
o \gls*{peer} e o \gls*{tracker}, não impedindo ataques de homem no meio por espiões que
saibam desses \glspl{hashvalue}.

%!TEX root = ../../tcc.tex

\subsubsection*{Comunicação com trackers}

Quando criptografia for habilitada, as comunicações com \glspl*{tracker}, ou seja, as
requisições de \glspl{announce}, não devem enviar o \bverb|info_hash| do \gls*{torrent}.
Ao invés disso, deve enviar o \bverb|sha_ih|, que é o \gls*{hashvalue} SHA-1 do
\bverb|info_hash| (que também é um \gls*{hashvalue} SHA-1) na forma \gls*{urlencode}.

Já a resposta de \glspl*{tracker} a \glspl*{announce} se mantêm no mesmo formato,
exceto pela lista de \glspl*{peer}, que será ofuscada.

Para isso, a requisição do \gls*{announce} deve passar como parâmetros

\begin{description}
    \item[supportcrypto:] valor 1 indica que o \gls*{peer} pode criar e receber
        conexões criptografadas. Neste caso, se o \gls*{tracker} aceitar esta extensão
        do BitTorrent, as listas de \glspl*{peer} que enviará em suas respostas
        priorizarão outros \glspl*{peer} que também enviaram \bverb|supportcrypto=1|
        antes dos que não o fizeram.

    \item[requirecrypto:] valor 1 indica que o \gls*{peer} irá criar e aceitar somente
        conexões criptografadas. Neste caso, as listas de \glspl*{peer} que o
        \gls*{tracker} enviará conterão somente \glspl*{peer} que também enviaram
        \bverb|supportcrypto=1| e \bverb|requirecrypto=1|.

    \item[cryptoport:] quando o parâmetro de \bverb|requirecrypto=1|, é inteiro que
        representa a porta na qual o cliente irá utilizar para conexões criptografadas.
\end{description}

\cfile[label="./libtransmission/announcer-http.c:58"]{./Codes/chap4/007-cripto-announce.c}

%!TEX root = ../../tcc.tex

\subsubsection*{Comunicação com peers}

A comunicação entre \glspl*{peer} é criptografada usando RC4 e troca de chaves
Diffie-Hellman-Merkle \cite{wikivuze:encription}. Para isso, o protocolo de
\emph{handshake} para mensagens entre \glspl*{peer} é estendido, de forma a efetuar
esses cinco procedimentos criptográficos:

\cfile[label="./libtransmission/crypto.c:60-79"]{./Codes/chap4/008-cripto-peer1.c}

1) $A$ envia $Y_A$ para $B$:

O Transmission, nesse caso o \gls*{peer} $A$, envia sua chave pública ($Y_A$) com um
trecho de dados aleatórios, com comprimento qualquer entre 0 e 512 bytes;

\cfile[label="./libtransmission/handshake.c:313"]{./Codes/chap4/009-cripto-peer2-send-ya.c}

2) $A$ recebe $Y_B$ de $B$:

O outro \gls*{peer}, $B$, responde com sua chave pública ($Y_B$). Assim, a chave
compartilhada $S$ já pode ser calculada;

\cfile[label="./libtransmission/handshake.c:391"]{./Codes/chap4/010-cripto-peer3-read-yb.c}

3) $A$ envia para $B$ as opções de criptografia e a mensagem de \emph{handshake}:

Então, $A$ envia dados de forma criptografada: a mensagem de \emph{handshake} e outras
informações sobre a criptografia para $B$, na forma:

\begin{verbatim}
HASH('req1', S), HASH('req2', SKEY) xor HASH('req3', S),
    ENCRYPT(VC, crypto_provide, len(PadC), PadC, len(IA)), ENCRYPT(IA)
\end{verbatim}

\newpage
onde:

\begin{description}
    \item[HASH():] é a função que calcula o \gls*{hashvalue} SHA-1 de todos os
        parâmetros de entrada concatenados;

    \item[ENCRYPT():] é a função RC4 com chave \bverb|HASH('keyA', S, SKEY)| (se
        $A \rightarrow B$), ou \bverb|HASH('keyB', S, SKEY)| (se $B \rightarrow A$). Os
        primeiros 1024 bytes da encriptação RC4 são descartados. O uso seguido desta
        função por uma das partes encripta o fluxo de dados, sem reinicializações ou
        trocas;

    \item[VC:] é uma constante de verificação (\emph{verification constant}), que é uma
        string de 8 bytes de valor 0x00, usada para verificar se a outra parte conhece
        $S$ e SKEY, evitando ataques de repetição do SKEY;

    \item[crypto\_provide/crypto\_select:] são bitfield de 32 bits. Dois valores são
        usados atualmente, com o restante sendo reservado para uso futuro. O \gls*{peer}
        $A$ deve ligar os bits de todos os métodos suportados por ele, enquanto o
        \gls*{peer} $B$ deve ligar o bit do método escolhido dentre os oferecidos, e
        enviar como resposta. Por enquanto, 0x01 indica sem encriptação, e 0x02 indica o
        RC4;

    \item[PadC/PadD:] reservados para futuras extensões do handshake. Hoje, possuem 0
         bytes;

    \item[IA:] conjunto de dados inicial de $A$. Pode ser 0 bytes, se quiser esperar por
        negociação de encriptação;
\end{description}

\cfile[label="./libtransmission/handshake.c:391", samepage=false]{./Codes/chap4/011-cripto-peer4-hashashash.c}

4) $A$ recebe de $B$ a opção de criptografia escolhida e a mensagem de resposta ao
\emph{handshake} encriptada por RC4:

Aqui, $B$ envia como resposta:

\begin{verbatim}
ENCRYPT(VC, crypto_select, len(padD), padD), ENCRYPT2(Payload Stream)
\end{verbatim}

\cfile[label="./libtransmission/handshake.c:493", samepage=false]{./Codes/chap4/012-cripto-peer5-readvc.c}

5) $A$ envia para $B$ o fim da mensagem de \emph{handshake} encriptada por RC4:

\cfile[label="./libtransmission/handshake.c:587"]{./Codes/chap4/013-cripto-peer6-handshake.c}
\cfile[label="./libtransmission/peer-io.c:1083"]{./Codes/chap4/014-cripto-peer7-write.c} % RC4

%!TEX root = ../../tcc.tex

\section{Bitfields e o traffic shaping}
\label{sec:bitfield}

No BitTorrent, os bitfields são usados como uma tabela de controle de quais partes de um
\gls*{torrent} o programa cliente já recebeu de outros \glspl*{peer}, permitindo
conhecer quais são as partes mais raras. Com eles, o \gls*{peer} consegue utilizar a
estratégia de \emph{Rarest First} (\pageref{subsubsec:rarest-first}), priorizando
partes raras antes das mais comuns.

Com o poder crescente da tecnologia de banda larga e a disseminação do uso do
BitTorrent em todo o mundo, as \glspl{isp} têm tido trabalho em manter a qualidade de
suas infraestruturas com a imensa quantidade de dados que trafegam pelos seus cabos.
Para evitar que o tráfego em excesso cause perda de desempenho no oferecimento do
serviço, as empresas realizam o chamado \emph{traffic shaping} (modelagem de tráfego),
que foi especificado em 1998 \cite{site:rfcshaping}.

Para conseguir controlar o grande volume da sua rede, as \glspl*{isp} tentam
classificar os dados de acordo com seus protocolos, as portas utilizadas e as
informações de cabeçalho dos pacotes. Feito isso, os pacotes entram na fila
correspondente ao seu tipo, até esta fila ser enviada depois de seguir o contrato de
tráfego da fornecedora da conexão. É dessa forma que os pacotes enviados durante o uso
de BitTorrent se atrasam ou, até mesmo, se perdem, causando limitação das taxas de
transmissão.

Especificamente, o controle do tráfego é admissível porque, durante a análise dos
pacotes, é possível perceber o distinto protocolo de \emph{handshake} do BitTorrent,
onde ocorre o envio dos bitfields. Para se evitar isso, foi desenvolvida a técnica de
\emph{lazy bitfield} (bitfield preguiçoso): o \gls*{peer} envia um bitfield indicando
que possui menos partes que a realidade, enviando a diferença em seguida, utilizando
mensagens \bverb|have|. Assim, o \gls*{peer} finge não ser um potencial \gls{seeder},
por exemplo, evitando ter sua rede regulada.

A proposta de extensão \emph{Fast} do protocolo BitTorrent \cite{site:bittorrent-fast},
que surgiu em 2008, permite o uso de uma mensagem simples para avisar outros
\glspl*{peer}, que também suportarem o protocolo, que o remetente completou o
\gls*{torrent} sem a necessidade de enviar o bitfield. Por conta disso, a probabilidade
da conexão ser controlada é menor.

%!TEX root = ../../tcc.tex

\section{Protocolos de redes}

Aqui vou explicar o que são os protocolos de rede TCP e UDP, apontar suas diferenças e
mostrar os motivos pelos quais o UDP é preferido ao TCP no uso de endereços de
\gls*{announce} de \glspl*{tracker}. % HTTP e UDP

%!TEX root = ../../tcc.tex

\newpage
\section{Multicast}

Na área de redes de computadores, \emph{multicast} é um conceito de entrega de
informação simultânea para um grupo de computadores a partir de uma única fonte,
utilizando algoritmos específicos para seu roteamento. Geralmente, é implementado sobre
os protocolos da camada de rede ou de enlace, e é usado em aplicações que necessitam de
distribuição de ``um para muitos'' ou ``muitos para muitos'', como por exemplo as de
distribuição de dados em massa (atualização de servidores), transmissão de áudio e vídeo
contínuo (transmissões de vídeo ao vivo), aplicações de dados compartilhados
(teleconferência), fontes de dados (bolsa de valores), atualização de cache de Internet
e jogos multi-jogadores interativos \cite{book:kurose}.

Os endereços de \emph{multicast} são definidos pelo protocolo IP como a sub-rede formada
entre os endereços IP \sverb|224.0.0.0| até \sverb|239.255.255.255|, antigamente chamada
de \textbf{rede classe D}, onde cada trecho desse intervalo é reservado a um uso
específico, controlado pela IANA (\emph{Internet Assigned Numbers Authority})
\cite{site:iana-multicast}. Dentre estes, endereços contidos no intervalo de
\sverb|239.0.0.0| a \sverb|239.255.255.255|, chamados de
\textbf{endereços de multicast de escopo administrativo}, são destinados a uso privado
dentro dos limites de organizações, porém, podendo não ser globalmente únicos
\cite{site:rfcmulticast}. Cada endereço deles é de um grupo específico de
\emph{multicast}, cujas mensagens encaminhadas a esse endereço serão repassadas a todos
os nós de rede que estiverem conectados a ele.

Um tipo de mensagem enviada em \emph{multicast} é a que utiliza o padrão SSDP
(\emph{Simple Service Discovery Protocol}), que serve para anúncio e descoberta de
equipamentos conectados a uma rede, sem a necessidade de mecanismos de configuração
baseados em servidor. Esse protocolo envia mensagens semelhantes às HTTP utilizando o
protocolo \gls{udp}. Apesar de não ser especificado por nenhum órgao técnico, o SSDP é
amplamente utilizado e faz parte da especificação do UPnP
(\emph{Universal Plug and Play}) \cite{site:upnp}.

O BitTorrent utiliza o \emph{multicast} de forma não oficial (que não possui
especificação formal) para a \textbf{descoberta de \glspl*{peer} local}. Inicialmente
desenvolvido pelo programa $\mu$Torrent e adotado por outros, um \gls*{peer} se
conecta a um grupo de \emph{multicast} no endereço \sverb|239.192.152.143:6771| e
espera por mensagens SSDP \cite{site:utorrent-forum}, onde estão indicadas o endereço
IP e porta do \gls*{peer} remetente, adicionando à sua lista de \glspl*{peer}.

\cfile[label="./libtransmission/tr-lpd.c:425"]{./Codes/chap4/016-rec-lpd.c}

\newpage
Da mesma forma, ele envia mensagens frequentemente para o grupo, para avisar da sua
existência na rede.

\cfile[label="./libtransmission/tr-lpd.c:425"]{./Codes/chap4/017-send-lpd.c}

%!TEX root = ../../tcc.tex

\section{Configuração e roteamento de pacotes em rede}

\begin{comment}
Em alguns roteadores que são ponte entre a Internet e a rede que ele gerencia, existe
uma função de se configurar portas de comunicação de rede automaticamente usando-se o
\emph{Network Address Translation Port Mapping Protocol}. Assim, não é necessário
realizar uma configuração específica somente pare esse fim.
\end{comment}

Atualmente, o uso de roteadores de rede para uso doméstico é bastante comum, servindo
para distribuição de uma conexão de Internet para vários equipamentos eletrônicos que
conseguem acessá-la. Para que ocorra essa distribuição de dados de Internet é
necessário que no roteador funcione um protocolo chamado NAT
(\emph{Network Address Translator}).

Um roteador em que funciona um serviço de NAT parece, externamente (para a rede
exterior), que é um único dispositivo com o seu endereço IP \cite{book:kurose}, enquanto
internamente, enquanto internamente é visto como o responsável por rotear os dados e
que abstrai a conexão de Internet externa. Do ponto de vista do roteador, ele conhece os
endereços IP de cada dispositivo da rede interna e o endereço IP do modem do \gls{isp}.
Assim, ele contrói uma tabela de tradução de endereços onde, para cada endereço da rede
interna, associa a uma porta de rede interna e outra externa.

Então, quando um dos dispositivos envia dados para a Internet, esse pacote passa pelo
roteador, que troca o endereço IP e porta do dispositivo de origem, contidos no
datagrama, pela respectiva tradução da rede externa, e então repassa o pacote para a
Internet. Analogamente, quando um pacote da Internet chega ao roteador, este verifica a
porta de conexão de destino, contida no datagrama, e a procura em sua tabela de
tradução. Se houver um dispositivo da rede interna associado a essa porta, o roteador
troca o endereço e porta de destinos do datagrama e repassa o pacote para tal
dispositivo.

Enquanto o serviço de NAT parece solucionar um problema, ele cria outro: \glspl*{peer}
de redes \gls*{p2p} necessitam saber as portas com que estão se comunicando para
informar a outros \glspl*{peer}. Porém, um programa cliente que esteja sendo executado
em um dispositivo que está numa sub-rede atendida por um serviço NAT deve saber informar
qual a porta externa na tabela de NAT está associada a ele, e não a que utiliza no
dispositivo.

Para resolver esse problema, existem dois protocolos diferentes de configuração de
portas: o NAT PMP (NAT \emph{Port Mapping Protocol}), que atualmente é chamado de PCP
(\emph{Port Control Protocol}) \cite{site:rfcpcp}, e o uPnP
(\emph{Universal Plug and Play}) \cite{site:rfcupnp}. Ambos esses protocolos têm a mesma
função: configurar um serviço de NAT e conhecer as portas externas que ele lhes
reservou, fazendo a travessia de NAT (\emph{NAT Traversal}).

O UPnP \cite{wiki:upnp} é um conjunto de protocolos que possibilita a comunicação entre
dispositivos variados a partir da conexão destes a uma rede, estabelecendo serviços.
Entre os diversos protocolos, está o IGDP (\emph{Internet Gateway Device Protocol}), no
qual é possível realizar várias ações, desde o conhecimento do endereço IP externo do
roteador, conhecer o mapeamento de portas internas e externas, e adicionar ou remover
entradas nesse mapeamento. Ao adicionar uma entrada, é possível realizar a travessia de
NAT. O UPnP utiliza as portas 1900 para o pacotes \gls*{udp} e a 2869 para portas
\gls*{tcp}.

Já o PCP \cite{wiki:pcp}, introduzido em 2005 pela Apple como alternativa ao IGDP,
serve somente para configuração de travessia de NAT e conhecimento do endereço externo
do gateway NAT, automatizando a configuração de redirecionamento de portas em
roteadores. Para isso, utiliza a porta 5351 para pacotes \gls*{udp}.

Essa praticidade de não precisar se configurar manualmente um roteador tem um preço.
Ambos os protocolos não são totalmente seguros, fazendo com que um roteador que permita
configurações através deles possa oferecer meios de se invadir essas redes por conexões
externas. Assim, para efeitos de segurança, um serviço NAT não substitui
\emph{firewalls}.

O Transmission utiliza duas bibliotecas, uma para cada protocolo: o MiniUPnP
\cite{site:miniupnp} e o libnatpmp \cite{site:libnatpmp}. Ambas são desenvolvidas por
Thomas Bernard e de código aberto. %NAT PMP

\input{Chapters/chap4/08-000-sec-retomada-downloads}

%!TEX root = ../../tcc.tex

\section{Conexão com a Internet}

Programas que se conectam à redes ou à Internet são comuns atualmente. Para isso, a
linguagem C permite escrever programas que criam conexões de Internet via \glspl{socket}
\cite{site:beej}.

Através de um \gls*{socket}, a aplicação consegue enviar mensagens para a camada de
transporte, que as transforma em segmentos e as repassa para as camadas inferiores, que
transmitem os dados.

Quando o protocolo \gls{tcp} for usado, os passos para se criar uma conexão serão:

\begin{enumerate}
    \item configuração do \gls*{socket} em uma variável \textbf{sadd} do tipo
        \sverb|struct sockaddr_in| (para IPv4) ou \sverb|struct sockaddr_in6| (para
        IPv6), que conterá informações do computador de destino da conexão.

        Enquanto isso, ocorre a criação de um \gls*{socket} usando a função
        \sverb|socket()|. Essa função, passando o parâmetro \bverb|SOCK_STREAM|, pede
        para o SO (sistema operacional) a criação de um descritor de arquivo especial
        para \gls*{socket} \gls*{tcp}, que será usado para o fluxo de dados da conexão;

    \item ligação do \gls*{socket} à variável \textbf{sadd} usando a função
        \sverb|bind()|, que serve para registrar no SO que ele deve manipular os dados
        que chegam na porta indicada em \textbf{sadd}, usando o \gls*{socket} indicado;

    \item aguardar conexões no \gls*{socket} usando a função \sverb|listen()|; e

    \item receber dados pelo \gls*{socket} usando a função \sverb|accept()|.
\end{enumerate}

\cfile[label="./libtransmission/net.c:317"]{./Codes/chap4/018-net-tcp.c}

\cfile[label="./libtransmission/net.c:417"]{./Codes/chap4/019-net-tcp2.c}

Já quando o protocolo \gls{udp} for usado, os passos para se criar uma conexão são
exatamente os mesmos que os do protocolo \gls*{tcp}, exceto pelo parâmetro
\bverb|SOCK_DGRAM| na função \sverb|socket()|.

\cfile[label="./libtransmission/tr-udp.c:190"]{./Codes/chap4/020-net-udp.c}

As funções \sverb|send()| e \sverb|recv()|, que enviam e recebem dados, respectivamente,
possuem as suas versões mais complexas \sverb|sendto()| e \sverb|recvfrom()|, recebendo
outro \gls*{socket} como parâmetro de entrada.

\input{Chapters/chap4/10-000-sec-ipv6}

\input{Chapters/chap4/11-000-sec-threads}

\input{Chapters/chap4/12-000-sec-engsoft}

\afterpage{\clearpage}