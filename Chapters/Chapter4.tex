%!TEX root = ../tcc.tex

\chapter{Conceitos de Computação no BitTorrent}

O BitTorrent é um protocolo cuja existência depende de vários componentes das mais
variadas áreas de estudo da Computação. Neste capítulo, mostraremos alguns desses
componentes e de que forma o Transmission os implementa, a fim de conseguir desempenhar
bem sua função de cliente BitTorrent.

%!TEX root = ../../tcc.tex

\section{Estruturas de dados}

Conjuntos são tão fundamentais na Ciência da Computação quando na matemática. Os
conjuntos manipulados por algoritmos são, em geral, alterados ao longo do tempo, sendo
chamados de dinâmicos. Alguns algoritmos utilizam conjuntos, realizando diversas
operações sobre eles, como inserção, remoção e testes de pertinência de elementos.
Conjuntos dinâmicos que permitem essas operações são chamados de dicionários
\cite{book:clrs}.

Na linguagem C, geralmente são implementados usando estruturas (\emph{structs}), que são
coleções de variáveis (membros), independentes de tipo, agrupadas sobre o mesmo nome.
Com isso, várias implementações de dicionário foram criadas, cada uma com suas
peculiaridades. Dentre as estruturas utilizadas no Transmission encontramos:

\begin{itemize}
    \item vetores (\emph{arrays});
    \item listas ligadas (\emph{linked lists});
    \item filas (\emph{queues}); e
    \item \glspl*{hashtable} (\emph{hash tables}).
\end{itemize}

%!TEX root = ../../tcc.tex

\subsection*{Vetores}

Vetores (ou \emph{arrays}) são a implementação de vetores matemáticos de maneira
virtual. Na prática, consistem de listas de variáveis do mesmo tipo. Na linguagem C,
podem ser declaradas de forma estática ou dinâmica.

Vetores estáticos têm tamanho fixo estabelecido na sua declaração em tempo de
compilação, tendo espaços de memória reservados na pilha de memória de acordo com o
tipo, não podendo ser alterado em tempo de execução.

\begin{ccode}
    char announce[1024]; // URL de announce
\end{ccode}

Por outro lado, vetores dinâmicos são construídos por ponteiros para o tipo (ao invés do
tipo) juntamente com alocação e liberação de memória programáticas. Com isso, a memória
é reservada em tempo de execução na memória heap. Assim, pode ter seu tamanho
redimensionado conforme a necessidade.

\begin{ccode}
    int *a = malloc( 3*sizeof(int) ); // aloca memória para um vetor de 3 inteiros
    free(a);                          // desaloca a memória alocada
\end{ccode}

Apesar dessa diferença, ambos os tipos de vetores funcionam da mesma maneira, usufruindo
da aritmética de ponteiros e acesso instantâneo ao valor armazenado.

\begin{ccode}
    int b[3];
    int *c = malloc( 3*sizeof(int) );

    b[0] = 1; b[1] = 3; b[2] = 5;
    c[0] = 2; c[1] = 4; c[2] = 6;

    printf("b[1] = \%d, *(c+2) = \%d\n", b[1], *(c+2)); // b[1] = 3, *(c+2) = 6
\end{ccode}

O Transmission não aloca seus vetores dinâmicos literalmente desta forma, pois possui
suas funções próprias onde encapsula o código mostrado.

%!TEX root = ../../tcc.tex

\newpage
\subsection*{Listas ligadas}

Listas ligadas é uma estrutura de dados que organiza os objetos de forma linear, assim
como os vetores. Porém, enquanto estes possuem índices que determinam a sua posição, em
listas, cada elemento possuem um ponteiro para o elemento seguinte. Por causa disso,
seu comprimento se altera organicamente conforme novos elementos vão sendo criados e
inseridos, consumindo somente a memória necessária.

Os nós de listas ligadas são definidos usando-se estruturas.

\cfile[label="./libtransmission/list.h:28"]{./Codes/chap4/001-lista-struct.c}

Para a estrutura ser usada, o Transmission utiliza uma função que aloca memória
dinamicamente para um objeto com valor nulo em todos os seus campos.

\cfile[label="./libtransmission/list.c:19"]{./Codes/chap4/002-lista-code.c}

Existem vários tipos de listas ligadas:

\begin{description}
    \item[simplesmente ligada:] possui somente um ponteiro para o próximo elemento;
    \item[duplamente ligada]: possui 2 ponteiros, um para o elemento anterior e outro
        para o próximo elemento;
    \item[multiplamente ligada]: possui ponteiros vários elementos, porém ligando-os em
        ordens diferentes;
    \item[circularmente ligada]: quando o último elemento liga a lista de volta ao
        1º elemento; e
    \item[com cabeça]: quando possui um elemento falso somente para ajudar a manipular as
        listas.
\end{description}

Comparando-se vetores e listas ligadas, cada um tem suas vantagens e desvantagens em
relação à complexidade de seus algoritmos de manipulação.

\begin{table}
    \centering
    \begin{tabular}{| l | c | c | c |}
        \hline
        \textbf{Operação} & \textbf{Vetor} & \textbf{Lista ligada} \\
        \hline
        Busca por posição & $\Theta(1)$ & $\Theta(n)$ \\
        \hline
        Inserção/Remoção (início) & $\Theta(n)$ & $\Theta(1)$ \\
        \hline
        Inserção/Remoção (fim) & $\Theta(1)$ & \parbox[t]{.3\textwidth}{\centering $\Theta(1)$ (c/ cabeça) \\ $\Theta(n)$ (s/ cabeça)} \\
        \hline
        Inserção/Remoção (meio) & $\Theta(n)$ & $\Theta(n)$ \\
        \hline
        Redimensionamento & \parbox[t]{.25\textwidth}{\centering $\Theta(n)$ (estático) \\ ? (dinâmico)} & não necessita \\
        \hline
    \end{tabular}
    \caption{tabela com os consumos de tempo de voperações sobre vetores e listas
    ligadas. OBS: tempos de buscas são considerados lineares. Redimensionamento de vetor
    dinâmico depende da implementação da linguagem C.}
\end{table}

Por conta da agilidade que é conseguida na manipulação de listas ligadas de tamanhos
imprevisíveis, o Transmission as utiliza em várias partes do seu código.

%!TEX root = ../../tcc.tex

\subsection*{Tabelas hash}

\Glspl{hashtable} são estruturas de dados eficientes na implementação de dicionários.
Apesar de buscas demorarem tanto quanto procurar um elemento em uma lista ligada -
$\Theta(n)$ no pior caso -, o espalhamento é bastante eficiente. Isso faz com que o
tempo médio de uma busca seja $O(1)$ \cite{book:clrs}.

Uma \gls*{hashtable} generaliza a noção do vetor de elementos comum. Nele, o
endereçamento direto nos permite avaliar o conteúdo de uma posição em $O(1)$. O que
torna esta tabela especial é a vantagem de transformar um certo conteúdo possuir uma
chave específica e exclusiva, fornecendo um meio de se encontrar essa chave. Esse meio é
uma \gls{hashfunction}.

Às vezes, \glspl*{hashfunction} fazem com que 2 conteúdos possuam a mesma chave, ou
seja, as chaves colidem. Para esses casos, existem várias técnicas de solução de
conflitos, porém colisões podem ser evitadas com boa \gls*{hashfunction}, descritas a
seguir.

A \gls*{hashtable} usada pelo Transmission aparece no \gls*{dht}, porém de uma forma
mais simples: não existe ``a \gls*{hashfunction} do \gls*{dht}'' como de costume, onde
existe uma função característica para uma modelagem de tabela. Ao invés disso, as chaves
já estão calculadas, sendo os IDs do \glspl*{torrent} e dos \glspl*{peer} do Kademlia.

%!TEX root = ../../tcc.tex

\section{Funções de hash}
\label{sec:sha1}

\Glspl{hashfunction} são funções matemáticas usadas para gerar conteúdo de comprimento
fixo para referência ao conteúdo original.

Isso é útil quando existe grandes quantidades de dados a serem indexados. Por exemplo,
numa busca em uma tabela de dados ou tarefas de comparação de dados tal como detecção de
duplicatas ou de trechos de sequências de DNA semelhantes. Outro uso é na criptografia,
quando é utilizado para comparar um conjunto de dados recebido com outro já existente,
verificando sua igualdade.

Em geral, \glspl*{hashfunction} não são inversíveis, ou seja, não é possível recuperar o
valor de entrada para um dado \gls{hashvalue}. Quando usado para fins criptográficos,
são construídas de forma que essa reconstrução seja impossível sem que um imenso poder
computional seja utilizado. Por conta disso, é igualmente difícil fingir um
\gls*{hashvalue} para esconder dados maliciosos, sendo então usado pelo algoritmo
\gls{pgp}.

Outra característica importante é o determinismo. Quando a função é executada para dois
dados de entrada iguais, ela deve produzir o mesmo valor. Essa condição é fundamental
no caso de uma \gls*{hashtable}, pois a busca deve encontrar o mesmo local onde o
algoritmo de inserção armazenou o dado, logo precisando do mesmo \gls*{hashvalue}.

Outros usos para \glspl*{hashfunction} são em \glspl{checksum}, códigos de correção de
erros e cifras.

%!TEX root = ../../tcc.tex

\subsection*{SHA-1}

O SHA-1 é uma \gls*{hashfunction} criada pela NSA, a Agência de Segurança Nacional
americana, em 1995, e tem seu nome da abreviação de \emph{Secure Hash Algorithm}
(algoritmo de hash seguro). Seu uso foi difundido depois que seu predecessor, o
algoritmo MD5, foi constatado com colisão de \gls*{hashvalue} prática realizada em um
computador comum \cite{report:md5-attack}.

Pertencente a uma família de algoritmos, que conta ainda com as versões SHA-0, SHA-2
(com funcionamento para vários comprimentos de bits de saída) e SHA-3, o SHA-1 teve
falhas expostas comprovadas por colisão, aidan que de difícil realização atualmente.
Essa família é caracterizada por possuir algoritmos iterativos, baseada no desenho do
algoritmo MD4 \cite{report:md4}.

O resulta dessa função é um \gls*{hashvalue} de 160 bits (ou 20 bytes), que forma um
número hexadecimal de 40 caracteres. A função de compressão do algoritmo consiste de
três partes:

\begin{enumerate}
    \item expansão da mensagem: a mensagem de entrada é expandida para que o bloco de
        dado total seja múltiplo de 512 bits;

    \item transformação de estado: consiste de passos simples de operações de números
        binários usando alguns valores pré-definidos, onde uma variável de encadeamento
        é usada de mensagem de entrada para iteração seguinte e os blocos da mensagem
        expandida se tornam as novas chaves de iteração;

    \item retroalimentação: ao final do processamento de um bloco de 512 bits, a
        mensagem de entrada da transformação de estado é adicionada ao valor de saída.
        Esta operação é chamada de construção de Davies-Meyer e garante que se a
        mensagem de entrada for fixada então a função de compressão é náo-inversível na
        variável de encadeamento.
\end{enumerate}

Uma das implementações conhecidos para a linguagem C é a biblioteca OpenSSL
\cite{site:openssl}, de código aberto, e é usada pelo Transmission no desenvolvimento de
códigos de \glspl*{hashfunction}, criptografia de dados e dados pseudo-aleatórios. O
OpenSSL foi programado de forma otimizada, possuindo um código bastante diferente do
usual.

O Transmission, seguindo a \gls{api} do OpenSSL, possui a sua \gls*{hashfunction}
SHA-1. Na sua versão, calcula o \gls*{hashvalue} de um bloco de dados em partes,
utilizando-na para obter o \gls*{hashvalue} do \gls*{torrent}, na criação de um ID para
o \gls*{dht} que ele possui, e na verificação da integridade das partes obtidas de
outros \glspl*{peer}.

\cfile[label="\<openssl/sha.h\>:101"]{./Codes/chap4/003-sha-ctx.c}

\cfile[label="./libtransmission/crypto.c:38"]{./Codes/chap4/004-trsha.c} % SHA1

%!TEX root = ../../tcc.tex

\section{Criptografia}

Aqui vou explicar como este algoritmo de chave simétrica funciona e como é utilizado
pelo Transmission para criptografar pacotes de dados. % RC4

%!TEX root = ../../tcc.tex

\section{Bitfields}
\label{sec:bitfield}

Apesar de ser um simples array de bits usado no gerenciamento de partes que o programa
já baixou ou não, foi percebido que o seu uso de forma não-convencional, chamado de
\emph{lazy bitfield}, pode ajudar a evitar o controle de banda (chamado de modelagem de
tráfego, ou \emph{traffic shaping} em inglês), feito por \glspl{isp}.

%!TEX root = ../../tcc.tex

\section{Protocolos de redes}

Aqui vou explicar o que são os protocolos de rede TCP e UDP, apontar suas diferenças e
mostrar os motivos pelos quais o UDP é preferido ao TCP no uso de endereços de
\gls*{announce} de \glspl*{tracker}. % HTTP e UDP

%!TEX root = ../../tcc.tex

\section{Multicast}

Apesar de não ser utilizado pelo protocolo BitTorrent, o multicast, que é uma forma de
entregar dados a um grupo de computadores simultaneamente numa só transmissão, é usado
pelo Transmission para tentar descobrir \glspl*{peer} que estão na mesma rede local,
otimizando as conexões.

%!TEX root = ../../tcc.tex

\section{Roteamento de pacotes}

Em alguns roteadores que são ponte entre a Internet e a rede que ele gerencia, existe
uma função de se configurar portas de comunicação de rede automaticamente usando-se o
\emph{Network Address Translation Port Mapping Protocol}. Assim, não é necessário
realizar uma configuração específica somente pare esse fim. %NAT PMP

%!TEX root = ../../tcc.tex

\section{Retomada de downloads}

No Transmission e em algums outros softwares que realizam downloads, existe a função de
se pausar a transferência do arquivo para que seja retomado em outro momento. Nesta
seção, mostrarei qual a idéiapor trás desse mecanismo e a forma como foi implementado no
Transmission.

%!TEX root = ../../tcc.tex

\section{Conexão com a Internet}

Aqui mostrarei a parte técnica da programação em linguagem C para utilização de
transmissão de dados por rede.t

%!TEX root = ../../tcc.tex

\section{IPv6}

O IPv6 é a versão mais recente do protocolo de Internet (IP), que foi criado para
substituir o IPv4, que atualmente é mais o usado porém sofre de exaustão de endereços.
Nesta seção, falarei sobre o novo protocolo e quais as implicações na programação de
softwares com comunicação de redes.

%!TEX root = ../../tcc.tex

\section{Threads}

Dentro de um contexto de programas que utilizam a Internet e funcionalidades que chegam
próximas ao tempo real, processamentos pesados devem ser tratados com cautela a fim de
se manter a instantaneidade do processo. Aqui explicarei como o Transmission usa o
conceito de \emph{threads} para paralelizar esses processamentos e, com isso, conseguir
utilizar as informações de rápida mudança antes que seja necessário obtê-las novamente.

%!TEX root = ../../tcc.tex

\section{Engenharia de Software}

O Transmission é um programa extenso e complexo, desenvolvido por vários programadores
que estão espalhados pelo globo. Nesta seção, abordarei alguns pontos utilizados pelos
desenvolvedores na manutenção do código aberto de qualidade e em funcionamento.

\afterpage{\clearpage}