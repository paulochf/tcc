%!TEX root = ../tcc.tex

\chapter{Comentários Finais}

Este trabalho abordou o protocolo BitTorrent, entendendo a sua especificação, e usar o
código do Transmission foi um excelente exemplo, servindo como ponte entre teoria e
prática. Esse tipo de estudo foi muito esclarecedor em vários aspectos, permitindo
conhecer melhor o protocolo e saber um pouco mais de cada um dos tópicos abordados, que
ainda são bastante atuais. Entre estes, estão desde a própria linguagem C até as
teorias relacionadas, como a teoria dos jogos, métodos criptográficos utilizados,
estruturas de dados e os conhecimentos de redes.

As disciplinas da grade curricular que foram mais importantes para o entendimento do
BitTorrent em sua totalidade foram:

\begin{itemize}
    \item MAC0211 - Laboratório de Programação 1
    \item MAC0242 - Laboratório de Programação 2
    \item MAC0323 - Estruturas de Dados
    \item MAC0332 - Engenharia de Software
    \item MAC0448 - Criptografia para Segurança de Dados
    \item MAC0438 - Programação Concorrente
    \item MAC0448 - Programação para Redes
\end{itemize}

\section*{Trabalhos futuros}

Como atividade que não foi realizada, podendo ser realizada em trabalhos futuros, está a
análise mais aprofundada dos estudos que analisaram o protocolo. Apesar do fato de terem
sido encontrados em grande número e requererem um bom conhecimento prévio do protocolo e
seu funcionamento, não foi possível avaliar os impactos desses estudos individualmente
ou como um todo.

\afterpage{\clearpage}