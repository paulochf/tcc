%!TEX root = ../tcc.tex

\chapter{Comentários Finais}

Este trabalho abordou o protocolo BitTorrent, entendendo a sua especificação, e usar o
código do Transmission foi um excelente exemplo, servindo como ponte entre teoria e
prática. Esse tipo de estudo foi muito esclarecedor em vários aspectos, permitindo
conhecer melhor o protocolo e saber um pouco mais de cada um dos tópicos abordados, que
ainda são bastante atuais. Entre estes, estão desde a própria linguagem C até as
teorias relacionadas, como a teoria dos jogos, métodos criptográficos utilizados,
estruturas de dados e os conhecimentos de redes.

As disciplinas da grade curricular que foram mais importantes para o entendimento do
BitTorrent em sua totalidade foram:

\begin{itemize}
    \item MAC0211 - Laboratório de Programação 1
    \item MAC0242 - Laboratório de Programação 2
    \item MAC0323 - Estruturas de Dados
    \item MAC0332 - Engenharia de Software
    \item MAC0336 - Criptografia para Segurança de Dados
    \item MAC0438 - Programação Concorrente
    \item MAC0448 - Programação para Redes
\end{itemize}

A disciplina Estruturas de Dados (MAC0323), apesar de sua importância, não é necessária
para compreender o protocolo, pois este independe da linguagem de programação. Essa
disciplina, por ter foco em linguagem C, auxilia somente programas escritos nessa
linguagem, assim como Laboratório de Programação 1 (MAC0211).

Paralelamente, as disciplinas de Laboratórios de Programação 1 (MAC0211) e 2 (MAC0242)
e Engenharia de Software (MAC0332) são fundamentais, pois desenvolvem a habilidade de
construir programas com muitos componentes e abstrações. Especificamente, as duas
últimas geralmente são ministradas em linguagens orientadas a objeto, como Java ou
ActionScript que justificam o fato da disciplina de Estruturas de Dados não ser
requerida.

Apesar dos estudos que ocorreram antes do início do trabalho, de artigos que analisaram
o protocolo, não foram feitas as análises dos resultados desses artigos e o impacto
deles no desenvolvimento do BitTorrent.

\todo[inline]{tentar melhorar isso aqui}

\afterpage{\clearpage}