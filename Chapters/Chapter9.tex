%!TEX root = ../tcc.tex

\chapter*{Visão Pessoal}

Aqui, apresento a minha visão sobre a experiência obtida neste trabalho, relacionando-a
o curso do BCC.

\section*{Desafios e frustrações}

Devo dizer que o tema BitTorrent não foi um assunto que eu desejei estudar a princípio.
Na verdade, eu pretendia mesmo era aplicar algum estudo de redes sociais usando grafos,
mas, quando eu contei meu plano para o professor Coelho, ele teve a (feliz e brilhante)
idéia de me sugerir este assunto. Eis que abracei o tema e o escolhi como orientador.

Após ter definido o tema, o primeiro desafio (que acredito ser comum a todos que cursam
esta disciplina) foi decidir o que escrever. Tive a mesma sensação que alguém deve ter
quando lhe dão uma folha em branco e pedem para desenhar algo que não conhece. Afinal,
como você vai fazer isso? Com que detalhes? Qual a abordagem? Foram muitas questões de
uma só vez.

Uma idéia inicial foi procurar na Internet trabalhos acadêmicos sobre o BitTorrent. Foi
aí que eu percebi que a área é abrangente, com artigos sobre muitos aspectos do
protocolo. Então, tracei como objetivo escrever um trabalho de análise do BitTorrent e
incluir um desses estudos já realizados. Durante o ano essa idéia caiu por terra,
pois percebi que a análise requeria muito mais tempo do que eu pensava, juntamente com
contratempos que foram ocorrendo, principalmente no segundo semestre.

Logo precisei decidir como escrever o trabalho. Uma idéia, que surgiu em uma das muitas
reuniões que tive com o professor Coelho, foi o de desenvolver o tema apresentando
código como se fizesse parte do texto. A sugestão soou estranha no começo mas, conforme
o trabalho foi crescendo, se consolidou como boa idéia.

Como o objetivo era apresentar código, avaliei os de alguns programas cliente
BitTorrent de código aberto. Não demorou muito para eu pensar no Transmission e
adotá-lo, dado que é o programa oficial da distribuição de Linux Ubuntu, que atualmente
é bastante usado.

Outro desafio que surgiu foi saber trabalhar com o \LaTeX. Eu não sabia usá-lo muito,
por isso tinha baixado um modelo preparado. Além disso, utilizava um editor online,
hospedado em um site, que logo mostrou que não era tão bom, quando percebi que o pacote
de formatação visual de código para \LaTeX\, não funcionava. Tive que decidir entre
manter o modelo de documento ou começar outro do arquivo em branco. Escolhi o segundo,
e até hoje não me arrependo, pois consegui deixar mais organizado e ao me gosto.

Durante minhas pesquisas, percebi que encontrava material sobre o BitTorrent na
Internet. Toda vez que buscava uma informação boa somente no Wikipedia, uma voz ecoava
dizendo que Wikipedia não é referência. Com o tempo, essa voz foi morrendo, pois cada
vez mais percebi que não era bem assim. Acredito que, com a rápida evolução tecnológica
que temos atualmente, é impossível depender somente de fontes estáticas, como livros.
Além disso, tive a oportunidade de tentar editar alguns artigos no Wikipedia, mais
precisamente melhorando-os com referências, e até para isso foi difícil. Foi aí que
percebi também que o antigo argumento de que ``qualquer um pode escrever qualquer coisa
no Wikipedia'' não é válido, e comecei a conviver com o fato de que eu não teria opções
melhores.

Um último desafio foi o de escrever o trabalho, dividindo o tempo entre ele e o
estágio, em um período bastante curto. Escrevo esta seção no dia anterior ao da entrega,
satisfeito com o resultado e com a sensação de dever cumprido.

\section*{Disciplinas relevantes ao trabalho}

Eu avaliei as disciplinas do BCC no capítulo \ref{chap:bcc}, na página \pageref{chap:bcc}.

\section*{Planos futuros}

A cada passo que eu dava no desenvolvimento do tema, percebi o quão fascinante é o
BitTorrent. Fiquei bastante contente com o fato dele se manter atual, mesmo com a certa
``mesmice'' que possui hoje em dia. Especificamente, a notícia de que é 7 vezes mais
rápido do que o Dropbox \cite{site:torrentvsdropbox} mostra como ele ainda é relevante.
Além disso, notícias como a que uma juíza americana estudou o protocolo antes de tomar
uma decisão em um julgamento envolvendo direitos autorais \cite{site:juizamanjona} me
fazem ter esperança de que, um dia, o BitTorrent deixe de ser sinônimo de pirataria.

A fascinação pelo assunto me fez querer entrar na comunidade BitTorrent e colaborar de
alguma forma, provavelmente no próprio Transmission, já que agora conheço boa parte do
seu código. Após algum descanso nas próximas semanas, espero entrar em contato com os
desenvolvedores e tentar ser voluntário no projeto.

%!TEX root = ../tcc.tex

\newpage
\section*{Agradecimentos}

Aqui vão alguns agradecimentos a pessoas que tornaram este trabalho possível.

Em especial, agradeço imensamente ao meu orientador, professor Coelho, pela paciência,
atenção e dedicação, me atendendo sempre que precisei de reuniões para este trabalho, e
até mesmo durante a graduação, quando achei que tudo estava distante. E obrigado por
ter sugerido o tema. Não acho que podia ter sido melhor! =)

Aos outros bons professores que tive no IME, cujas aulas marcaram a minha vida de
estudante e como profissional da Computação: Gubi, Roberto Hirata, Nina Hirata,
Carlinhos, Cristina, Ḿarcelo Queiroz, Marcelo Finger, Alfredo, Hitoshi e Reverbel.

Agradeço ao meu pai, minha mãe e minha irmã por terem compreendido mais uma fase de
ausência durante o curso, me incentivando em todas as decisões importantes que tive que
tomar durante toda a minha passagem pela USP, que se iniciou em 2004.

Agradeço à Tatiana, minha namorada, e à sua mãe Valéria, por me acolherem e me
acompanharem neste trabalho, sempre ao meu lado e de forma tão dedicada e paciente, me
apoiando nos momentos mais difíceis. Tati, te amo!

Aos meus gestores de trabalho: Prof. Mardel de Conti (LabNumeral - Poli-USP); Jason
Dyett (Harvard DRCLAS); Daniel Creão (UpLexis); Bruno Yoshimura e Allan Kajimoto
(Kekanto), pelas horas de trabalho flexíveis, que me permitiram chegar neste ponto do
curso do BCC, e em especial aos meus atuais colegas de LabArq da FAU-USP, Anderson
Valtriani, Ricardo Couto e amigos desenvolvedores pelas ausências permitidas para que
eu pudesse terminar este trabalho.

Aos meus amigos da turma de BCC 2009: Jackson, Renato, Samuel, Susana, Felipe, Rafael,
Thiago, Diogo, Gustavo Katague, Fernando, Henrique, Wilson, Gustavo Coelho, Wallace,
Jéssica, Jefferson, Tiago e Nádia (ex-IME), que nos momentos de desespero e de calmaria
me acompanharam por muitos dias durante o curso e, com certeza, suas amizades vão me
acompanhar pelo resto da vida.

Aos meus muitos amigos do BCC dos outros anos, em especial Luciano, Lucas, Marcos,
Edson, Rafael e Roberto, cuja amizade transcendeu o IME e hoje já fazem parte do meu
cotidiano.

E aos \textbf{muitos} outros nomes que conheço e com quem tive momentos no IME ou no
IF, nessa minha história pela USP, se estiver lendo este texto, saiba que também foi
muito importante! =)

E a todos, desculpem pela minha ausência. Eu estava cancelando o Apocalipse.

\afterpage{\clearpage}


\afterpage{\clearpage}