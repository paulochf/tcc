\chapter{Histórico}

Pra entender como o BitTorrent se tornou o que é hoje, devemos voltar um pouco no tempo e rever a história recente do compartilhamento de arquivos pela Internet. 

Até o fim dos anos 1990, os métodos usados para transferir arquivos tinham a forma de conexões cliente-servidor. Primeiro vieram as BBSs (1978) e a Usenet (1979), depois o protocolo FTP e seus servidores (1985). O IRC (1988) e o  Hotline (1997) permitia enabled users to communicate remotely through chat and to exchange files. The mp3 encoding, which was standardized in 1991 and which substantially reduced the size of audio files, grew to widespread use in the late 1990s. In 1998, MP3.com and Audiogalaxy were established, the Digital Millennium Copyright Act was unanimously passed, and the first mp3 player devices were launched.



, onde uma das pontas da conexão eram servidores de arquivos exclusivos para essa finalidade, com endereço IP fixo e conhecido. A pessoa que desejava dividir um arquivo com outra, fazia o upload dele para esse servidor; ao fim, avisava ao destinatário que acessasse o servidor e fazer o download do arquivo. 


Em 1999, surgiu o Napster, um serviço  de compartilhamento que começava a inovar por desfigurar o modelo cliente-servidor por possuir conexões entre usuários, criando a primeira rede P2P.

\subsection{Napster}

O Napster foi o primeiro sistema P2P, onde um usuário selecionava os arquivos de seu computador que gostaria de compartilhar com outras pessoas. Com isso, um servidor indexava os arquivos que