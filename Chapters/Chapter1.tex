%!TEX root = ../tcc.tex

\chapter{Introdução}

Desde o início da história da computação, o compartilhamento de dados é uma ação
naturalmente necessária e que passou a ser mais comum com a criação dos dispositivos de
armazenamento de dados sob a forma de arquivos, com a Internet anos depois e, mais
recentemente, com a computação em nuvem.

À partir de 1999, com o surgimento do Napster, as \glspl{p2p} passaram a ser mais
populares, sendo frequentemente utilizadas para transferir dados. Essas redes têm como
característica principal computadores transferindo dados entre si, ou seja, não
existindo funções fixas de fonte e de consumo de dados, mas sim de ambas essas funções.

O comunicação \gls*{p2p} veio se desenvolvendo ao longo dos anos. Em 2003, esse
desenvolvimento teve um grande impulso, quando Bram Cohen propôs o protocolo BitTorrent,
lançando juntamente com um programa cliente, e incentivando o seu uso por ``testadores''
simplesmente compartilhando material pornográfico. Com isso, pôde melhorar o seu
funcionamento, se tornando popular rapidamente através de seus usuários.

Desde então, muitos programas de compartilhamento BitTorrent passaram a ser
desenvolvidos, e o protocolo foi sendo estudado pela área acadêmica, passando por
melhorias. Em 2013, o protocolo foi o responsável por aproximadamente 10\% do tráfego
total de Internet nos Estados Unidos \cite{report:internet-usage-2013}, se tornando uma
das formas mais eficientes e utilizadas de se compartilhar arquivos via Internet
atualmente.

O BitTorrent contém conceitos de diversos tópicos em Ciência da Computação, tais como
teoria dos jogos, estruturas de dados, tabelas de dispersão, etc, contando ainda com
diversos estudos acadêmicos sobre topologias de rede formadas e otimizações de redes e
algoritmos, por exemplo.

Neste trabalho, estudamos o protocolo BitTorrent, analisando em profundidade a sua
aplicação pelo programa cliente Transmission \cite{site:transmission}, particularmente
interessados nos elementos de Ciência de Computação presentes no código. Este texto
contém uma descrição desses vários elementos encontrados, junto com trechos de código
que consideramos ilustrativos para exemplicá-los.

No final, relacionamos os conceitos de Ciência da Computação encontrados no
Transmission, e onde eles aparecem ou se aparecem na grade curricular do Bacharelado em
Ciência da Computação.

\afterpage{\clearpage}