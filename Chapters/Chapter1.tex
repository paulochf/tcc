\chapter{Histórico}

Pra entender como e por que o BitTorrent se tornou o que é hoje, devemos voltar um pouco no tempo e rever a história recente do compartilhamento de arquivos pela Internet. 

Primeiro vieram as BBSs (1978) e a Usenet (1979), depois o protocolo FTP (1985) e seus servidores. O IRC (1988) e o  Hotline (1997) ainda permitiam que seus usuários se comunicassem e enviassem arquivos. Até o fim dos anos 1990, esses métodos usados para transferir arquivos seguiam o modelo cliente-servidor. Neste modelo, o cliente é um computador que faz uma requisição de serviço ou recurso para um servidor, que se dedica a aguardar pedidos de clientes. 

Com o advento da codificação MP3, os arquivos de música em formato digital passaram a ser largamente utilizados na segunda metade dos anos 1990, devido à grande redução de tamanho de arquivos de áudio nesse formato quando comparado aos outros formatos contemporâneos. Então, em 1998, dois sites de compartilhamento de músicas foram criados: o AudioGalaxy.com e o MP3.com, ambos sites de busca de MP3 sobre arquivos que seus usuários faziam upload. Ambos os sites ajudaram na popularização do MP3 como forma de se escutar músicas no dia-a-dia, que foi catalisado quando indústrias de equipamentos eletrônicos anunciaram seus sucessores do toca-fitas: os tocadores de mídia portáteis, popularmente conhecidos como MP3 players.

Logo, a indústria da música, representada pela RIAA (Recording Industry Association of America, Associação da Indústria de Gravação da América), começou a ver nessa popularização do MP3 um perigo real de seus lucros com vendas de discos, e então passou a combater mais firmemente a troca de arquivos protegidos por direitos autorais pela Internet. Não demorou muito para que uma lei que protegesse o direito autoral surgisse no Senado americano e fosse assinado pelo presidente Bill Clinton, estabelecendo regras mais específicas sobre direitos autorais e se tornando obstáculo àqueles sites de compartilhamento por penalizar os usuários por quebra de direitos autorais.

Em 1999, surgiu o Napster, um serviço  de compartilhamento que começava a inovar por desfigurar o modelo cliente-servidor por possuir conexões entre usuários e não a servidores, criando assim a primeira rede P2P. Nele, um usuário selecionava os arquivos de seu computador que gostaria de compartilhar com outras pessoas, e um servidor os indexava e um algoritmo de busca de usuários.

<explicar um pouco mais do algoritmo>