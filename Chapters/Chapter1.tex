\chapter{Histórico}

Pra entender como e por que o BitTorrent se tornou o que é hoje, devemos voltar um pouco no tempo e rever a história recente do compartilhamento de arquivos pela Internet. 

Primeiro vieram as BBSs (1978) e a Usenet (1979), depois o protocolo FTP (1985) e seus servidores. O IRC (1988) e o  Hotline (1997) ainda permitiam que seus usuários se comunicassem e enviassem arquivos. Até o fim dos anos 1990, esses métodos usados para transferir arquivos seguiam o modelo cliente-servidor. Neste modelo, o cliente é um computador que faz uma requisição de serviço ou recurso para um servidor, que se dedica a aguardar pedidos de clientes. 

Com o advento da codificação MP3, os arquivos de música em formato digital passaram a ser largamente utilizados na segunda metade dos anos 1990, devido à grande redução de tamanho de arquivos de áudio nesse formato quando comparado aos outros formatos contemporâneos. Então, em 1998, dois sites de compartilhamento de músicas foram criados: o AudioGalaxy.com e o MP3.com, ambos sites de busca de MP3 sobre arquivos que seus usuários faziam upload. Ambos os sites ajudaram na popularização do MP3 como forma de se escutar músicas no dia-a-dia, que foi catalisado quando indústrias de equipamentos eletrônicos anunciaram seus sucessores do toca-fitas: os tocadores de mídia portáteis, popularmente conhecidos como MP3 players.

Não demorou muito tempo para a indústria da música, representada pela RIAA (Recording Industry Association of America, Associação da Indústria de Gravação da América) começar a enxergar essa popularização do MP3 como um perigo real de diminuição de seus lucros com vendas de discos, e então passou a encabeçar a frente de combate contra a troca de arquivos protegidos por direitos autorais pela Internet. Pouco tempo depois, durante o governo Clinton, o Senado americano aprova uma lei de proteção a conteúdos com direitos autorais, estabelecendo regras mais específicas sobre esses direitos autorais e se tornando obstáculo àqueles sites de compartilhamento por penalizar os usuários por quebrá-los. 

No segundo semestre de 1999 surgiu o Napster, um serviço  de compartilhamento de MP3 que inovou por desfigurar o modelo cliente-servidor ao possuir conexões entre usuários e não a servidores, criando assim a primeira rede P2P.

O sucesso foi rápido, e em dezembro de 1999 já enfrentava sua primeira ação judicial, vinda de várias grandes gravadoras. Sua popularidade foi crescendo, atingindo o auge de 13.6 milhões de usuários em fevereiro de 2001. A sua vida começou a complicar em julho de 2001, quando foi emitida uma ordem judicial de seu fechamento e o caso foi finalizado em 

  Case partially settled on September 24, 2001 
  Napster paid $26 million for past and future damages 
  Bertelsmann AG bought Napster on May 17, 2002 
  Napster filed Chapter 11 bankruptcy protection 
  On September 3, 2002, Napster forced to liquidate (Chapter 7) 
  On October 29, 2003 Napster came back as an online music store


Com o seu lançamento, o Napster atraiu a atenção da indústria, e em 2001 perdeu uma ação judicial por responsabilidade na distribuição de conteúdo protegido, sendo obrigado a desligar o serviço de indexação, o que tornou a rede P2P indisponível.

O sucesso do Napster, mesmo que por curto período tempo, mostrou a eficiência que redes P2P poderiam ter, e com isso novos softwares e modelos de redes foram sendo lançados, porém tentando contornar o ponto fraco do antecessor a fim de não serem novos alvos de medidas judiciais. A solução para isso foi tentar descentralizar o mecanismo de indexação e busca, que foi o calcanhar de Aquiles do Napster.

A pioneira nessa tentativa foi a rede Gnutella, que foi lançada em 2003 mas que em sua primeira versão não conseguiu manter o bom desempenho do Napster. A busca era demorada e inconsistente, pois era repassada aleatória e finitamente de peers para seus vizinhos, o que podia terminar em buscas sem resultados mesmo quando um arquivo estava sendo compartilhado por alguém conectado à rede. 

