%!TEX root = ../../tcc.tex

\subsubsection*{Protocolo}

O protocolo de mensagens \gls*{dht} utiliza o formato KRPC, que é um mecanismo de
chamada \gls{rpc} que envia dicionários \gls*{bencode} através de \gls*{udp}, uma única
vez por chamada (um pacote para a requisição, outro para a resposta), sem novas
tentativas.

Existem 3 tipos de mensagem: consulta (\emph{query}), resposta (\emph{response}) e erro
(\emph{error}). Para o protocolo \gls*{dht}, são 4 comandos \emph{query}: \bverb|ping|,
\bverb|find_node|, \bverb|get_peers| e \bverb|announce_peer|. Em todos, o nó sempre
enviará seu ID como valor da chave \bverb|id|.

Uma mensagem KRPC é um dicionário com 2 chaves comuns a todos os 4 comandos: \bverb|y|,
que especifica o tipo da mensagem, e \bverb|t|, que corresponde ao ID da transação.
Este é um número binário convertido para \gls*{string}, geralmente formada por 2
caracteres, possuindo valor até $2^{16}$, e devolvida nas respostas. Isso permite que
estas se relacionem a múltiplas consultas a um nó.

Cada tipo de mensagem possui formatos diferentes entre si, permitindo parâmetros
adicionais para cada chamada, possuindo as seguintes chaves e seus respectivos valores:

\newpage
\begin{itemize}
    \item \emph{query}
        \begin{itemize}
            \item \bverb|y|: caractere \sverb|q|
            \item \bverb|q|: string do comando desejado (\sverb|ping|,
                \sverb|find_node|, \sverb|get_peers|, \sverb|announce_peer|)
            \item \bverb|a|: dicionário contendo parâmetros adicionais, dependendo do
                comando passado na chave \bverb|q|
        \end{itemize}

    \item \emph{response}
        \begin{itemize}
            \item \bverb|y|: caractere \sverb|r|
            \item \bverb|r|: dicionário contendo valores da resposta, dependendo do
                comando passado na chave \bverb|q|
        \end{itemize}

    \item \emph{error}
        \begin{itemize}
            \item \bverb|y|: caractere \sverb|e|

            \item \bverb|e|: lista contendo 2 elementos: código (número inteiro) e
                mensagem para o erro (\gls*{string}). Os erros podem ser:
                \begin{itemize}
                    \item 201 (Generic Error): erros genéricos
                    \item 202 (Server Error): erros de servidor
                    \item 203 (Protocol Error): para pacote mal formado, argumento
                        inválido ou token incorreto
                    \item 204 (Method Unknown): comando não conhecido
                \end{itemize}

            \item exemplo: \\
                \bverb|d1:eli201e23:A Generic Error Ocurrede1:t2:aa1:y1:ee|
                (\gls*{bencode}) \\
                \sverb|{"t":"aa", "y":"e", "e":[201, "A Generic Error Ocurred"]}|
                (\gls*{string})
        \end{itemize}
\end{itemize}

As informações retornadas podem ser sobre \glspl*{peers} ou nós \gls{dht}: enquanto o
primeiro é a \enquote{informação compacta de endereço IP/porta} - string de 6 bytes (4
bytes iniciais para o endereço IP e 2 bytes finais para a porta de comunicação usada)
-, o segundo é a \enquote{informação compacta de nó} - string de 26 bytes (20 bytes
iniciais para o ID do nó e 6 bytes finais para a respectiva informação compacta de
endereço IP/porta).