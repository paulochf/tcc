%!TEX root = ../../tcc.tex

\newpage
\section{Tabelas Hash Distribuídas e o Kademlia}

Os \glspl{dht} surgiram quando buscas em \glspl{p2p} não eram eficientes, fosse pelos
problemas da supercentralização dos sistemas ou pela falta dela, tentando criar uma
maneira híbrida de buscar e localizar recursos nessas redes.

Existem duas formas de se encontrar objetos: busca e endereçamento. A primeira se baseia
no casamento de palavras-chave com as das descrições dos objetos, sendo mais amigável
para um usuário pois não necessita de formas complexas de identificação. Porém, é mais
difícil de se tornar eficiente, além de precisar conferir também os objetos para saber
se são o mesmo. As \glspl*{p2p} descentralizadas utilizavam desta abordagem.

A segunda, no entanto, utiliza endereços que possam identificá-los de forma única. Dessa
forma, um objeto é exclusivamente identificável, sendo possível encontrá-lo
eficientemente. Contudo, é necessário algum procedimento de se conhecer seu nome único,
além de ser requerido manter uma estrutura organizada por endereços. Historicamente, as
\glspl*{p2p} de estrutura centralizada se baseavam neste método.

Uma outra maneira de organizar esses objetos através da rede é de organizar as conexões
entre os diversos \glspl*{peer}, que acabam formando redes superpostas (\emph{overlay
networks}), que são redes virtuais que figuram sobre as tradicionais redes IP.

Anteriormente ao BitTorrent, nas \glspl*{p2p} de estrutura centralizada, o servidor
central tentava conhecer a situação de cada um dos \glspl*{peer} da rede, enviando
mensagens diretamente a eles. Gerenciar muitos \glspl*{peer} ao mesmo tempo
sobrecarregava o \emph{hardware} do servidor, ou forçava o seu desligamento. Por outro
lado, quando a rede tinha estrutura descentralizada, ocorria o fenômeno chamado de
inundação de mensagens, no qual estas eram repassadas entre \glspl*{peer}
intermediários até o \gls*{peer} de destino. Esse repasse excessivo acarretava
sobrecarga de processamento (\emph{overhead}) dos intermediários, além de gerar
mensagens falso-negativas.

Assim, as \glspl*{p2p} necessitam de funcionalidades eficientes para que um \gls*{peer}
possa entrar e sair do \emph{overlay}, assim como armazenar objetos e encontrá-los
(usando endereçamento). Tal eficiência é conseguida através de \glspl{dht}, onde todas
essas funcionalidades dependem da troca de mensagens eficiente entre \glspl*{peer}.

\begin{figure}[H]
    \centering
    \fbox{\includegraphics[width=0.85\textwidth]{graph-dht.png}}
    %\caption{amostra de uma rede de conexões BitTorrent}
    %\label{fig:torrent-universo}
\end{figure}

A funcionalidade do \gls*{dht} passou anos sendo utilizado extraoficialmente à
especificação do protocolo, quando, enfim, foi adicionada em 2008
\cite{site:bittorrent-dht}.

%!TEX root = ../../tcc.tex

\subsection*{Kademlia}

O Kademlia é um \gls*{dht} criado em 2002 \cite{artigo:kademlia} com o objetivo de
melhorar os métodos de busca atuais (Napster e\gls*{gnutella}), que eram ineficientes.
Assim como os outros algoritmos de \gls*{dht}, ele se baseou na estrutura informalmente
conhecida como \enquote{rede de Plaxton} (\emph{Plaxton mesh}), nome que remete a um
dos seus autores \cite{artigo:dht}. Por ter causado boas impressões, foi usado na
implementação da busca de arquivos no programa cliente eMule.

O algoritmo implementa uma rede \emph{overlay} cuja estrutura e comunicação se baseiam
na procura de seus nós. Cada um destes nós é identificado por um identificador único
(ID), que serve tanto para a identificação quanto para a localização de valores na
\gls*{hashtable}. Durante uma busca, o processo deve conhecer a chave (que é um
\gls*{hashvalue}) associado ao objeto - neste caso, o ID do \gls*{torrent}, que é seu
\gls*{hashvalue} - e explora a rede em passos, encontrando nós mais próximos da chave,
até encontrar o valor buscado ou não nós existirem mais próximos que o atual. Dessa
forma, para uma rede com $n$ nós, o algoritmo visita apenas $O(\log n)$ nós.