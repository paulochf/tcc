%!TEX root = ../../tcc.tex

\newpage
\subsection*{Resultados de announce e scrape}

Com a requisição do \gls*{announce}, o Transmission recebe a sua resposta e prepara
esses dados para utilização, carregando-os em memória.

\cfile[label="./libtransmission/announcer-http.c:288"]{./Codes/chap3/009-httpannounce.c}
\cfile[label="./libtransmission/announcer-http.c:189"]{./Codes/chap3/010-onhttpannouncedone.c}

Após essa preparação dos dados da resposta do \gls*{announce}, cuja informação
principal é a lista de \glspl*{peer}, o Transmission gerencia o tempo para a próxima
requisição periódica de \gls*{announce} ou \gls*{scrape}. Além disso, sinaliza para a
\gls{thread} principal, ou seja, a execução primária do programa, de que recebeu a
resposta do \gls*{tracker}.

\cfile[label="./libtransmission/announcer.c:1012"]{./Codes/chap3/011-onannouncedone.c}
\cfile[label="./libtransmission/announcer.c:523"]{./Codes/chap3/012-publicapeers.c}

A \gls*{thread} principal percebe que foi sinalizado o recebimento de uma resposta de
\gls*{tracker} e começa a utilizar a lista de \glspl*{peer} adquirida, adicionando-os ao
\gls{pool} do objeto em memória que representa o \gls{swarm}.

\cfile[label="./libtransmission/torrent.c:552"]{./Codes/chap3/013-ontrackerresponse.c}
\cfile[label="./libtransmission/peer-mgr.c:2113"]{./Codes/chap3/014-addpex.c}

Essa inclusão é condicionada ao fato de que o endereço de um \gls*{peer} ainda não está
contido na lista de \glspl*{peer}.

\cfile[label="./libtransmission/peer-mgr.c:1852"]{./Codes/chap3/015-verificapeer.c}

Assim, a lista de \glspl*{peer} do Transmission está pronta para ser utilizada nos
pedidos das partes do \gls*{torrent}.