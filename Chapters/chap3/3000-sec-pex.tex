%!TEX root = ../../tcc.tex

\section{Peer Exchange}

Outro mecanismo que um \gls*{peer} dentro da rede BitTorrent tem de encontrar outros
\glspl*{peer} além do \gls*{dht} é usando o protocolo PEX (\emph{Peer Exchange}, ou
troca de \glspl*{peer}). Ele permite enviar mensagens periodicamente para outros
membros do \gls*{swarm} contendo listas de \glspl*{peer} adicionados e \glspl*{peer}
removidos de sua lista de contatos, agilizando o processo de descobrimento de nós da
rede. Apesar disso, não é necessário para o funcionamento de uma transmissão BitTorrent,
sendo de uso opcional, onde o usuário do programa cliente pode escolher utilizar esse
método habilitando-o em suas configurações.

Este mecanismo faz parte das extensões do protocolo BitTorrent
\cite{site:bittorrent-extension}, porém ainda não é oficial, mesmo possuindo
especificações de mensagem, sua definição mais forte é uma convenção
\cite{wikitheory:pex}.

Existem 2 protocolos de extensão: o \emph{Azureus Messaging Protocol} (AZMP) e o
\emph{libtorrent Extension Protocol} (LTEP). Em ambos os casos, a convenção estabelece
que:

\begin{itemize}
    \item no máximo 100 \glspl*{peer} podem ser enviados numa mesma mensagem (50
        adicionados e 50 removidos).
    \item uma mensagem de \emph{peer exchange} não deve ser enviada com frequência maior
        do que 1 minuto.
\end{itemize}

No Transmission, quando o seu uso é habilitado pelo seu usuário, o programa executa
periodicamente uma função de envio de mensagem PEX, quando um \gls*{peer} se conecta a
ele ou quando recebe um pedido de mensagem PEX.

\cfile[label="./libtransmission/peer-msgs.c:2280"]{./Codes/chap3/026-send-pex.c}

Já quando o Transmission recebe uma mensagem PEX, ele só utiliza os endereços de
\glspl*{peer} adicionados, ignorando os que o \gls*{peer} remetente enviou na lista de
removidos.

\cfile[label="./libtransmission/peer-msgs.c:1159"]{./Codes/chap3/027-parse-pex.c}

\todoquestion{Tudo bem deixar deste jeito?}