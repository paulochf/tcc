%!TEX root = ../../tcc.tex

\section{Peer Exchange}

Outro mecanismo que um \gls*{peer}, dentro da rede BitTorrent, tem de encontrar outros
\glspl*{peer} (além da \gls*{dht}) é usando o protocolo PEX (\emph{Peer Exchange}, ou
troca de \glspl*{peer}). Ele permite enviar mensagens periodicamente para outros
membros do \gls*{swarm} contendo listas de \glspl*{peer} adicionados e \glspl*{peer}
removidos de sua lista de contatos, agilizando o processo de descobrimento de nós da
rede. Apesar disso, ele não é necessário para o funcionamento de uma transmissão
BitTorrent, sendo de uso opcional, com o usuário do programa cliente tendo a opção de
escolher utilizar esse método habilitando-o em suas configurações.

Este mecanismo faz parte das extensões do protocolo BitTorrent
\cite{site:bittorrent-extension}, porém, ainda não é oficial (mesmo possuindo
especificações de mensagem, sua definição mais forte é uma convenção)
\cite{wikitheory:pex}.

Existem dois protocolos de extensão: o \emph{Azureus Messaging Protocol} (AZMP) e o
\emph{libtorrent Extension Protocol} (LTEP) \cite{site:libtorrent}. Em ambos, a
convenção estabelece que:

\begin{itemize}
    \item no máximo 100 \glspl*{peer} podem ser enviados numa mesma mensagem (50
        adicionados e 50 removidos);
    \item uma mensagem de \emph{peer exchange} não deve ser enviada com frequência maior
        do que um minuto.
\end{itemize}

No Transmission, se o seu uso for habilitado pelo usuário, o programa executará
periodicamente uma função de envio de mensagens PEX, seja quando um \gls*{peer} se
conectar a ele ou quando receber um pedido de mensagem PEX.

\cfile[label="./libtransmission/peer-msgs.c:2280"]{./Codes/chap3/026-send-pex.c}

Já quando o Transmission recebe uma mensagem PEX, ele só utiliza os endereços de
\glspl*{peer} adicionados, ignorando os que o \gls*{peer} remetente enviou na lista de
removidos.

\cfile[label="./libtransmission/peer-msgs.c:1159"]{./Codes/chap3/027-parse-pex.c}