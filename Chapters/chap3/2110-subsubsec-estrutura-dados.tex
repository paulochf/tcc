%!TEX root = ../../tcc.tex

\subsubsection*{Modelagem e estrutura de dados}

O algoritmo implementa uma rede \emph{overlay}, cuja estrutura e comunicação se baseiam
na procura de seus nós. Cada um destes nós é identificado por um identificador único
(ID), que serve tanto para a identificação quanto para a localização de valores na
\gls*{hashtable}.

Essa \gls*{hashtable} é no formato de uma árvore binária, cujas folhas são os nós da
rede. Cada folha tem suas posições estabelecidas pelo menor prefixo comum de seus IDs,
organizando-os de forma que, para um dado nó $x$, a árvore é dividida em várias
subárvores menores que não o contém. Assim, a maior subárvore consiste de metade da
árvore que não contém $x$, a subárvore seguinte é feita da metade da árvore restante
onde $x$ também não está contido, etc. O Kademlia garante ainda que todo nó conhece um
outro que esteja em cada uma das subárvores, se estas contiverem algum nó.

\begin{figure}[ht!]
    \centering
    \fbox{\includegraphics[width=\textwidth]{dht1.png}}
    \caption{Árvore binária do Kademlia. O nó preto é a posição do ID 0011...; os ovais
    cinzas são as subárvores onde o nó preto deve possuir nós conhecidos. Fonte:
    \cite{artigo:kademlia}}
    \label{fig:dht-arvore}
\end{figure}

No Kademlia, objetos e nós possuem IDs únicos de 160 bits: enquanto o primeiro utiliza
o \gls*{hashvalue} de 20 bytes SHA-1 da chave \bverb|info_hash| do \gls*{torrentfile},
o segundo é um valor aleatório escolhido pelo próprio programa.

Durante uma busca por \glspl*{peer} de um \gls*{torrent}, o processo deve conhecer a
chave associada ao objeto, ou seja, o ID, e explorar a rede em passos. A cada passo,
encontrará nós mais próximos da chave, até chegar ao valor buscado ou até não existirem
nós mais próximos que o atual. Dessa forma, para uma rede com $n$ nós, o algoritmo
visita apenas $\Oh(\log n)$ nós.

\newpage
\begin{figure}[ht!]
    \centering
    \fbox{\includegraphics[width=0.8\textwidth]{dht2.png}}
    \caption{Exemplo de uma busca na árvore de nós do Kademlia usando-se um ID. O nó
    preto, de prefixo 0011, encontra o nó de prefixo 1110 através de sucessivas buscas
    (setas numeradas inferiores). As setas superiores mostram a convergência da
    busca durante a execução. Fonte: \cite{artigo:kademlia}}
    \label{fig:dht-arvore-busca}
\end{figure}

Para o conceito de proximidade, as distâncias são calculadas usando-se a função de
distância baseada em \gls{xor} bit a bit

\begin{equation*}
    d(x,y) = x \oplus y
\end{equation*}

que possui certas propriedades, algumas em comum com a equação de distância euclidiana
usual:

\begin{itemize}
    \item $d(x,x) = 0$;
    \item $x \neq y$, $d(x,y) > 0$;
    \item simetria: $\forall x,y$, $d(x,y) = d(y,x)$;
    \item desigualdade triangular: $d(x,y) + d(y,z) \geq d(x,z)$. \\
        Isto vem do fato de $d(x,z) = d(x,y) \oplus d(y,z)$ e que $\forall a \geq 0,
        \forall b \geq 0 : a + b \geq a \oplus b$;
    \item unidirecionalidade: para um dado ponto $x$ e uma distância $\Delta > 0$,
        existe exatamente um ponto $y$ tal que $d(x,y) = \Delta$. Isso garante que todas
        as procuras por uma mesma chave convirjam para um mesmo percurso, independente
        do ponto de partida;
    \item para um dado $x$ no espaço de números, o conjunto de distâncias $\Delta$ entre
        $x$ e os outros números possui distribuição uniforme, ao contrário da distância
        euclidiana \cite{bittorrentcom:dht}.
\end{itemize}

No Transmission, o Kademlia é usado a partir de uma biblioteca externa (não mantida
pelos seus desenvolvedores), criada por Juliusz Chroboczek \cite{repo:dht-c}, sendo
adicionada e adaptada para o uso no código do programa. Sua implementação não é na forma
de árvore, mas sim de listas ligadas.

\cfile[label="./third-party/dht/dht.c:129"]{./Codes/chap3/018-dht-structs.c}