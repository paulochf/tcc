%!TEX root = ../../tcc.tex

\subsection*{Magnet Link}

Além do \gls*{torrentfile}, existe uma outra forma de se obter os \glspl*{metadata}
necessários para se iniciar a transmissão, utilizando-se de \glspl{magnetlink}.

\Glspl*{magnetlink}, ao contrário dos \glspl*{torrentfile}, não estão gravados em
algum dispositivo de armazenamento. Basicamente, é um esquema de \gls{uri}, usado
exclusivamente para o protocolo.

No caso citado, do filme \enquote{A Noite dos Mortos Vivos}, o site de origem do
\gls*{torrentfile} que estamos usando não fornece um \gls*{magnetlink} oficialmente.
Porém, o Transmission consegue construir um \gls*{uri} a partir do arquivo original,
para fins de compartilhamento direto. O resultado, após decodificar o endereço para um
formato legível (retirando a \gls{urlencode}) \cite{wiki:urlencode}, foi o seguinte:

\begin{listing}[ht!]
    \begin{minted}[
        linenos,
        frame=single,
        numbersep=6pt,
        baselinestretch=1,
        fontfamily=courier,
        gobble=4,
        fontsize=\scriptsize
    ]{text}
    magnet:?xt=urn:btih:72d7a3179da3de7a76b98f3782c31843e3f818ee
    &dn=night_of_the_living_dead
    &tr=http://bt1.archive.org:6969/announce&tr=http://bt2.archive.org:6969/announce
    &ws=http://archive.org/download/
    &ws=http://ia600301.us.archive.org/22/items/&ws=http://ia700301.us.archive.org/22/items/
    \end{minted}
    \caption{link magnético do arquivo .torrent do filme
    \enquote{A Noite dos Mortos Vivos}, de 1960 \cite{torrent-file}, com parâmetros
    divididos entre linhas para melhor visualização}
    \label{lst:torrent-file-magnet-link}
\end{listing}

Esse endereço é composto por vários pares, constituídos por nomes de parâmetros e seus
respectivos valores, sem qualquer ordem específica, formando uma \gls{querystring}.
Podemos dividir esse endereço em partes, cada uma tendo o seu significado:

\begin{description}
    \item[xt:] parâmetro para \emph{exact topic}, ou tópico exato, que contém a
        informação mais importante do \gls*{magnetlink}: o identificador único de
        \glspl*{torrent}. Serve para encontrar e verificar os arquivos especificados.
        No caso, \bverb|urn:btih:<hash>| corresponde ao \gls{urn} \sverb|btih|
        (\emph{BitTorrent Info Hash}), que é a \gls*{string} \gls{hashvalue} resultado
        da \gls{hashfunction} SHA-1, convertida para hexadecimal;

    \item[dn:] parâmetro que contém o \emph{display name}, ou nome de visualização, que
        é um texto de apresentação amigável para o usuário;

    \item[tr:] o \emph{address tracker}, ou endereço do \gls*{tracker}, onde o programa
        cliente vai procurar as informações de \glspl*{peer}; e

    \item[ws:] endereço do arquivo para \emph{webseed}, ou fornecimento web,
        que é o endereço de Internet de um servidor HTTP ou FTP, que será utilizado como
        alternativa a um \gls*{swarm} problemático \cite{wiki:torrent}.
\end{description}