%!TEX root = ../../tcc.tex

\newpage
\subsection*{Algoritmos de enforcamento}

Além das estratégias mostradas, existem outras que são as formas com que um \gls*{peer}
se relacionará com seus vizinhos. Cada \gls*{peer} é responsável por melhorar suas
taxas de download e, para isso, baixam partes de quem eles conseguem e escolhem para
quais enviará outras partes, de modo a mostrar cooperação. Caso a escolha seja de não
cooperar, um \gls*{peer} ``enforca'' (\emph{choke}) outro, o que implica num
cancelamento temporário do envio de partes do primeiro para o segundo. O recebimento de
partes continuará normalmente e a conexão não precisará ser rediscutida quando o
enforcamento terminar.

O enforcamento (ou \emph{choking}) não faz parte do protocolo de mensagens, mas é
necessário para a boa performance do protocolo. Um bom algoritmo de \emph{choking}, que
se utilize de todos os recursos disponíveis, traz boas condições para todos os
\glspl*{peer} cooperativos e é resistente contra aqueles que só fazem download.

\subsubsection*{Eficiência de Pareto}

Sistemas \emph{pareto eficientes} \cite{wiki:pareto} são aqueles onde duas entidades
não podem fazer trocas e ficar mais ``felizes'' que antes. Em termos de Ciência da
Computação, buscar a eficiência de Pareto é usar um algoritmo de otimização local onde
as entidades procurem meios de melhorar mutuamente, que convergirão a uma situação ótima
global. No contexto do BitTorrent, se dois \glspl*{peer} não estão tendo vantagem
recíproca por enviar partes, eles podem começar a trocar partes entre si, de modo a
conseguir taxas de download melhores do que as anteriores.

\subsubsection*{Algoritmo de choking}

Cada \gls*{peer} fará \emph{unchoke} de quatro \glspl*{peer} (geralmente), ou seja,
sempre deixará livres de enforcamento cerca de quatro \glspl*{peer}. Com isso, o
problema passará a ser escolher quais deles fazer, deixando que o \gls*{tcp} controle o
congestionamento de banda. Essa escolha é baseada estritamente na taxa de download,
calculando-se a média dessa taxa durante vinte segundos. Antigamente, esse cálculo era
feito sobre quantidades transferidas a longos prazos, mas notou-se que era uma medida
fraca por causa das variações de largura da banda de rede.

Para evitar que recursos sejam desperdiçados pelo rápido \emph{choke} e \emph{unchoke}
de \glspl*{peer}, um cálculo é realizado a cada dez segundos a fim de saber quem
sofrerá o \emph{choke}, deixando a situação atual como está até o próximo intervalo
acabar. Esse tempo é suficiente para que o \gls*{tcp} acelere transferências até sua
capacidade total.

\subsubsection*{Optimistic Unchoking}

Fazer upload para \glspl*{peer} que possuem as melhores taxas de download não é tarefa
fácil, pois não há meios de se descobrir conexões melhores. Para tentar contornar esse
problema, um \gls*{peer} será escolhido para ser alvo de um ``\emph{unchoke} otimista''
(\emph{Optimistic Unchoking}), que é um \emph{unchoke} independentemente da taxa de
download que ele provê.

A escolha do \gls*{peer} ocorrerá a cada trinta segundos (após o terceiro ciclo de
\emph{choke}). Esse tempo é suficiente para que o \gls*{peer} escolhido tome a
iniciativa de colaborar enviando partes, sendo recompensado com novas, e essas
transmissões alcancem altas taxas de upload e download.

Este algoritmo objetiva, portanto, incentivar a cooperação entre \glspl*{peer}.

\subsubsection*{Anti-snubbing}

Eventualmente, um \gls*{peer} sofrerá \emph{choke} de todos os \glspl*{peer} de onde
estava fazendo download. Nesses casos, terá taxas de download ruins até que um
\emph{unchoke} otimista seja executado.

Quando ficar mais de um minuto sem receber nenhuma parte de um outro \gls*{peer}, o
\gls*{peer} que não recebeu nada perceberá que foi censurado (\emph{snubbed}), e
retaliará deixando de enviar partes para quem o censurou \emph{Anti-snubbing}, a menos
que ele seja sorteado em um \emph{unchoke} otimista. Dessa forma, vários \emph{unchokes}
otimistas serão executados simultaneamente, o que implicará na recuperação das taxas de
download.

\subsubsection*{Upload Only}

Enquanto um \gls*{peer} ainda não recebeu todas as partes do \gls*{torrent}, ele
priorizará para quem enviará as que ele possui, de acordo com as taxas de download que
ele atinge com os outros \glspl*{peer}. Isso acelerará a disseminação do \gls*{torrent}.
Porém, quando tiver recebido todas as partes, o \gls*{peer} manterá a rápida
disseminação do \gls*{torrent}, priorizando os \glspl*{peer} que alcançam altas taxas de
download com ele, maximizando sua taxa de upload.

\subsubsection*{Implementação do Transmission}

Quando um \gls*{torrent} é executado, e em intervalos de dez segundos, o Transmission
executa uma função que faz os \emph{chokes} de upload e também verifica os downloads.

\cfile[label="./libtransmission/peer-mgr.c:3167"]{./Codes/chap3/047-rechoke-pulse.c}

Primeiro, verifica os uploads: a cada execução da função, verificará se será um turno de
\emph{unchoke} otimista, que durará quatro execuções. Além disso, o vetor de
\glspl*{peer} será analisado para se obter informações, guardadas em um vetor
temporário, que ajudarão na decisão de quais desses \glspl*{peer} passarão a ser
\emph{choked}.

\cfile[label="./libtransmission/peer-mgr.c:3057"]{./Codes/chap3/048-rechoke-up1.c}

A ordenação é feita com \sverb|qsort()| usando uma função comparadora que classifica os
\glspl*{peer} considerando, nesta ordem:

\begin{enumerate}
    \item velocidade de transferência de dados (download e upload);
    \item estado de \emph{choke}: preferência por \emph{unchoked}; e
    \item aleatoriedade.
\end{enumerate}

\cfile[label="./libtransmission/peer-mgr.c:2993"]{./Codes/chap3/049-compare-choke.c}

Então, trata o vetor temporário: os melhores \glspl*{peer} mudarão seu estado para
\emph{unchoked}, enquanto o restante permanecerá \emph{choked}. Além disso, no caso de
um rodada de \emph{unchoke} otimista, sorteará um dos \glspl*{peer} para ter a sua vez.

\cfile[label="./libtransmission/peer-mgr.c:3057"]{./Codes/chap3/050-rechoke-up2.c}

Depois, trata os downloads: primeiro, calculará para quantos \glspl*{peer} mandará
mensagens \bverb|interested|, usando como base as estatísticas de pedidos de blocos e
quais deles foram cancelados, dentro de limites de bom funcionamento.

\cfile[label="./libtransmission/peer-mgr.c:2833"]{./Codes/chap3/051-rechoke-down1.c}

Após saber o limite de \glspl*{peer}, o Transmission procurará quais deles contatará, de
acordo com as partes que lhe faltam. Para isso, montará um vetor com informações dos
\glspl*{peer}, ordenando-o pela função \sverb|qsort()|, e fornecerá uma função
comparadora que avaliará os critérios:

\begin{enumerate}
    \item estado de \emph{choke}: se um \gls*{peer} tiver cancelado 10\% (ou menos) dos
        pedidos de blocos, terá preferência sobre os outros; e
    \item aleatoriedade.
\end{enumerate}

\cfile[label="./libtransmission/peer-mgr.c:2813"]{./Codes/chap3/053-enum-rechoke.c}

\cfile[label="./libtransmission/peer-mgr.c:2820"]{./Codes/chap3/052-compare-rechoke.c}

\cfile[label="./libtransmission/peer-mgr.c:2833"]{./Codes/chap3/054-rechoke-down2.c}
