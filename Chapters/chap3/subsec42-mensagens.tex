%!TEX root = ../../tcc.tex

\subsection*{Mensagens}

O protocolo é definido por 12 mensagens e 2 tipos de assinaturas. Essas mensagens são
enviadas entre \glspl*{peer} e serve para estes tomarem conhecimento da situação de
download de um \gls*{torrent}. A primeira assinatura é exclusiva da mensagem de
handshake, enquanto todas as outras seguem o mesmo padrão.

\subsubsection*{handshake}

assinatura: \bverb|<tam. header><header><bytes reservados><info_hash><peer_id>|

O \emph{handshake} (aperto de mãos) é a primeira mensagem a ser enviada por um
\gls*{peer} que recém-chegado à rede.

\begin{itemize}
    \item \bverb|<tam. header>|: tamanho da string \bverb|<header>|,
        representado em binário por 1 byte. O comprimento oficial é 19.

    \item \bverb|<header>|: \gls*{string} identificadora do protocolo. Na versão 1.0 do
        protocolo BitTorrent, a \gls*{string} oficial é \sverb|BitTorrent protocol|.

    \item \bverb|<bytes reservados>|: seção de 8 bytes (= 64 bits) reservados para a
        habilitação de funcionalidades extras do protocolo. Um e-mail enviado pelo
        criador do BitTorrent Bram Cohen \cite{wikitheory:reserved-bytes} sugere que os
        bits menos significativos sejam usados primeiro, para que os mais significativos
        possam ser usados para alterar o significado dos bits finais. A implementação de
        cada uma das funcionalidades não-oficiais depende do programa cliente. A tabela
        abaixo mostra os bits e seus respectivos usos, oficiais (*) ou não-oficiais.

        \begin{center}
            \begin{tabular}{ | c | c |}
            \hline
            \textbf{Bit} & \textbf{Uso}                         \\ \hline
            1       & Azureus Extended Messaging                \\ \hline
            1-16    & BitComet Extension protocol               \\ \hline
            21      & BitTorrent Location-aware Protocol 1.0    \\ \hline
            44      & Extension protocol                        \\ \hline
            47-48   & Extension Negotiation Protocol            \\ \hline
            61      & NAT Traversal                             \\ \hline
            62      & Fast Peers*                               \\ \hline
            63      & XBT Peer Exchange                         \\ \hline
            64      & DHT* ou XBT Metadata Exchange             \\ \hline
            \end{tabular}
        \end{center}

    \item \bverb|<info_hash>|: o ID do \gls*{torrent}, que é a \gls*{string}
        \gls*{hashvalue} de 20 bytes resultante da \gls*{hashfunction} SHA-1, com
        \gls*{urlencode}, do valor da chave \bverb|info| do arquivo \gls*{torrentfile};

    \item \bverb|<peer_id>|: ID único do cliente, que é uma \gls*{string} de 20 bytes,
        geralmente sendo o mesmo valor \bverb|peer_id| enviado nas requisições ao
        \gls*{tracker}, prefixado pelas informações como o nome do programa cliente e a
        sua versão. Por exemplo, o Transmission envia o prefixo \sverb|-TR2820-...|
\end{itemize}

\cfile[label="./libtransmission/handshake.c:191"]{./Codes/chap3/028-handshake.c}

Esse mensagem é enviada imediatamente pelo \gls*{peer} que inicia uma conexão. O
receptor deve responder, assim que ver o ID do \gls*{torrent} na seção de
\bverb|info_hash| da mensagem, com o seu \bverb|peer_id|. A conexão deve ser fechada em
dois casos: pelo receptor, se ele receber a mensagem para um ID de \gls*{torrent}
desconhecido para si, ou pelo iniciador, caso o \bverb|peer_id| recebido como resposta
seja diferente daquele indicado na lista de \glspl*{peer} recebida do \gls*{dht}.

\subsubsection*{keep-alive}

keep-alive: \bverb|<tamanho=0000>|

A mensagem de \emph{keep-alive} (``mantenha vivo'' em português literal) serve para
manter uma conexão aberta caso nenhuma outra mensagem seja enviada num período de tempo
(geralmente, 2 minutos).

Assim como as outras mensagens, esta mensagem usa a assinatura

\bverb|<tamanho><ID da mensagem><dados>|

\begin{itemize}
    \item \bverb|<tamanho>|: valor de 4 bytes em \emph{big} \gls{endian}

    \item \bverb|<ID da mensagem>|: decimal de 1 byte

    \item \bverb|<dados>|: dados a serem enviados ao outro \gls*{peer}, dependente da
        mensagem
\end{itemize}

Porém, não possui ID da mensagem nem dados a serem enviados, possuindo tamanho 0.

\cfile[label="./libtransmission/peer-msgs.c:1093"]{./Codes/chap3/029-keepalive.c}

\subsubsection*{choke e unchoke}

\hspace*{-\parindent} % alinha a tabela à margem esquerda
\begin{tabular}{r l}
choke: & \bverb|<tamanho=0001><ID da mensagem=0>| \\
unchoke: & \bverb|<tamanho=0001><ID da mensagem=1>|
\end{tabular}

Estas mensagens servem para indicar que a mudança de estado de choking que o \gls*{peer}
remetente tratará o receptor. Ou seja, o receptor \emph{choked} não terá suas
requisições atendidas, ao contrário de quando estiver \emph{unchoked}.

\cfile[label="./libtransmission/peer-msgs.c:431"]{./Codes/chap3/030-choke-unchoke.c}

\subsubsection*{interested e not interested}

\hspace*{-\parindent} % alinha a tabela à margem esquerda
\begin{tabular}{r l}
interested: & \bverb|<tamanho=0001><ID da mensagem=2>| \\
not interested: & \bverb|<tamanho=0001><ID da mensagem=3>|
\end{tabular}

Assim como as mensagens de \emph{choke}, estas 2 mensagens também servem para indicar a
mudança de estado, desta vez sendo a mudança de interesse que o \gls*{peer} remetente
terá no receptor.

\cfile[label="./libtransmission/peer-msgs.c:769"]{./Codes/chap3/031-interest.c}

\subsubsection*{have}

have: \bverb|<tamanho=0005><ID da mensagem=4><dados=i-ésima parte>|

A definição é que esta mensagem avisa um \gls*{peer} que a $i$-ésima parte do
\gls*{torrent} foi baixada pelo remetente e verificada através de \gls*{hashvalue}.
Porém, não necessariamente é usada dessa forma.

\cfile[label="./libtransmission/peer-msgs.c:399"]{./Codes/chap3/032-have.c}

Uma implementação de algoritmo do jogo da troca pode fazer com que um \gls*{peer}, que
acabou de adquirir a parte, não emitir aviso para todos os \glspl*{peer} vizinhos que já
a possuem, diminuindo o \gls{overhead} de mensagens do protocolo. Por outro lado, enviar
esse aviso pode ajudar na determinação de qual parte é mais rara.

\subsubsection*{bitfield}

bitfield: \bverb|<tamanho=0001+X><ID da mensagem=5><dados=mapa de bits>|

O BitTorrent usa bitfields (mapas de bits) em \emph{big} \gls*{endian} para representar
na forma de \emph{flags} quais partes de um \gls*{torrent} já possui.

Esta mensagem só deve ser enviada imediatamente após o processo de \emph{handshake}
terminar e antes de qualquer outra mensagem. Se o \gls*{peer} não tiver nenhuma parte o
campo de dados pode ser omitido. Como bitfields variam de acordo com o tamanho total do
\gls*{torrent}, o comprimento da mensagem é variável, onde $X$ é o comprimento do
bitfield, acrescentado de alguns bits 0 de sobra no seu final (\emph{spare bits}).

Um bitfield de comprimento errado ou sem bits de sobra no final são considerados erros,
fazendo com que as respectivas conexões devam ser fechadas.

\cfile[label="./libtransmission/peer-msgs.c:2120"]{./Codes/chap3/033-bitfield.c}

\subsubsection*{request}

request: \bverb|<tamanho=0013><ID da mensagem=6><dados=<índice><início><tamanho>>|

Conforme foi explicado anteriormente (página \pageref{subsec:partes}), um
\gls*{torrent} divide os dados em partes, que por sua vez são divididas em blocos, que
são os conteúdos trocados entre os \glspl*{peer}. Assim, o tamanho das partes é um
múltiplo do tamanho dos blocos.

\begin{figure}[H]
    \centering
    \fbox{\includegraphics[width=0.64\textwidth]{partes.png}}
    \caption{trecho da seção de dados do torrent, com as divisões das partes e dos
    blocos}
    \label{fig:partes}
\end{figure}

Com esta mensagem, um \gls*{peer} pede por uma parte do \gls*{torrent}. A seção de dados
contém os seguintes números inteiros:

\begin{figure}[ht!]
    \centering
    \fbox{\includegraphics[width=0.64\textwidth]{request.png}}
    \caption{parâmetros da mensagem \emph{request} e seus significados}
    \label{fig:request}
\end{figure}

\begin{itemize}
    \item índice: índice da parte base $i$
    \item início: deslocamento, em bytes, da posição do bloco da parte $i$ pedido
    \item tamanho: tamanho, em bytes, do bloco pedido
\end{itemize}

\cfile[label="./libtransmission/peer-msgs.c:356"]{./Codes/chap3/037-request.c}

\subsubsection*{piece}

piece: \bverb|<tamanho=0009+X><ID da mensagem=7><dados=<índice><início><bloco>>|

Esta mensagem é a resposta para a requisição de um bloco por meio da mensagem
\emph{request}, tendo assinatura análoga com exceção do trecho com os dados do bloco.
Assim, um \gls*{peer} envia um bloco de dados a outro.

\newpage
O tamanho da mensagem é variável, onde $X$ é o tamanho do segmento de dados, já que
este pode ter tamanhos diferentes entre mensagens diferentes. A seção de dados possui
os seguintes números inteiros:

\begin{itemize}
    \item índice: índice da parte base $i$
    \item início: deslocamento, em bytes, da posição do bloco da parte $i$ pedido
    \item bloco: conteúdo dos dados do bloco
\end{itemize}

\cfile[label="./libtransmission/peer-msgs.c:1093"]{./Codes/chap3/034-piece.c}

\newpage
\subsubsection*{cancel}

cancel: \bverb|<tamanho=0013><ID da mensagem=8><dados=<índice><início><tamanho>>|

A mensagem \emph{cancel} serve para cancelar a mensagem \emph{request} de mesmos
parâmetros. É mais utilizada durante o algoritmo de fim de jogo.

\todo[inline]{referenciar seção do algoritmo do fim de jogo}
\cfile[label="./libtransmission/peer-msgs.c:372"]{./Codes/chap3/035-cancel.c}

\subsubsection*{port}

cancel: \bverb|<tamanho=0003><ID da mensagem=9><dados=porta>|

Para os casos do \gls*{peer} estar usando a função de \gls*{dht}, esta mensagem serve
para avisar ao receptor qual a porta de comunicação \gls*{tcp} que o remetente está
usando para receber mensagens de \gls*{dht}; Com isso, é adicionado à tabela de
roteamento do receptor.

\cfile[label="./libtransmission/peer-msgs.c:387"]{./Codes/chap3/036-port.c}
