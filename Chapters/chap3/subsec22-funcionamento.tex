%!TEX root = ../../tcc.tex

\subsection*{Funcionamento}

No Kademlia, objetos e nós possuem IDs únicos de 160 bits: enquanto o primeiro utiliza
o \gls*{hashvalue} de 20 bytes SHA-1 da chave \bverb|info_hash| do \gls*{torrentfile},
o segundo é um valor aleatório escolhido pelo próprio programa.

Como este \gls*{dht} é construído com base em distâncias entre nós, a função de medida
escolhida foi

\begin{equation}
    d(x,y) = x \oplus y
\end{equation}

pois possui algumas propriedades em comum com a equação de distância euclidiana usual.
Assim, seguem suas propriedades:

\begin{itemize}
    \item $d(x,x) = 0$
    \item $x \neq y$, $d(x,y) > 0$
    \item simetria: $\forall x,y$, $d(x,y) = d(y,x)$
    \item desigualdade triangular: $d(x,y) + d(y,z) \geq d(x,z)$. \\
        Isto vem do fato de $d(x,z) = d(x,y) \oplus d(y,z)$ e que $\forall a \geq 0,
        \forall b \geq 0 : a + b \geq a \oplus b$
    \item unidirecionalidade: para um dado ponto $x$ e uma distância $\Delta > 0$,
        existe exatamente um ponto $y$ tal que $d(x,y) = \Delta$. Isso garante que todas
        as procuras por uma mesma chave convirjam para um mesmo percurso, independente
        do ponto de partida.
\end{itemize}

%!TEX root = ../../tcc.tex

\subsubsection*{Estado dos nós}

Cada nó do Kademlia armazena informações sobre outros nós para rotear mensagens de
pesquisa. Para cada bit $i$ dos IDs (cada ID tem 160 bits) é mantido um \gls{kbucket},
que contém os nós cuja distância até ele está entre $2^i$ e $2^{i+1}$. Esses
\glspl*{kbucket} são listas de endereço IP, porta de comunicação \gls*{udp} e ID de
nós, ordenadas pelo horário da última notícia destes. Para distâncias pequenas, essas
listas geralmente serão vazias, enquanto para distâncias maiores poderão ser de tamanho
$k$. Este valor, que é o de replicação do sistema, é escolhido de tal forma que esses
$k$ nós possuam grande probabilidade de não falharem na próxima hora.

Quando um nó \textbf{A} recebe uma mensagem de outro nó \textbf{B}, o \gls*{bucket}
correspondente ao ID do remetente (nó \textbf{B}) é atualizado. Disto, podem ocorrer as
seguintes situações:

\newpage
\begin{itemize}
    \item \textbf{B} já existe no \gls*{bucket}: passa a ser o primeiro da lista, pois
        existiu mensagem recente.

    \item \textbf{B} não existe no \gls*{bucket}:
        \begin{itemize}
            \item \gls*{bucket} não está cheio: \textbf{B} é adicionado no começo da
                lista.
            \item \gls*{bucket} cheio: é enviado um \emph{ping} para o nó do final da
                lista (nó \textbf{C}), contatado há mais tempo

            \begin{itemize}
                \item \textbf{C} não responde ao \emph{ping}: \textbf{C} é retirado da
                lista e \textbf{B} é inserido no início
                \item \textbf{C} responde ao \emph{ping}: \textbf{C} é movido para o
                início da lista e \textbf{B} é ignorado
            \end{itemize}
        \end{itemize}

\end{itemize}

Por conta disso, ocorre que nós mais antigos e funcionais são preferidos, pois quanto
mais tempo um nó está conectado, mais provável ele se manterá conectado por mais 1 hora
\cite{artigo:gnutella-uptime}. Outra vantagem disso é a resistência a alguns ataques de
negação de serviço, pois mesmo que ocorra uma inundação de novos nós, estes só seriam
inseridos nos \glspl*{kbucket} se os antigos fossem excluídos.

\input{Chapters/chap3/subsubsec222-protocolo}

%!TEX root = ../../tcc.tex

\subsubsection*{Comandos}

Os 4 comandos de \emph{query} do \gls*{dht} (\bverb|ping|, \bverb|find_node|,
\bverb|get_peers| e \bverb|announce_peer|) estão definidos da seguinte forma.

%!TEX root = ../../tcc.tex

% \newpage
\subsubsubsection{ping}

É o comando mais simples, que verifica se o nó está online. Possui um único argumento,
que é uma chave \bverb|id|, que é o ID do nó consultante (na requisição) ou do nó
consultado (na resposta).

\begin{itemize}
    \item requisição \\
        \bverb|d1:ad2:id20:abcdefghij0123456789e1:q4:ping1:t2:aa1:y1:qe|
        (\gls*{bencode}) \\
        \sverb|{"t":"aa", "y":"q", "q":"ping", "a":{"id":"abcdefghij0123456789"}}|
        (\gls*{string})

    \item resposta \\
        \bverb|d1:rd2:id20:mnopqrstuvwxyz123456e1:t2:aa1:y1:re|
        (\gls*{bencode}) \\
        \sverb|{"t":"aa", "y":"r", "r": {"id":"mnopqrstuvwxyz123456"}}|
        (\gls*{string})
\end{itemize}

\cfile[label="./third-party/dht/dht.c:2291"]{./Codes/chap3/016-dht-macros.c}
\cfile[label="./third-party/dht/dht.c:2341"]{./Codes/chap3/017-dht-ping.c}

%!TEX root = ../../tcc.tex

\newpage
\subsubsubsection{find\_node}

Este comando, que equivale à mensagem de \bverb|FIND\_NODE| do artigo do Kademlia
\cite{artigo:kademlia}, é usado para encontrar as informações do nó dado seu ID.
Necessita enviar 2 argumentos: a chave \bverb|id| e o ID do nó consultante, e a chave
\bverb|target| e o ID do nó cujas informações o consultante está procurando (ou nó
alvo).

\begin{itemize}
    \item formato dos argumentos da requisição \\
        \sverb|{"id":"<IDs dos nós consultantes>", "target":"<ID do nó alvo>"}|

    \item formato da resposta \\
        \sverb|{"id":"<IDs dos nós consultados>", "nodes":"<info compacta do(s) nó(s)>"}|
\end{itemize}

O nó consultado deve responder com a chave \bverb|nodes| contendo uma
\gls*{string} com a informação compacta (6 bytes) do nó alvo ou dos $k$ nós bons (que
fizeram contato recentemente) mais próximos que estão contidos em sua tabela de
roteamento, de 1 ou mais \glspl*{kbucket}. O funcionamento do algoritmo da busca é
explicado no trabalho:

\blockquote{O procedimento mais importante que um participante do Kademlia deve realizar
é encontrar os $k$ nós próximos a um dado ID de nó. Nós chamamos esse procedimento de
\emph{busca por nós}. Kademlia utiliza de um algoritmo recursivo nas buscas por nós. O
disparador das buscas começa escolhendo $\alpha$ nós do \gls*{bucket} não-vazio mais
próximo (ou, se esse \gls*{bucket} tiver menos que $\alpha$ entradas, utiliza desses
$\alpha$ nós mais próximos que conhece). Então, o disparador envia chamadas \gls*{rpc}
assíncronas paralelas de comandos \textbf{find\_node} para esses $\alpha$ nós
escolhidos. $\alpha$ é um parâmetro de concorrência geral ao sistema, assumindo valor
como 3.

No passo recursivo, o disparador reenvia chamadas a \textbf{find\_node} para os nós que
conheceu das chamadas \gls*{rpc} passadas. (Esta recursão pode começar antes que todos
os $\alpha$ nós anteriores tenham respondido). Dos $k$ nós que o disparador concluiu
serem mais próximos ao alvo, ele pega $\alpha$ que ainda não foram consultados e envia
chamadas \gls*{rpc} \textbf{find\_node}. Nós que falharem em responder rapidamente são
desconsiderados até que respondam. Se uma rodada de comandos \textbf{find\_node} não
retornar algum nó mais próximo do que os nós já conhecidos, o disparador reenvia
comandos \textbf{find\_node} para todos os $k$ nós mais próximos que ainda não foram
consultados. A busca termina quando o disparador tiver consultado e obtido respostas de
todos os $k$ nós mais próximos conhecidos.}

Porém, o Transmission implementa essa busca de forma mais flexível e simples. De início,
busca o \gls*{bucket} no qual o ID procurado está ou que contém nós mais próximos.

\cfile[label="./third-party/dht/dht.c:2536"]{./Codes/chap3/020-dht-findclosestnodes.c}

A busca do \gls*{bucket} itera sobre a lista ligada de \glspl*{bucket}.

\cfile[label="./third-party/dht/dht.c:129"]{./Codes/chap3/018-dht-structs.c}
\cfile[label="./third-party/dht/dht.c:464"]{./Codes/chap3/019-dht-findbucket.c}

Caso retorne o \gls*{bucket} mais provável, efetua buscas internas nele. Se ele possuir
elementos vizinhos anteriores ou posteriores, também busca por nós neles.

\cfile[label="./third-party/dht/dht.c:2523"]{./Codes/chap3/021-dht-bufferclosestnodes.c}
\cfile[label="./third-party/dht/dht.c:2476"]{./Codes/chap3/022-dht-insertclosestnode.c}

Ao fim da busca, envia a lista de nós que encontrou como resposta à mensagem de
\bverb|find_node| recebida. Um exemplo de requisição e resposta para este comando é

\begin{itemize}
    \item requisição \\
        \bverb|d1:ad2:id20:abcdefghij01234567896:target20:mnopqrstuvwxyz123456e1:q| \\
        \bverb|9:find_node1:t2:aa1:y1:qe| (\gls*{bencode}) \\
        \sverb|{"t":"aa", "y":"q", "q":"find_node", "a":{"id":"abcdefghij0123456789",| \\
        \sverb|"target":"mnopqrstuvwxyz123456"}}| (\gls*{string})

    \item resposta \\
        \bverb|d1:rd2:id20:0123456789abcdefghij5:nodes9:def456...e1:t2:aa1:y1:re| \\
        (\gls*{bencode}) \\
        \sverb|{"t":"aa", "y":"r", "r":{"id":"0123456789abcdefghij", "nodes":| \\
        \sverb|"def456..."}}| (\gls*{string})
\end{itemize}

%!TEX root = ../../tcc.tex

\subsubsubsection{get\_peers}

É o comando \gls*{rpc} da mensagem \bverb|FIND\_VALUE|, serve para buscar \glspl*{peer}
para um dado um \gls*{hashvalue} identificador de \gls*{torrentfile}, enviado como
valor da chave \bverb|info\_hash|, além do recorrente ID do nó consultante como valor
da chave \bverb|id|.

O funcionamento é equivalente ao comando \bverb|find\_node|, com um detalhe extra: se o
nó que recebeu a mensagem possuir \glspl*{peer} para o \gls*{hashvalue} dado, eles são
informados na forma compacta (6 bytes para cada \gls*{peer}) numa lista \gls*{bencode}
de \glspl*{string}, devolvida como valor da chave \bverb|values|. Por outro lado, caso o
receptor da mensagem não conhecer nós para o \gls*{hashvalue} especificado, a resposta
conterá a chave \bverb|nodes| com os $k$ nós mais próximos desse \gls*{hashvalue}. Em
ambos os casos, uma chave \bverb|token| é informada na resposta, cujo valor é uma
\gls*{string} binária curta, que deverá ser utilizada em futuras mensagens de
\bverb|announce\_peer|.

\begin{comment}
Get peers associated with a torrent infohash. "q" = "get_peers" A get_peers query has two arguments, "id" containing the node ID of the querying node, and "info_hash" containing the infohash of the torrent. If the queried node has peers for the infohash, they are returned in a key "values" as a list of strings. Each string containing "compact" format peer information for a single peer. If the queried node has no peers for the infohash, a key "nodes" is returned containing the K nodes in the queried nodes routing table closest to the infohash supplied in the query. In either case a "token" key is also included in the return value. The token value is a required argument for a future announce_peer query. The token value should be a short binary string.

arguments:  {"id" : "<querying nodes id>", "info_hash" : "<20-byte infohash of target torrent>"}

response: {"id" : "<queried nodes id>", "token" :"<opaque write token>", "values" : ["<peer 1 info string>", "<peer 2 info string>"]}

or: {"id" : "<queried nodes id>", "token" :"<opaque write token>", "nodes" : "<compact node info>"}
Example Packets:

get_peers Query = {"t":"aa", "y":"q", "q":"get_peers", "a": {"id":"abcdefghij0123456789", "info_hash":"mnopqrstuvwxyz123456"}}
bencoded = d1:ad2:id20:abcdefghij01234567899:info_hash20:mnopqrstuvwxyz123456e1:q9:get_peers1:t2:aa1:y1:qe
Response with peers = {"t":"aa", "y":"r", "r": {"id":"abcdefghij0123456789", "token":"aoeusnth", "values": ["axje.u", "idhtnm"]}}
bencoded = d1:rd2:id20:abcdefghij01234567895:token8:aoeusnth6:valuesl6:axje.u6:idhtnmee1:t2:aa1:y1:re
Response with closest nodes = {"t":"aa", "y":"r", "r": {"id":"abcdefghij0123456789", "token":"aoeusnth", "nodes": "def456..."}}
bencoded = d1:rd2:id20:abcdefghij01234567895:nodes9:def456...5:token8:aoeusnthe1:t2:aa1:y1:re

When a node wants to find peers for a torrent, it uses the distance metric to compare
the infohash of the torrent with the IDs of the nodes in its own routing table. It then
contacts the nodes it knows about with IDs closest to the infohash and asks them for the
contact information of peers currently downloading the torrent. If a contacted node
knows about peers for the torrent, the peer contact information is returned with the
response. Otherwise, the contacted node must respond with the contact information of
the nodes in its routing table that are closest to the infohash of the torrent. The
original node iteratively queries nodes that are closer to the target infohash until it
cannot find any closer nodes. After the search is exhausted, the client then inserts
the peer contact information for itself onto the responding nodes with IDs closest to
the infohash of the torrent.

The return value for a query for peers includes an opaque value known as the "token."
For a node to announce that its controlling peer is downloading a torrent, it must
present the token received from the same queried node in a recent query for peers. When
a node attempts to "announce" a torrent, the queried node checks the token against the
querying node's IP address. This is to prevent malicious hosts from signing up other
hosts for torrents. Since the token is merely returned by the querying node to the same
node it received the token from, the implementation is not defined. Tokens must be
accepted for a reasonable amount of time after they have been distributed. The
BitTorrent implementation uses the SHA1 hash of the IP address concatenated onto a
secret that changes every five minutes and tokens up to ten minutes old are accepted.
\end{comment}

\input{Chapters/chap3/subsubsubsec2234-announce-peer}