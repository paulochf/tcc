%!TEX root = ../../tcc.tex

\newpage
\subsubsubsection{find\_node}

Este comando, que equivale à mensagem de \bverb|FIND\_NODE| do artigo do Kademlia
\cite{artigo:kademlia}, é usado para encontrar as informações do nó dado seu ID.
Necessita enviar 2 argumentos: a chave \bverb|id| e o ID do nó consultante, e a chave
\bverb|target| e o ID do nó cujas informações o consultante está procurando (ou nó
alvo).

\begin{itemize}
    \item formato dos argumentos da requisição \\
        \sverb|{"id":\"<IDs dos nós consultantes>", "target":\"<ID do nó alvo>"}|

    \item formato da resposta \\
        \sverb|{"id":\"<IDs dos nós consultados>", "nodes":\"<info compacta do(s) nó(s)>"}|
\end{itemize}

O nó consultado deve responder com a chave \bverb|nodes| contendo uma
\gls*{string} com a informação compacta (6 bytes) do nó alvo ou dos $k$ nós bons (que
fizeram contato recentemente) mais próximos que estão contidos em sua tabela de
roteamento, de 1 ou mais \glspl*{kbucket}. O funcionamento do algoritmo da busca é
explicado no trabalho:

\blockquote{O procedimento mais importante que um participante do Kademlia deve realizar
é encontrar os $k$ nós próximos a um dado ID de nó. Nós chamamos esse procedimento de
\enquote{\emph{lookup} de nós}. Kademlia utiliza de um algoritmo recursivo nas buscas
por nós. O disparador das buscas começa escolhendo $\alpha$ nós do \gls*{bucket}
não-vazio mais próximo (ou, se esse \gls*{bucket} tiver menos que $\alpha$ entradas,
utiliza desses $\alpha$ nós mais próximos que conhece). Então, o disparador envia
chamadas \gls*{rpc} assíncronas paralelas de comandos \textbf{find\_node} para esses
$\alpha$ nós escolhidos. $\alpha$ é um parâmetro de concorrência geral ao sistema,
assumindo valor como 3.

No passo recursivo, o disparador reenvia chamadas a \textbf{find\_node} para os nós que
conheceu das chamadas \gls*{rpc} passadas. (Esta recursão pode começar antes que todos
os $\alpha$ nós anteriores tenham respondido). Dos $k$ nós que o disparador concluiu
serem mais próximos ao alvo, ele pega $\alpha$ que ainda não foram consultados e envia
chamadas \gls*{rpc} \textbf{find\_node}. Nós que falharem em responder rapidamente são
desconsiderados até que respondam. Se uma rodada de comandos \textbf{find\_node} não
retornar algum nó mais próximo do que os nós já conhecidos, o disparador reenvia
comandos \textbf{find\_node} para todos os $k$ nós mais próximos que ainda não foram
consultados. O \emph{lookup} termina quando o disparador tiver consultado e obtido
respostas de todos os $k$ nós mais próximos conhecidos.}

Porém, o Transmission implementa essa busca de forma mais flexível e simples. De início,
busca o \gls*{bucket} no qual o ID procurado está ou que contém nós mais próximos.

\cfile[label="./third-party/dht/dht.c:2536"]{./Codes/chap3/020-dht-sendclosestnodes.c}

A busca do \gls*{bucket} itera sobre a lista ligada de \glspl*{bucket}.

\cfile[label="./third-party/dht/dht.c:464"]{./Codes/chap3/019-dht-findbucket.c}

Caso retorne o \gls*{bucket} mais provável, efetua buscas internas nele. Se ele possuir
elementos vizinhos anteriores ou posteriores, também busca por nós neles.

\cfile[label="./third-party/dht/dht.c:2523"]{./Codes/chap3/021-dht-bufferclosestnodes.c}
\cfile[label="./third-party/dht/dht.c:2476"]{./Codes/chap3/022-dht-insertclosestnode.c}

Ao fim da busca, envia a lista de nós que encontrou como resposta ao comando de
\bverb|find_node| recebida. O Transmission também utiliza a função para enviar os nós
encontrados pelo comando \bverb|get_peers| (pág. ~\pageref{subsubsubsec:getpeers}).

\cfile[label="./third-party/dht/dht.c:2409"]{./Codes/chap3/023-dht-sendnodespeers.c}

Um exemplo de requisição e resposta para este comando é

\begin{itemize}
    \item exemplo de requisição \\
        \bverb|d1:ad2:id20:abcdefghij01234567896:target20:mnopqrstuvwxyz123456e1:q| \\
        \bverb|9:find_node1:t2:aa1:y1:qe| (\gls*{bencode}) \\
        \sverb|{"t":"aa", "y":"q", "q":"find_node", "a":{"id":"abcdefghij0123456789",| \\
        \sverb|"target":"mnopqrstuvwxyz123456"}}| (\gls*{string})

    \item exemplo de resposta \\
        \bverb|d1:rd2:id20:0123456789abcdefghij5:nodes9:def456...e1:t2:aa1:y1:re| \\
        (\gls*{bencode}) \\
        \sverb|{"t":"aa", "y":"r", "r":{"id":"0123456789abcdefghij", "nodes":| \\
        \sverb|"def456..."}}| (\gls*{string})
\end{itemize}