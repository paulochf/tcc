%!TEX root = ../../tcc.tex

\subsection*{Kademlia}

O Kademlia é um \gls*{dht} criado em 2002 \cite{artigo:kademlia} com o objetivo de
melhorar os métodos de busca atuais (Napster e\gls*{gnutella}), que eram ineficientes.
Assim como os outros algoritmos de \gls*{dht}, ele se baseou na estrutura informalmente
conhecida como \enquote{rede de Plaxton} (\emph{Plaxton mesh}), nome que remete a um
dos seus autores \cite{artigo:dht}. Por ter causado boas impressões, foi usado na
implementação da busca de arquivos no programa cliente eMule.

O algoritmo implementa uma rede \emph{overlay} cuja estrutura e comunicação se baseiam
na procura de seus nós. Cada um destes nós é identificado por um identificador único
(ID), que serve tanto para a identificação quanto para a localização de valores na
\gls*{hashtable}. Durante uma busca, o processo deve conhecer a chave (que é um
\gls*{hashvalue}) associado ao objeto - neste caso, o ID do \gls*{torrent}, que é seu
\gls*{hashvalue} - e explora a rede em passos, encontrando nós mais próximos da chave,
até encontrar o valor buscado ou não nós existirem mais próximos que o atual. Dessa
forma, para uma rede com $n$ nós, o algoritmo visita apenas $O(\log n)$ nós.