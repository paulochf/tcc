%!TEX root = ../../tcc.tex

\newpage
\subsection*{Algoritmos de seleção de partes}

Por se tratar de um protocolo de troca de dados em partes, uma escolha ruim de quais
destas se adquirir primeiro faz com que seja grande a possibilidade de um \gls*{peer}
baixar alguma parte que seja sempre ofertada por outros \glspl*{peer}. Isso acarretará
em não se ter nenhuma das partes que se deseja. Assim, durante a troca das partes, um
\gls*{peer} adota estratégias diferentes para fornecer e receber blocos de dados, com o
intuito de tentar otimizar a obtenção do conjunto total de dados dos \glspl*{torrent} e
ajudando a difundir o conteúdo deste ao resto do \gls*{swarm}.

As estratégias a seguir são medidas que um \gls*{peer} BitTorrent pode adotar com
efeitos em si mesmo, sem afetar outros \glspl*{peer}.

\subsubsection*{Random First Piece}

No início do download de um \gls*{torrent}, um \gls*{peer} não possui partes. Para que
comece a receber partes, ele avisa que é recém-chegado, e assim algum membro do
\gls*{swarm} envia-no uma parte comum aleatoriamente. Dessa forma, ele possuirá uma
``moeda de troca'', podendo então ajudar outros \glspl*{peer} e, assim, conseguirá
melhores condições de ser atendido. É importante que a parte seja comum, pois assim
será possível conseguir blocos de locais diferentes; se fosse rara, seria mais difícil
conseguir completá-la

Após completar a primeira parte, o algoritmo passa para a estratégia de \emph{Rarest
First}.

\subsubsection*{Rarest First}

Nesta fase, o \gls*{peer} passa a pedir as partes mais raras antes. Para isso, utiliza
os bitfields recebidos dos outros \glspl*{peer}, mantendo-os atualizados a cada mensagem
\bverb|have| que recebe. Feito isso, das partes que são menos frequentes nos bifields,
escolhe uma aleatoriamente, devido ao fato de que uma parte rara poderia ser muito
requisitada, o que seria improdutivo. Da mesma forma, a estratégia de deixar partes mais
comuns para serem baixadas mais tarde não é prejudicial, pois a probabilidade de que um
\gls*{peer} que está disponibilizando essas partes num dado momento deixará de ser
interessante é reduzida.

É fácil ver que, enquanto um \gls*{seeder} não enviar todas as partes do \gls*{torrent}
que está fornecendo, não haverá nenhum \gls*{peer} que possa ter terminado de baixá-lo.
Assim, quando o \gls*{seeder} possuir capacidade de upload menor do que de seus
\glspl*{leecher}, a melhor situação será se cada um destes baixar partes diferentes dos
outros, que maximiza o espalhamento dos dados e alivia a carga sobre o \gls*{seeder},
justificando a prioridade em se fazer download das partes raras.

Em uma outra situação, o \gls*{seeder} pode sair da rede, tornando os \glspl*{leecher}
responsáveis pela distribuição do \gls*{torrent}. Assim, corre-se o risco de alguma
parte se tornar indisponível. O \emph{rarest first} também ajuda neste caso, pois
replica as partes raras o mais rápido possível, reduzindo o risco do \gls*{torrent} se
tornar incompleto.

\subsubsection*{Strict Priority}

Cada parte é composta de blocos, que são os pacotes de dados trocados entre
\glspl*{peer}. Esta política de prioridade faz com que, se um bloco for requisitado, o
restante dos blocos da mesma parte serão pedidos antes dos blocos de outras partes, a
fim de se ter partes completas o mais rápido possível.

\subsubsection*{Endgame Mode}

Próximo ao fim do download de um \gls*{torrent}, a tendência é que os últimos blocos de
dados demorem, chegando aos poucos. Para agilizar isso, o cliente pede todos os blocos
faltantes para todos os \glspl*{peer}, enviando uma mensagem \bverb|cancel| para todos
que não tiverem respondido à requisição assim que o bloco é recebido para evitar
despedício de banda de rede na recepção redundante de dados.

\subsubsection*{Implementação do Transmission}

O Transmission implementa as partes desejadas e os blocos do \gls*{torrent} a serem
baixados como vetores. O vetor de partes é ordenado, utilizando seus campos auxiliares,
de forma que a parte mais importante que se deseja receber será requisitada antes. Ele
é usado para decidir quais blocos serão requisitados.

\cfile[label="./libtransmission/peer-mgr.c:167"]{./Codes/chap3/038-struct-weighted-piece.c}

Conforme são recebidas mensagens \bverb|have| e \bverb|bitfield| de outros \glspl*{peer},
ocorre o processamente da carga de dados dessas mensagens e a atualização das
informações das partes de cada remetente. Assim, o Transmission estima quais partes do
\gls*{torrent} são mais raras que outras, guardando essa informação no vetor de
replicações.

\cfile[label="./libtransmission/peer-mgr.c:1699"]{./Codes/chap3/041-peer-callback.c}

Já o vetor de blocos mantém a lista dos blocos pedidos com o momento da requisição e
para quem o foi feito. É usada para cancelar requisições que ficaram pendentes por muito
tempo ou para evitar requisições duplicadas antes do modo de fim de jogo.

\cfile[label="./libtransmission/peer-mgr.c:161"]{./Codes/chap3/039-struct-block-request.c}

Ambos os vetores são armazenados em uma estrutura que guarda essas e outras informações
sobre o \gls*{swarm} do \gls*{torrent} que está sendo baixado.

\cfile[label="./libtransmission/peer-mgr.c:178"]{./Codes/chap3/040-struct-swarm.c}

Eventualmente, o Transmission verifica a necessidade e a capacidade de enviar pedidos
de blocos. Caso seja possível, realiza o processamento da lista de partes, montando uma
lista de blocos a serem requisitados.

\newpage
\cfile[label="./libtransmission/peer-mgr.c:1324"]{./Codes/chap3/042-peer-mgr-getreqs.c}

\newpage
Tendo criado a lista de blocos, os requisita.

\cfile[label="./libtransmission/peer-msgs.c:1874"]{./Codes/chap3/043-update-block-reqs.c}