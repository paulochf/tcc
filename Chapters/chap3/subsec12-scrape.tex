%!TEX root = ../../tcc.tex

\subsection*{Scrape}

Além do \gls*{announce}, outra forma de troca de informação entre \glspl*{peer} e
\glspl*{tracker} se dá pelo \gls{scrape}. Geralmente usado pelos programas cliente para
decidir quando realizar um \gls*{announce}, informa o número de \glspl*{peer},
\glspl*{leecher} e \glspl*{seeder} de uma lista de um ou mais \glspl*{torrent}. É dessa
forma que os sites de indexação sabem dessas informações e as apresentam nas em páginas.

A requisição de \gls*{scrape} pode ser sobre todos os \glspl*{torrent} que o
\gls*{tracker} gerencia ou sobre um ou mais \glspl*{torrent} em específico, quando são
passados seus respectivos \glspl*{hashvalue}. Sua resposta é um dicionário na forma
\gls*{bencode} contendo as chaves

\begin{itemize}
    \item \textbf{files}: um dicionário contendo um par chave-valor para cada
    \gls*{torrent} especificado na requisição do \gls*{scrape} através do
    \gls*{hashvalue} do \gls*{torrent} de 20 bytes:

    \begin{itemize}
        \item \textbf{complete}: quantidade de \glspl*{seeder};

        \item \textbf{incomplete}: quantidade de \glspl*{leecher};

        \item \textbf{downloaded}: quantidade total de vezes que o \gls*{tracker}
        registrou uma finalização (\sverb|event=complete| ao término de um download);

        \item \textbf{name} (opcional): o nome interno do \gls*{torrent}, como
        especificado pelo respectivo \gls*{torrentfile}, na sua seção \bverb|info|;
    \end{itemize}
\end{itemize}

Da mesma forma que o \gls*{announce}, o \gls*{scrape} é um endereço do \gls*{tracker}
usando-se o \gls*{tcp} ou o \gls*{udp}, e é tratado de forma semelhante pelo
Transmission.

\cfile[label="./libtransmission/announcer-common.h:37"]{./Codes/chap3/007-scrapestruct.c}
\cfile[label="./libtransmission/announcer.c:1397"]{./Codes/chap3/008-scrape.c}

\begin{listing}[H]
    \begin{minted}[
        linenos,
        frame=single,
        numbersep=6pt,
        baselinestretch=1,
        fontfamily=courier,
        gobble=4,
        fontsize=\scriptsize
    ]{text}
    * About to connect() to www.mvgroup.org port 2710 (#1)
    *   Trying 88.129.153.50...
    * Connected to www.mvgroup.org (88.129.153.50) port 2710 (#1)
    > GET /scrape?info_hash=\%83F\%24\%b62\%e5Q\%cc4\%10h\%ba\%1e8\%e2C\%f7\%80\%01\%87 HTTP/1.1
    User-Agent: Transmission/2.82
    Host: www.mvgroup.org:2710
    Accept: */*
    Accept-Encoding: gzip;q=1.0, deflate, identity

    * HTTP 1.0, assume close after body
    < HTTP/1.0 200 OK
    <
    * Closing connection 1
    Scrape response:
    < {
        "files": {
            \"<binary>\": {
                "complete": 9,
                "downloaded": 59074,
                "incomplete": 2
            }
        }
    }
    \end{minted}

    \caption{Logs do Transmission sobre uma requisição de scrape e a respectiva
    resposta, com o conteúdo binário truncado}
    \label{lst:scrape}
\end{listing}