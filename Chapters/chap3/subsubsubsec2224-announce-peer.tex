%!TEX root = ../../tcc.tex

\subsubsubsection{announce\_peer}

O último comando é o da mensagem \bverb|STORE|, pelo qual um nó avisa outros que está
baixando um \gls*{torrentfile}, passando 4 argumentos: o ID do nó consultante como
valor da chave \bverb|id|, o \gls*{hashvalue} identificador do \gls*{torrent} na
chave \bverb|info_hash|, \bverb|port| contém um número inteiro de porta, e o
\bverb|token| recebido como resposta de uma mensagem \bverb|get_peers| anterior. O nó
consultado deve verificar que esse token foi enviado anteriormente para o mesmo endereço
IP que o nó consultante, e então o nó consultado armazena usando o \bverb|info_hash| do
torrent como chave e a informação compacta de endereço IP/porta do nó como valor.

Existe ainda mais um argumento, opcional, que é o \bverb|implied_port|, cujo valor pode
ser 0 ou 1. Se este for 1, o argumento da porta deve ser ignorado e então a fonte de
pacotes \gls*{udp} deve ser usada como a porta do \gls*{peer}. Isso é útil para
\glspl*{peer} que estão em redes internas a \gls{nat}, que podem não saber quais são
suas portas externas, e que suportam uTP, aceitando conexões na mesma porta que o
\gls*{dht}.

O token tem papel fundamental para a segurança neste comando, pois serve para prevenir
que um \gls*{peer} malicioso registre outros \glspl*{peer} para um \gls*{torrent}. No
BitTorrent, esse token é a \gls*{string} do \gls*{hashvalue} SHA-1 do endereço IP
concatenado a uma chave secreta, criada pelo programa cliente, que varia a cada 5
minutos e são aceitos até 10 minutos depois de serem criados.

Para armazenar um valor sob uma chave, um nó busca os $k$ nós mais próximos a ela
(usando \emph{lookup} de nós) e envia o comando \bverb|announce_peer|.

\newpage
\begin{itemize}
    \item formato dos argumentos da requisição \\
        \sverb|{\"id": \"<IDs dos nós consultantes>",| \\
        \sverb| \"implied_port": <0 ou 1>,| \\
        \sverb| \"info_hash": \"<hash de 20 bytes do torrent>",| \\
        \sverb| \"port": <número da porta>,| \\
        \sverb| \"token": \"<token>"}|

    \item formato da resposta \\
        \sverb|{"id": \"<IDs dos nós consultados>"}|
\end{itemize}

\cfile[label="./third-party/dht/dht.c:1277"]{./Codes/chap3/025-storage-store.c}

\begin{itemize}
    \item exemplo de requisição \\
        \bverb|d1:ad2:id20:abcdefghij01234567899:info_hash20:mnopqrstuvwxyz123456| \\
        \bverb|4:porti6881e5:token8:aoeusnthe1:q13:announce_peer1:t2:aa1:y1:qe| \\
        (\gls*{bencode}) \\
        \sverb|{"t":"aa", \"y":"q", \"q":"announce_peer",| \\
        \sverb|"a":{"id":"abcdefghij0123456789", \"implied_port": 1,| \\
        \sverb|"info_hash":"mnopqrstuvwxyz123456", \"port": 6881, \"token": \"aoeusnth"}}|
        (\gls*{string})

    \item exemplo de resposta \\
        \bverb|d1:rd2:id20:mnopqrstuvwxyz123456e1:t2:aa1:y1:re| \\
        (\gls*{bencode}) \\
        \sverb|{"t":"aa", \"y":"r", \"r": {"id":"mnopqrstuvwxyz123456"}}|
        (\gls*{string})
\end{itemize}