%!TEX root = ../../tcc.tex

\subsection*{Kademlia}

O \gls{kademlia} é uma \gls*{dht}, criada em 2002 \cite{artigo:kademlia}, com o objetivo
de melhorar os métodos de busca atuais (Napster e \gls*{gnutella}), que eram
ineficientes. Assim como os outros algoritmos de \gls*{dht}, ele se baseou na estrutura
informalmente conhecida como \enquote{rede de Plaxton} (\emph{Plaxton mesh}), nome que
remete a um dos seus autores \cite{artigo:dht}. Por ter mostrado bons resultados, foi
usado na implementação da busca de arquivos no programa cliente eMule.

A sua modelagem computacional monta um mapa no formato de \gls*{hashtable} onde IDs de
\glspl*{peer} ou de \glspl*{torrent} são chaves para listas de outros \glspl*{peer}.

%!TEX root = ../../tcc.tex

\subsubsection*{Modelagem e estrutura de dados}

O algoritmo implementa uma rede \emph{overlay}, cuja estrutura e comunicação se baseiam
na procura de seus nós. Cada um destes nós é identificado por um identificador único
(ID), que serve tanto para a identificação quanto para a localização de valores na
\gls*{hashtable}.

Essa \gls*{hashtable} é no formato de uma árvore binária, cujas folhas são os nós da
rede. Cada folha tem suas posições estabelecidas pelo menor prefixo comum de seus IDs,
organizando-os de forma que, para um dado nó $x$, a árvore é dividida em várias
subárvores menores que não o contém. Assim, a maior subárvore consiste de metade da
árvore que não contém $x$, a subárvore seguinte é feita da metade da árvore restante
onde $x$ também não está contido, etc. O Kademlia garante ainda que todo nó conhece um
outro que esteja em cada uma das subárvores, se estas contiverem algum nó.

\begin{figure}[ht!]
    \centering
    \fbox{\includegraphics[width=\textwidth]{dht1.png}}
    \caption{Árvore binária do Kademlia. O nó preto é a posição do ID 0011...; os ovais
    cinzas são as subárvores onde o nó preto deve possuir nós conhecidos. Fonte:
    \cite{artigo:kademlia}}
    \label{fig:dht-arvore}
\end{figure}

No Kademlia, objetos e nós possuem IDs únicos de 160 bits: enquanto o primeiro utiliza
o \gls*{hashvalue} de 20 bytes SHA-1 da chave \bverb|info_hash| do \gls*{torrentfile},
o segundo é um valor aleatório escolhido pelo próprio programa.

Durante uma busca por \glspl*{peer} de um \gls*{torrent}, o processo deve conhecer a
chave associada ao objeto, ou seja, o ID, e explorar a rede em passos. A cada passo,
encontrará nós mais próximos da chave, até chegar ao valor buscado ou até não existirem
nós mais próximos que o atual. Dessa forma, para uma rede com $n$ nós, o algoritmo
visita apenas $\Oh(\log n)$ nós.

\newpage
\begin{figure}[ht!]
    \centering
    \fbox{\includegraphics[width=0.8\textwidth]{dht2.png}}
    \caption{Exemplo de uma busca na árvore de nós do Kademlia usando-se um ID. O nó
    preto, de prefixo 0011, encontra o nó de prefixo 1110 através de sucessivas buscas
    (setas numeradas inferiores). As setas superiores mostram a convergência da
    busca durante a execução. Fonte: \cite{artigo:kademlia}}
    \label{fig:dht-arvore-busca}
\end{figure}

Para o conceito de proximidade, as distâncias são calculadas usando-se a função de
distância baseada em \gls{xor} bit a bit

\begin{equation*}
    d(x,y) = x \oplus y
\end{equation*}

que possui certas propriedades, algumas em comum com a equação de distância euclidiana
usual:

\begin{itemize}
    \item $d(x,x) = 0$;
    \item $x \neq y$, $d(x,y) > 0$;
    \item simetria: $\forall x,y$, $d(x,y) = d(y,x)$;
    \item desigualdade triangular: $d(x,y) + d(y,z) \geq d(x,z)$. \\
        Isto vem do fato de $d(x,z) = d(x,y) \oplus d(y,z)$ e que $\forall a \geq 0,
        \forall b \geq 0 : a + b \geq a \oplus b$;
    \item unidirecionalidade: para um dado ponto $x$ e uma distância $\Delta > 0$,
        existe exatamente um ponto $y$ tal que $d(x,y) = \Delta$. Isso garante que todas
        as procuras por uma mesma chave convirjam para um mesmo percurso, independente
        do ponto de partida;
    \item para um dado $x$ no espaço de números, o conjunto de distâncias $\Delta$ entre
        $x$ e os outros números possui distribuição uniforme, ao contrário da distância
        euclidiana \cite{bittorrentcom:dht}.
\end{itemize}

No Transmission, o Kademlia é usado a partir de uma biblioteca externa (não mantida
pelos seus desenvolvedores), criada por Juliusz Chroboczek \cite{repo:dht-c}, sendo
adicionada e adaptada para o uso no código do programa. Sua implementação não é na forma
de árvore, mas sim de listas ligadas.

\cfile[label="./third-party/dht/dht.c:129"]{./Codes/chap3/018-dht-structs.c}

\input{Chapters/chap3/2120-subsubsec-tabela-roteamento}

\input{Chapters/chap3/2130-subsubsec-protocolo}

%!TEX root = ../../tcc.tex

\subsubsection*{Entrada e saída na rede e manutenção}

Um nó que queira se juntar à rede deve se preparar, numa fase que é chamada de
\gls{bootstrap}.

No início dessa fase, o novo nó deve conhecer o endereço IP e a porta de outro nó que já
esteja dentro da rede (o \gls*{bootstrap} \emph{node}, ou nó de \gls*{bootstrap}). Ao
entrar, o novo nó escolherá um ID aleatório, ainda não utilizado, que durará até que ele
saia da rede. Feito isso, o novo nó adicionará o nó de \gls*{bootstrap} em um de seus
\glspl*{kbucket} e iniciará uma consulta \bverb|FIND_NODE| em si mesmo. Isso fará com
que \glspl*{kbucket} em outros nós sejam populados com o novo ID, assim como os
\glspl*{kbucket} do novo nó serão preenchidos com os nós entre ele e o nó de
\gls*{bootstrap}. Em seguida, o novo nó atualizará todos os \glspl*{kbucket} mais
distantes que o do nó de \gls*{bootstrap}, para procurar por uma chave aleatória que
estará nesse intervalo.

Vale notar que os \glspl*{bucket} são sempre mantidos atualizados, devido ao grande
número de mensagens que viajam pelos nós. Para evitar problemas quando não houver esse
tráfego constante, cada nó deverá atualizar um \gls*{bucket} em que não tiver feito um
\emph{lookup} de nó na última uma hora. Assim, deverá escolher um ID aleatório do
intervalo correspondente e efetuar uma busca por esse ID.

Uma implementação ingênua pode ser feita usando-se um vetor de 160 \glspl*{bucket}, um
para cada possibilidade de diferença de bits. Porém, como no Kademlia a tendência é de
se conhecer mais sobre \glspl*{peer} mais próximos, muitos \glspl*{bucket} ficarão
vazios. Uma forma mais otimizada de tratar isso é fazendo com que, inicialmente, os nós
possuam somente um \gls*{bucket}. Eventualmente, estes ficarão cheios, sendo então
divididos. Nesse caso, dois novos \glspl*{bucket} são criados, onde o conteúdo do
\gls*{bucket} original será dividido entre ambos, e ocorrerá uma nova tentativa de
inserção. Se falhar novamente, o novo contato será descartado. Somente o \gls*{bucket}
mais recente será divisível.

\newpage

Além dessa divisão normal, existe a possibilidade de que árvores muito desbalanceadas
atrapalhem a notificação de um novo nó. Supondo que um nó esteja entrando na rede que
já possui mais do que $k$ nós com o mesmo prefixo, estes conseguirão adicioná-lo às
suas respectivas tabelas num \gls*{bucket} apropriado, porém, o novo nó só conseguirá
adicionar $k$ nós à sua tabela. Para evitar isso, os nós mantêm todos os contatos
válidos numa subárvore de tamanho $\geq k$, mesmo que deva dividir \glspl*{bucket} que
não contenham o próprio ID. Assim, quando o novo nó dividir \glspl*{bucket}, todos os
nós de mesmo prefixo saberão de sua existência.

Do Transmission, vale destacar que, quando ele efetua a saída da rede, salva a tabela
de nós num arquivo \sverb|dht.bootstrap|, cujos \glspl*{peer} salvos serão utilizados na
fase de \gls*{bootstrap} do próximo início da \gls*{dht}. Caso esses arquivos não
existam ou não tenham sido suficientes, o programa utilizará \glspl*{peer} fornecidos
pela \gls*{dht} ``oficial'', localizado no endereço \url{dht.transmissionbt.com:6881}.

%!TEX root = ../../tcc.tex

\subsubsection*{Otimizações}

As atualizações periódicas de tabelas ocorrem para evitar dois problemas no
\emph{lookup} por chaves válidas: nós que receberam anteriormente algum valor a ser
guardado naquele chave podem ter saído da rede; ou outros nós novos podem ter entrado
na rede com IDs mais próximos à uma chave já armazenada. Em ambos os casos, os nós que
possuem aquela entrada de chave-valor devem republicá-la, de forma a garantir que
esteja disponível nos $k$ nós mais próximos dessa chave.

Para compensar as saídas de nós, a republicação de cada chave-valor acontece uma vez por
hora. Porém, uma implementação ingênua necessitaria de muitas mensagens: cada um dos $k$
nós contendo o par de chave-valor executaria um \emph{lookup de nó} seguido de $k - 1$
comandos \bverb|announce_peer| por hora. Entretando, essa implementação pode ser
otimizada.

Primeiramente, quando um nó receber um comando \bverb|announce_peer| para um par de
chave-valor, assumirá que também já foi feito para os outros $k - 1$ nós próximos, não
necessitando republicar esse par na próxima hora. Assim, a menos que todos os horários
de republicação estejam sincronizados, somente um nó executará a republicação do par de
chave-valor.

Em segundo lugar, no caso de árvores muito desbalanceadas, os nós dividirão
\glspl*{kbucket} de acordo com o necessário para garantir o conhecimento total de uma
subárvore de tamando $\geq k$. Se antes de republicar pares de chave-valor, um nó $x$
atualizar todos os \glspl*{kbucket} dessa subárvore de $k$ nós, automaticamente será
capaz de descobrir os $k$ nós mais próximos para a chave dada. Para entender o motivo,
deve-se considerar dois casos:

\begin{enumerate}
    \item se a chave republicada cair no intervalo de ID da subárvore, considerando que
        a subárvore é de tamanho $\geq k$ e o novo nó já conhecerá os nós dessa
        subárvore, então o nó $x$ já conhecerá os $k$ nós mais próximos à chave;

    \item se a chave a ser atualizada estiver fora do intervalo da subárvore, mas o nó
        $x$ for um dos $k$ nós mais próximos da chave, então todos os \glspl*{kbucket}
        de $x$, para intervalos mais próximos da chave do que a subárvore, terão menos
        que $k$ nós.
\end{enumerate}

Então, o nó $x$ conhecerá todos os nós dentro desses \glspl*{kbucket}, o que,
juntamente do conhecimento da subávore, incluirá os $k$ nós mais próximos à chave.