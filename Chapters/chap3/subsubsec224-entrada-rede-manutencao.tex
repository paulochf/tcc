%!TEX root = ../../tcc.tex

\subsubsection*{Entrada na rede e manutenção}

Um nó que queira se juntar à rede deve se preparar, numa fase que é chamada de
\gls{bootstrap}.

No início dessa fase, o novo nó deve conhecer o endereço IP e a porta de outro nó que
já esteja dentro da rede Kademlia, chamado de \gls*{bootstrap} \emph{node} (nó de
\gls*{bootstrap}). Caso não esteja, um ID ainda não utilizado é computado e será usado
até que esse nó saia da rede.

Feito isso, o novo nó adiciona o nó de \gls*{bootstrap} em um de seus \glspl*{kbucket}
e inicia uma consulta \bverb|FIND_NODE| em si mesmo. Isso faz com que \glspl*{kbuckets}
em outros nós sejam populados com o novo ID, assim como os \glspl*{kbuckets} do novo nó
será preenchido com os nós entre ele e o nó de \gls*{bootstrap}. Em seguida, o novo
nó atualiza todos os \glspl*{kbuckets} mais distantes que o do nó de \gls*{bootstrap},
para procurar por uma chave aleatória que está nesse intervalo.

Vale notar que os \glspl*{bucket} são sempre mantidos atualizados, devido ao grande
número de mensagens que viajam pelos nós. Para evitar problemas quando não houver esse
tráfego constante, cada nó deverá atualizar um \gls*{bucket} em que não tiver feito um
\emph{lookup} de nó na última 1 hora. Assim, deverá escolher um ID aleatório do
intervalo correspondente e efetuar uma busca por esse ID.

Durante todo esse processo pode haver separação de \glspl*{kbucket}. Inicialmente, nós
possuem somente um \gls*{kbucket}, que eventualmente ficarão cheios, quando serão
divididos. Essa divisão ocorre sempre que o intervalo de nós em um \gls*{kbucket}
incluir o ID do prório nó. Nesse caso, 2 novos \glspl*{bucket} são criados, onde o
conteúdo do anterior é dividido entre ambos, e ocorre uma nova tentativa de inserção. Se
falhar novamente, o novo contato é descartado.

Além dessa divisão normal, pode ocorrer de árvores muito desbalanceadas atrapalharem a
notificação de um novo nó. Supondo que um nó esteja entrando na rede que já possui mais
do que $k$ nós com o mesmo prefixo, estes conseguirão adicioná-lo à suas respectivas
tabelas num \gls*{bucket} apropriado, porém o novo nó só conseguirá adicionar $k$ nós à
sua tabela. Para evitar isso, os nós mantém todos os contatos válidos numa subárvore de
tamanho $\geq k$, mesmo que deva dividir \glspl*{bucket} que não contenham o próprio
ID. Assim, quando o novo nó dividir \glspl*{bucket}, todos os nós de mesmo prefixo
saberão de sua existência.