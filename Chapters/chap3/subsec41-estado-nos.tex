%!TEX root = ../../tcc.tex

\subsection*{Estados dos nós e informações}

Existem 2 características independentes que formam as possibilidades de estados que um
\gls*{peer} pode assumir enquanto participa de um \gls*{swarm}:

\begin{itemize}
    \item \textbf{choking} (estrangulamento): se um \gls*{peer} \textbf{A} estrangulará
        a conexão com outro \gls*{peer} \textbf{B} (\emph{choked}) ou a deixará livre
        (\emph{unchoked}).

    \item \textbf{interested} (interesse): se um \gls*{peer} \textbf{A} terá interesse
        em um \gls*{peer} \textbf{B} (\emph{interested}) ou não (\emph{not interested})
\end{itemize}

Uma nova conexão entre \glspl*{peer} inicia em \emph{choked} e \emph{not interested} em
ambos os sentidos, ou seja, com \textbf{A} e \textbf{B} estrangulando suas conexões
mutuamente e sem interesse no outro. Esses estados ditarão todas as estratégias de troca
de partes entre \glspl*{peer}.

Outra informação utilizada é o \emph{bitfield}, que é um mapa de bits onde cada bit
representa uma parte que o \gls*{peer} já possui.