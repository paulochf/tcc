%!TEX root = ../../tcc.tex

\newpage
\subsection*{Arquivo .torrent}

Ao se adicionar um \gls*{torrentfile} em um programa cliente, ocorrem muitas
transmissões de dados antes do download de fato. Para demonstrar isso, usaremos um
\gls*{torrentfile} do filme \enquote{A Noite dos Mortos Vivos}, de 1960
\cite{torrent-file}, que é de domínio público e livre de direitos autorais.

Se abrirmos esse arquivo, veremos uma grande \gls{string}, caracteres diferentes e
incomuns, formando um conteúdo ilegível (na seção binária) e sob uma forma compacta,
mostrado abaixo.

\begin{listing}[H]
    \begin{minted}[
        linenos,
        frame=single,
        numbersep=6pt,
        baselinestretch=1,
        fontfamily=courier,
        gobble=4,
        fontsize=\scriptsize
    ]{text}
    d8:announce36:http://bt1.archive.org:6969/announce13:announce-listll36:http://bt1.
    archive.org:6969/announceel36:http://bt2.archive.org:6969/announceee7:comment13:crea
    tiondatei1343715473e4:infod5:filesld5:crc328:030208fe6:lengthi4127671704e3:md532:627
    f5a428f9e454ccfcb29d31b87169a5:mtime10:10794024804:pathl29:night_of_the_living_dead.
    mpege4:sha140:5e44bb1b3f700240249a5287c64dc02dc56d034bee4:name24:night_of_the_living
    _dead12:piecelengthi4194304e6:pieces23720:<binary>

    (...)

    e6:locale2:en5:title24:night_of_the_living_dead8:url-listl28:http://archive.org/
    download/39:http://ia600301.us.archive.org/22/items39:http://ia700301.us.archive.
    org/22/itemsee
    \end{minted}

    \caption{trecho do conteúdo do arquivo .torrent do filme \enquote{A Noite dos Mortos
    Vivos}, de 1960 \cite{torrent-file}, com a parte binária truncada}
    \label{lst:torrent-file-raw}
\end{listing}

Esse conteúdo está organizado usando a \gls{bencode}, que é uma codificação compacta de
arquivos, especial para \glspl*{torrentfile}, e ininteligível. Com alguma formatação,
podemos enxergar os componentes separadamente, como mostra o código
~\ref{lst:torrent-file-code}.

Esse conteúdo tem significado, sendo utilizado da seguinte forma
\cite{wikitheory:bencoding}:

\begin{itemize}
    \item \textbf{\glspl*{string}} são prefixos de números na base 10, que representam
        comprimentos, seguidos por um caractere \bverb|:| e então o conteúdo. Por
        exemplo, na linha 2, \bverb|8:announce| corresponde à \gls*{string}
        \sverb|"announce"|.

    \item \textbf{números} são representados por um \bverb|i|, seguidos do valor na
        base 10 (sem qualquer limite de tamanho, mas sem zeros precedentes - como em
        \bverb|0003| - e pode ser negativo), terminados por um \bverb|e|. Por exemplo,
        na linha 11, \bverb|i1343715473e| corresponde ao número \sverb|1343715473|.

    \item \textbf{listas} são formadas por \bverb|l|, seguidos por seus elementos
        (também no formato \gls*{bencode}), e então terminados por \bverb|e|. Por
        exemplo, \bverb|l3:foo3:bare| corresponde a \sverb|["foo", "bar"]|. No código
        ~\ref{lst:torrent-file-code}, é presente entre as linhas 43 e 47.

    \item \textbf{dicionários} são definidos por \bverb|d|, seguidos de uma lista
        alternada de chaves e seus valores correspondentes, terminando com \bverb|e|,
        onde as chaves devem estar ordenadas usando-se comparação binária, ao invés da
        usual alfanumérica. Por exemplo, a \gls*{string}
        \bverb|d3:foo3:bar6:foobar6:bazbare| corresponde ao dicionário puro \\
        \sverb|{"foo": "bar", "foobar": "bazbar"}|, e a estrutura mais complexa dada por
        \\ \bverb|d3:fool6:foobar3:bazee| equivale a \sverb|{"foo": ["foobar", "baz"]}|.
\end{itemize}

\begin{listing}[H]
    \begin{minted}[
        linenos,
        frame=single,
        numbersep=6pt,
        baselinestretch=1,
        fontfamily=courier,
        gobble=4,
        fontsize=\scriptsize
    ]{text}
    d
        8:announce
        36:http://bt1.archive.org:6969/announce
        13:announce-list
        l
            l36:http://bt1.archive.org:6969/announcee
            l36:http://bt2.archive.org:6969/announcee
        e
        7:comment
        13:creation date
        i1343715473e
        4:info
        d
            5:files
            l
                d
                    5:crc32
                    8:030208fe
                    6:length
                    i4127671704e
                    3:md5
                    32:627f5a428f9e454ccfcb29d31b87169a
                    5:mtime
                    10:1079402480
                    4:path
                    l29:night_of_the_living_dead.mpege
                    4:sha1
                    40:5e44bb1b3f700240249a5287c64dc02dc56d034b
                e
            e
            4:name
            24:night_of_the_living_dead
            12:piece length
            i4194304e
            6:pieces
            23720:<binary>
        e
        6:locale
        2:en
        5:title
        24:night_of_the_living_dead
        8:url-list
        l
            28:http://archive.org/download/
            39:http://ia600301.us.archive.org/22/items
            39:http://ia700301.us.archive.org/22/items
        e
    e
    \end{minted}
    \caption{trechos formatados de forma legível do conteúdo do arquivo .torrent do
    filme \enquote{A Noite dos Mortos Vivos}, de 1960 \cite{torrent-file}, com a parte
    binária truncada}
    \label{lst:torrent-file-code}
\end{listing}