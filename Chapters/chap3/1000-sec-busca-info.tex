%!TEX root = ../../tcc.tex

\newpage
\section{Busca por informações}

Quando adicionamos um \gls*{torrent} ao Transmission, o programa salva as informações em
disco durante todo o período em que estas estiverem sendo gerenciadas por ele. Caso
tenha sido por meio de um \gls*{torrentfile}, uma cópia deste é salva em uma pasta pré-
definida, para seu controle interno; caso seja por \gls*{magnetlink}, um novo arquivo é
criado contendo as informações obtidas através dele (por questões de praticidade), para
que haja necessidade de se fazer essa aquisição dos dados novamente ao ser aberto, ou
quando alguma transferência for pausada e depois continuada.

Após esse arquivamento, o programa processa as informações salvas para deixá-las
carregadas em memória, a fim de obter o \gls*{hashvalue} que identifica o
\gls*{torrentfile}.

\cfile[label="./libtransmission/metainfo.c:367"]{./Codes/chap3/001-leiturametadata.c}

Se aquele arquivo para controle interno tiver sido criado por conta de um
\gls*{magnetlink}, não possuirá a chave \bverb|info| em seu dicionário, mas deverá
conter as chaves \bverb|urn:btih:<hash>| e \bverb|info_hash|.

\cfile[label="./libtransmission/metainfo.c:367"]{./Codes/chap3/002-leiturametadata2.c}

Porém, se o arquivo para controle interno tiver sido criado como cópia do
\gls*{torrentfile}, possuirá o mesmo dicionário, que contém a chave \bverb|info_hash|,
que é utilizada para calcular o \gls*{hashvalue} do arquivo.

\cfile[label="./libtransmission/metainfo.c:367"]{./Codes/chap3/003-leiturametadata3.c}

Outras informações podem ser recuperadas, dependendo da origem do \gls*{torrentfile},
tais como a privacidade do \gls*{torrent}, a lista de arquivos e seus respectivos
tamanhos, \gls*{hashvalue} de cada parte, entre outras. Por fim, termina coletando os
\glspl{announce}.

\cfile[label="./libtransmission/metainfo.c:367"]{./Codes/chap3/004-leiturametadata4.c}

%!TEX root = ../../tcc.tex

\subsection*{Announce}

Para cada \gls*{swarm} gerenciado, o \gls*{tracker} possui uma lista dos \glspl*{peer}
que participam dele, que é enviada ao \gls*{peer} que a requer por meio de uma
\gls{httpget}. Quando essa requisição é recebida pelo \gls*{tracker}, este incluirá ou
atualizará um registro para o \gls*{peer} solicitante, e devolverá uma lista de 50
\glspl*{peer} aleatórios, de forma uniforme, que fazem parte do \gls*{swarm}. Não
havendo essa quantidade total, a lista toda será enviada ao requisitante. Caso
contrário, a aleatoriedade proporcionará uma diversidade de listas enviadas,
ocasionando robustez ao sistema \cite{wikitheory:tracker-response}.

Esse contato entre um \gls*{peer} e um \gls*{tracker} é chamado de \gls{announce}, que
pode ser feito usando-se tanto o \gls{tcp}, bem como o \gls{udp}, e é o meio como
\glspl*{peer} podem passar várias informações, usando um dicionário no formato
\gls*{bencode}:

\begin{description}
    \item[info\_hash:] \gls*{hashvalue} de 20 bytes resultante da \gls*{hashfunction}
        SHA-1, com \gls*{urlencode}, do valor da chave \bverb|info| do
        \gls*{torrentfile};

    \item[peer\_id:] \gls*{string} de 20 bytes, com \gls*{urlencode}, usado como
        identificador único do programa cliente, gerado no ínício da sua execução. Para
        isso, provavelmente deverá incorporar informações do computador, a fim de se
        gerar um valor único;

    \item[uploaded:] a quantidade total de dados, em bytes, transmitidos desde o
        momento em que o cliente enviou o primeiro aviso ao \gls*{tracker};

    \item[downloaded:] a quantidade total de dados, em bytes, recebidos desde o momento
        em que o cliente enviou o primeiro aviso ao \gls*{tracker};

    \item[left:] a quantidade total de dados, em bytes, que faltam para o requisitante
        terminar o download do \gls*{torrent} e passar a ser um \gls*{seeder};

    \item[compact] (opcional): se o valor passado for 1, significa que o requisitante
        aceita respostas compactas. A lista de \glspl*{peer} enviada é substituída por
        uma única \gls*{string} de \glspl*{peer}, sendo que cada \gls*{peer} terá 6
        bytes, onde os 4 bytes iniciais serão o host e os 2 bytes finais serão a porta
        de transmissão. Por exemplo, o endereço IP 10.10.10.5:80 seria transmitido como
        \bverb|0A 0A 0A 05 00 80|. Deve-se observar que alguns \glspl*{tracker} suportam
        somente conexões deste tipo para otimização da utilização da banda de rede e,
        desse modo, ou recusarão requisições sem \bverb|compact=1| ou, caso não as
        recusem, enviarão respostas compactas (a menos que a requisição possua
        \bverb|compact=0|);

    \item[no\_peer\_id] (opcional): sinaliza que o \gls*{tracker} pode omitir o
        campo \textcolor{Bittersweet}{\texttt{peer\_id}} no dicionário de \glspl*{peer},
        porém será ignorado caso o modo compacto esteja habilitado;

    \item[event] (opcional): pode possuir os valores \sverb|started| (iniciado),
        \sverb|completed| (terminado), \sverb|stopped| (parado), ou vazio, para não
        especificar:

        \begin{itemize}
            \item \emph{started} : a primeira requisição para o \gls*{tracker} deve
                enviar este valor;
            \item \emph{stopped} : avisa que o programa cliente está fechando; e
            \item \emph{completed} : quando o download que estava ocorrendo termina numa
                mesma execução do programa cliente (não é enviado quando o programa
                cliente é iniciado com o \gls*{torrent} em 100\%).
        \end{itemize}

    \item[port] (opcional): o número da porta de conexão que o programa cliente está
        aguardando (ou ``escutando'') por transmissões de dados. Em geral, portas
        reservadas para BitTorrent estão entre 6881 e 6889. Se esse for o caso, pode
        ser omitido;

    \item[ip] (opcional): o endereço IP verdadeiro do requisitante, no formato legível
        do IPv4 (4 conjuntos de números de 0 a 255 separados por \bverb|.|) ou do IPv6
        (8 conjuntos de números hexadecimais de 4 dígitos separados por \bverb|:|). Não
        é sempre necessário, pois o endereço pode ser conhecido através da requisição.
        Assim, é usado quando o programa cliente está se comunicando com o
        \gls*{tracker} através de um \gls{proxy} ou quando ambos (cliente e
        \gls*{tracker}) estão na mesma sub-rede (no mesmo lado de um roteador) ---
        com \gls{nat} ---, pois, nesse caso, o endereço IP não é roteável;

    \item[numwant] (opcional): quantidade de \glspl*{peer} que o requisitante gostaria
        de receber do \gls*{tracker}. É permitido valor zero. Se omitido, assumirá
        valor padrão de 50;

    \item[key] (opcional): mecanismo de identificação adicional para o programa cliente
        provar sua identidade, caso tenha ocorrido mudança no seu endereço IP; e

    \item[trackerid] (opcional): se a resposta de um \gls*{announce} anterior continha
        o endereço IP de um \gls*{tracker}, deve ser enviado neste campo.
\end{description}

\cfile[label="./libtransmission/announcer-common.h:127"]{./Codes/chap3/005-announcestruct.c}
\cfile[label="./libtransmission/announcer.c:1200"]{./Codes/chap3/006-announce.c}

Como resposta, é recebido um outro dicionário em \gls*{bencode}, podendo conter as
seguintes chaves:

\begin{description}
    \item[failure\_reason:] se presente, não podem existir outras chaves no dicionário.
        Seu valor é uma \gls*{string} de mensagem de erro legível sobre a causa da falha
        da requisição;

    \item[warning\_message] (opcional): similar à chave \bverb|failure_reason|, mas com
        a requisição tendo sido processada normalmente. A mensagem é mostrada como um
        erro;

    \item[interval:] intervalo, em segundos, que o cliente deve esperar emtre
        requisições de \gls*{announce} ao \gls*{tracker};

    \item[min\_interval] (opcional): intervalo mínimo, em segundos, entre requisições
        de \gls*{announce}. Se presente, o programa cliente não deve efetuar essas
        requisições acima da frequência estipulada;

    \item[tracker\_id:] \gls*{string} que o programa cliente deve enviar junto às
        próximas requisições. Se ausente, e um valor tiver sido passado anteriormente, o
        uso desse valor antigo é continuado;

    \item[complete:] quantidade de \glspl*{seeder};

    \item[incomplete:] quantidade de \glspl*{leecher}; e

    \item[\glspl*{peer}]: pode ser uma das seguintes opções:

        \begin{enumerate}
            \item lista de dicionários \gls*{bencode}, com as seguintes chaves:

            \begin{itemize}
                \item \textbf{peer\_id}: identificador de um \gls*{peer} na forma de
                    \gls*{string}, escolhido por si próprio da mesma forma que a
                    descrita pela definição de requisição;

                \item \textbf{ip}: endereço IP do \gls*{peer} nos formatos IPv4
                    (4 octetos) ou IPv6 (valores hexadecimais), ou ainda o nome de
                    domínio DNS (string); e

                \item \textbf{port}: número da porta utilizada pelo \gls*{peer}.
            \end{itemize}

            \item \gls*{string} binária, cujo tamanho é de 6 bytes para cada \gls*{peer}
                , onde os 4 primeiros representam o endereço IP e os 2 últimos são o
                número da porta, em notação de rede (\emph{big} \gls{endian});
        \end{enumerate}
\end{description}

\begin{listing}[H]
    \begin{minted}[
        linenos,
        frame=single,
        numbersep=6pt,
        baselinestretch=1,
        fontfamily=courier,
        gobble=4,
        fontsize=\scriptsize
    ]{text}
    * About to connect() to exodus.desync.com port 6969 (#8)
    *   Trying 208.83.20.164...
    *
    * Connected to exodus.desync.com (208.83.20.164) port 6969 (#8)
    > GET /announce?info_hash=\%88\%15\%8c\%7bW\%e0\%85\%21\%86~\%d0\%b5\%de\%06\%5b\%
    7dWI\%cf\%d7&peer_id=-TR2820-ne1joqgh8z9o&port=51413&uploaded=0&downloaded=0&left=52
    406288292&numwant=80&key=6ee99240&compact=1&supportcrypto=1&requirecrypto=1&event=
    started HTTP/1.1
    User-Agent: Transmission/2.82
    Host: exodus.desync.com:6969
    Accept: */*
    Accept-Encoding: gzip;q=1.0, deflate, identity

    < HTTP/1.1 200 OK
    < Content-Type: text/plain
    < Content-Length: 136
    <
    * Connection #8 to host exodus.desync.com left intact
    Announce response:
    < {
        "complete": 1,
        "downloaded": 11,
        "incomplete": 6,
        "interval": 1732,
        "min interval": 866,
        "peers": \"<binary>\"
    }
    \end{minted}

    \caption{Logs do Transmission sobre uma requisição de announce e a respectiva
    resposta, com o conteúdo binário truncado}
    \label{lst:announce}
\end{listing}

Essa comunicação ocorre nas seguintes situações:

\begin{itemize}
    \item no primeiro contato do \gls*{peer}, para que ele tenha acesso a um
        \gls*{swarm};

    \item a cada período de tempo, estipulado pelo tracker, para que o \gls*{peer}
        continue mostrando que ainda está conectado, além de poder receber endereços de
        \glspl*{peer} novos;

    \item quando a quantidade de \glspl*{peer} conhecidos que estiverem ativos for
        menor do que 5;

    \item quando terminar o download, notificando que o \gls*{peer} passou a ser um
        \gls*{seeder}; e

    \item quando o \gls*{peer} sair do \gls*{swarm}, seja por desconexão ou por
        encerramento do programa cliente.
\end{itemize}


%!TEX root = ../../tcc.tex

\subsection*{Scrape}

Além do \gls*{announce}, outra forma de troca de informação entre \glspl*{peer} e
\glspl*{tracker} se dá pelo \gls{scrape}. Geralmente usado pelos programas cliente para
decidir quando realizar um \gls*{announce}, informa o número de \glspl*{peer},
\glspl*{leecher} e \glspl*{seeder} de uma lista de um ou mais \glspl*{torrent}. É dessa
forma que os sites de indexação sabem dessas informações e as apresentam em suas
páginas.

A requisição de \gls*{scrape} pode ser sobre todos os \glspl*{torrent} que o
\gls*{tracker} gerencia ou sobre um ou mais \glspl*{torrent} em específico, quando são
passados seus respectivos \glspl*{hashvalue}. Sua resposta é um dicionário na forma
\gls*{bencode} contendo a chave:

\begin{description}
    \item[files:] um dicionário contendo um par chave-valor para cada \gls*{torrent}
        especificado na requisição do \gls*{scrape}, através do \gls*{hashvalue} do
        \gls*{torrent} de 20 bytes:

        \begin{itemize}
            \item \emph{complete}: quantidade de \glspl*{seeder};

            \item \emph{incomplete}: quantidade de \glspl*{leecher};

            \item \emph{downloaded}: quantidade total de vezes que o \gls*{tracker}
                registrou uma finalização (\bverb|event=complete| ao término de um
                download); e

            \item \textbf{name} (opcional): o nome interno do \gls*{torrent}, como
                especificado pelo respectivo \gls*{torrentfile}, na sua seção
                \bverb|info|.
        \end{itemize}
\end{description}

Assim como o \gls*{announce}, o \gls*{scrape} é um endereço do \gls*{tracker}
usando-se o \gls*{tcp} ou o \gls*{udp}, e é tratado de forma semelhante pelo
Transmission.

\cfile[label="./libtransmission/announcer-common.h:37"]{./Codes/chap3/007-scrapestruct.c}
\cfile[label="./libtransmission/announcer.c:1397"]{./Codes/chap3/008-scrape.c}

\begin{listing}[H]
    \begin{minted}[
        linenos,
        frame=single,
        numbersep=6pt,
        baselinestretch=1,
        fontfamily=courier,
        gobble=4,
        fontsize=\scriptsize
    ]{text}
    * About to connect() to www.mvgroup.org port 2710 (#1)
    *   Trying 88.129.153.50...
    * Connected to www.mvgroup.org (88.129.153.50) port 2710 (#1)
    > GET /scrape?info_hash=\%83F\%24\%b62\%e5Q\%cc4\%10h\%ba\%1e8\%e2C\%f7\%80\%01\%87 HTTP/1.1
    User-Agent: Transmission/2.82
    Host: www.mvgroup.org:2710
    Accept: */*
    Accept-Encoding: gzip;q=1.0, deflate, identity

    * HTTP 1.0, assume close after body
    < HTTP/1.0 200 OK
    <
    * Closing connection 1
    Scrape response:
    < {
        "files": {
            \"<binary>\": {
                "complete": 9,
                "downloaded": 59074,
                "incomplete": 2
            }
        }
    }
    \end{minted}

    \caption{logs do Transmission sobre uma requisição de scrape, e a respectiva
    resposta, com o conteúdo binário truncado}
    \label{lst:scrape}
\end{listing}

%!TEX root = ../../tcc.tex

\subsection*{Convenção de scrape de trackers}

O endereço de \gls*{scrape} pode ser obtido a partir do endereço de \gls*{announce},
seguindo-se a seguinte convenção:

\begin{enumerate}
    \item comece a partir do endereço de \gls*{announce};

    \item encontre o último caractere \sverb|/|;

    \item se o texto que segue a última \sverb|/| não for \sverb|announce|, é um sinal
        de que o \gls*{tracker} não segue a convenção; e

    \item caso contrário, substituir \sverb|announce| por \sverb|scrape|.
\end{enumerate}

Alguns exemplos:

\begin{itemize}
    \item suportam scrape:
        \begin{enumerate}
            \item \url{http://example.com/announce} $\rightarrow$
                \url{http://example.com/scrape};
            \item \url{http://example.com/announce.php} $\rightarrow$
                \url{http://example.com/scrape.php};
            \item \url{http://example.com/x/announce} $\rightarrow$
                \url{http://example.com/x/scrape}; e
            \item \url{http://example.com/announce?x2\%0644} $\rightarrow$
                \url{http://example.com/scrape?x2\%0644}.
        \end{enumerate}

    \item não suportam scrape:
        \begin{enumerate}
            \item \url{http://example.com/a};
            \item \url{http://example.com/announce?x=2/4}; e
            \item \url{http://example.com/x\%064announce}.
        \end{enumerate}
\end{itemize}

%!TEX root = ../../tcc.tex

\subsection*{Resultados de announce e scrape}

Com a requisição do \gls*{announce}, o Transmission recebe a sua resposta e prepara
esses dados para utilização, carregando-os em memória.

\cfile[label="./libtransmission/announcer-http.c:288"]{./Codes/chap3/009-httpannounce.c}
\cfile[label="./libtransmission/announcer-http.c:189"]{./Codes/chap3/010-onhttpannouncedone.c}

Após essa preparação dos dados da resposta do \gls*{announce}, cuja informação
principal é a lista de \glspl*{peer}, o Transmission gerencia o tempo para a próxima
requisição periódica de \gls*{announce} ou \gls*{scrape}. Além disso, sinaliza para a
\gls{thread} principal, ou seja, a execução primária do programa, de que recebeu a
resposta do \gls*{tracker}.

\cfile[label="./libtransmission/announcer.c:1012"]{./Codes/chap3/011-onannouncedone.c}
\cfile[label="./libtransmission/announcer.c:523"]{./Codes/chap3/012-publicapeers.c}

A \gls*{thread} principal percebe que foi sinalizado o recebimento de uma resposta de
\gls*{tracker} e começa a utilizar a lista de \glspl*{peer} adquirida, adicionando-os ao
\gls{pool} do objeto em memória que representa o \gls{swarm}.

\cfile[label="./libtransmission/torrent.c:552"]{./Codes/chap3/013-ontrackerresponse.c}
\cfile[label="./libtransmission/peer-mgr.c:2113"]{./Codes/chap3/014-addpex.c}

Essa inclusão é condicionada ao fato de que o endereço de um \gls*{peer} ainda não está
contido na lista de \glspl*{peer}.

\cfile[label="./libtransmission/peer-mgr.c:1852"]{./Codes/chap3/015-verificapeer.c}

Assim, a lista de \glspl*{peer} do Transmission está pronta para ser utilizada nos
pedidos das partes do \gls*{torrent}.