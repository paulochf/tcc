%!TEX root = ../../tcc.tex

% \newpage
\subsubsubsection{ping}

É o comando mais simples, que verifica se o nó está online. Possui um único argumento,
que é uma chave \bverb|id|, ou seja, o ID do nó consultante (na requisição) ou do nó
consultado (na resposta).

\begin{itemize}
    \item formato da requisição: \\
        \bverb|d1:ad2:id20:abcdefghij0123456789e1:q4:ping1:t2:aa1:y1:qe|
        (\gls*{bencode}) \\
        \sverb|{"t":"aa", "y":"q", "q":"ping", "a":{"id":"abcdefghij0123456789"}}|
        (\gls*{string})

    \item formato da resposta: \\
        \bverb|d1:rd2:id20:mnopqrstuvwxyz123456e1:t2:aa1:y1:re|
        (\gls*{bencode}) \\
        \sverb|{"t":"aa", "y":"r", "r":{"id":"mnopqrstuvwxyz123456"}}|
        (\gls*{string})
\end{itemize}

\cfile[label="./third-party/dht/dht.c:2291"]{./Codes/chap3/016-dht-macros.c}
\cfile[label="./third-party/dht/dht.c:2341"]{./Codes/chap3/017-dht-ping.c}