%!TEX root = ../../tcc.tex

\subsubsection*{Entrada e saída na rede e manutenção}

Um nó que queira se juntar à rede deve se preparar, numa fase que é chamada de
\gls{bootstrap}.

No início dessa fase, o novo nó deve conhecer o endereço IP e a porta de outro nó que já
esteja dentro da rede (o \gls*{bootstrap} \emph{node}, ou nó de \gls*{bootstrap}). Ao
entrar, o novo nó escolherá um ID aleatório, ainda não utilizado, que durará até que ele
saia da rede. Feito isso, o novo nó adicionará o nó de \gls*{bootstrap} em um de seus
\glspl*{kbucket} e iniciará uma consulta \bverb|FIND_NODE| em si mesmo. Isso fará com
que \glspl*{kbucket} em outros nós sejam populados com o novo ID, assim como os
\glspl*{kbucket} do novo nó serão preenchidos com os nós entre ele e o nó de
\gls*{bootstrap}. Em seguida, o novo nó atualizará todos os \glspl*{kbucket} mais
distantes que o do nó de \gls*{bootstrap}, para procurar por uma chave aleatória que
estará nesse intervalo.

Vale notar que os \glspl*{bucket} são sempre mantidos atualizados, devido ao grande
número de mensagens que viajam pelos nós. Para evitar problemas quando não houver esse
tráfego constante, cada nó deverá atualizar um \gls*{bucket} em que não tiver feito um
\emph{lookup} de nó na última uma hora. Assim, deverá escolher um ID aleatório do
intervalo correspondente e efetuar uma busca por esse ID.

Uma implementação ingênua pode ser feita usando-se um vetor de 160 \glspl*{bucket}, um
para cada possibilidade de diferença de bits. Porém, como no Kademlia a tendência é de
se conhecer mais sobre \glspl*{peer} mais próximos, muitos \glspl*{bucket} ficarão
vazios. Uma forma mais otimizada de tratar isso é fazendo com que, inicialmente, os nós
possuam somente um \gls*{bucket}. Eventualmente, estes ficarão cheios, sendo então
divididos. Nesse caso, dois novos \glspl*{bucket} são criados, onde o conteúdo do
\gls*{bucket} original será dividido entre ambos, e ocorrerá uma nova tentativa de
inserção. Se falhar novamente, o novo contato será descartado. Somente o \gls*{bucket}
mais recente será divisível.

\newpage

Além dessa divisão normal, existe a possibilidade de que árvores muito desbalanceadas
atrapalhem a notificação de um novo nó. Supondo que um nó esteja entrando na rede que
já possui mais do que $k$ nós com o mesmo prefixo, estes conseguirão adicioná-lo às
suas respectivas tabelas num \gls*{bucket} apropriado, porém, o novo nó só conseguirá
adicionar $k$ nós à sua tabela. Para evitar isso, os nós mantêm todos os contatos
válidos numa subárvore de tamanho $\geq k$, mesmo que deva dividir \glspl*{bucket} que
não contenham o próprio ID. Assim, quando o novo nó dividir \glspl*{bucket}, todos os
nós de mesmo prefixo saberão de sua existência.

Do Transmission, vale destacar que, quando ele efetua a saída da rede, salva a tabela
de nós num arquivo \sverb|dht.bootstrap|, cujos \glspl*{peer} salvos serão utilizados na
fase de \gls*{bootstrap} do próximo início da \gls*{dht}. Caso esses arquivos não
existam ou não tenham sido suficientes, o programa utilizará \glspl*{peer} fornecidos
pela \gls*{dht} ``oficial'', localizado no endereço \url{dht.transmissionbt.com:6881}.