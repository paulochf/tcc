%!TEX root = ../../tcc.tex

\subsubsection*{Tabela de roteamento}

Cada nó do \gls*{kademlia} armazena informações sobre outros nós para rotear mensagens
de pesquisa. Para cada bit $i$ dos IDs (cada ID tem 160 bits) é mantido um \gls{kbucket},
que contém os nós cuja distância até ele está entre $2^i$ e $2^{i+1}$. Esses
\glspl*{kbucket} são listas de endereço IP, porta de comunicação \gls*{udp} e ID de
nós, ordenadas pelo horário da última notícia destes. Para distâncias pequenas, essas
listas geralmente serão vazias, enquanto que para distâncias maiores, poderão ser de
tamanho $k$. Este valor, que é o de replicação do sistema, é escolhido de tal forma que
haja grande probabilidade desses $k$ nós não falharem na próxima hora.

Quando um nó \textbf{A} recebe uma mensagem de outro nó \textbf{B}, o \gls*{bucket}
correspondente ao ID do remetente (nó \textbf{B}) é atualizado. Disto, podem ocorrer as
seguintes situações:

\begin{itemize}
    \item \textbf{B} já existe no \gls*{bucket}: passa a ser o primeiro da lista, pois
        existiu mensagem recente;

    \item \textbf{B} não existe no \gls*{bucket}:
        \begin{itemize}
            \item \gls*{bucket} não está cheio: \textbf{B} é adicionado no começo da
                lista;
            \item \gls*{bucket} cheio: é enviado um \emph{ping} para o nó do final da
                lista (nó \textbf{C}), contatado há mais tempo:

                \begin{itemize}
                    \item \textbf{C} não responde ao \emph{ping}: \textbf{C} é retirado
                        da lista e \textbf{B} é inserido no início;
                    \item \textbf{C} responde ao \emph{ping}: \textbf{C} é movido para o
                        início da lista e \textbf{B} é ignorado.
                \end{itemize}
        \end{itemize}
\end{itemize}

Por conta disso, ocorre que nós mais antigos e funcionais são preferidos, pois quanto
mais tempo um nó está conectado, mais provável que ele se mantenha conectado por mais
uma hora \cite{artigo:gnutella-uptime}. Outra vantagem disso é a resistência a alguns
ataques de negação de serviço, pois mesmo que ocorra uma inundação de novos nós, estes
só seriam inseridos nos \glspl*{kbucket} se os antigos fossem excluídos.