%!TEX root = ../../tcc.tex

\subsubsubsection{get\_peers}
\label{subsubsubsec:getpeers}

É o comando \gls*{rpc} da mensagem \bverb|FIND_VALUE|, serve para buscar \glspl*{peer}
para um dado um \gls*{hashvalue} identificador de \gls*{torrentfile}, enviado como
valor da chave \bverb|info_hash|, além do ID do nó consultante como valor da chave
\bverb|id|.

O funcionamento é equivalente ao comando \bverb|find_node|, com um detalhe extra: se o
nó que recebeu a mensagem possuir \glspl*{peer} para o \gls*{hashvalue} dado, eles são
informados imediatamente na forma compacta (6 bytes para cada \gls*{peer}) numa lista
\gls*{bencode} de \glspl*{string}, devolvida como valor da chave \bverb|values|. Por
outro lado, caso o receptor da mensagem não conhecer nós para o \gls*{hashvalue}
especificado, a resposta conterá a chave \bverb|nodes| com os $k$ nós mais próximos
desse \gls*{hashvalue}. O nó original consulta outros nós próximos ao \gls*{torrentfile}
iterativamente. Ao fim da busca, o programa cliente insere o contato para si mesmo na
lista de nós próximos ao \gls*{torrentfile}.

Em ambos os casos, uma chave \bverb|token| é informada na resposta, cujo valor é uma
\gls*{string} binária curta, que deverá ser utilizada em futuras mensagens de
\bverb|announce_peer|.

\todoquestion{referencio o código do comando anterior?}

\begin{itemize}
    \item formato dos argumentos da requisição \\
        \sverb|{"id":\"<IDs dos nós consultantes>",| \\
        \sverb| "info_hash":\"<hash de 20 bytes do torrent buscado>"}|

    \item formato da resposta \\
        \sverb|{"id":\"<IDs dos nós consultados>", "token":\"<token>",| \\
        \sverb| "values":[\"<info peer 1>", \"<info peer 2>", ...]}| \\
        ou \\
        \sverb|{"id":\"<IDs dos nós consultados>", "token":\"<token>",| \\
        \sverb| "nodes":\"<info compacta do(s) nó(s)>"}| \\ \\
\end{itemize}

\cfile[label="./third-party/dht/dht.c:1866"]{./Codes/chap3/024-dht-periodic.c}

\newpage
\begin{itemize}
    \item exemplo de requisição \\
        \bverb|d1:ad2:id20:abcdefghij01234567899:info_hash20:mnopqrstuvwxyz123456e|\\
        \bverb|1:q9:get_peers1:t2:aa1:y1:qe| (\gls*{bencode}) \\
        \sverb|{"t":"aa", "y":"q", "q":"get_peers",|\\
        \sverb| "a":{"id":"abcdefghij0123456789", "info_hash":"mnopqrstuvwxyz123456"}}| (\gls*{string})

    \item exemplo de resposta
        \begin{itemize}
            \item com peers \\
                \bverb|d1:rd2:id20:abcdefghij01234567895:token8:aoeusnth|\\
                \bverb|6:valuesl6:axje.u6:idhtnmee1:t2:aa1:y1:re| (\gls*{bencode}) \\
                \sverb|{"t":"aa", "y":"r", "r":{"id":"abcdefghij0123456789",|\\
                \sverb| "token":"aoeusnth", "values":["axje.u", "idhtnm"]}}|
                (\gls*{string})

            \item com nós próximos \\
                \bverb|d1:rd2:id20:abcdefghij01234567895:nodes9:def456...5:token|\\\bverb|8:aoeusnthe1:t2:aa1:y1:re| (\gls*{bencode}) \\
                \sverb|{"t":"aa", "y":"r", "r":{"id":"abcdefghij0123456789",|\\
                \sverb| "token":"aoeusnth", "nodes":"def456..."}}| (\gls*{string})
        \end{itemize}
\end{itemize}