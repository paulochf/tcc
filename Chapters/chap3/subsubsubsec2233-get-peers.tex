%!TEX root = ../../tcc.tex

\subsubsubsection{get\_peers}

É o comando \gls*{rpc} da mensagem \bverb|FIND\_VALUE|, serve para buscar \glspl*{peer}
para um dado um \gls*{hashvalue} identificador de \gls*{torrentfile}, enviado como
valor da chave \bverb|info\_hash|, além do recorrente ID do nó consultante como valor
da chave \bverb|id|.

O funcionamento é equivalente ao comando \bverb|find\_node|, com um detalhe extra: se o
nó que recebeu a mensagem possuir \glspl*{peer} para o \gls*{hashvalue} dado, eles são
informados na forma compacta (6 bytes para cada \gls*{peer}) numa lista \gls*{bencode}
de \glspl*{string}, devolvida como valor da chave \bverb|values|. Por outro lado, caso o
receptor da mensagem não conhecer nós para o \gls*{hashvalue} especificado, a resposta
conterá a chave \bverb|nodes| com os $k$ nós mais próximos desse \gls*{hashvalue}. Em
ambos os casos, uma chave \bverb|token| é informada na resposta, cujo valor é uma
\gls*{string} binária curta, que deverá ser utilizada em futuras mensagens de
\bverb|announce\_peer|.

\begin{comment}
Get peers associated with a torrent infohash. "q" = "get_peers" A get_peers query has two arguments, "id" containing the node ID of the querying node, and "info_hash" containing the infohash of the torrent. If the queried node has peers for the infohash, they are returned in a key "values" as a list of strings. Each string containing "compact" format peer information for a single peer. If the queried node has no peers for the infohash, a key "nodes" is returned containing the K nodes in the queried nodes routing table closest to the infohash supplied in the query. In either case a "token" key is also included in the return value. The token value is a required argument for a future announce_peer query. The token value should be a short binary string.

arguments:  {"id" : "<querying nodes id>", "info_hash" : "<20-byte infohash of target torrent>"}

response: {"id" : "<queried nodes id>", "token" :"<opaque write token>", "values" : ["<peer 1 info string>", "<peer 2 info string>"]}

or: {"id" : "<queried nodes id>", "token" :"<opaque write token>", "nodes" : "<compact node info>"}
Example Packets:

get_peers Query = {"t":"aa", "y":"q", "q":"get_peers", "a": {"id":"abcdefghij0123456789", "info_hash":"mnopqrstuvwxyz123456"}}
bencoded = d1:ad2:id20:abcdefghij01234567899:info_hash20:mnopqrstuvwxyz123456e1:q9:get_peers1:t2:aa1:y1:qe
Response with peers = {"t":"aa", "y":"r", "r": {"id":"abcdefghij0123456789", "token":"aoeusnth", "values": ["axje.u", "idhtnm"]}}
bencoded = d1:rd2:id20:abcdefghij01234567895:token8:aoeusnth6:valuesl6:axje.u6:idhtnmee1:t2:aa1:y1:re
Response with closest nodes = {"t":"aa", "y":"r", "r": {"id":"abcdefghij0123456789", "token":"aoeusnth", "nodes": "def456..."}}
bencoded = d1:rd2:id20:abcdefghij01234567895:nodes9:def456...5:token8:aoeusnthe1:t2:aa1:y1:re

When a node wants to find peers for a torrent, it uses the distance metric to compare
the infohash of the torrent with the IDs of the nodes in its own routing table. It then
contacts the nodes it knows about with IDs closest to the infohash and asks them for the
contact information of peers currently downloading the torrent. If a contacted node
knows about peers for the torrent, the peer contact information is returned with the
response. Otherwise, the contacted node must respond with the contact information of
the nodes in its routing table that are closest to the infohash of the torrent. The
original node iteratively queries nodes that are closer to the target infohash until it
cannot find any closer nodes. After the search is exhausted, the client then inserts
the peer contact information for itself onto the responding nodes with IDs closest to
the infohash of the torrent.

The return value for a query for peers includes an opaque value known as the "token."
For a node to announce that its controlling peer is downloading a torrent, it must
present the token received from the same queried node in a recent query for peers. When
a node attempts to "announce" a torrent, the queried node checks the token against the
querying node's IP address. This is to prevent malicious hosts from signing up other
hosts for torrents. Since the token is merely returned by the querying node to the same
node it received the token from, the implementation is not defined. Tokens must be
accepted for a reasonable amount of time after they have been distributed. The
BitTorrent implementation uses the SHA1 hash of the IP address concatenated onto a
secret that changes every five minutes and tokens up to ten minutes old are accepted.
\end{comment}