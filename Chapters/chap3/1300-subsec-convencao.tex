%!TEX root = ../../tcc.tex

\newpage
\subsection*{Convenção de scrape de trackers}

O endereço de \gls*{scrape} pode ser obtido a partir do endereço de \gls*{announce},
seguindo-se a seguinte convenção:

\begin{enumerate}
    \item comece a partir do endereço de \gls*{announce};

    \item encontre o último caractere \sverb|/|;

    \item se o texto que segue a última \sverb|/| não for \sverb|announce|, é um sinal
        de que o \gls*{tracker} não segue a convenção; e

    \item caso contrário, substituir \sverb|announce| por \sverb|scrape|.
\end{enumerate}

Alguns exemplos:

\begin{itemize}
    \item suportam scrape:
        \begin{enumerate}
            \item \url{http://example.com/announce} $\rightarrow$
                \url{http://example.com/scrape};
            \item \url{http://example.com/announce.php} $\rightarrow$
                \url{http://example.com/scrape.php};
            \item \url{http://example.com/x/announce} $\rightarrow$
                \url{http://example.com/x/scrape}; e
            \item \url{http://example.com/announce?x2\%0644} $\rightarrow$
                \url{http://example.com/scrape?x2\%0644}.
        \end{enumerate}

    \item não suportam scrape:
        \begin{enumerate}
            \item \url{http://example.com/a};
            \item \url{http://example.com/announce?x=2/4}; e
            \item \url{http://example.com/x\%064announce}.
        \end{enumerate}
\end{itemize}