%!TEX root = ../tcc.tex

\newpage
\chapter*{Sobre este trabalho}

Algumas observações devem ser feitas sobre este trabalho, para conhecimento antes da
leitura.

\section*{Cores no texto}

Em alguns momentos percebeu-se que poderia haver confusão semântica. Na tentativa de se
resolver isso, um padrão de escrita foi adotado para o trabalho e, utilizando cores,
dividiu-se em duas funções semânticas;

\begin{itemize}
    \item \bverb|texto|: foi usado quando o significado do trecho destacado era de
        conteúdo não processado pelo programa, ou seja, como foi recebido; e

    \item \sverb|texto|: usado quando o texto destacado já foi processado, já sendo na
        forma de \gls{string}.
\end{itemize}

Essa diferença é notada na seção que trata de dicionários bencoded (capítulo
\ref{sec:bencode}, página \pageref{sec:bencode}), quando são mostrados conteúdos que
representam um dicionário em duas formas diferentes: enquanto
\bverb|d3:foo3:bar6:foobar6:bazbare| é o que se recebe em uma mensagem, possui
representação diferente no computador após ser processado, que é
\sverb|{"foo": "bar", "foobar":| \\ \sverb|"bazbar"}|.

\section*{Termos em inglês}

Foi preferido o uso dos termos técnicos de BitTorrent em inglês às suas traduções, para
que o usuário se habitue com os originais, que são bastante utilizados na área. Por
isso, estes não aparecem em itálico.

Para os outros termos, são escritos em \emph{itálico}.

\section*{Trechos de código e comentários}

Como o objetivo deste trabalhar é apresentar código da linguagem C usado no
Transmission, tiveram momentos que comentários originais foram mantidos para agregar seu
valor à apresentação. Todos esses códigos originais estão comentados entre
\sverb|/* texto */|.

Porém, existem momentos em que não basta a leitura do código do Transmission, então
foram feitos comentários extras. Estes estão escritos usando comentários em
\sverb|// texto|.

\begin{ccode}
    /* Comentarios originais do codigo do Transmission. */

    // comentario de codigo
    // Comentarios extras adicionados posteriormente.
\end{ccode}

\afterpage{\clearpage}