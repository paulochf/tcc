%!TEX root = ../tcc.tex

\chapter{Anatomia do BitTorrent}

O BitTorrent é uma rede \gls{p2p}, onde cada um de seus usuários assume o papel híbrido
de servidor (que fornece os arquivos) e de cliente (que adquire os arquivos). Cada
computador é chamado de \gls{peer}.

Cada transferência por BitTorrent está associada a um \gls{torrentfile}, que contém
\glspl{metadata}, que são informações sobre arquivos que constituem um pacote chamado
\gls{torrent}. Além disso, possui um ou mais endereços de \glspl{tracker}, que são
servidores que mantém listas atualizadas dos \glspl*{peer} que estão compartilhando
aqueles arquivos, atualizadas em curtos períodos de tempo (usualmente 30 minutos).

\begin{comment}
    \begin{figure}[H]
        \centering
        \fbox{\includegraphics[width=0.64\textwidth]{funcionamento.png}}
        \caption{esquema básico do funcionamento do BitTorrent}
        \label{fig:torrent-basics}
    \end{figure}
\end{comment}

Enquanto um \gls*{peer} estiver fazendo download de um \gls*{torrent}, ele é chamado de
\gls{leecher}, pois ainda consumirá dados de outros \glspl*{peer}; quando o download
acabar, passará a ser um \gls{seeder}, que somente enviará dados.

\begin{figure}[ht]
    \centering
    \fbox{\includegraphics[width=0.85\textwidth]{universobt.png}}
    \caption{amostra de uma rede de conexões BitTorrent}
    \label{fig:torrent-universo}
\end{figure}

Os \glspl*{torrentfile} ficam disponíveis em vários sites de índice (às vezes, chamados
de comunidades), como o \href{http://thepiratebay.sx/}{ThePirateBay}, o
\href{http://kickass.to/}{Kickass} ou \href{https://torrentz.eu/}{Torrentz}, muitas
vezes em mais de um deles ao mesmo tempo. Apesar de todo conteúdo compartilhado possuir
um \gls*{torrentfile}, não necessariamente um \gls*{torrentfile} está sendo
compartilhado naquele momento, podendo até mesmo estar extinto.

\Glspl*{peer} que participam do compartilhamento de um \gls*{torrentfile} específico
fazem parte do \gls{swarm}, onde os dados contidos no pacote desse arquivo são
compartilhados com outros por partes.

\begin{figure}[hp]
    \newlength{\myvsize}
    \newlength{\myhsize}
    \setlength{\myvsize}{5mm}
    \setlength{\myhsize}{0.28\textwidth}

    \centering

    \begin{subfigure}[H]{\myhsize}
        \fbox{\includegraphics[width=\textwidth]{Torrentcomp_small-0.png}}
        \caption{}
        \label{fig:torrent-repr-0}
    \end{subfigure}%
    \quad %add desired spacing between images (~, \quad, \qquad or blank line)
    \begin{subfigure}[H]{\myhsize}
        \fbox{\includegraphics[width=\textwidth]{Torrentcomp_small-1.png}}
        \caption{}
        \label{fig:torrent-repr-1}
    \end{subfigure}%
    \quad
    \begin{subfigure}[H]{\myhsize}
        \fbox{\includegraphics[width=\textwidth]{Torrentcomp_small-2.png}}
        \caption{}
        \label{fig:torrent-repr-2}
    \end{subfigure}

    \vspace{\myvsize}

    \begin{subfigure}[H]{\myhsize}
        \fbox{\includegraphics[width=\textwidth]{Torrentcomp_small-3.png}}
        \caption{}
        \label{fig:torrent-repr-3}
    \end{subfigure}%
    \quad
    \begin{subfigure}[H]{\myhsize}
        \fbox{\includegraphics[width=\textwidth]{Torrentcomp_small-4.png}}
        \caption{}
        \label{fig:torrent-repr-4}
    \end{subfigure}%
    \quad
    \begin{subfigure}[H]{\myhsize}
        \fbox{\includegraphics[width=\textwidth]{Torrentcomp_small-5.png}}
        \caption{}
        \label{fig:torrent-repr-5}
    \end{subfigure}

    \vspace{\myvsize}

    \begin{subfigure}[H]{\myhsize}
        \fbox{\includegraphics[width=\textwidth]{Torrentcomp_small-6.png}}
        \caption{}
        \label{fig:torrent-repr-6}
    \end{subfigure}%
    \quad
    \begin{subfigure}[H]{\myhsize}
        \fbox{\includegraphics[width=\textwidth]{Torrentcomp_small-7.png}}
        \caption{}
        \label{fig:torrent-repr-7}
    \end{subfigure}%
    \quad
    \begin{subfigure}[H]{\myhsize}
        \fbox{\includegraphics[width=\textwidth]{Torrentcomp_small-8.png}}
        \caption{}
        \label{fig:torrent-repr-8}
    \end{subfigure}

    \vspace{\myvsize}

    \begin{subfigure}[H]{\myhsize}
        \fbox{\includegraphics[width=\textwidth]{Torrentcomp_small-9.png}}
        \caption{}
        \label{fig:torrent-repr-9}
    \end{subfigure}

    % TODO: trocar menos (-) por hífen (--)
    \caption{simulação de uma transferência torrent: o seeder, na parte
    inferior das figuras, possui todas as 5 partes de um arquivo, que os outros
    computadores - os leechers - baixam de forma independente e paralela. Fonte:
    \cite{fig:torrent-dl}}
    \label{fig:torrent-repr}
\end{figure}

Essas partes variam de acordo com cada \gls*{torrentfile}. O tamanho total do
conjunto de arquivos contidos nesse arquivo é dividido em partes de tamanho fixo
(geralmente 256kB) e transmitido de forma independente das outras, seguindo a ordem
estabelecida pelo algoritmo de troca de partes (explicado na seção~\ref{titfortat}),
ordem essa que varia de acordo com o estado atual do \gls*{swarm} para o arquivo
\gls*{torrentfile} dado.

Todos esses agentes possuem relações de múltiplas entre si. Por exemplo, um mesmo
\gls*{torrentfile} pode estar indexado por vários sites indexadores. Como veremos
nos capítulos seguintes, eles contém uma informação que os identificam unicamente entre
si, mantendo a consistência de dados através desses vários sites de busca. Outra
observação a ser feita é que um \gls*{peer} pode estar baixando mais de um
\gls*{torrentfile} simultaneamente, ou seja, participando de dois \glspl*{swarm} ao mesmo
tempo. Por fim, em alguns casos um \gls*{torrentfile} possui grande quantidade de
\glspl*{peer}, havendo necessidade de se dividir o \gls*{swarm} em algumas partes para
fins de escalabilidade da rede formada.

\newpage
\subsection*{Arquivo .torrent}

Ao se adicionar um \gls*{torrentfile} em um programa cliente, ocorrem muitas transmissões de
dados antes do download de fato. Para demonstrar isso, usaremos um arquivo torrent do
filme ``A Noite dos Mortos Vivos'' de 1960 \cite{torrent-file}, que é de domínio público
e livre de direitos autorais.

Se abrirmos esse arquivo, veremos uma grande \gls{string}, caracteres diferentes e
incomuns, formando um conteúdo ilegível (na seção binária) e sob uma forma compacta,
mostrado abaixo.

\begin{listing}[H]
    \begin{minted}[
        linenos,
        frame=single,
        numbersep=6pt,
        baselinestretch=1,
        fontfamily=courier,
        gobble=4,
        fontsize=\scriptsize
    ]{text}
    d8:announce36:http://bt1.archive.org:6969/announce13:announce-listll36:http://bt1.
    archive.org:6969/announceel36:http://bt2.archive.org:6969/announceee7:comment13:crea
    tiondatei1343715473e4:infod5:filesld5:crc328:030208fe6:lengthi4127671704e3:md532:627
    f5a428f9e454ccfcb29d31b87169a5:mtime10:10794024804:pathl29:night_of_the_living_dead.
    mpege4:sha140:5e44bb1b3f700240249a5287c64dc02dc56d034bee4:name24:night_of_the_living
    _dead12:piecelengthi4194304e6:pieces23720:<binary>

    (...)

    e6:locale2:en5:title24:night_of_the_living_dead8:url-listl28:http://archive.org/
    download/39:http://ia600301.us.archive.org/22/items39:http://ia700301.us.archive.
    org/22/itemsee
    \end{minted}

    \caption{trecho do conteúdo do arquivo .torrent do filme ``A Noite dos Mortos
    Vivos'', de 1960 \cite{torrent-file}, com a parte binária truncada}
    \label{lst:torrent-file-raw}
\end{listing}

Esse conteúdo está organizado usando o formato \gls{bencode}, que é um formato de
codificação compacta de arquivos especial para arquivos \gls*{torrentfile} e não é legível
ao ser humano. Com alguma formatação, podemos enxergar os componentes separadamente,
como mostra o código~\ref{lst:torrent-file-code}.

Esse conteúdo tem significado, sendo utilizado da seguinte forma
\cite{wikitheory:bencoding}:

\begin{itemize}
    \item \textbf{\glspl*{string}} são prefixos de números na base 10 que representam
        comprimentos, seguidos por um caractere \bverb|:| e então o conteúdo. Por
        exemplo, na linha 2, \bverb|8:announce| corresponde à \gls*{string}
        \sverb|"announce"|.

    \item \textbf{números} são representados por um \bverb|i|, seguidos do valor na
        base 10 (sem qualquer limite, mas sem zeros precedentes como em \bverb|0003|, e
        podendo ser negativo), e terminados por um \bverb|e|. Por exemplo, na linha 11,
        \bverb|i1343715473e| corresponde ao número \sverb|1343715473|.

    \item \textbf{listas} são formadas por \bverb|l|, seguidos por seus elementos
        (também no formato \gls*{bencode}), e então terminados por \bverb|e|. Por
        exemplo, \bverb|l3:foo3:bare| corresponde a \sverb|["foo", "bar"]|. No código
        acima, é presente entre as linhas 43 e 47.

    \item \textbf{dicionários} são definidos por \bverb|d|, seguidos de uma lista
        alternada de chaves e seus valores correspondentes, terminando com \bverb|e|,
        onde as chaves devem estar ordenadas usando-se comparação binária ao invés da
        usual alfanumérica. Por exemplo, a \gls*{string}
        \bverb|d3:foo3:bar6:foobar6:bazbare| corresponde ao dicionário puro \\
        \sverb|{"foo": "bar", "foobar": "bazbar"}|, e a estrutura mais complexa dada por
        \\ \bverb|d3:fool6:foobar3:bazee| equivale a \sverb|{"foo": ["foobar", "baz"]}|.
\end{itemize}

\begin{listing}[H]
    \begin{minted}[
        linenos,
        frame=single,
        numbersep=6pt,
        baselinestretch=1,
        fontfamily=courier,
        gobble=4,
        fontsize=\scriptsize
    ]{text}
    d
        8:announce
        36:http://bt1.archive.org:6969/announce
        13:announce-list
        l
            l36:http://bt1.archive.org:6969/announcee
            l36:http://bt2.archive.org:6969/announcee
        e
        7:comment
        13:creation date
        i1343715473e
        4:info
        d
            5:files
            l
                d
                    5:crc32
                    8:030208fe
                    6:length
                    i4127671704e
                    3:md5
                    32:627f5a428f9e454ccfcb29d31b87169a
                    5:mtime
                    10:1079402480
                    4:path
                    l29:night_of_the_living_dead.mpege
                    4:sha1
                    40:5e44bb1b3f700240249a5287c64dc02dc56d034b
                e
            e
            4:name
            24:night_of_the_living_dead
            12:piece length
            i4194304e
            6:pieces
            23720:<binary>
        e
        6:locale
        2:en
        5:title
        24:night_of_the_living_dead
        8:url-list
        l
            28:http://archive.org/download/
            39:http://ia600301.us.archive.org/22/items
            39:http://ia700301.us.archive.org/22/items
        e
    e
    \end{minted}
    \caption{trechos formatados de forma legível do conteúdo do arquivo .torrent do
    filme ``A Noite dos Mortos Vivos'', de 1960 \cite{torrent-file}, com a parte
    binária truncada}
    \label{lst:torrent-file-code}
\end{listing}

\subsection*{Magnet Link}

Além do \gls*{torrentfile}, existe uma outra forma de se transmitir os
\glspl*{metadata} necessários para se iniciar a transmissão, que é utilizando
\gls{magnetlink}.

\Glspl*{magnetlink}, ao contrário dos arquivos \gls*{torrentfile}, não estão gravados em
algum dispositivo de armazenamento. Basicamente, é um esquema de \gls{uri} usado
exclusivamente para o protocolo.

No caso citado, o site de origem do \gls*{torrentfile} que estamos usando não fornece um
\gls*{magnetlink} oficialmente. Porém, o Transmission consegue construir uma \gls*{uri}
a partir do arquivo original, para fins de compartilhamento direto. O resultado, após
decodificar o endereço para um formato legível (retirando a codificação de caracteres
especiais \cite{wiki:urlencode}) foi o seguinte:

\begin{listing}[ht!]
    \begin{minted}[
        linenos,
        frame=single,
        numbersep=6pt,
        baselinestretch=1,
        fontfamily=courier,
        gobble=4,
        fontsize=\scriptsize
    ]{text}
    magnet:?xt=urn:btih:72d7a3179da3de7a76b98f3782c31843e3f818ee
    &dn=night_of_the_living_dead
    &tr=http://bt1.archive.org:6969/announce&tr=http://bt2.archive.org:6969/announce
    &ws=http://archive.org/download/
    &ws=http://ia600301.us.archive.org/22/items/&ws=http://ia700301.us.archive.org/22/items/
    \end{minted}
    \caption{link magnético do arquivo .torrent do filme ``A Noite dos Mortos Vivos''
    , de 1960 \cite{torrent-file}, com parâmetros divididos entre linhas para melhor
    visualização}
    \label{lst:torrent-file-magnet-link}
\end{listing}

Esse endereço é composto por vários pares de nomes de parâmetros (sem qualquer ordem
específica) e seus respectivos valores, formando uma \gls{querystring}. Podemos
dividir esse endereço em partes, cada uma tendo o seu significado:

\begin{itemize}
    \item \textbf{xt}: parâmetro para \emph{exact topic}, ou tópico exato, contém a
        informação mais importante do \gls*{magnetlink}: o identificador único de
        \gls*{torrentfile}. Serve para encontrar e verificar os arquivos especificados.
        No caso, \bverb|urn:btih:<hash>| corresponde ao \gls{urn} \sverb|btih|
        (\emph{BitTorrent Info Hash}), que é a \gls*{string} \gls{hashvalue} resultado
        da \gls{hashfunction} SHA-1, convertida para hexadecimal

    \item \textbf{dn}: parâmetro que contém o \emph{display name}, ou nome de
        visualização, contém um nome que é mostrado para o usuário, por conveniência

    \item \textbf{tr}: o \emph{address tracker}, ou endereço do \gls*{tracker}, onde o
        programa cliente vai procurar as informações de \glspl*{peer}

    \item \textbf{ws}: endereço do arquivo para \emph{webseed}, ou fornecimento web,
        que é o endereço do arquivo em um servidor HTTP ou FTP, sendo utilizado como
        alternativa a um \gls*{swarm} problemático \cite{wiki:torrent}
\end{itemize}

\section{Busca por informações}

Quando adicionamos um \gls*{torrentfile} ao Transmission, o programa salva as informações
em disco durante todo o período que estiverem sendo gerenciados por ele. Caso tenha
sido por meio de um arquivo de \glspl*{metadata}, uma cópia deste é salva em uma pasta
pré-definida para seu controle interno; caso seja por \gls*{magnetlink}, um novo
arquivo é criado contendo as informações obtidas por ele, por questões de praticidade,
para que não necessite fazer essa aquisição dos dados novamente ao ser aberto ou quando
alguma transferência for pausada e depois continuada.

Após esse processo de arquivamento, o programa processa essas informações salvas para
deixá-las carregadas em memória a fim de obter o \gls*{hashvalue} que identifica o
\gls*{torrentfile}.

\cfile[label="./libtransmission/metainfo.c:367"]{./Codes/chap3/001-leiturametadata.c}

Se aquele arquivo para controle interno tiver sido criado por conta de um
\glspl*{magnetlink}, não possuirá a chave \bverb|info| em seu conteúdo, mas deverá
conter as chaves \bverb|urn:btih:<hash>| e \bverb|info_hash|.

\cfile[label="./libtransmission/metainfo.c:367"]{./Codes/chap3/002-leiturametadata2.c}

Já se o arquivo para controle interno tiver sido gerado como cópia do arquivo torrent,
possuirá o mesmo dicionário original, que contém a chave \bverb|info_hash|, que é
utilizada para calcular o \gls*{hashvalue} do arquivo.

\cfile[label="./libtransmission/metainfo.c:367"]{./Codes/chap3/003-leiturametadata3.c}

Outras informações podem ser recuperadas dependendo do formato de origem do
\gls*{torrentfile}, como a privacidade, lista de arquivos e seus respectivos tamanhos,
\gls*{hashvalue} de cada parte, entre outras. Por fim, termina coletando os endereços de
\gls{announce}.

\cfile[label="./libtransmission/metainfo.c:367"]{./Codes/chap3/004-leiturametadata4.c}

\subsection*{Announce}

Para cada \gls*{swarm} gerenciado, o \gls*{tracker} possui uma lista dos \glspl*{peer}
que participam dele, que é enviada ao \gls*{peer} que a requer por meio de uma
requisição \gls{httpget}. Quando essa requisição é recebida pelo \gls*{tracker}, este
inclui ou atualiza um registro para o \gls*{peer} solicitante e devolve uma lista de 50
\glspl*{peer} aleatórios, de forma uniforme, que fazem parte do \gls*{swarm}. Não
havendo essa quantidade total, a lista toda é enviada ao requisitante; caso contrário,
a aleatoriedade proporciona uma diversidade de listas enviadas, proporcionando robustez
ao sistema \cite{wikitheory:tracker-response}.

Esse contato entre um \gls*{peer} e um \gls*{tracker} é chamado de \gls{announce}, que
pode ser feito usando-se tanto com o protocolo \gls{tcp} bem como \gls{udp}, e é onde
\glspl*{peer} podem passam várias informações

\begin{itemize}
    \item \textbf{info\_hash}: \gls*{hashvalue} de 20 bytes resultante da
    \gls*{hashfunction} SHA-1 com \gls*{urlencode} do valor da chave \bverb|info| do
    arquivo de \gls*{metadata}

    \item \textbf{peer\_id}: \gls*{string} com \gls*{urlencode} de 20 bytes usado como
    identificador único do programa cliente, gerado no ínício da sua execução. Para
    isso, provavelmente deve incorporar informações do computador, a fim de se gerar um
    valor único para o computador

    \item \textbf{uploaded}: a quantidade total de dados, em bytes, enviados desde
    quando o cliente enviou o primeiro aviso ao \gls*{tracker}

    \item \textbf{downloaded}: a quantidade total de dados, em bytes, recebidos desde
    quando o cliente enviou o primeiro aviso ao \gls*{tracker}

    \item \textbf{left}: a quantidade total de dados, em bytes, que faltam para o
    requisitante terminar o download do \gls*{torrentfile} e passe a ser um \gls*{seeder}

    \item \textbf{compact} (opcional): se o valor passado for 1, significa que o
    requisitante aceita respostas compactas. A lista de \glspl*{peer} enviada é
    substituída por uma \gls*{string} de peers, cada um com 6 bytes, onde os 4 bytes
    iniciais são o host e os 2 bytes finais são a porta de transmissão. Por exemplo, o
    endereço IP 10.10.10.5:80 seria transmitido como \bverb|0A 0A 0A 05 00 80|. Deve-se
    observar que alguns \glspl*{tracker} suportam somente conexões deste tipo para
    otimização da utilização da banda de rede e, pra isso, ou recusam requisições sem
    \bverb|compact=1| ou, caso não as recusem, enviam respostas compactas a menos que a
    requisição possua \bverb|compact=0|

    \item \textbf{no\_peer\_id} (opcional): sinaliza que o \gls*{tracker} por omitir o
    campo \textcolor{Bittersweet}{\texttt{peer\_id}} no dicionário de \glspl*{peer}. É
    ignorado caso o modo compacto esteja habilitado

    \item \textbf{event} (opcional): pode possuir os valores \sverb|started|
    (iniciado), \sverb|completed| (terminado), \sverb|stopped| (parado), ou vazio para
    não especificar.

    \begin{itemize}
        \item \emph{started} : a primeira requisição para o \gls*{tracker} deve enviar
        este valor
        \item \emph{stopped} : avisa que o programa cliente está fechando
        \item \emph{completed} : quando o download que estava ocorrendo termina, ou
        seja, não é enviado quando o programa cliente inicia com o \gls*{torrentfile} em
        100\%
    \end{itemize}

    \item \textbf{port} (opcional): o número da porta de conexão que o programa cliente
    está escutando por transmissões de dados. Em geral, portas reservadas para
    BitTorrent estão entre a 6881 e a 6889. Se esse for o caso, pode ser omitido

    \item \textbf{ip} (opcional): o endereço IP verdadeiro do requisitante no formato
    legível do IPv4 (4 conjuntos de número de 0 a 255 separados por \bverb|.|) ou do
    IPv6 (8 conjuntos de números hexadecimais de 4 dígitos separados por \bverb|:|).
    Não é sempre necessário, pois o endereço pode ser conhecido através da requisição.
    Assim, é usado quando o programa cliente está se comunicando com o \gls*{tracker}
    através de um \gls{proxy} ou quando ambos cliente e \gls*{tracker} estão no mesmo
    lado local de um \gls{nat}, pois nesse caso o endereço IP não é roteável

    \item \textbf{numwant} (opcional): quantidade de \glspl*{peer} que o requisitante
    gostaria de receber do \gls*{tracker}. É permitido valor zero. Se omitido, assume
    valor padrão de 50

    \item \textbf{key} (opcional): mecanismo de idetificação adicional para o programa
    cliente provar sua identidade caso tenha ocorrido mudança no seu endereço IP

    \item \textbf{trackerid} (opcional): se a resposta de um \gls*{announce} anterior
    continha o endereço IP de um \gls*{tracker}, deve ser enviado neste campo
\end{itemize}

%\newpage
%\cfile{./Codes/chap3/002-announce.c}

Essa comunicação ocorre nas seguintes situações:

\begin{itemize}
    \item no primeiro contato do \gls*{peer}, para que ele tenha acesso a um
        \gls*{swarm}

    \item a cada período de tempo, estipulado pelo tracker, para que o \gls*{peer}
        continue mostrando que ainda está conectado, além de pode receber endereços de
        \glspl*{peer} novos

    \item quando a quantidade de \glspl*{peer} conhecidos que estão ativos é menor do 5

    \item quando terminar o download, notificando que passou a ser um \gls*{seeder}

    \item quando sai do \gls*{swarm}, seja por desconexão ou por encerramento do
        programa cliente
\end{itemize}

Além do \gls*{announce}, outra forma de troca de informação entre \glspl*{peer} e
\glspl*{tracker} é pelo \gls{scrape}. Geralmente usado pelos programas cliente para
decidir quando realizar um \gls*{announce}, informa o número de \glspl*{peer},
\glspl*{leecher} e \glspl*{seeder} de uma lista de um ou mais \glspl*{torrent}. É dessa
forma que os sites de indexação sabem dessas informações e as apresentam nas páginas.

\todo[inline]{explicar como o Transmission usa o torrent}

\section{Fontes de arquivos}

Mostrarei o processamento dos dados adquiridos na seção anterior e como ele organiza a lista das fontes de arquivos usando a tabela hash DHT Kademlia.

\section{Jogo da troca de arquivos}
\label{titfortat}

Explicarei o algoritmo tit-for-tat padrão do protocolo BitTorrent, que vem da Teoria dos Jogos, e como o Transmission o implementa.

\afterpage{\clearpage}