%% -------------------------------------------------------------------------------------
%% tcc.tex -- MAIN FILE
%% -------------------------------------------------------------------------------------
\documentclass[a4paper,12pt,oneside]{book}

%% -------------------------------------------------------------------------------------
%% PACKAGES
%% -------------------------------------------------------------------------------------

\usepackage[T1]{fontenc}
\usepackage[utf8]{inputenc}
\usepackage[brazilian]{babel}
\usepackage[brazilian]{translator}

% Cor
\usepackage[table,dvipsnames]{xcolor}

\usepackage{amsmath}                                % \begin{align*}
\usepackage{upquote}                                % trocar aspa por apóstrofo em tt
\usepackage{csquotes}                               % aspas espertas (?)
\usepackage[font={small,it},labelfont=bf]{caption}  % \begin{caption}
\usepackage{subcaption}                             % \begin{subfigure}
\usepackage{graphicx}                               % \includegraphics
\usepackage{verbatim}                               % \begin{comment}
\usepackage{fullpage}                               % margens menores
\usepackage{indentfirst}                            % indentação dos 1ºs parágrafos
\usepackage{setspace}                               % \setstretch
\usepackage{afterpage}                              % \afterpage
\usepackage{newverbs}                               % \newverbcommand

% \usepackage{layouts}                              % debug dos tamanhos do documento
% \afterpage http://tex.stackexchange.com/q/88657

% \newminted{cpp}
% [chapter] é para numerar usando o capítulo
\usepackage[chapter]{minted}

% Links azuis
\usepackage[unicode,colorlinks=true,linkcolor=blue]{hyperref}

% Desligando a ligatura do 'fi'
% http://www.latex-community.org/forum/viewtopic.php?f=5&t=953#p13896
\usepackage[babel=true]{microtype}
\DisableLigatures[f]{encoding = *}

% Usa estilo "Natbib" para as referências bibliográficas
\usepackage[backend=biber,bibencoding=utf8,style=numeric-comp]{biblatex}

% \makeglossaries
% http://en.wikibooks.org/wiki/LaTeX/Glossary#Using_defined_terms
% sanitize=none: http://tex.stackexchange.com/a/14930/5125
\usepackage[xindy,toc,sanitize=none]{glossaries}

% anotações TODO
\usepackage[colorinlistoftodos,portuguese]{todonotes}

%% -------------------------------------------------------------------------------------
%% CONFIGURATIONS
%% -------------------------------------------------------------------------------------

\setcounter{secnumdepth}{5}
\setcounter{tocdepth}{5}

%%%%%%%%%%%%%%%%%%%%%
%  ???
%%%%%%%%%%%%%%%%%%%%%
% padding de imagens
% \setlength\fboxsep{2mm}
%
% Links só nos primeiros usos de gls etc
% http://tex.stackexchange.com/a/109137/5125
%
% from http://tex.stackexchange.com/q/57151
% \usepackage{accsupp}
% \newcommand{\emptyaccsupp}[1]{\BeginAccSupp{ActualText={}}#1\EndAccSupp{}}

\newcommand{\Oh}{\textup{O}}

\newcommand{\subsubsubsection}[1]{\indent \textbf{#1}}

\newcommand{\glink}[1]{\glslink{#1}{\glsentryname{#1}}}

\newcommand{\glinkpl}[1]{\glslink{#1}{\glsentryplural{#1}}}

\addbibresource{bibliografia.bib}

\loadglsentries{glossario.tex}
\makeglossaries

% Seta pasta de figuras
\graphicspath{{Figures/}{Figures/chap3/}{Figures/chap4/}{Figures/chap5/}}

\includeonly{
    Chapters/Chapter0,
    Chapters/Chapter1,     % 1. Introdução
    Chapters/Chapter2,     % 2. Histórico
    Chapters/Chapter3,     % 3. Anatomia do BitTorrent
    Chapters/Chapter4,     % 4. Ciência da Computação no Transmission
    Chapters/Chapter5,     % 5. Transmission e o BCC
    Chapters/Chapter6,     % 6. Comentários Finais
    Chapters/Chapter7,     % 7. Glossário
    Chapters/Chapter8,     % 8. Bibliografia
    Chapters/Chapter9      % 9. Parte Subjetiva
}

\hypersetup{
     colorlinks   = true,
     citecolor    = Sepia
}

% Fonte 'Times' do sistema (mais bonita)
% http://www.tug.org/pracjourn/2006-1/schmidt/schmidt.pdf
\renewcommand{\rmdefault}{ptm}

% cor de fundo do \verb
% http://tex.stackexchange.com/a/120985/5125
\newverbcommand{\bverb}
{\begin{lrbox}{\verbbox}}
{\end{lrbox}\colorbox{gray!30}{\box\verbbox}}

\newverbcommand{\sverb}{\color{Bittersweet}}{}

%%%%%%%%%%%%%%%%%%%%%%%%%%%%%%%%%
% Do manual do todonote
%%%%%%%%%%%%%%%%%%%%%%%%%%%%%%%%%
\newcommand{\todorefs}[1]{
    \todo[
        color=Orchid,
        bordercolor=Plum,
        inline
    ]{#1}
}

\newcommand{\todoquestion}[1]{
    \todo[
        color=LimeGreen,
        bordercolor=OliveGreen,
        inline
    ]{#1}
}

\newcommand{\todoerrors}[1]{
    \todo[
        color=BrickRed,
        bordercolor=BrickRed,
        inline
    ]{#1}
}

\newcommand{\todoimg}[1]{
    \missingfigure{#1}
}
%%%%%%%%%%%%%%%%%%%%%%%%%%%%%%%%%

%%%%%%%%%%%%%%%%%%%%%
%  Minted
%%%%%%%%%%%%%%%%%%%%%

% Do manual do minted
\renewcommand\listingscaption{Conteúdo textual}
% \renewcommand\listoflistingscaption{Lista de Códigos-fonte}

% http://tex.stackexchange.com/a/99656/5125
\renewcommand{\listoflistings}{%
    \cleardoublepage
    \phantomsection
    \addcontentsline{toc}{chapter}{\listoflistingscaption}%
    \listof{listing}{\listoflistingscaption}%
}

% usando valores predefinidos
% \begin{cppcode}
% template <typename T>
%   T id(T value) {
%     return value;
%   }
% \end{cppcode}
%
%
% sobrescrevendo valores
% \begin{cppcode*} {linenos=false,frame=single}
% template <typename T>
%   T id(T value) {
%     return value;
%   }
% \end{cppcode*}
%
% labels: http://citeseerx.ist.psu.edu/viewdoc/
%      download?doi=10.1.1.169.9130&rep=rep1&type=pdf
\newminted{c}{
    linenos,
    mathescape,
    frame=single,
    samepage=true,
    numbersep=6pt,
    baselinestretch=1,
    fontfamily=courier,
    fontsize=\scriptsize
}

\newmintedfile[cfile]{c}{
    linenos,
    mathescape,
    frame=single,
    samepage=true,
    numbersep=6pt,
    baselinestretch=1,
    fontfamily=courier,
    fontsize=\scriptsize
}

%% -------------------------------------------------------------------------------------
%% DOCUMENT
%% -------------------------------------------------------------------------------------

\begin{document}

\thispagestyle{empty}

% Numeração das primeiras páginas em números romanos
\frontmatter
\pagenumbering{Roman}

% OBS:
% \clearpage vs \newpage
% http://tex.stackexchange.com/a/45619/5125
%
% \afterpage{\clearpage}
% http://tex.stackexchange.com/q/88657


%%%%%%%%%%%%%%%%%%%%%
%  Capa
%  https://www.sharelatex.com/blog/2013/08/09/thesis-series-pt5.html
%%%%%%%%%%%%%%%%%%%%%
\begin{titlepage}
    \begin{center}
        %\addcontentsline{toc}{chapter}{Capa}
        \vspace*{1cm}

        \Huge
        \textbf{Anatomia do BitTorrent}

        \vspace{0.5cm}
        \LARGE
        a Ciência da Computação no Transmission

        \vspace{2.5cm}

        \textbf{
            Paulo Cheadi Haddad Filho \\
            Orientador: José Coelho de Pina
        }

        \vfill

        Trabalho de Formatura Supervisionado

        \vspace{3.5cm}

        \includegraphics[scale=0.1]{logo-ime.png}

        \vspace{2cm}

        \Large
        %Instituto de Matemática e Estatística\\
        Universidade de São Paulo\\
        São Paulo, 2013

    \end{center}
    \afterpage{\clearpage}

\end{titlepage}

% Melhor ter fonte menor e espaçamento de linha maior
\setstretch{1.3}



%%%%%%%%%%%%%%%%%%%%%
% Páginas frontais
%%%%%%%%%%%%%%%%%%%%%

% Sumário
\tableofcontents

% Lista de códigos fonte
% \listoflistings

% Lista de figuras
\listoffigures

% Lista de TODOs
\listoftodos
\todototoc

\afterpage{\clearpage}



%%%%%%%%%%%%%%%%%%%%%
% Conteúdo
%%%%%%%%%%%%%%%%%%%%%

% Tamanhos / Distâncias
% http://tex.stackexchange.com/a/24468/5125
% http://www.las.ic.unicamp.br/pub/ctan/macros/latex/contrib/layouts/layman.pdf
\setlength{\baselineskip}{15pt}                % Entre linhas do texto
%\setlength{\parindent}{10pt}                  % Entre parágrafos
\setlength{\parskip}{15pt}                     % Entre parágrafos
\setlength{\textfloatsep}{1.25\baselineskip}   % Entre floats [t] ou [b] e parágrafos
\setlength{\intextsep}{1.25\baselineskip}      % Entre floats [h] e parágrafos
\setlength{\abovecaptionskip}{.3\parskip}      % Acima das legendas
\setlength{\belowcaptionskip}{-\baselineskip}  % Abaixo das legendas
\addtolength{\belowcaptionskip}{1.2ex}
\setlength{\topsep}{0pt}                       % Espaço superior de listas 1
\setlength{\partopsep}{-\baselineskip}         % Espaço superior de listas 2
\addtolength{\partopsep}{1.2ex}

% Numeração normal em indo-arábico
\mainmatter

% Inclusões dos capítulos
%!TEX root = ../tcc.tex

\newpage
\chapter*{Sobre este trabalho}

Algumas observações devem ser feitas sobre este trabalho, para conhecimento antes da
leitura.

\section*{Cores no texto}

Em alguns momentos percebeu-se que poderia haver confusão semântica. Na tentativa de se
resolver isso, um padrão de escrita foi adotado para o trabalho e, utilizando cores,
dividiu-se em duas funções semânticas;

\begin{itemize}
    \item \bverb|texto|: foi usado quando o significado do trecho destacado era de
        conteúdo não processado pelo programa, ou seja, como foi recebido; e

    \item \sverb|texto|: usado quando o texto destacado já foi processado, já sendo na
        forma de \gls{string}.
\end{itemize}

Essa diferença é notada na seção que trata de dicionários bencode (capítulo
\ref{sec:bencode}, página \pageref{sec:bencode}), quando são mostrados conteúdos que
representam um dicionário em duas formas diferentes: enquanto
\bverb|d3:foo3:bar6:foobar6:bazbare| é o que se recebe em uma mensagem, possui
representação diferente no computador após ser processado, que é
\sverb|{"foo": "bar", "foobar":| \\ \sverb|"bazbar"}|.

\section*{Termos em inglês}

Foi preferido o uso dos termos técnicos de BitTorrent em inglês às suas traduções, para
que o usuário se habitue com os originais, que são bastante utilizados na área. Por
isso, estes não aparecem em itálico.

Para os outros termos, são escritos em \emph{itálico}.

\section*{Trechos de código e comentários}

Como o objetivo deste trabalho é apresentar código da linguagem C usado no Transmission,
em alguns momentos, o código original que seria mostrado não era muito legível. Por
causa disso, eles foram modificados apenas para melhorar sua ilustração, não perdendo
funcionalidade. Essas alterações envolveram omissões de trechos de código irrelevantes
(por exemplo, verificação de erros) e trocas de valores, pré-definidos como constantes,
para seus valores absolutos.

Além disso, alguns comentários originais foram mantidos para garantir a essência do
código apresentado. Todos esses originais estão comentados entre \sverb|/* ... */|.

Porém, existem momentos em que a leitura do código do Transmission não é suficiente, e,
nesses casos, foram feitos comentários extras usando \sverb|// ...|.

\begin{ccode}
    /* Comentarios originais do codigo do Transmission. */

    // comentario de codigo
    // Comentarios extras adicionados posteriormente.
\end{ccode}

\afterpage{\clearpage}     % 0. Disclaimer
%!TEX root = ../tcc.tex

\chapter{Parte Objetiva}

\lhead{\emph{Introdução}}
\section{Introdução}

Aqui vou explicar o objetivo do trabalho e o que será mostrado ao longo dele.

\clearpage     % 1. Introdução
%!TEX root = ../tcc.tex

\chapter{Napster, Gnutella, eDonkey e BitTorrent}

Para entendermos como e por que o BitTorrent se tornou o que é hoje, devemos voltar um
pouco no tempo e rever a história que precedeu à sua criação, que é no fim da década dos
anos 1990.

\section{Período pré-torrent}

Entre o final dos anos 80 e o início dos 90 \cite{wiki:fs,wiki:fs-timeline}, a
Internet deixou de ser uma rede de computadores usada somente por entidades
governamentais, laboratórios de pesquisa e universidades, passando a ter seu acesso
comercializado para o público em geral pelos \glspl{isp} \cite{wiki:isp}. Com o
advento do \gls{mp3} \cite{wiki:mp3}, no final de 1991, e do seu primeiro reprodutor
de áudio \gls*{mp3} Winamp, o tráfego da Internet cresceu devido ao aumento da troca
direta desse tipo de arquivo.

Entre 1998 e 1999, dois sites de compartilhamento gratuito de músicas foram criados: o
MP3.com \cite{wiki:mp3.com}, que era um site de divulgação de bandas independentes,
e o \gls{audiogalaxy} \cite{wiki:audiogalaxy.com,revista:pnp}. Mais popular que o
primeiro, o \gls*{audiogalaxy} era um site de busca de músicas, sendo que o download e
upload eram feitos a partir de um software cliente. A lista de músicas procuradas era
enviada pelo site para o computador onde o usuário tinha instalado o cliente, que então
conectava com o computador do outro usuário, que era indicado pelo servidor. A lista
possuía todos os arquivos que um dia passaram pela sua rede. Se algum arquivo fosse
requisitado, mas o usuário que o possuísse não estivesse conectado, o servidor central
do \gls*{audiogalaxy} fazia a ponte, pegando o arquivo para si e enviando-o para o
cliente do requisitante em seu próximo login.

Os 3 anos seguintes à criação desses dois sites foram muito produtivo ao mundo das
\glspl{p2p} de modo geral, onde surgiram alguns protocolos desse paradigma e inúmeros
softwares que os implementavam. Os mais relevantes foram o Napster, o Gnutella, o
eDonkey e o BitTorrent.

\subsection{Napster}

Em maio de 1999, surgiu o Napster \cite{wiki:napster}, um programa de compartilhamento
de \gls*{mp3} que inovou por desfigurar o usual modelo cliente-servidor, no qual um
servidor central localizava os arquivos nos usuários e fazia a conexão entre estes,
onde ocorriam as transferências. O Napster foi contemporâneo ao \gls*{audiogalaxy}, e
ambos fizeram muito sucesso por cerca de 2 anos, até que começaram as ações judiciais
contra ambos os programas.

Não demorou muito tempo para a indústria da música entrar em ação contra a troca de
arquivos protegidos por direitos autorais sem autorização dos detentores de tais
direitos pela Internet. Seu primeiro alvo foi o Napster, em dezembro de 1999, quando a
\gls{riaa} entrou com processo judicial representando várias gravadoras, alegando
quebra de direitos autorais \cite{site:napster-riaa}. Em abril de 2000, foi a vez da
banda Metallica processar, como retaliação à descoberta de que uma música ainda não
lançada oficialmente já circulava na rede
\cite{site:napster-metallica,site:napster-metallica-orig}. Um mês depois, outra ação
judicial, agora encabeçada pelo rapper Dr. Dre, que tinha feito um pedido formal para a
retirada de seu material de circulação \cite{site:napster-drdre-orig}. Isso fez com que
o Napster recebesse atenção da mídia, ganhando popularidade e atingindo os 20 milhões
de usuários em meados do ano 2000 \cite{site:napster-use-2000}.

Em 2001, esses imbróglios judiciais resultaram numa liminar federal que ordenou que o
Napster retirasse o conteúdo protegido das entidades representadas pela \gls*{riaa}. O
Napster tentou cumprir a ordem judicial, mas a juíza do caso não ficou satisfeita,
ordenando então, em julho daquele ano, o desligamento da rede enquanto não conseguisse
controlar o conteúdo que trafegava ali \cite{wiki:napster}. Em setembro, o Napster fez
um acordo \cite{wiki:napstervsriaa}, onde pagou 26 milhões de dólares pelos danos já
causados, pelo uso indevido de músicas, e mais 10 milhões de dólares pelos danos futuros
envolvendo royalties. Para pagar esse valor, o Napster tentou cobrar o serviço que
prestava de seus usuários, que acabaram migrando de rede \gls*{p2p}, inclusive para o
\gls*{audiogalaxy}. Não conseguindo quitar o acordo, em 2002, o Napster decreta
falência e é forçado a liquidar seus ativos. De lá para cá, foi negociado algumas vezes,
e, atualmente, pertence ao site Rhapsody \cite{site:napster-rhapsody}.

O sucesso do Napster, mesmo que por curto período tempo, mostrou o potencial que as
redes \gls*{p2p} poderiam ter, e com isso, novos softwares e protocolos de redes foram
sendo lançados, sempre tentando se diferenciar dos seus antecessores, a fim de não serem
novos alvos de ações judiciais. A solução para isso foi tentar descentralizar os
mecanismos de indexação e de busca, que foram os pontos fracos do Napster.

\subsection{Gnutella}

O sucessor foi o \gls{gnutella}, que em março de 2000 \cite{wiki:gnutella}, surgiu como
uma resposta de domínio público, feita com ``gambiarras'', para os problemas que o
Napster encontrou com relação às acusações de violação de direitos autorais. Enquanto o
Napster possuía em sua estrutura um servidor central, fato este que foi explorado em
seu julgamento como prova de que o sistema encorajava a violação de direitos autorais,
o \gls*{gnutella} foi modelado como um sistema \gls*{p2p} puro, onde todos os
\glspl*{peer} são completamente iguais, sendo responsáveis pelos seus próprios atos.

O \gls*{gnutella} disponibiliza arquivos da mesma forma que o Napster \cite{book:birman}
, mas sem a limitação de ser de em formato de música, ou seja, qualquer arquivo pode
ser compartilhado. A diferença mais significativa entre os dois protocolos é o
algoritmo de busca: a abordagem do \gls*{gnutella} é baseada numa forma de \gls{anycast}
. Isso envolve duas partes: a primeira, sobre como cada usuário é conectado a outros
nós e mantém a lista dessas conexões atualizada; a segunda, sobre como ele trata as
buscas e trabalha inundando de pedidos para todos os nós que estão a uma certa
distância do usuário (nó-cliente). Por exemplo, se a distância limite for de 4, então
todos os nós que estiverem a 4 passos a partir do cliente serão verificados, começando
a partir dos mais próximos. Eventualmente, algum nó possuirá o arquivo requisitado e
responderá, e assim será feita a transferência desse arquivo. Muitos softwares que
implementam o protocolo vão além dessa funcionalidade básica de download simples,
tentando transferir de forma paralela partes diferentes do arquivo desejado de nós
diferentes, de forma a amenizar eventuais problemas de velocidade de rede.

Assim, experiências sugerem que o sistema escala para um tamanho maior, tornando o
mecanismo de \gls*{anycast} extremamente caro, e, em alguns casos, até proibitivo. O
problema ocorre nas buscas por arquivos menos populares, onde será necessário um maior
número de nós perguntados.

O \gls*{gnutella} ainda teve uma segunda versão \cite{wiki:gnutella2}, no final de 2002,
onde utilizou o mesmo protocolo que o original, porém, organizando a rede de
\glspl*{peer} em folhas (\emph{leafs}) e \emph{hubs}. Um \emph{hub} poderia
ter centenas de conexões à outras folhas, mas apenas 7 (em média) a outros \emph{hubs},
enquanto uma folha se conectaria a apenas 2 \emph{hubs} simultaneamente. Essa nova
topologia, somada com uma nova tabela de índice de arquivos das folhas mantida pelos
\emph{hubs} onde estavam conectados, melhorou o desempenho das buscas, que era ruim na versão antiga.

\subsection{eDonkey}

O protocolo \gls{edonkey} inovou em muitos aspectos em relação aos seus precursores,
tendo papel fundamental na história das redes \gls*{p2p}, consolidando-se como
ferramenta de compartilhamento especializado em arquivos grandes.

O \gls*{edonkey} implementou o primeiro método de download por \gls{swarming}, onde
\glspl*{peer} fazem downloads de diferentes partes de um arquivo e de \glspl*{peer}
diferentes, utilizando de forma efetiva a largura de banda de rede para todos os
\glspl*{peer}, ao invés de ficar limitado somente à banda de um único \gls*{peer}.

Outra melhoria deu-se na busca de arquivos: no seu lançamento, os servidores eram
separados entre si, porém, nas versões seguintes, permitiu-se que eles formassem uma
rede de buscas. Isso possibilitou que os servidores repassassem buscas de seus clientes
conectados localmente a outros servidores, facilitando a localização de \glspl*{peer}
conectados em qualquer servidor da rede de buscas, aumentando a capacidade de download
do enxame.

Diferentemente do Napster, o \gls*{edonkey} utilizou-se de \glspl{hashvalue} de arquivos
nos resultados de busca ao invés dos simples nomes dos arquivos. As buscas geradas
pelos usuários eram baseadas em palavras-chave e comparadas com a lista de nomes de
arquivos armazenada no servidor, mas o servidor retornava uma lista de pares de nomes
de arquivos com seus respectivos \glspl*{hashvalue}. Enfim, quando o usuário
selecionasse o arquivo desejado, o cliente iniciaria o download do arquivo usando o seu
\gls*{hashvalue}. Desse modo, um arquivo poderia ter muitos nomes entre os diferentes
\glspl*{peer} e servidores, mas seria considerado idêntico para download se possuísse o
mesmo \gls*{hashvalue}.

A arquitetura da rede em dois níveis, usando cliente e servidor, alcançou um meio termo
entre as redes centralizadas (como o Napster) e as descentralizadas (como o
\gls*{gnutella}), já que o servidor central no primeiro era um alvo garantido para ações
judiciais, enquanto o segundo mostrou-se inviável à propositura de tais ações, devido
ao tráfego massivo de buscas entre \glspl*{peer}.

Por fim, a inovação mais importante foi o uso de \glspl{dht}, em específico o
\gls{kademlia}, como algoritmo de indexação e busca nos servidores centrais dos
arquivos através da rede \gls*{edonkey}. Além de ser uma das razões da melhoria no
desempenho das pesquisas, os \glspl*{dht} possuem ainda outras características, tais
como tolerância a falhas e escalabilidade.

\section{Nascimento do BitTorrent}

Em meados dos anos 1990, Bram Cohen era um programador que tinha largado a faculdade no
segundo ano do curso de Ciência da Computação, da Universidade de Buffalo -- Nova
Iorque, para trabalhar em empresas ``ponto com''. A última delas foi a MojoNation, uma
empresa que desenvolvia um software de distribuição de arquivos criptografados por
\gls*{swarming}.

Em abril de 2001, Bram saiu da MojoNation e começou a modelar o protocolo BitTorrent,
lançando a primeira implementação usando a linguagem Python, em julho de 2001. Em
fevereiro de 2002, ele apresentou o seu trabalho na CodeCon \cite{site:codecon}, e na
mesma época começou a testá-lo, usando como chamariz uma coleção de material
pornográfico para atrair \glspl{betatester} \cite{site:bramcohen}. Assim, o software
começou a ser usado imediatamente.

Nesse meio tempo, Bram ainda passou pela Valve \cite{wiki:bramcohen}, empresa de
desenvolvimento de jogos, trabalhando no sistema de distribuição online do jogo Half
Life 2. Em 2004, saiu da Valve e voltou o foco ao Torrent. Em setembro, fundou a
BitTorrent Inc. com seu irmão Ross Cohen e o parceiro de negócios Ashwin Navin, sendo
então responsável pelo desenvolvimento do protocolo. Ainda naquele ano, surgiram os
primeiros programas de televisão e filmes compartilhados na rede através do BitTorrent,
o que popularizou o protocolo.

Em maio de 2005, a empresa lançou uma nova versão do BitTorrent, que não precisava de
\glspl{tracker}, juntamente com um site de buscas de conteúdo torrent na Internet. Em
setembro, a empresa recebeu investimento na ordem de \$8.5 milhões de dólares. No final
desse ano, a BitTorrent Inc. e a MPAA (\emph{Motion Picture Association of America},
associação americana de produtoras de filmes) fizeram um acordo \cite{wiki:mpaa}
visando a retirada dos conteúdos não autorizados dos representados pela associação, o
que não evitou a pirataria, pois já havia outros sites de busca de torrent sem
restrições de conteúdo, como o TorrentSpy, Mininova, The Pirate Bay, etc.

\section{Mundo pós-torrent}

Desde o fechamento de seu site de buscas, a BitTorrent Inc. tem desenvolvido outros
softwares baseados na tecnologia \gls*{p2p} \cite{site:bittorrent}, como transmissão de
vídeos ao vivo (BitTorrent Live), sincronização de arquivos entre computadores ligados
à Internet (BitTorrent Sync), publicação e distribuição de conteúdo de artistas a seus
fãs (BitTorrent Bundles), entre outros serviços comerciais.

Como protocolo, o BitTorrent criou um novo paradigma de transmissão de informações pela
Internet, sendo utilizado de inúmeras formas e motivos, tais como:

\begin{itemize}
    \item alguns softwares de podcasting, como o Miro \cite{site:miro}, passaram a
        usar o protocolo como forma de lidar com a grande quantidade de downloads de
        programas online;

    \item o site da gravadora DGM Live fornece o conteúdo via torrent após a venda
        \cite{site:dgm};

    \item VODO \cite{site:vodo} é um site de divulgação e distribuição de filmes sob a
        licença Creative Commons e que faz a publicação em outros sites de busca de
        torrents;

    \item canais como a americana CBC \cite{site:cbc} e a holandesa VPRO
        \cite{site:vpro} já disponibilizaram programas de sua grade para download. A
        norueguesa NRK o faz para conteúdos em HD \cite{site:nrk} e, apesar de algumas
        restrições de direitos, tem aumentado a oferta;

    \item o serviço Amazon S3, de armazenamento de conteúdo via web service, permite o
        uso de torrent para a transmissão de arquivos \cite{site:aws-s3};

    \item as empresas de desenvolvimento de jogos CCP Games (Eve Online) e Blizzard
        (Diablo III, StarCraft II e World of Warcraft) usaram o protocolo para
        distribuir o instalador de seu jogo \cite{site:eve}, e distribuir os jogos e
        suas eventuais atualizações \cite{site:blizzard}, respectivamente;

    \item o governo britânico distribuiu os detalhes de seus gastos \cite{site:gov-uk},
        enquanto a Universidade do Estado da Flórida utiliza para transmitir grandes
        conjuntos de dados científicos aos seus pesquisadores \cite{site:univ-fl};

    \item Facebook \cite{site:facebook-torrent} e Twitter \cite{site:twitter-torrent}
        o usam para atualizar os seus sites, enviando de forma eficiente o código novo
        para seus servidores de aplicação \cite{site:twitter-torrent-power}.
\end{itemize}

Em 2013, o BitTorrent é um dos maiores geradores de tráfego de rede do mundo, de forma
crescente, ao lado do NetFlix, Youtube, Facebook e acessos HTTP
\cite{report:internet-usage-2013}.

\subsection{Questões legais}

Desde que surgiu, o BitTorrent, bem como os outros protocolos \gls*{p2p}, chamou a
atenção dos defensores de direitos autorais, por conta do compartilhamento não
autorizado de arquivos protegidos por tais direitos, sendo alvo de medidas judiciais.
Porém, assim como o \gls*{gnutella}, e ao contrário do Napster, por possuir uma
estrutura descentralizada e não armazenar dados sobre os compartilhamentos realizados,
dificulta o trabalho de identificação das pessoas que compatilham esses dados.

Ainda assim, não existe um consenso sobre os efeitos financeiros do compartilhamento de
arquivos protegidos por direitos autorais, onde o principal argumento utilizado pelos
reclamantes é que estes têm grandes prejuízos e, por isso, entram com ações
indenizatórias de valores vultosos. Existem alguns estudos que tentam medir esses
prejuízos; um dos mais recentes, mostrou que não existem evidências de diminuição das
receitas das empresas cujo conteúdo é pirateado, e que o combate aos usuários
infratores não tem o impacto esperado, que é o de reduzir o compartilhamento desses
arquivos \cite{report:lse-piracy}.

\subsection{Estudos acadêmicos}

%\todoquestion{tá beeem resumido; desenvolve mais??}

Academicamente, o protocolo é bastante estudado desde o seu surgimento, sendo focos de
pesquisa os efeitos do algoritmo original e ajustes finos de seu funcionamento. Os
pontos principais são a parte algorítmica da troca de pedaços dos arquivos, estudos
sobre as topologias das redes formadas pelos \glspl*{peer} e melhoria da eficiência
do protocolo com alterações nessas topologias.

\afterpage{\clearpage}     % 2. Histórico
%!TEX root = ../tcc.tex

\chapter{Anatomia do BitTorrent}

Aqui mostrarei como funciona o BitTorrent de forma linear, como se fosse um humano o
utilizando.

\section{Busca por informações}

Falarei quais informações um arquivo torrent pode conter e como o Transmission as
utiliza, tentando mostrar detalhes da troca de mensagens dele com servidores de
rastreamento.

\section{Fontes de arquivos}

Mostrarei o processamento dos dados adquiridos na seção anterior e como ele organiza a lista das fontes de arquivos usando a tabela hash DHT Kademlia.

\section{Jogo da troca de arquivos}

Explicarei o algoritmo tit-for-tat padrão do protocolo BitTorrent, que vem da Teoria dos Jogos, e como o Transmission o implementa.

\afterpage{\clearpage}     % 3. Anatomia do BitTorrent
%!TEX root = ../tcc.tex

\chapter{BitTorrent e a Ciência da Computação}

Aqui mostrarei detalhes técnicos sobre as partes coadjuvantes do BitTorrent e do Transmission.


\section{Estruturas de dados, listas ligadas e árvores}
\section{Funções de hash} % SHA1
\section{Criptografia} %RC4
\section{Bitfields}
\section{Protocolos de redes} % HTTP e UDP
\section{Multicast}
\section{Roteamento de pacotes} %NAT PMP
\section{Retomada de downloads}
\section{Conexão com a Internet}
\section{Threads}
\section{Engenharia de Software}

\clearpage
     % 4. Ciência da Computação no Transmission
%!TEX root = ../tcc.tex

\chapter{Transmission e o BCC}
\label{chap:bcc}

Neste texto, descrevemos que são encontrados vários elementos de Ciência da Computação
no programa cliente Transmission. A intenção agora é encontrar quais disciplinas da
grade curricular do Bacharelado em Ciência da Computação (BCC) tem conhecimentos
utilizados no Transmission.

\begin{figure}[H]
    \centering
    \fbox{\includegraphics[width=.75\textwidth]{mapa-bcc.png}}
    \caption{disciplinas do BCC utilizadas diretamente na programação do Transmission,
    ligadas pelos seus pré-requisitos. As bordas arredondadas indicam que o
    conhecimento auxilia, mas não é necessário.}
    \label{fig:bcc}
\end{figure}

Das disciplinas do BCC, as reconhecidas como necessárias para o entendimento do
BitTorrent e o Transmission foram:

\begin{itemize}
    \item \textbf{MAC0110 --- Introdução à Computação} \\
        \textbf{MAC0122 --- Princípios de Desenvolvimento de Algoritmos} \\
        linguagem C e algoritmos básicos.

    \item \textbf{MAC0211 --- Laboratórios de Programação 1} \\
        organização de código, trabalho gerenciado em grupos de desenvolvedores,
        controle de versão de código fonte (Git), portabilidade de código entre
        plataformas (Autoconf e Automake), ferramentas de linha de comando UNIX (grep,
        find, sed), documentação (\LaTeX);

    \item \textbf{MAC0242 --- Laboratório de Programação 2} \\
        \textbf{MAC0332 --- Engenharia de Software} \\
        uso de IDE de desenvolvimento (Eclipse), criação de programas complexos,
        uso de bibliotecas (GTK+);

    \item \textbf{MAC0323 --- Estruturas de Dados} \\
        utilização e manipulação de estruturas de dados, como vetores e listas ligadas;

    \item \textbf{MAC0328 --- Algoritmos em Grafos} \\
        entendimento do algoritmo de busca de nós no \gls{dht};

    \item \textbf{MAC0328 --- Análise de Algoritmos} \\
        \textbf{MAC0446 --- Teoria dos Jogos Algorítmica} \\
        algoritmo da troca das partes entre \glspl{peer};

    \item \textbf{MAT0138 --- Álgebra 1} \\
        \textbf{MAC0336 --- Criptografia para Segurança de Dados} \\
        noções de Álgebra modular e seus usos nos métodos criptográficos;

    \item \textbf{MAC0448 --- Programação para Redes} \\
        desenvolvimento de código de conexão via Internet, como \gls{tcp} e \gls{udp},
        multicast, \gls{nat}, IPv6, UPnP e \gls*{nat}-PMP;

    \item \textbf{MAC0414 --- Linguagens Formais e Autômatos} \\
        uso de autômatos para entendimento dos estados que \glspl*{peer} podem assumir
        e suas transições com as trocas de mensagens \cite{conf:swarming}; e

    \item \textbf{MAC0422 --- Sistemas Operacionais} \\
        \textbf{MAC0438 --- Programação Concorrente} \\
        uso de \glspl{thread} e métodos para computação com seus usos, como acesso à
        memória compartilhada (\emph{mutex}) e travas, e entendimento de caches de
        memória na leitura e escrita de dados.
\end{itemize}

\afterpage{\clearpage}     % 5. Transmission e o BCC
%!TEX root = ../tcc.tex

\chapter{Referências bibliográficas}


\clearpage     % 6. Comentários Finais
%!TEX root = ../tcc.tex

% Glossário
% \glsaddall
\printglossary[type=main,style=altlist]

\afterpage{\clearpage}     % 7. Glossário
%!TEX root = ../tcc.tex

\chapter*{Visão Pessoal}


\afterpage{\clearpage}     % 8. Bibliografia
%!TEX root = ../tcc.tex

\chapter*{Visão Pessoal}

Aqui, apresento a minha visão sobre a experiência obtida neste trabalho, relacionando-a
o curso do BCC.

\section*{Desafios e frustrações}

Devo dizer que o tema BitTorrent não foi um assunto que eu desejei estudar a princípio.
Na verdade, eu pretendia mesmo era aplicar algum estudo de redes sociais usando grafos,
mas, quando eu contei meu plano para o professor Coelho, ele teve a (feliz e brilhante)
idéia de me sugerir este assunto. Eis que abracei o tema e o escolhi como orientador.

Após ter definido o tema, o primeiro desafio (que acredito ser comum a todos que cursam
esta disciplina) foi decidir o que escrever. Tive a mesma sensação que alguém deve ter
quando lhe dão uma folha em branco e pedem para desenhar algo que não conhece. Afinal,
como você vai fazer isso? Com que detalhes? Qual a abordagem? Foram muitas questões de
uma só vez.

Uma idéia inicial foi procurar na Internet trabalhos acadêmicos sobre o BitTorrent. Foi
aí que eu percebi que a área é abrangente, com artigos sobre muitos aspectos do
protocolo. Então, tracei como objetivo escrever um trabalho de análise do BitTorrent e
incluir um desses estudos já realizados. Durante o ano essa idéia caiu por terra,
pois percebi que a análise requeria muito mais tempo do que eu pensava, juntamente com
contratempos que foram ocorrendo, principalmente no segundo semestre.

Logo precisei decidir como escrever o trabalho. Uma idéia, que surgiu em uma das muitas
reuniões que tive com o professor Coelho, foi o de desenvolver o tema apresentando
código como se fizesse parte do texto. A sugestão soou estranha no começo mas, conforme
o trabalho foi crescendo, se consolidou como boa idéia.

Como o objetivo era apresentar código, avaliei os de alguns programas cliente
BitTorrent de código aberto. Não demorou muito para eu pensar no Transmission e
adotá-lo, dado que é o programa oficial da distribuição de Linux Ubuntu, que atualmente
é bastante usado.

Outro desafio que surgiu foi saber trabalhar com o \LaTeX. Eu não sabia usá-lo muito,
por isso tinha baixado um modelo preparado. Além disso, utilizava um editor online,
hospedado em um site, que logo mostrou que não era tão bom, quando percebi que o pacote
de formatação visual de código para \LaTeX\, não funcionava. Tive que decidir entre
manter o modelo de documento ou começar outro do arquivo em branco. Escolhi o segundo,
e até hoje não me arrependo, pois consegui deixar mais organizado e ao me gosto.

Durante minhas pesquisas, percebi que encontrava material sobre o BitTorrent na
Internet. Toda vez que buscava uma informação boa somente no Wikipedia, uma voz ecoava
dizendo que Wikipedia não é referência. Com o tempo, essa voz foi morrendo, pois cada
vez mais percebi que não era bem assim. Acredito que, com a rápida evolução tecnológica
que temos atualmente, é impossível depender somente de fontes estáticas, como livros.
Além disso, tive a oportunidade de tentar editar alguns artigos no Wikipedia, mais
precisamente melhorando-os com referências, e até para isso foi difícil. Foi aí que
percebi também que o antigo argumento de que ``qualquer um pode escrever qualquer coisa
no Wikipedia'' não é válido, e comecei a conviver com o fato de que eu não teria opções
melhores.

Um último desafio foi o de escrever o trabalho, dividindo o tempo entre ele e o
estágio, em um período bastante curto. Escrevo esta seção no dia anterior ao da entrega,
satisfeito com o resultado e com a sensação de dever cumprido.

\section*{Disciplinas relevantes ao trabalho}

Eu avaliei as disciplinas do BCC no capítulo \ref{chap:bcc}, na página \pageref{chap:bcc}.

\section*{Planos futuros}

A cada passo que eu dava no desenvolvimento do tema, percebi o quão fascinante é o
BitTorrent. Fiquei bastante contente com o fato dele se manter atual, mesmo com a certa
``mesmice'' que possui hoje em dia. Especificamente, a notícia de que é 7 vezes mais
rápido do que o Dropbox \cite{site:torrentvsdropbox} mostra como ele ainda é relevante.
Além disso, notícias como a que uma juíza americana estudou o protocolo antes de tomar
uma decisão em um julgamento envolvendo direitos autorais \cite{site:juizamanjona} me
fazem ter esperança de que, um dia, o BitTorrent deixe de ser sinônimo de pirataria.

A fascinação pelo assunto me fez querer entrar na comunidade BitTorrent e colaborar de
alguma forma, provavelmente no próprio Transmission, já que agora conheço boa parte do
seu código. Após algum descanso nas próximas semanas, espero entrar em contato com os
desenvolvedores e tentar ser voluntário no projeto.

%!TEX root = ../tcc.tex

\newpage
\section*{Agradecimentos}

Aqui vão alguns agradecimentos a pessoas que tornaram este trabalho possível.

Em especial, agradeço imensamente ao meu orientador, professor Coelho, pela paciência,
atenção e dedicação, me atendendo sempre que precisei de reuniões para este trabalho, e
até mesmo durante a graduação, quando achei que tudo estava distante. E obrigado por
ter sugerido o tema. Não acho que podia ter sido melhor! =)

Aos outros bons professores que tive no IME, cujas aulas marcaram a minha vida de
estudante e como profissional da Computação: Gubi, Roberto Hirata, Nina Hirata,
Carlinhos, Cristina, Ḿarcelo Queiroz, Marcelo Finger, Alfredo, Hitoshi e Reverbel.

Agradeço ao meu pai, minha mãe e minha irmã por terem compreendido mais uma fase de
ausência durante o curso, me incentivando em todas as decisões importantes que tive que
tomar durante toda a minha passagem pela USP, que se iniciou em 2004.

Agradeço à Tatiana, minha namorada, e à sua mãe Valéria, por me acolherem e me
acompanharem neste trabalho, sempre ao meu lado e de forma tão dedicada e paciente, me
apoiando nos momentos mais difíceis. Tati, te amo!

Aos meus gestores de trabalho: Prof. Mardel de Conti (LabNumeral - Poli-USP); Jason
Dyett (Harvard DRCLAS); Daniel Creão (UpLexis); Bruno Yoshimura e Allan Kajimoto
(Kekanto), pelas horas de trabalho flexíveis, que me permitiram chegar neste ponto do
curso do BCC, e em especial aos meus atuais colegas de LabArq da FAU-USP, Anderson
Valtriani, Ricardo Couto e amigos desenvolvedores pelas ausências permitidas para que
eu pudesse terminar este trabalho.

Aos meus amigos da turma de BCC 2009: Jackson, Renato, Samuel, Susana, Felipe, Rafael,
Thiago, Diogo, Gustavo Katague, Fernando, Henrique, Wilson, Gustavo Coelho, Wallace,
Jéssica, Jefferson, Tiago e Nádia (ex-IME), que nos momentos de desespero e de calmaria
me acompanharam por muitos dias durante o curso e, com certeza, suas amizades vão me
acompanhar pelo resto da vida.

Aos meus muitos amigos do BCC dos outros anos, em especial Luciano, Lucas, Marcos,
Edson, Rafael e Roberto, cuja amizade transcendeu o IME e hoje já fazem parte do meu
cotidiano.

E aos \textbf{muitos} outros nomes que conheço e com quem tive momentos no IME ou no
IF, nessa minha história pela USP, se estiver lendo este texto, saiba que também foi
muito importante! =)

E a todos, desculpem pela minha ausência. Eu estava cancelando o Apocalipse.

\afterpage{\clearpage}


\afterpage{\clearpage}     % 9. Parte Subjetiva

% Espaço no ToC, por estética
% \addtocontents{toc}{\vspace{2em}}

\end{document}